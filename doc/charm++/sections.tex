\charmpp{} supports defining and communicating with subsets of a chare
array or group.  This entity is called a chare array section or a group section (\emph{section}).
Section elements are addressed via a section proxy.
\charmpp{} also supports sections which are a subset of elements of
multiple chare arrays/groups of the same type (see \ref{cross array section}).

Multicast operations, a broadcast to all members of a section, are directly
supported by the section proxy. For array sections, multicast operations by default
use optimized spanning trees via the CkMulticast library in \charmpp{}.
For group sections, multicast operations by default use an unoptimized
direct-sending implementation.
To optimize messaging, group sections need to be manually delegated to CkMulticast
(see \ref{Manual Delegation}).
Reductions are also supported for both arrays and group sections via the CkMulticast library.

Array and group sections work in mostly the same way.
Check examples/charm++/groupsection for a group section example and
examples/charm++/arraysection for an array section example.

\section{Section Creation}
\label{section creation}

\subsection{Array sections}

For each chare array ``A'' declared in a ci file, a section proxy
of type ``CProxySection\_A'' is automatically generated in the decl and def
header files.
You can create an array section in your application by invoking ckNew() function of the CProxySection.
The user will need to provide array indexes
of all the array section members through either explicit enumeration, or an index range expression.
For example, for a 3D array:

\begin{alltt}
  CkVec<CkArrayIndex3D> elems;    // add array indices
  for (int i=0; i<10; i++)
    for (int j=0; j<20; j+=2)
      for (int k=0; k<30; k+=2)
         elems.push_back(CkArrayIndex3D(i, j, k));
  CProxySection_Hello proxy = CProxySection_Hello::ckNew(helloArrayID, elems.getVec(), elems.size());
\end{alltt}

Alternatively, one can do the same thing by providing the index range [lbound:ubound:stride]
for each dimension:

\begin{alltt}
  CProxySection_Hello proxy = CProxySection_Hello::ckNew(helloArrayID, 0, 9, 1, 0, 19, 2, 0, 29, 2);
\end{alltt}

The above code creates a section proxy that contains array elements
[0:9, 0:19:2, 0:29:2].

For user-defined array index other than CkArrayIndex1D to CkArrayIndex6D,
one needs to use the generic array index type: CkArrayIndex.

\begin{alltt}
  CkArrayIndex *elems;    // add array indices
  int numElems;
  CProxySection_Hello proxy = CProxySection_Hello::ckNew(helloArrayID, elems, numElems);
\end{alltt}

\subsection{Group sections}

Group sections are created in the same way as array sections.
A group ``A'' will have an associated ``CProxySection\_A'' type which is used to create a section
and obtain a proxy.
In this case, ckNew() will receive the list of PE IDs which will form the section.
See examples/charm++/groupsection for an example.

It is important to note that \charmpp{} does not automatically delegate group sections to the
internal CkMulticast library,
and instead defaults to a point-to-point implementation of multicasts.
To use CkMulticast with group sections, the user must manually delegate after invoking group creation.
See \ref{Manual Delegation} for information on how to do this.

\subsection{Creation order restrictions}

Important: Array sections should be created in post-constructor entry methods to avoid race conditions.

If the user wants to invoke section creation from a group, special care must be taken
that the collection for which we are creating a section (array or group) already exists.

For example, suppose a user wants to create a section of array ``A'' from an entry
method in group ``G''. Because groups are created before arrays in \charmpp{}, and
there is no guarantee of creation order of groups, there is a risk that array A's
internal structures have not been initialized yet on every PE, causing section creation to fail.
As such, the application must ensure that A has been created
before attempting to create a section.

If the section is created from inside an array element there is no such risk.

\section{Section Multicasts}
\label {array_section_multicast}

Once the proxy is obtained at section creation time, the user can broadcast to all the
section members, or send messages to one member using its offset index within the section, like this:

\begin{alltt}
  CProxySection_Hello proxy;
  proxy.someEntry(...)          // section broadcast
  proxy[0].someEntry(...)       // send to the first element in the section
\end{alltt}

See examples/charm++/arraysection for an example on how sections are used.

You can send the section proxy in a message to another processor, and still
safely invoke the entry functions on the section proxy.

\subsection{Optimized multicast via CkMulticast}

\charmpp{} has an inbuilt CkMulticast library that optimizes section communications.
By default, charm RTS will use this library for array sections. For group sections,
the user must manually delegate to CkMulticast (see \ref{Manual Delegation}).

By default, CkMulticast builds a spanning tree for multicast/reduction
with a factor of 2 (binary tree). One can specify a different branching factor when creating the
section.
\begin{alltt}
  CProxySection_Hello sectProxy = CProxySection_Hello::ckNew(..., 3); // factor is 3
\end{alltt}

Note, to use CkMulticast library, all multicast messages must inherit from
CkMcastBaseMsg, as the following example shows.
Note that CkMcastBaseMsg must come first, this is IMPORTANT for CkMulticast
library to retrieve section information out of the message.


\begin{alltt}
class HiMsg : public CkMcastBaseMsg, public CMessage_HiMsg
\{
public:
  int *data;
\};
\end{alltt}

Due to this restriction, when using CkMulticast you must define messages explicitly for multicast
entry functions and no parameter marshalling can be used.

\section{Section Reductions}

Reductions over the elements of a section are supported through the CkMulticast library.
As such, to perform reductions, the section must have been delegated to
CkMulticast, either automatically (which is the default case for array sections),
or manually for group sections.

Since an array element can be a member of multiple array sections,
it is necessary to disambiguate between which array
section reduction it is participating in each time it contributes to one. For this purpose, a data structure
called ``CkSectionInfo'' is created by CkMulticast library for each
array section that the array element belongs to.
During a section reduction, the array element must pass the
\kw{CkSectionInfo} as a parameter in the \kw{contribute()}.
The \kw{CkSectionInfo} for a section can be retrieved
from a message in a multicast entry point using function call
\kw{CkGetSectionInfo}:

\begin{alltt}
  CkSectionInfo cookie;

  void SayHi(HiMsg *msg)
  \{
    CkGetSectionInfo(cookie, msg);     // update section cookie every time
    int data = thisIndex;
    CProxySection_Hello::contribute(sizeof(int), &data, CkReduction::sum_int, cookie, cb);
  \}
\end{alltt}


Note that the cookie cannot be used as a one-time local variable in the 
function, the same cookie is needed for the next contribute. This is 
because the cookie includes some context sensive information (e.g., the 
reduction counter). Subsequent invocations of \kw{CkGetSectionInfo()} only updates 
part of the data in the cookie, rather than creating a brand new one.

Similar to array reductions, to use section-based reductions, a
reduction client CkCallback object must be created. You may pass the
client callback as an additional parameter to \kw{contribute}. If
different contribute calls to the same reduction operation pass
different callbacks, some (unspecified, unreliable) callback will be
chosen for use. 

See the following example:

\begin{alltt}
    CkCallback cb(CkIndex_myArrayType::myReductionEntry(NULL),thisProxy); 
    CProxySection_Hello::contribute(sizeof(int), &data, CkReduction::sum_int, cookie, cb);
\end{alltt}

As in an array reduction, users can use built-in reduction
types (Section~\ref{builtin_reduction}) or define his/her own reducer functions
(Section~\ref{new_type_reduction}).

\section{Section Operations and Migration}

When using a section reduction, you don't need to worry about migrations of array elements.
When migration happens, an array element in the array section can still use
the \kw{CkSectionInfo} it stored previously for doing a reduction.
Reduction messages will be correctly delivered but may not be as efficient 
until a new multicast spanning tree is rebuilt internally 
in the \kw{CkMulticastMgr} library.
When a new spanning tree is rebuilt, an updated \kw{CkSectionInfo} is
passed along with a multicast message,
so it is recommended that 
\kw{CkGetSectionInfo()} function is always called when a multicast 
message arrives (as shown in the above SayHi example).

In the case where a multicast root migrates, the library must reconstruct the
spanning tree to get optimal performance. One will get the following
warning message if this is not done:
``Warning: Multicast not optimized after multicast root migrated.''
In the current implementation, the user needs to initiate the rebuilding process
using \kw{resetSection}.


\begin{alltt}
void Foo::pup(PUP::er & p) {
    // if I am multicast root and it is unpacking
   if (ismcastroot && p.isUnpacking()) {
      CProxySection_Foo   fooProxy;    // proxy for the section
      fooProxy.resetSection(fooProxy);
        // you may want to reset reduction client to root
      CkCallback *cb = new CkCallback(...);
   }
}
\end{alltt}

\section{Cross Array Sections}
\label{cross array section}
\experimental{}

Cross array sections contain elements from multiple arrays.
Construction and use of cross array sections is similar to normal
array sections with the following restrictions.

\begin{itemize}

\item Arrays in a section must all be of the same type.

\item Each array must be enumerated by array ID.

\item The elements within each array must be enumerated explicitly.

\item No existing modules currently support delegation of cross
  section proxies. Therefore reductions are not currently supported.

\end{itemize}

Note: cross section logic also works for groups with analogous characteristics.

Given three arrays declared thusly:

\begin{alltt}
	  CkArrayID *aidArr= new CkArrayID[3];
	  CProxy\_multisectiontest\_array1d *Aproxy= new CProxy\_multisectiontest\_array1d[3];
	  for(int i=0;i<3;i++)
	    \{
	      Aproxy[i]=CProxy\_multisectiontest\_array1d::ckNew(masterproxy.ckGetGroupID(),ArraySize);	  
	      aidArr[i]=Aproxy[i].ckGetArrayID();
	    \}
\end{alltt}

One can make a section including the  lower half elements of all three
arrays as follows:

\begin{alltt}
	  int aboundary=ArraySize/2;
	  int afloor=aboundary;
	  int aceiling=ArraySize-1;
	  int asectionSize=aceiling-afloor+1;
	  // cross section lower half of each array
	  CkArrayIndex **aelems= new CkArrayIndex*[3];
	  aelems[0]= new CkArrayIndex[asectionSize];
	  aelems[1]= new CkArrayIndex[asectionSize];
	  aelems[2]= new CkArrayIndex[asectionSize];
	  int *naelems=new int[3];
	  for(int k=0;k<3;k++)
	    \{
	      naelems[k]=asectionSize;
	      for(int i=afloor,j=0;i<=aceiling;i++,j++)
	        aelems[k][j]=CkArrayIndex1D(i);
	    \}
	  CProxySection\_multisectiontest\_array1d arrayLowProxy(3,aidArr,aelems,naelems);
\end{alltt}

The resulting cross section proxy, as in the example \uw{arrayLowProxy},
can then be used for multicasts in the same way as a normal array
section.

Note: For simplicity the example has all arrays and sections of uniform
size.  The size of each array and the number of elements in each array
within a section can all be set independently.




%=============================================================================================
\section{Manual Delegation}
\label{Manual Delegation}

By default \charmpp{} uses the CkMulticast library for optimized broadcasts and
reductions on array sections, but advanced \charmpp{} users can choose to
delegate\footnote{See chapter \ref{delegation} for general information on message delegation in \charmpp{}.}
sections to custom libraries (called delegation managers).
Note that group sections are not automatically
delegated to CkMulticast and hence must be manually delegated to this library
to benefit from the optimized multicast tree implementation.
This is explained in this section, and see examples/charm++/groupsection for an example.

While creating a chare array one can set the auto delegation flag to
false in CkArrayOptions and the runtime system will not use the default CkMulticast library.
A CkMulticastMgr (or any other delegation manager) group can then be created by the user, and any
section delegated to it.

One only needs to create one delegation manager group, and it
can serve all multicast/reduction delegations for different array/group sections in an application.
In the following we show a manual delegation example using CkMulticast (the same can be applied
to custom delegation managers):

\begin{alltt}
  CkArrayOptions opts(...);
  opts.setSectionAutoDelegate(false); // manual delegation
  CProxy_Hello arrayProxy = CProxy_Hello::ckNew(opts,...);
  CProxySection_Hello sectProxy = CProxySection_Hello::ckNew(...);
  CkGroupID mCastGrpId = CProxy_CkMulticastMgr::ckNew();
  CkMulticastMgr *mCastGrp = CProxy_CkMulticastMgr(mCastGrpId).ckLocalBranch();

  sectProxy.ckSectionDelegate(mCastGrp);  // initialize section proxy

  sectProxy.someEntry(...)           // multicast via delegation library as before

\end{alltt}

One can also set the default branching factor when creating a CkMulticastMgr group.
Sections created via this manager will use the specified branching factor for their multicast tree.
For example,

\begin{alltt}
  CkGroupID mCastGrpId = CProxy_CkMulticastMgr::ckNew(3);   // factor is 3
\end{alltt}

Contributing using a custom CkMulticastMgr group:

\begin{alltt}
  CkSectionInfo cookie;

  void SayHi(HiMsg *msg)
  \{
    CkGetSectionInfo(cookie, msg);     // update section cookie every time
    int data = thisIndex;
    CkCallback cb(CkIndex_myArrayType::myReductionEntry(NULL),thisProxy);
    mcastGrp->contribute(sizeof(int), &data, CkReduction::sum_int, cookie, cb);
  \}
\end{alltt}


Setting default reduction client for a section when using manual delegation:

\begin{alltt}
  CProxySection_Hello sectProxy;
  CkMulticastMgr *mcastGrp = CProxy_CkMulticastMgr(mCastGrpId).ckLocalBranch();
  mcastGrp->setReductionClient(sectProxy, new CkCallback(...));
\end{alltt}

Writing the pup method:

\begin{alltt}
void Foo::pup(PUP::er & p) {
    // if I am multicast root and it is unpacking
   if (ismcastroot && p.isUnpacking()) {
      CProxySection_Foo   fooProxy;    // proxy for the section
      CkMulticastMgr *mg = CProxy_CkMulticastMgr(mCastGrpId).ckLocalBranch();
      mg->resetSection(fooProxy);
        // you may want to reset reduction client to root
      CkCallback *cb = new CkCallback(...);
      mg->setReductionClient(mcp, cb);
   }
}
\end{alltt}
