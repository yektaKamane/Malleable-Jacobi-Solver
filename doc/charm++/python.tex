The Python scripting language in \charmpp{} allows the user to dynamically
execute pieces of code inside a running application, without the need to
recompile. This is performed through the CCS (Converse Client Server) framework
(see ``Converse Manual'' for more information about this). The user specifies
which elements of the system will be accessible through the interface, as we
will see later, and then run a client which connects to the server.

In order to exploit this functionality, Python interpreter needs to be installed
into the system, and \charmpp{} LIBS need to be built with:\\
\texttt{./build LIBS $<$arch$>$ $<$options$>$}

The interface provides three different types of requests:

\begin{description}
\item[Execute] requests to execute a code, it will contain the code to be executed on the server, together with the instructions on how to handle the environment;
\item[Print] asks the server to send back all the strings which has been printed by the script until now;
\item[Finished] asks the server if the current script has finished or it is still running.
\end{description}

There are three modes to run code on the server, ordered here by increase of
functionality, and decrease of dynamic flexibility:
\begin{itemize}
\item \textbf{simple read/write} By implementing the \kw{read} and \kw{write} methods
of the object exposed to python, in this way single variables may be exposed,
and the code will have the possibility to modify them individually as desired.
(see section~\ref{pythonServerRW})
\item \textbf{iteration} By implementing the iterator functions in the server (see
\ref{pythonServerIterator}), the user can upload the code of a Python function
and a user-defined iterator structure, and the system will apply the specified
function to all the objects reflected by the iterator structure.
\item \textbf{high level} By implementing \kw{python} entry methods, the Python code uploaded can access them and activate complex, parallel operations that will be performed by the \charmpp{} application. (see section~\ref{pythonHighLevel})
\end{itemize}

The description will follow the client implementation first, and continuing then
on the server implementation.


\subsubsection{The client side}

\label{pythonClient}

In order to facilitate the interface between the client and the server, some
classes are available to the user to include into the client. Currently \CC{} and
java interfaces are provided.

\CC{} programs need to include \texttt{PythonCCS-client.h} into their
code. This file is among the \charmpp{} include files. For java, the package
\texttt{charm.ccs} needs to be imported. This is located under the java
directory on the \charmpp{} distribution, and it provides both the Python and
CCS interface classes.

There are three main classes provided: \texttt{PythonExecute},
\texttt{PythonPrint}, and \texttt{PythonFinished} which are used for the three
different types of request.

All of them have two common methods to enable communication across different platforms:

\begin{description}

\item[int size();]
Returns the size of the class, as number of bytes that will be
transmitted through the network (this includes the code and other dynamic
variables in the case of \texttt{PythonExecute}).

\item[char *pack();]
Returns a new memory location containing the data to be sent to the server, this
is the data which has to be passed to the \texttt{CcsSendRequest} function. The
original class will be unmodified and can be reused in subsequent calls.

\end{description}

A tipical invocation to send a request from the client to the server has the
following format:

\begin{alltt}
CcsSendRequest (&server, "pyCode", 0, request.size(), request.pack());
\end{alltt}

\subsubsection{PythonExecute}

\label{pythonExecute}

To execute a Python script on a running server, the client has to create an
instance of \texttt{PythonExecute}, the two constructors have the following
signature (java has a correspondent functionality):

\begin{alltt}
PythonExecute(char *code, bool persistent=false, bool highlevel=false, CmiUInt4 interpreter=0);
PythonExecute(char *code, char *method, PythonIterator *info, bool persistent=false,
              bool highlevel=false, CmiUInt4 interpreter=0);
\end{alltt}

The second one is used for iterative requests (see~\ref{pythonIterator}). The
only required argument is the code, a null terminated string, which will not be
modified by the system. All the other parameters are optional. They refer to the
possible variants that an execution request can be. In particular, this is a
list of all the options present:

\begin{description}

\item[iterative]
If the request is a single code (false) or if it represents a function over
which to iterate (true) (see~\ref{pythonIterator} for more details).

\item[persistent]
It is possible to store information on the server which will be retained across
different client calls (such as simple data or complete libraries). True means
that the information will be retained on the server, false means that the
information will be deleted when the script terminates to run. In order to
properly dispose the memory, when the last call is made (and the data is not
anymore needed), this flag should be set to false. When information has been
stored on the server, in order to reuse it, the interpreter field of the request
should be set to the correct value (which was returned by the previous call, see
later in this subsection).

\item[high level]
In order to have the ability to call high level \charmpp{} functions (available
through the keyword \kw{python}) this flag must be set to true. If it is false,
the entire module ``charm'' will not be present, but the startup of the script
will be faster.

\item[print retain]
When the client decides to print some output back to the client, this data can be
retrieved with a PythonPrint request. If the output is not desired, this flag
can be set to false, and the output will be discarded. If it is set to true the
output will be saved waiting to be retrieved by the client. The data will
survive also after the termination of the Python script, and if not retrieved
will waste memory on the server.

\item[busy waiting]
Instead of returning immediately to the client a handle that can be used to
retrieve prints and check if the script has finished, the server will answer to
the client only when the script has terminated to run (and it will effectively
work as a PythonFinished request).

\end{description}

These flags can be set and checked with the following routines (CmiUInt4 represent a 4
byte unsigned integer):

\begin{alltt}
void setCode(char *set);
void setPersistent(bool set);
void setIterate(bool set);
void setHighLevel(bool set);
void setKeepPrint(bool set);
void setWait(bool set);
void setInterpreter(CmiUInt4 i);

bool isPersistent();
bool isIterate();
bool isHighLevel();
bool isKeepPrint();
bool isWait();
CmiUInt4 getInterpreter();
\end{alltt}

From a PythonExecute request, the server will answer with a 4 byte integer
value, which is a handle for the interpreter that is running. It can be used to
request for prints, check if the script has finished, and for reusing the same
interpreter (if it was persistent).

A value of 0 means that there was an error and the script didn't run. This is
typically due to a request to reuse of an existing interpreter which is not
available, either because it was not persistent or because another script is
still running on that interpreter.


\subsubsection{Auto-imported modules}

\label{pythonModules}

When a Python script is run inside a \charmpp{} application, two Python modules
are made available by the system. One is \textbf{ck}, the other is
\textbf{charm}. The first one is always present and it represent basic
functions, the second is related to high level scripting and it is present only
when this is enabled (see \ref{pythonExecute} for how to enable it, and
\ref{pythonHighLevel} for a description on how to implement charm functions).

The methods present in the \texttt{ck} module are the following:

\begin{description}

\item[printstr]
It accepts a string as parameter. It will write into the server stdout that string
using the \texttt{CkPrintf} function call.

\item[printclient]
It accepts a string as parameter. It will forward the string back to the client when it
issues a PythonPrint request. It will buffer the strings if the \texttt{KeepPrint}
option is true, otherwise it will discard them.

\item[mype]
It requires no parameters, and it will return an integer representing the
current processor where the code is executing. It is equivalent to the \charmpp{}
function \texttt{CkMyPe()}.

\item[numpes]
It requires no parameters, and it will return an integer representing the
total number of processors that the application is using. It is equivalent to
the \charmpp{} function \texttt{CkNumPes()}.

\item[myindex]
It requires no parameters, and it will return the index of the current element
inside the array, if the object under which Python is running is an array, or
None if it is running under a Chare, a Group or a Nodegroup. The index will be a
tuple containing as many numbers as the dimension of the array.

\item[read]
It accepts one object parameter, and it will perform a read request to the
\charmpp{} object connected to the Python script, and return an object
containing the data read (see \ref{pythonServerRW} for a description of this
functionality). An example of a call can be:
\function{value = ck.read((number, param, var2, var3))}
where the double parenthesis are needed to create a single tuple object
containing four values passed as a single paramter, instead of four different
parameters.

\item[write]
It accepts two object parameters, and it will perform a write request to the
\charmpp{} object connected to the Python script. For a description of this
method, see \ref{pythonServerRW}. Again, only two objects need to be passed, so
extra parenthesis may be needed to create tuples from individual values.

\end{description}

\subsubsection{Iterate mode}

\label{pythonIterator}

Sometimes some operations need to be iterated over all the elements in the
system. This ``iterative'' functionality provides a shortcut for the client user
to do this. As an example, suppose we have a system which contains particles,
with their position, velocity and mass. If we implement \texttt{read} and
\texttt{write} routines which allow us to access single particle attributes, we may
upload a script which doubles the mass of the particles with velocity greater
than 1:

\begin{alltt}
size = ck.read((``numparticles'', 0));
for i in range(0, size):
    vel = ck.read((``velocity'', i));
    mass = ck.read((``mass'', i));
    mass = mass * 2;
    if (vel > 1): ck.write((``mass'', i), mass);
\end{alltt}

Instead of all these read and writes, it will be better to be able to write:

\begin{alltt}
def increase(p):
    if (p.velocity > 1): p.mass = p.mass * 2;
\end{alltt}

This is what the ``iterative'' functionality provides. In order for this to
work, the server has to implement two additional functions
(see~\ref{pythonServerIterator}), and the client has to pass some more
information together with the code. This information is the name of the function
that has to be called (which can be defined in the ``code'' or have already been
uploaded to a persistent interpreter), and a user defined structure which
specifies over what data the function should be invoked. These values can be
specified either while constructing the PythonExecute variable (see the second
constructor in section~\ref{pythonExecute}), or with the following methods:

\begin{alltt}
void setMethodName(char *name);
void setIterator(PythonIterator *iter);
\end{alltt}

As for the PythonIterator object, it has to be a class defined by the user, and
the user has to insure that the same definition is present inside both the
client and the server. The \charmpp{} system will simply pass this structure as
a void pointer. This structure needs to inherit from \texttt{PythonIterator}. It
is recommended that no pointers are used inside this class, and no dynamic
memory allocation. If this is the case, nothing else needs to be done.

If instead pointers and dynamic memory allocation is used, the following methods
have to be reimplemented:

\begin{alltt}
int size();
char * pack();
void unpack();
\end{alltt}

The first returns the size of the class/structure after being packed. The second
returns a pointer to a newly allocated memory containing all the packed data,
the returned memory must be compatible with the class itself, since later on
this same memory a call to unpack will be performed. Finally, the third will do
the work opposite to pack and fix all the pointers. This method will not return
anything and is supposed to fix the pointers ``inline''.

\subsubsection{PythonPrint}

\label{pythonPrint}

In order to receive the output printed by the Python script, the client needs to
send a PythonPrint request to the server. The constructor is:

\function{PythonPrint(CmiUInt4 interpreter, bool Wait=true, bool Kill=false);}

The interpreter for which the request is made is mandatory. The other parameters
are optional. The wait parameter represents whether a reply will be sent back
immediately to the client even if there is no output (false), or if the answer
will be delayed until there is an output (true). The \kw{kill} option set to
true means that this is not a normal request, but a signal to unblock the latest
print request which was blocking.

The returned data will be a non null-terminated string if some data is present
(or if the request is blocking), or a 4 byte zero data if nothing is present.
This zero reply can happen in different situations:

\begin{itemize}
\item If the request is non blocking and no data is available on the server;
\item If a kill request is sent, the previous blocking request is squashed;
\item If the Python code ends without any output and it is not persistent;
\item If another print request arrives, the previous one is squashed and the second one is kept.
\end{itemize}

As for a print kill request, no data is expected to come back, so it is safe to
call \texttt{CcsNoResponse(server)}.

The two options can also be dynamically set with the following methods:

\begin{alltt}
void setWait(bool set);
bool isWait();

void setKill(bool set);
bool isKill();
\end{alltt}

\subsubsection{PythonFinished}

\label{pythonFinished}

In order to know when a Python code has finished executing, especially when
using persistent interpreters, and a serialization of the scripts is needed, a
PythonFinished request is available. The constructor is the following:

\function{PythonFinished(CmiUInt4 interpreter, bool Wait=true);}

The interpreter corresponds to the handle for which the request was sent, while
the wait option refers to a blocking call (true), or immediate return (false).

The wait option can be dynamically modified with the two methods:

\begin{alltt}
void setWait(bool set);
bool isWait();
\end{alltt}

This request will return a 4 byte integer containing the same interpreter value
if the Python script has already finished, or zero if the script is still
running.

\subsubsection{The server side}

\label{pythonServer}

In order for a \charmpp{} object (chare, array, node, or nodegroup) to receive
python requests, it is necessary to define it as python-compliant. This is done
through the keyword \kw{python} placed in square brackets before the object name
in the .ci file. Some examples follow:

\begin{alltt}
mainchare [python] main \{\ldots\}
array [1D] [python] myArray \{\ldots\}
group [python] myGroup \{\ldots\}
\end{alltt}

In order to register a newly created object to receive Python scripts, the
method \texttt{registerPython} of the proxy should be called. As an example,
the following code creates a 10 element array myArray, and then registers it to
receive scripts directed to ``pycode''. The argument of \texttt{registerPython}
is the string that CCS will use to address the Python scripting capability of
the object.

\begin{alltt}
Cproxy_myArray localVar = CProxy_myArray::ckNew(10);
localVar.registerPython(``pycode'');
\end{alltt}


\subsubsection{Server \kw{read} and \kw{write} functions}

\label{pythonServerRW}

As explained previously in subsection~\ref{pythonModules}, some functions are
automatically made available to the scripting code through the {\em ck} module.
Two of these, \textbf{read} and \textbf{write} are only available if redefined
by the object. The signatures of the two methods to redefine are:

\begin{alltt}
PyObject* read(PyObject* where);
void write(PyObject* where, PyObject* what);
\end{alltt}

The read function receives as a parameter an object specifying from where the data
will be read, and returns an object with the information required. The write
function will receive two parameters: where the data will be written and what
data, and will perform the update. All these \texttt{PyObject}s are generic, and
need to be coherent with the protocol specified by the application. In order to
parse the parameters, and create the value of the read, please refer to the
manual \htmladdnormallink{``Extending and Embedding the Python Interpreter''}{http://docs.python.org/}, and in particular to the functions
\texttt{PyArg\_ParseTuple} and \texttt{Py\_BuildValue}.

\subsubsection{Server iterator functions}

\label{pythonServerIterator}

In order to use the iterative mode as explained in
subsection~\ref{pythonIterator}, it is necessary to implement two functions
which will be called by the system. These two functions have the following
signatures:

\begin{alltt}
int buildIterator(PyObject*, void*);
int nextIteratorUpdate(PyObject*, PyObject*, void*);
\end{alltt}

The first one is called once before the first execution of the Python code, and
receives two parameters. The first is a pointer to an empty PyObject to be filled with
the data needed by the Python code. In order to manage this object, some utility
functions are provided. They are explained in subsection~\ref{pythonUtilityFuncs}.

The second is a void pointer containing information of what the iteration should
run over. This parameter may contain any data structure, and an agreement between the
client and the user object is necessary. The system treats it as a void pointer
since it has no information of what user defined data it contains.

The second function (\texttt{nextIteratorUpdate}) has three parameters. The
first contains the object to be filled like in \texttt{buildIterator}, but this
time the object contains the PyObject which was provided for the last iteration,
potentially modified by the Python function. Its content can be read with the
provided routines, used to retrieve the next logical element in the iterator
(with which to update the parameter itself), and possibly update the content of
the data inside the \charmpp{} object. The second parameter is the object
returned by the last call to the Python function, and the third parameter is the
same data structure passed to \texttt{buildIterator}.

Both functions return an integer which will be interpreted by the system as follows:
\begin{description}
\item[1] - a new iterator in the first parameter has been provided, and the Python function should be called with it;
\item[0] - there are no more elements to iterate.
\end{description}

\subsubsection{Server utility functions}

\label{pythonUtilityFuncs}

They are inherited when declaring an object as Python-compliant, and therefore
they are available inside the object code. All of them accept a PyObject pointer
where to read/write the data, a string with the name of a field, and one or two
values containing the data to be read/written (note that to read the data from
the PyObject, a pointer needs to be passed). The strings used to identify the
fields will be the same strings that the Python script will use to access the
data inside the object.

The name of the function identifies the type of Python object stored inside the
PyObject container (i.e String, Int, Long, Float, Complex), while the parameter
of the functions identifies the \CC object type.

\begin{alltt}
void pythonSetString(PyObject*, char*, char*);
void pythonSetString(PyObject*, char*, char*, int);
void pythonSetInt(PyObject*, char*, long);
void pythonSetLong(PyObject*, char*, long);
void pythonSetLong(PyObject*, char*, unsigned long);
void pythonSetLong(PyObject*, char*, double);
void pythonSetFloat(PyObject*, char*, double);
void pythonSetComplex(PyObject*, char*, double, double);

void pythonGetString(PyObject*, char*, char**);
void pythonGetInt(PyObject*, char*, long*);
void pythonGetLong(PyObject*, char*, long*);
void pythonGetLong(PyObject*, char*, unsigned long*);
void pythonGetLong(PyObject*, char*, double*);
void pythonGetFloat(PyObject*, char*, double*);
void pythonGetComplex(PyObject*, char*, double*, double*);
\end{alltt}

To handle more complicated structures like Dictionaries, Lists or Tuples, please refer to \htmladdnormallink{``Python/C API Reference Manual''}{http://docs.python.org/}.

\subsubsection{High level scripting}

\label{pythonHighLevel}

When in addition to the definition of the \charmpp{} object as \kw{python}, an
entry method is also defined as \kw{python}, this entry method can be accessed
directly by a Python script through the {\em charm} module. For example, the
following definition will be accessible with the python call:
\function{result = charm.highMethod(var1, var2, var3)}
It can accept any number of parameters (even complex like tuples or
dictionaries), and it can return an object as complex as needed.

The method must have the following signature:

\begin{alltt}
entry [python] void highMethod(int handle);
\end{alltt}

The parameter is a handle that is passed by the system, and can be used in
subsequent calls to return values to the Python code. %Thus, if the method
%does not return immediately but it sends out messages to other \charmpp{}
%objects, the handle must be saved somewhere. \textbf{Note:} if another Python
%script is sent to the server, this second one could also call the same function.
%If this is possible, the handle should be saved in a non-scalar variable.

The arguments passed by the Python caller can be retrieved using the function:

\function{PyObject *pythonGetArg(int handle);}

which returns a PyObject. This object is a Tuple containing a vector of all
parameters. It can be parsed using \texttt{PyArg\_ParseTuple} to extract the
single parameters.

When the \charmpp's entry method terminates (by means of \texttt{return} or
termination of the function), control is returned to the waiting Python script.
Since the \kw{python} entry methods execute within an user-level thread, it is
possible to suspend the entry method while some computation is carried on in
\charmpp. To start parallel computation, the entry method can send regular messages,
as every other threaded entry method (see~\ref{libraryInterface} for more
information on how this can be done using CkCallbackResumeThread callbacks). The
only difference with other threaded entry methods is that here the callback
\texttt{CkCallbackPython} must be used instead of CkCallbackResumeThread. The
more specialized CkCallbackPython callback works exactly like the other one,
except that it handles correctly Python internal locks.

At the end of the computation, if a value needs to be returned to the Python script,
the following special returning function has to be used:

\function{void pythonReturn(int handle, PyObject* result);}

where the second parameter is the Python object representing the returned value.
The function \texttt{Py\_BuildValue} can be used to create this value. This
function in itself does not terminate the entry method, but only sets the
returning value for Python to read when the entry method terminates.

A characteristic of Python is that in a multithreaded environment (like the one
provided in \charmpp{}), the running thread needs to keep a lock to prevent
other threads to access any variable. When using high level scripting, and the
Python script is suspended for long periods of time while waiting for the
\charmpp{} application to perform the required task, the Python internal locks
are automatically released and re-acquired by the \texttt{CkCallbackPython}
class when it suspends.

% This can be done using the two functions:

%\begin{alltt}
%void pythonAwake(int handle);   // to acquire the lock
%void pythonSleep(int handle);   // to release the lock
%\end{alltt}

%Important to remember is that before any Python value is accessed, the Python
%interpreter must be awake. This include the functions \texttt{Py\_BuildValue} and
%\texttt{PyArg\_ParseTuple}. \textbf{Note:} it is an error to call these functions
%more than once before the other one is called.
