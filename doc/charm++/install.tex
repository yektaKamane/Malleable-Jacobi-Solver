\section{Installing \charmpp{}}

You can install \charmpp{} as either source code or a precompiled
binary package.  Downloading source code is more flexible, since you
can choose the options you want; but a precompiled binary is slightly
easier to get running.

You begin by downloading \charmpp{} from our web site:
	http://charm.cs.uiuc.edu/download.html

Unpack \charmpp{} using a tool capable of extracting gzip'd tar
files, such as tar (on Unix) or WinZIP (under Windows).  \charmpp{}
will be extracted to a directory called ``charm''.
If you choose the source distribution, read the included 
``charm/README'' file for detailed instructions on building \charmpp{}
from source.

The main directories in a \charmpp{} installation are:

\begin{description}
\item[\kw{charm/bin}]
Executables, such as charmc and charmrun,
used by \charmpp{}.

\item[\kw{charm/doc}]
Documentation for \charmpp{}, such as this
document.  Distributed as LaTeX source code; HTML and PDF versions
can be built or downloaded from our web site.

\item[\kw{charm/include}]
The \charmpp{} C++ and Fortran user include files (.h).

\item[\kw{charm/lib}]
The libraries (.a) that comprise \charmpp{}.

\item[\kw{charm/pgms}]
Example \charmpp{} programs.

\item[\kw{charm/src}]
Source code for \charmpp{} itself.

\item[\kw{charm/tmp}]
Directory where \charmpp{} is built.

\item[\kw{charm/tools}]
Visualization tools for \charmpp{} programs.

\end{description}


\subsection{Security Issues}

On most computers, \charmpp{} programs are simple binaries, and they pose
no more security issues than any other program would.  The only exception
is the network version {\tt net-*}, which has the following issues. 

The network versions utilize many unix processes communicating with
each other via UDP.  Only a simple attempt is currently made to filter out
unauthorized packets.  Therefore, it is theoretically possible to
mount a security attack by sending UDP packets to an executing
\converse{} or \charmpp{} program's sockets.

The second security issue associated with networked programs is
associated with the fact that we, the \charmpp{} developers, need evidence
that our tools are being used.  (Such evidence is useful in convincing
funding agencies to continue to support our work.)  To this end, we
have inserted code in the network {\tt charmrun} program (described
later) to notify us that our software is being used.
This notification is a single {\tt UDP} packet sent by {\tt charmrun}
to {\tt charm.cs.uiuc.edu}.  This data is put
to one use only: it is gathered into tables recording the internet
domains in which our software is being used, the number of individuals
at each internet domain, and the frequency with which it is used.

We recognize that some users may have objections to our notification
code.  Therefore, we have provided a second copy of the {\tt
charmrun} program with the notification code removed.  If you look
within the charm {\tt bin} directory, you will find these programs:

\begin{alltt}
    % cd charm/bin
    % ls charmrun*
    charmrun
    charmrun-notify
    charmrun-silent
\end{alltt}

The program {\tt charmrun.silent} has the notification code removed.  To
permanently deactivate notification, you may use the version without the
notification code:

\begin{alltt}
    % cd charm/bin
    % cp charmrun.silent charmrun
\end{alltt}

The {\em only} versions of \charmpp{} that ever notify us are 
the network versions.


\subsection{Reducing disk usage}

This section describes how you may delete parts of the distribution to
save disk space.  

The charm directory contains a collection of example-programs and
test-programs.  These may be deleted with no other effects:

\begin{alltt}
    % rm -r charm/pgms
\end{alltt}

You may also {\tt strip} all the binaries in {\tt charm/bin}.





