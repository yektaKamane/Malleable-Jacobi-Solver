\subsection{Chare Arrays}

Chare arrays \index{arrays} are arbitrarily-sized collections of chares.
The entire array is has a globally unique identifier of type 
\kw{CkArrayID}, and each element has a unique index of type
\kw{CkArrayIndex}.  A \kw{CkArrayIndex} can be a single integer (i.e., 1D
array), several integers (i.e., a multidimentional array), or 
an arbitrary string of bytes (e.g., a binary tree index).

In most respects, array elements are identical to chares.  Differences
include variations in the creation and remote invocation syntax, to
include the array index.  

Array elements can be migrated, allowing arrays to be efficiently
load balanced.  Array elements can also receive array broadcasts and
contribute to array reductions.

\subsubsection{Creating an Array}

You can declare a one-dimentional \index{array}\index{chare array}chare array
as:

\begin{verbatim}
//In the .ci file:
array [1D] A {
  entry A();
  entry void someEntry(M1 *);
};

//In the .h file:
class M1:public CMessage_M1 {
//...message fields...
};

class A : public ArrayElement1D {
  public:
    A(void);
    A(CkMigrateMessage *);

    void someEntry(M1 *);
};
\end{verbatim}

Note A's odd migration constructor, which is normally empty:

\begin{verbatim}
//In the .C file:
A::A(void)
{
  //...your constructor code...
}
A::A(CkMigrateMessage *m) { }
\end{verbatim}

Read the section ``Migratable Array Elements'' for more
information on the \kw{CkMigrateMessage}
constructor. 


\subsubsection{Inserting and Destroying Array Elements}

You create an array using the \kw{CProxy\_Array::ckNew}
routine.  You insert elements into the array with the
\kw{CProxy\_Array::insert} call, and start using
the array with \kw{CProxy\_Array::doneInserting}.

To create a 1D \index{array}array containing elements indexed 
(0, 1, ..., num\_elements-1), you can use the shorter:

\begin{verbatim}
CProxy_A a = CProxy_A::ckNew(num_elements);
\end{verbatim}

If A has a constructor that takes a message, you can use

\begin{verbatim}
CProxy_A a = CProxy_A::ckNew(msg,num_elements);
\end{verbatim}


If you don't have a 1D array or don't want a contiguous range
of elements, you insert array elements dynamically using the 
\kw{insert} and \kw{doneInserting} methods:
\begin{verbatim}
CProxy_A a = CProxy_A::ckNew();//<- note no arguments
a[17].insert();//<- could also pass a message to constructor
a[43].insert();//<- could use a different message or constructor here
a.doneInserting(); //<- needed because we used insert.
\end{verbatim}

Insertion allows you to create sparse or multidimentional arrays,
send a different message to each element, and specify the PE to create 
the element on.  

The \kw{doneInserting} call starts the the reduction manager (see ``Array
Reductions'') and load balancer (see ``Load Balancer'')-- since
these objects need to know about all the array's elements, they
must be started after the initial elements are inserted.
However, if you forget to call \kw{doneInserting}, the reduction manager 
and load balancer will never start!

You may call \kw{doneInserting} multiple times, but only the first
call actually does anything.  You may even \kw{insert} or \kw{destroy}
elements after a call to \kw{doneInserting}, with different semantics-- 
see the reduction manager and load balancer sections for details.


To summarize, if your constructor takes no message, you can use \kw{ckNew} and \kw{insert} as:
\begin{verbatim}
With 1D elements: ckNew(numInitial); //<- doesn't need doneInserting
No elements:      ckNew(); //<- will need doneInserting
Default PE:       insert();
Specific PE:      insert(onPE);
\end{verbatim}

And if your constructor takes a message:
\begin{verbatim}
With 1D elements: ckNew(msg, numInitial); //<- doesn't need doneInserting
No elements:      ckNew(); //<- will need doneInserting
Default PE:       insert(msg);
Specific PE:      insert(msg,onPE);
\end{verbatim}

The system chooses the default PE to create an array element on
based on the element's index: 1D indices map to processors in
a round-robin fashion; other indices map based on a hash function
applied to the index.

To destroy an array element-- detach it from the array
call its destructor, and release its memory--invoke its ``destroy'' 
entry point, as:
\begin{verbatim}
a3(x,y,z).destroy();
\end{verbatim}

After destroying an element, you may insert a new element at
its index.


\subsubsection{Messages and Broadcasts}

\charmpp contains direct support for 1D, 2D, 3D, and user-defined
index arrays.  These arrays can be declared as:

\begin{verbatim}
//In the .ci file:
message MyMsg;
array [1D] A1 {entry A1();entry void e(MyMsg *m);}
array [2D] A2 {entry A2();entry void e(MyMsg *m);}
array [3D] A3 {entry A3();entry void e(MyMsg *m);}
array [Foo] AF {entry AF();entry void e(MyMsg *m);}
\end{verbatim}

The last declaration expects an array index of type CkArrayIndexFoo,
which must be defined before including the .decl.h file 
(see ``Defining a new array index type'' below).  Also, charm++.h must be included before the definition of the CkArrayIndexFoo type. 

\begin{verbatim}
//In the .h file:
class MyMsg: public CMessage_MyMsg {
public: 
  int val;
  MyMsg(int i) {val=i;}
}
class A1:public ArrayElement1D {public: A1(){} ...};
class A2:public ArrayElement2D {public: A2(){} ...};
class A3:public ArrayElement3D {public: A3(){} ...};
class AF:public ArrayElement {public: AF(){} ...};
//(the migration constructors have been omitted for clarity)
\end{verbatim}

You can insert elements or invoke methods on the elements of these
arrays by specifying the array index in parentheses.  For 1D 
arrays or user-defined arrays, you can also use brackets.

You may invoke methods on array elements that have not yet
been created-- the system will buffer the message until the
element is created.  However, the element must eventually be 
created (currently, there is a 3-minute buffering period).
Messages sent to migrating elements will be delivered after
the migrating element arrives.  It is an error to send 
a message to a deleted array element.

\begin{verbatim}
//In the .C file:
int x,y,z;
CProxy_A1 a1=CProxy_A1::ckNew();
a1[x  ].insert(); 
a1(x+1).insert();
a1.doneInserting();
a1[x].e(new MyMsg(7));

CProxy_A2 a2=CProxy_A2::ckNew(); 
a2(x,y).insert();
a2.doneInserting();
a2(x,y).e(new MyMsg(7));

CProxy_A3 a3=CProxy_A3::ckNew();
a3(x,y,z).insert();
a3.doneInserting();
a3(x,y,z).e(new MyMsg(7);

CProxy_AF aF=CProxy_AF::ckNew();
aF(new CkArrayIndexFoo(...)).insert();
aF[new CkArrayIndexFoo(...)].insert();
aF.doneInserting();
aF[new CkArrayIndexFoo(...)].e(new MyMsg(7);

\end{verbatim}
With user-defined index types, you must pass a fresh index 
(allocated with new) to a [] or () call in order to create an element proxy.
System will delete the index in the destructor for the element proxy.

To broadcast a message to all the current elements of an array, 
simply omit the index, as:

\begin{verbatim}
a3.e(new MyMsg(7)); //<- invokes e(7) on each array element
\end{verbatim}

\subsubsection{Migratable Array Elements}
Array objects can \index{migrate}migrate from one PE to another.
For example, the load balancer (see section ``Load Balancing Chare Arrays'')
might migrate array elements to better balance the load between
processors.  For an array element to migrate, it must implement
a pack/unpack or ``pup'' method:

\begin{verbatim}
//In the .h file:
class A2:public ArrayElement2D {
private: //My data members:
    int nt;
    unsigned char chr;
    float flt[7];
    int numDbl;
    double *dbl;
public:	
    //...other declarations

    virtual void pup(PUP::er &p);
};

//In the .C file:
void A2::pup(PUP::er &p)
{
    ArrayElement2D::pup(p);//<- call superclass's pup routine
    p(nt);
    p(chr);
    p(flt,7);
    p(numDbl);
    if (p.isUnpacking()) dbl=new double[numDbl];
    p(dbl,numDbl);
}
\end{verbatim}

You can apply a \kw{PUP::er} to any primitive variable
(as with ``p(nt)'') or array of primitive variables-- there,
the second parameter gives the length of the array (as with
``p(flt,7)'').  You can pass a simple run of bytes 
by calling the \kw{PUP::er} with a ``void *'' and number of bytes
(packing bytes, however, prevents cross-platform pack/unpacking).

The system uses your one pup routine to do both packing and
unpacking by passing different types of \kw{PUP::er}s to it. 
You can determine what type of \kw{PUP::er} has been passed to you 
with the \kw{isPacking()}, \kw{isUnpacking()}, and \kw{isSizing()} calls.

An array element can migrate by calling the ``migrateMe(destination
processor)'' member function-- this call must be the last action
in an element entry point.  The system can also migrate array elements
for load balancing (see the ``Load Balancing Chare Arrays'' chapter).

To migrate your array element to another processor, the \charmpp 
runtime will:

\begin{itemize}
\item Call your \kw{pup} method with a sizing \kw{PUP::er} to determine how 
big a message it needs to hold your element.
\item Call your \kw{pup} method again with a packing \kw{PUP::er} to pack 
your element into a message.
\item Call your element's destructor (killing off the old copy).
\item Send the message (containing your element) across the network.
\item Call your element's migration constructor on the new processor.
\item Call your \kw{pup} method on with an unpacking \kw{PUP::er} to unpack 
the element.
\end{itemize}

Migration constructors, then, are normally empty-- all the
unpacking and allocation of the data items is done in the 
element's \kw{pup} routine.  Deallocation is done in the
element destructor as usual.

The \kw{PUP::er} overhead is very small-- one 
virtual function call for each item to be packed/unpacked.
The actual packing/unpacking is a simple memory-to-memory 
binary copy.  In addition, the \kw{pup} routine may also be called to 
pack your array 
element to disk (e.g., for a ``checkpoint'' operation), and unpack 
it from disk afterwards (e.g., for ``restart'' operation).  
During unpacking, a ``translating'' \kw{PUP::er} can be used to perform binary
format conversion if needed.

To easily pack user-defined types, we recommend adding
a \kw{pup} method to the user-defined type:

\begin{verbatim}
class myData 
{
private:
    int a,b;
    char c;
public:
    ...
    void pup(PUP::er &p) {
        p(a);
        p(b);
        p(c);
    }
};

//... array element can now easily pack myData as:
class myElement:public ArrayElement1D 
{
private:
    myData d;
public:
    ...
    void pup(PUP::er &p)
    {
        ArrayElement1D::pup(p);//Call superclass
        d.pup(p); //<- pack/unpacks myData
    }
};
\end{verbatim}


\subsubsection{Load Balancing Chare Arrays}
\charmpp includes a run-time load balancer which works
on array elements.  The idea is to run programs more efficiently
by migrating array elements away from overloaded processors.
To use the load balancer, you must make your elements migratable
(see migration section above) and chose a load balancing 
``strategy'' (see the load balancing section for a description
of load balancing strategy routines).

There are two ways to use the load balancer-- the AtSync
method, for elements that only want to be migrated at certain
times; and the non-AtSync method, for elements that can migrate
any time.

The non-AtSync method is the default.  Your elements may be
asked to migrate any time they are not currently executing
an entry method.

For the AtSync method, set \kw{usesAtSync} to true in your 
array element constructor.  When an element is ready to migrate,
call \kw{ArrayElement::AtSync}.  Once all elements have reached AtSync, 
the load balancer runs and may migrate elements.  Once
all migrations are complete, the load balancer calls 
\kw{ArrayElement::ResumeFromSync}.  You must ensure no messages are
sent to an array element between its calls to AtSync and ResumeFromSync.
AtSync/ResumeFromSync is currently a implemented as a global barrier.



\subsubsection{Reductions on Chare Arrays}
A \index{reduction}reduction applies a single operation (e.g. add, max, min, ...) to data items scattered across many processors and collects the result in one place.  \charmpp supports reductions on the elements of a Chare array.

The data to be reduced comes from each array element, which must call the contribute method:
\begin{verbatim}
ArrayElement::contribute(int nBytes,void *data,ReducerType type);
\end{verbatim}

This call contributes \kw{nBytes} bytes starting at \kw{data} to the reduction \kw{type} (see reduction types, below).  Unlike sending a message, you may use \kw{data} after the call to \kw{contribute}.  All array elements must call contribute, and all must use the same reduction type.  When you create a new array element, it is expected to contribute to the next reduction not already in progress.  The reduction will complete properly even if array elements are migrated or deleted.

If we want to sum each array element's single integer myInt, we would use:
\begin{verbatim}
    //Inside any array element method
    int myInt=get_myInt();
    contribute(sizeof(int),(void *)&myInt,CkReduction::sum_int);
\end{verbatim}

The built-in reduction types (see below) can also handle arrays of numbers.  For example, if each element of an array has a pair of doubles \kw{forces}[2] which need to be summed up (separately) across all array elements, from each array element call:
\begin{verbatim}
    //Inside any array element method
    double forces[2]=get_my_forces();
    contribute(2*sizeof(double),(void *)forces,CkReduction::sum_double);
\end{verbatim}
Note that since C++ arrays (like forces[2]) are already pointers, we don't use (void *)\&forces.


After the data is reduced, it is passed to a ``client'' function on node zero.  This is an ordinary C or C++ function that you pass to \kw{CProxy\_Array::setReductionClient}.  Reduction clients are typically registered from your main Chare's constructor, as in:

\begin{verbatim}
//Outside any class definition:
void myClient(void *param,int dataSize,void *data)
{
  double *forceSum=(double *)data;
  cout<<``First force sum is ``<<forceSum[0]<<endl;
  cout<<``Second force sum is ``<<forceSum[1]<<endl;
}

...
    //Inside main::main
    CProxy_A a=CProxy_A::ckNew();
    a.setReductionClient(myClient,(void *)NULL);
\end{verbatim}

This simple client prints out the double values received, and does not use the \kw{param} value passed in (\kw{NULL}, in this case).  The handler can use param, dataSize, and the data in any fashion; but the data will be deleted after the handler returns.


\subsubsection{Built-in Array Reduction Types}
\charmpp includes several built-in reduction types, used to combine the contributions from different array elements.  Any of them may be passed as an \kw{ArrayReductionFn} type to \kw{ArrayElement::contribute}.

The first four reductions (sum, product, max, and min) work on int, float, or double data as indicated by the suffix.  The logical reductions (and, or) only work on integer data.  All the built-in reductions work on single numbers (pass a pointer) or arrays-- just pass the correct number of bytes to \kw{ArrayElement}::\kw{contribute}.

\begin{enumerate}
\item \kw{CkReduction::sum\_int}, \kw{\_float}, \kw{\_double}-- the result will be the sum of the given numbers.

\item \kw{CkReduction::product\_int}, \kw{\_float}, \kw{\_double}-- the result will be the product of the given numbers.

\item \kw{CkReduction::max\_int}, \kw{\_float}, \kw{\_double}-- the result will be the largest of the given numbers.

\item \kw{CkReduction::min\_int}, \kw{\_float}, \kw{\_double}-- the result will be the smallest of the given numbers.

\item \kw{CkReduction::and}-- the result will be the logical AND of the given integers.  0 is false, nonzero is true.

\item \kw{CkReduction::or}-- the result will be the logical OR of the given integers.

\item \kw{CkReduction::set}-- the result will be a verbatim concatenation of all the contributed data, separated into \kw{CkReduction::setElement} records.  The data contributed can be of any length, and can vary across array elements or reductions.  To extract the data from each element, see the description below.

\end{enumerate}


\kw{CkReduction::set} returns a collection of \kw{CkReduction::setElement} objects, one per contribution.  This class has definition:
\begin{verbatim}
class CkReduction::setElement {
public:
  int dataSize;//The length of the data array below
  char data[];//The (dataSize-long) array of data
  CkReduction::setElement *next(void);
};
\end{verbatim}

To extract the contribution of each array element from a reduction set, use the \kw{next} routine repeatedly:
\begin{verbatim}
  //Inside a reduction handler-- 
  //  data is our reduced data from CkReduction_set
  CkReduction::setElement *cur=(CkReduction::setElement *)data;
  while (cur!=NULL)
  {
    ... //Use cur->dataSize and cur->data
    //Now advance to the next element's contribution
    cur=cur->next();
  }
\end{verbatim}

The reduction set order is undefined.  Add a source field to your contribution if you need to know which array element gave a particular contribution.

\subsubsection{Defining a new array index type}
\charmpp array indices are arbitrary collections of bytes.
To define a new array index, you create an ordinary C++ class 
which inherits from \kw{CkArrayIndex} and overrides the \kw{getKey} method.  
For example, if you have a structure or class named ``foo'', you 
can use a foo object as an array index by defining the class:

\begin{verbatim}
class CkArrayIndexFoo:public CkArrayIndex {
public:
    foo f;

    CkArrayIndexFoo(const foo &in) {f=in;}

    //This method returns the length of and a pointer to the index data.
    //The returned pointer must be aligned to at least an integer boundary.
    virtual const unsigned char *getKey(/*out*/ int &nBytes) const {
      	nBytes=sizeof(foo);
        return (const unsigned char *)&f;
    }
};
\end{verbatim}

You can then declare an array indexed by foo objects with

\begin{verbatim}
//in the .ci file:
array [Foo] AF {entry AF(); ... }

//in the .h file:
class AF:public ArrayElement {public: AF() {} ... }

//in the .C file:
    foo f;
    CProxy_AF a=CProxy_AF::ckNew();
    a[new CkArrayIndexFoo(f)].insert();
    ...
\end{verbatim}

Inside an \kw{ArrayElement}, you can get the bytes of your index
from the ``thisindex'' variable, which has type \kw{CkArrayIndexGeneric*}.  
It may be more convenient, however, to inherit from \kw{ArrayElementT} a 
class templated in the index type foo (stored as ``thisIndex'').
For example:

\begin{verbatim}
//.ci file unchanged
//in the .h file:
class AF:public ArrayElementT<foo> {public: AF() {} ... }

//in the .C file:
AF::AF()
{
    foo myF=thisIndex;
    functionTakingFoo(myF);
}
\end{verbatim}


\subsubsection{Array Syntax Changes}
The syntax for arrays changed in April, 2000 in the following ways:

\begin{itemize}

\item CkAID is obsolete, and replaced with CkArrayID.  In general,
you should now use CProxy's instead of CkArrayID's because the syntax
is more palatable and the compiler can do static type checking.

\item The translator syntax for arrays now includes the index type,
e.g. ``array [1D] foo { ... }''.

\item Messages sent to array elements need no longer inherit from 
ArrayMessage.

\item Array elements for 1D arrays must inherit from ArrayElement1D
to use a 1D thisIndex.  You may still inherit from ArrayElement, but you
won't have a thisIndex.

\item You may have multiple array constructors; each can now take a user-defined message, or no message.  

\item The ``ArrayElementCreateMessage'' constructor is obsolete;
as are the finishConstruction and finishMigration calls. No constructor
need explicitly call its superclass.  

\item ``ArrayElementMigrateMessage'' is now depricated-- use 
``CkMigrateMessage''

\item For migration, the old packsize, pack, and unpacking portion of
the migration constructor have been replaced by a unified pack/unpack 
``pup'' routine.

\item For load balancing, the uses\_at\_sync variable is now called usesAtSync.

\item CkReduction\_sum\_int is now an enumeration, not a function pointer, and
now called CkReduction::sum\_int.

\item CkRegisterArrayReductionHandler(proxy,fn,param)
is now proxy.setReductionClient(fn,param).
\end{itemize}



