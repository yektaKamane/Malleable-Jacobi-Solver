\documentclass[11pt]{article}

\newif\ifpdf
\ifx\pdfoutout\undefined
  \pdffalse
\else
  \pdfoutput=1
  \pdftrue
\fi

\ifpdf
  \pdfcompresslevel=9
  \usepackage[pdftex,colorlinks=true,plainpages=false]{hyperref}
\else
\fi

\usepackage{fullpage}
\pagestyle{headings}
\setlength{\parskip}{0.1in}
\setlength{\textheight}{9.5in}
\setlength{\textwidth}{6.5in}
\setlength{\parindent}{0in}
\setlength{\topmargin}{-.5in}
\parskip 0.1in

%
% Constants
%
\newcommand{\version}{5.0}		%%% The current version number
\newcommand{\prevversion}{4.9}	%%% The previous version number

%
% Commands
%
\newcommand{\zap}[1]{ }
\newcommand{\fcmd}{\bf}		%%% Font for Charm commands
\newcommand{\fparm}{\it\sf}	%%% Font for parameters to Charm commands
\newcommand{\fexec}{\bf}	%%% Font for compile/execute cmds/options
\newcommand{\atitle}[1]{{\it #1}}
\newcommand{\keyword}[1]{{\textbf{#1}}}
\newcommand{\userword}[1]{{\fparm \textsf{#1}}}
\newcommand{\constraint}[1]{Note: {\it #1}}
\newcommand{\note}[1]{Note: {\it #1}}

%
% Conveniences
%
\newcommand{\uw}[1]{\userword{#1}}
\newcommand{\kw}[1]{\keyword{#1}}
\newcommand{\CLocalBranch}{\keyword{CLocalBranch}}

%
%       \CC gives "C++" that looks good.
%
\newcommand{\CC}{C\kern -0.0em\raise 0.5ex\hbox{\normalsize++}}
\newcommand{\emCC}{C\kern -0.0em\raise 0.4ex\hbox{\normalsize\em++}}
\newcommand{\charmpp}{{\sc Charm++}}

\makeindex

\begin{document}

\begin{titlepage}
\vspace*{2in}
\Huge
\begin{center}
The \\
\charmpp \\
Programming Language \\
Manual\\
\vspace*{0.5in}
Version 5.0\\
\vspace*{0.7in}
\today
\end{center}
\normalsize
\end{titlepage}

\section*{Acknowlegements}

\large
The Charm software was developed as a group effort.  The earliest
prototype, Chare Kernel(1.0), was developed by Wennie Shu and Kevin
Nomura working with Laxmikant Kale.  The second prototype, Chare
Kernel(2.0), a complete re-write with major design changes, was
developed by a team consisting of Wayne Fenton, Balkrishna Ramkumar,
Vikram Saletore, Amitabh B. Sinha and Laxmikant Kale. The translator
for Chare Kernel(2.0) was written by Manish Gupta.  Charm(3.0), with
significant design changes, was developed by a team consisting of
Attila Gursoy, Balkrishna Ramkumar, Amitabh B.  Sinha and Laxmikant
Kale, with a new translator written by Nimish Shah.  The \charmpp\ 
implementation was done by Sanjeev Krishnan.  Charm(4.0) included
\charmpp\ and was released in fall 1993.  Charm(4.5) was developed by
Attila Gursoy, Sanjeev Krishnan, Milind Bhandarkar, Joshua Yelon,
Narain Jagathesan and Laxmikant Kale.  Charm(4.8), developed by the
same team included Converse, a parallel runtime system that allows
interoperability among modules written using different paradigms
within a single application. \charmpp\ runtime system was re-targetted
at Converse. Syntactic extensions in \charmpp\ were dropped, and a
simple interface translator was developed that, along with the \charmpp\
runtime, became the \charmpp\ language.  The current version (5.0)
includes a complete rewrite of the \charmpp\ runtime system (using \CC)
and the interface translator (done by Milind Bhandarkar).  It also
includes several new features such as Chare Arrays (developed by
Robert Brunner), and various libraries (written by Terry Wilmarth,
Gengbin Zheng, Laxmikant Kale, Zehra Sura, Milind Bhandarkar, Robert
Brunner, and Krishnan Varadarajan.) A coordination language
``Structured Dagger'' has been implemented on top of \charmpp\ (Milind
Bhandarkar), and included in this version.  IDL bindings for \charmpp\ 
were developed by Jayant Desouza, Milind Bhandarkar and Gengbin
Zheng. Several features have also been added to Converse. Dynamic
seed-based load balancing has been implemented (Terry Wilmarth and
Joshua Yelon), a client-server interface for Converse programs, and
debugging support has been added (Parthasarathy Ramachandran, Jeff
Wright, and Milind Bhandarkar).  Converse has been ported to new
platforms including ASCI Red (Joshua Yelon), Cray T3E (Robert
Brunner), and SGI Origin2000 (Milind Bhandarkar).  The test suite for
\charmpp\ was developed by Michael Lang, Jackie Wang, and Fang
Hu. Projections, the performance visualization and analysis tool, was
redesigned and rewritten using Java by Michael Denardo.

\normalsize


\newpage
\tableofcontents

\newpage
This manual describes \charmpp\ and Converse libraries.\cite{InterOpIPPS96}
This is a work in progress towards a standard library for parallel
programming on top of the Converse and \charmpp\ system.

For a description of the C-based {\sc Charm} parallel programming system,
please refer to the {\sl Charm Programming Language Manual} and the
{\sl Tutorial Introduction to Charm}\footnote{{\sc Charm} is no longer
actively supported and maintained, and these manuals are kept only for
offering the historical perspectives.}.


\newpage
\section{Big Questions}

\subsubsection{What is Charm++?}

\htmladdnormallink{Charm++}{http://charm.cs.uiuc.edu/research/charm/} is
a runtime library to let C++ objects communicate with each other efficiently.
The programming model is thus very much like CORBA, Java RMI, or RPC; but
it is targeted towards tightly coupled, high-performance parallel machines.
It uses the ``single program, multiple data'' (SPMD) programming model made
popular by MPI.

Charm++ has demonstrated scalability up to thousands of processors,
and provides extremely advanced load balancing and object migration facilities.

\subsubsection{Can Charm++ parallelize my serial program automatically?}

No.

Charm++ is used to write ``explicitly parallel'' programs--we don't have
our own compiler, so we don't do automatic parallelization. We've found
automatic parallelization useful only for a small range of very regular
numerical applications.

However, you should { \em not} have to throw away your serial code;
normally only a small fraction of a large program needs to be changed to
enable parallel execution. In particular, Charm++'s support for object-oriented
programming and high-level abstractions such as Charm++ Arrays make it
simpler and more expressive than many other parallel languages. So you
will have to write some new code, but not as much as you might think. This
is particularly true when using one of the Charm++ \htmladdnormallink{frameworks}{http://charm.cs.uiuc.edu/research/}.

\subsubsection{I can already write parallel applications in MPI. Why should I use Charm++?}

Charm++ provides several extremely sophisticated features, such as
application-independent object migration, that are very difficult to provide
in MPI. If you have a working MPI code but have scalability problems
because of dynamic behavior, load imbalance, or communication costs, Charm++
might dramatically improve your performance. You can even run your
MPI code on Charm++ unchanged using \htmladdnormallink{AMPI}{http://charm.cs.uiuc.edu/research/ampi/}.

\subsubsection{Will Charm++ run on my machine?}

Yes.

Charm++ supports both shared-memory and distributed-memory machines,
SMPs and non-SMPs. In particular, we support serial machines, Windows
machines, clusters connected via Ethernet, Myrinet or Infiniband, IBM
SP series and BlueGene, Cray XT series, and any machine that supports MPI
or SHMEM. We normally do our development on Linux workstations, and
our testing on large parallel machines. Programs written using Charm++
will run on any supported machine.

No one has ported Charm++ to a vector supercomputer, but it should be
possible.

\subsubsection{Does anybody actually use Charm++?}

Yes.

The large, production-quality molecular dynamics application \htmladdnormallink{NAMD}{http://charm.cs.uiuc.edu/research/moldyn/}
is built on Charm++.\\
The \htmladdnormallink{Center for Simulation of Advanced Rockets}{http://www.csar.uiuc.edu/}
has a large physical simulation code built using Charm++.\\
We have significant collaborations with groups in Materials Science,
Chemistry, and Astrophysics in Illinios, New York, and Washington.

\subsubsection{Who created Charm++?}

Prof. \htmladdnormallink{L.V. Kale}{mailto:kale@cs.uiuc.edu}, of the
\htmladdnormallink{Computer Science Department}{http://www.cs.uiuc.edu/}
of the \htmladdnormallink{University of Illinois at Urbana-Champaign}{http://www.uiuc.edu/},
and his research group, the \htmladdnormallink{Parallel Programming Lab}{http://charm.cs.uiuc.edu/">Parallel}.
Nearly a hundred people have contributed something
to the project over the course of aproximately 15 years; a partial list
of contributors appears in the \htmladdnormallink{people}{http://charm.cs.uiuc.edu/people/}'s page.

\subsubsection{What is the future of Charm++?}

Our research group of approximately twenty people are actively engaged
in maintaining and extending Charm++; and in particular the Charm++ 
\htmladdnormallink{frameworks}{http://charm.cs.uiuc.edu/research/}.
Several other groups are dependent on Charm++, so we expect to continue
improving Charm++ indefinitely.

\subsubsection{How is Charm++ Licensed?}

Charm++ is open-source and free for research, educational, and academic
use. The University of Illinois retains the copyright to the software,
and requires a license for any commercial redistribution of our software.
The actual, legal license is included with Charm++ (in charm/LICENSE).

\subsubsection{I have a suggestion/feature request/bug report. Who should I send
it to?}

Our mailing list is \htmladdnormallink{ppl@cs.uiuc.edu}{mailto:ppl@cs.uiuc.edu}.
We're always glad to get feedback on our software.


\section{The \charmpp\ Language}
  \subsection{Modules}

\subsubsection{Structure of a \charmpp\ Program}

A \charmpp\ program is structurally similar to a \CC{} program.  Most of a
\charmpp\ program {\em is} \CC{} code \footnote{\bf Constraint: The \CC{} code
cannot, however, contain global or static variables.}. The main syntactic units
in a \charmpp\ program are class definitions. A \charmpp\ program can be
distributed across several source code files.

There are five disjoint categories of objects (classes) in \charmpp:

\begin{itemize}
\item Sequential objects: as in \CC{}
\item Chares (concurrent objects) \index{chare}
\item Chare Groups \index{chare groups} (a form of replicated objects)
\index{group}
\item Chare Arrays \index{chare arrays} (an indexed collection of chares)
\index{array}
\item Messages (communication objects)\index{message}
\end{itemize}

The user's code is written in \CC{} and interfaces with the \charmpp\ system as
if it were a library containing base classes, functions, etc.  A translator is
used to generate the special code needed to handle \charmpp\ constructs.  This
translator generates \CC{} code that needs to be compiled with the user's code.

Interfaces to the \charmpp\ objects (such as messages, chares, readonly
variables etc.) \index{message}\index{chare}\index{readonly} have to be
declared in \charmpp\ interface files. Typically, such entities are grouped
\index{module} into {\em modules}. A \charmpp\ program may consists of multiple
modules.  One of these modules is declared to be a \kw{mainmodule}. All the
modules that are ``reachable'' from the \kw{mainmodule} via the \kw{extern}
construct are included in a \charmpp\ program.

The \charmpp\ interface file has the suffix ``.ci''.  The \charmpp\ interface
translator parses this file and produces two files (with suffixes ``.decl.h''
and ``.def.h'', {\em for each modules declared in the ``.ci'' file}), that
contain declarations (interface) and definitions (implementation)of various
translator-generated entities. If the name of a module is \uw{MOD}, then the
files produced by the \charmpp\ interface translator are named \uw{MOD.decl.h}
and \uw{MOD.def.h}\footnote{Note that the interface file for module \uw{MOD}
need not be named \uw{MOD.ci}. Indeed one ``.ci'' file may contain interface
declarations for multiple modules, and the translator will produce one pair of
declaration and definition files for each module.}.  We recommend that the
declarations header file be included at the top of the header file (\uw{MOD.h})
for module \uw{MOD}, and the definitions file be included at the bottom of the
code for module (\uw{MOD.C}) \footnote{In the earlier version of interface
translator, these files used to be suffixed with ``.top.h'' and ``.bot.h'' for
this reason.}.

A simple \charmpp\ program is given below:

\begin{alltt}
///////////////////////////////////////
// File: pgm.ci

mainmodule Hello \{
  readonly CkChareID mid;
  mainchare HelloMain \{
    entry HelloMain(); // implicit CkArgMsg * as argument
    entry void Wait(void);
  \};
  group HelloGroup \{
    entry HelloGroup(void);
  \};
\};

////////////////////////////////////////
// File: pgm.h
#include "Hello.decl.h" // Note: not pgm.decl.h

class HelloMain: public Chare \{
  public:
    HelloMain(CkArgMsg *);
    void Wait(void);
  private:
    int count;
\};

class HelloGroup: public Group \{
  public:
    HelloGroup(void);
\};

/////////////////////////////////////////
// File: pgm.C
#include "pgm.h"

CkChareID mid;

HelloMain::HelloMain(CkArgMsg *msg) \{
  delete msg;
  count = 0;
  mid = thishandle;
  CProxy_HelloGroup::ckNew(); // Create a new "HelloGroup"
\}

void HelloMain::Wait(void) \{
  count++;
  if (count == CkNumPes()) \{ // Wait for all group members to finish the printf
    CkExit();
  \}
\}

HelloGroup::HelloGroup(void) \{
  ckout << "Hello World from processor " << CkMyPe() << endl;
  CProxy_HelloMain pself(mid);
  pself.Wait();
\}

#include "Hello.def.h" // Include the Charm++ object implementations

/////////////////////////////////////////
// File: Makefile

pgm: pgm.ci pgm.h pgm.C
      charmc -c pgm.ci
      charmc -c pgm.C
      charmc -o pgm pgm.o -language charm++

\end{alltt}

\uw{HelloMain} is designated a \kw{mainchare}. Thus the Charm Kernel starts
execution of this program by creating an instance of \uw{HelloMain} on
processor 0. The HelloMain constructor creates a chare group \uw{HelloGroup},
and stores a handle to itself and returns. The call to create the group returns
immediately after directing Charm Kernel to perform the actual creation and
invocation.  Shortly after, the Charm Kernel will create an object of type
\uw{HelloGroup} on each processor, and call its constructor. The constructor
will then print ``Hello World...'' and then call the \uw{Wait} method of
\uw{HelloMain}. The Wait method calls CkExit() after all group members have
called it (i.e., they have finished printing ``Hello World...''), and the
\charmpp program exits.

\subsubsection{Functions in the ``decl.h'' and ``def.h'' files}

The \texttt{decl.h} file provides declarations for the proxy classes of the
concurrent objects declared in the ``.ci'' file (from which the \texttt{decl.h}
file is generated). So the \uw{Hello.decl.h} file will have the declaration of
the class CProxy\_HelloMain. Similarly it will also have the declaration for
the HelloGroup class. 

This class will have functions to create new instances of the chares and
groups, like the function \kw{ckNew}. For \uw{HelloGroup} this function creates
an instance of the class \uw{HelloGroup} on all the processors. 

The proxy class also has functions corresponding to the entry methods defined
in the ``.ci'' file. In the above program the method wait is declared in
\uw{CProxy\_HelloMain} (proxy class for \uw{HelloMain}).

The proxy class also provides static registration functions used by the
\charmpp{} runtime.  The \texttt{def.h} file has a registration function
(\uw{\_\_registerHello} in the above program) which calls all the registration
functions corresponding to the readonly variables and entry methods declared in
the module.

	
  \subsection{Messages}

In \charmpp, \index{chare}chares, \index{group}groups and \index{nodegroup}
nodegroups communicate using 
messages. Sending a message \index{message} to an object corresponds to an 
asynchronous method invocation.
Several variations on messaging are available, regular \charmpp messages are 
objects of \textit{fixed size}. If you want your message object to contain 
pointers to dynamically allocated memory and need these to be valid when 
messages are sent across processors you must declare your messages to be 
\kw{varsize} (variable size) or \kw{packed} messages, which need some 
additional processing. Though similar in nature, they vary in the amount 
of flexibilty they provide in terms of representing arbitrary data structures.
 
Also available is a mechanism for assigning \textit{priorities} to messages 
that applies all kinds of messages, a detailed discussion
of priorities appears later in this section.

Like all 
other entities involved in asynchronous method invocation, messages need to be
declared in the {\tt .ci} file.

Message definitions that may appear in the {\tt .h} or {\tt .C} file are very 
similar to class definitions in C++.

In the {\tt .ci} file (the interface file), a message is declared as: 

\begin{tabbing}
~~~~ \=~~~~ \=~~~~ \=~~~~ \=~~~~ \=~~~~ \=~~~~ \=~~~~ \=~~~~ \=~~~~ \kill
\> \kw{message} \uw{MessageType};
\end{tabbing}

a variable sized or packed message is declared as:

\begin{tabbing}
~~~~ \=~~~~ \=~~~~ \=~~~~ \=~~~~ \=~~~~ \=~~~~ \=~~~~ \=~~~~ \=~~~~ \kill
\> \kw{message} [varsize|packed] \uw{MessageType};
\end{tabbing}


If the name of the message class is \uw{MessageType}, the class must inherit 
publicly from a class whose name is \kw{CMessage}\_\uw{MessageType}. This class
is generated by the charm translator. Then message definition has the form:

\begin{tabbing}
~~~~ \=~~~~ \=~~~~ \=~~~~ \=~~~~ \=~~~~ \=~~~~ \=~~~~ \=~~~~ \=~~~~ \kill
\> \kw{class} \uw{MessageType} : \kw{public CMessage}\_\uw{MessageType} \{ \\
\> \>  // List of data and function members as in C++ \\
\> \};
\end{tabbing}


\subsubsection{Message Creation and Deletion}
\label{memory allocation}

\index{message}Messages are allocated using the \CC\ \kw{new}\index{new}
operator:

\begin{tabbing}
~~~~ \=~~~~ \=~~~~ \=~~~~ \=~~~~ \=~~~~ \=~~~~ \=~~~~ \=~~~~ \=~~~~ \kill
\> \uw{MessageType} *{\it msgptr} = \\
\> \> \kw{new} [((\kw{int} *){\it sizes\_array}, \kw{int} {\it priobits}=0)] 
\uw{MessageType}[(constructor arguments)]; \\
\end{tabbing}

The optional arguments to the new operator are used when allocating 
\kw{varsize} or \kw{prioritized} messages. The {\it sizes\_array} 
used for varsize messages is an 
array of integers elements of which denote the size (in appropriate
units) of the memory blocks that need to be allocated and assigned to the
pointers that the message contains. The optional {\it priobits} argument 
denotes the size of a bitfield that will be used to store the message 
priority.

For example, to allocate a \index{varsize message}varsize message whose 
class declaration is:

\begin{verbatim}
class VarsizeMessage : public CMessage_VarsizeMessage{
  int length;
  int *firstArray;
  double *secondArray;
};
\end{verbatim}

assign the size array and allocate the message:
\begin{verbatim}
int size[2];
size[0] = 10;
size[1] = 20;
VarsizeMessage *msg = new (size) VarsizeMessage;
\end{verbatim}

This allocates a {\it VarsizeMessage}, in which (typically) {\it firstArray}
points to an array of 10 ints and {\it secondArray} points to an array
of 20 doubles. We say typically because the programmer has full control over 
how the sizes array is used. The meaning of this will be made clear in the 
section on variable sized messages.


To add a \index{priority}priority bitfield to this message, 
\begin{verbatim}
VarsizeMessage *msg = new (size, sizeof(int)*8) VarsizeMessage;
\end{verbatim}
Note, you must provide number of bits which is used to store the priority as 
the {\it priobits} parameter. The section on prioritized execution describes
how this bitfield is used.

In Section~\ref{message packing} we explain how messages can
contain arbitrary pointers, and how the validity of such pointers can be
maintained across processors in a distributed memory machine.

When a message \index{message} is sent to a \index{chare}chare, the programmer
relinquishes control of it; the space allocated to the message is
freed by the system.  When a message is received at an entry point it
is not freed by the runtime system.  It may be reused or deleted by
the programmer.  Messages can be deleted using the standard C++
\kw{delete}\index{delete} operator.  


\subsubsection{Message Packing}
\label{message packing}
\index{message packing}
\index{packed messages}

When one declares \charmpp\ messages in the ``.ci'' file, the
translator generates code to register those messages along with their
size with the runtime system. This information is later used to
efficiently allocate messages by the runtime system.

In many cases, the messages store {\em non-linear} data structures
using pointers.  Examples of these are binary trees, hash tables
etc. Thus, the message itself contains only a pointer to the actual
data. When the message is sent to the same processor, these pointers
point to the original locations, which are within the address space of
the same processor, however, when such a message is sent to other
processors, these pointers will point to invalid locations.

Thus, the programmer needs a way to ``serialize'' these messages
\index{serialized messages}\index{message serialization}{\em
only if} the message crosses the address-space boundary. This is
provided in \charmpp\ using
\kw{packed} \index{packed} messages.

Such messages are declared in the {\tt .ci} file as:

\verb+message [packed] PMessage;+

The class \uw{PMessage} needs to inherit from \uw{CMessage\_PMessage}
and needs to provide two {\em static} methods: \kw{pack}, \index{pack}
and \kw{unpack}\index{unpack}. These methods are called by the
\charmpp\ runtime system, when the message is determined to be
crossing address-space boundary. The prototypes for these methods are
as follows:

\begin{verbatim}
static void *PMessage::pack(PMessage *in);
static PMessage *PMessage::unpack(void *in);
\end{verbatim}

Typically, the following tasks are done in \kw{pack} method:

\begin{itemize}
\item Determine size of the buffer needed to serialize message data.
\item Allocate buffer using the \kw{CkAllocBuffer}
\index{CkAllocBuffer} function. This
function takes in two parameters: input message, and size of the
buffer needed, and returns the buffer.
\item Serialize message data into buffer (alongwith any control
information needed to de-serialize it on the receiving side.
\item Free resources occupied by message (including message itself.)  
\end{itemize}

On the receiving processor, the \kw{unpack} method is
called. Typically, the following tasks are done in the \kw{unpack}
method:

\begin{itemize}
\item Allocate message using \kw{CkAllocBuffer} function. {\em Do not
use \kw{new} to allocate message here. If the message constructor has
to be called, it could be done using the in-place \kw{new} \index{new}
operator.}
\item De-serialize message data from input buffer into the allocated message.
\item Free the input buffer using \kw{CkFreeMsg}\index{CkFreeMsg}.
\end{itemize}

Here is an example of a {\em packed} message implementation:

\begin{verbatim}
// File: pgm.ci
module PackExample {
  ...
  message [packed] PackedMessage;
  ...
};

// File: pgm.h
...
class PackedMessage : public CMessage_PackedMessage
{
  public:
    BinaryTree<NodeType> *btree; // Non-linear data structure
    static void *pack(PackedMessage *);
    static PackedMessage *unpack(void *);
    ...
};
...

// File: pgm.C
...
void *
PackedMessage::pack(PackedMessage *inmsg)
{
  int numnodes = inmsg->btree->numNodes();
  int totalsize = numnodes*sizeof(NodeType) + sizeof(int);
  int *buf = (int *)CkAllocBuffer(inmsg, totalsize);
  *buf++ = numnodes;
  inmsg->btree->flatten((void *)buf); // copies nodes, and 
                                      // replaces pointers by indices
  delete inmsg;
  return (void *)(--buf);
}

PackedMessage *
PackedMessage::unpack(void *inbuf)
{
  int *buf = (int *)inbuf;
  int numnodes = *buf++;
  PackedMessage *pmsg = CkAllocBuffer(inbuf, sizeof(PackedMessage));
  pmsg = new ((void *)pmsg) PackedMessage(inbuf, numnodes); // constructs btree
  CkFreeMsg(inbuf);
}
... 
}
\end{verbatim}


While serializing an arbitrary data structure into a flat buffer, one
must be very wary of any possible alignment problems.  Thus, if
possible, the buffer itself should be declared to be a flat struct.
This will allow the \CC\ compiler to ensure proper alignment of all
its member fields.


\subsubsection{Variable Size Messages}
\label{varsize messages}
\index{variable size messages}
\index{varsize message}

An ordinary message in \charmpp\ is a fixed size message that is
allocated internally with an envelope which encodes the size of the
message. Very often, the size of the data contained in a message is
not known until runtime. One can use \kw{packed}\index{packed
messages} messages to alleviate 
this problem. However, it requires multiple memory allocations (one
for the message, and another for the buffer.) This can be avoided by
making use of a \kw{varsize}\index{varsize} message.

A \kw{varsize} message is declared as {\tt message [varsize] mtype} in
\charmpp\ interface file. The class \uw{mtype} has to inherit from
\uw{CMessage\_mtype} AND has to provide \kw{pack}\index{pack},
\kw{unpack}\index{unpack}, AND
\kw{alloc}\index{alloc} class methods.

Thus, a varsize message is declared in the interface as: 

\verb+message [varsize] MyVarsizeMessage;+ 

In the \CC\ header file, it has to be declared as: 

\begin{verbatim}
class MyVarsizeMessage : public CMessage_MyVarsizeMessage { 
// Other methods & data members 
  static void *alloc(int msgnum, size_t size, int *array, int priobits); 
  static void* pack(msg* in); 
  static msg* unpack(void* in); 
}; 
\end{verbatim}

The \kw{alloc} method will be called when the message is allocated using
the \CC\ \kw{new} operator. The programmer never needs to explicitly
call it.
The \kw{alloc} method should actually allocate the message using
\kw{CkAllocMsg}\index{CkAllocMsg}, whose signature is given below:

\begin{verbatim}
void *CkAllocMsg(int msgnum, int size, int priobits); 
\end{verbatim}  

{\bf An Example:}

Suppose a \charmpp\ message contains two varsize fields:

\begin{verbatim} 
class VarsizeMessage : public CMessage_VarsizeMessage { 
  int length; 
  int *firstArray; 
  double *secondArray; 
}; 
\end{verbatim}

Then in the {\tt .ci} file, this can be represented as: 

\begin{verbatim}
message [varsize] NewVarsizeMessage;
\end{verbatim}

and in the {\tt .h} or {\tt .C} file as:

\begin{verbatim} 

class NewVarsizeMessage : public CMessage_NewVarsizeMessage { 
  public: 
    int len; 
    int *firstArray; 
    double *secondArray; 

  static void *alloc(int msgnum, int size, int *array, int priobits) 
  { 
    int totalsize; 
    totalsize = size + array[0]*sizeof(int) + array[1]*sizeof(double); 
    NewVarsizeMessage *newMsg = 
      (NewVarsizeMessage *) CkAllocMsg(msgnum, totalsize, priobits); 
    newMsg->firstArray = (int *) ((char *)newMsg+size); 
    newMsg->secondArray = (double *) ((char *)newMsg + size + 
                                      sizeof(int)*array[0]); 
    return (void *)newMsg; 
  } 

  static void *pack(NewVarsizeMessage *in) 
  { 
    in->firstArray = (int *) ((char *) in->firstArray - 
                              (char *)&in->firstArray); 
    in->secondArray = (double *) ((char *) in->secondArray - 
                                  (char *) &in->secondArray); 
    return in; 
  } 

  static NewVarsizeMessage* unpack(void *in) 
  { 
    NewVarsizeMessage *me = (NewVarsizeMessage *) in; 
    me->firstArray = (int *) ((char *) &(me->firstArray) + 
                              (size_t)(me->firstArray)); 
    me->secondArray = (double *) ((char *) &(me->secondArray) + 
                                  (size_t)(me->secondArray)); 
    return me;
  } 
}; 
\end{verbatim}

The pointers in a varsize message can exist in two states.  
At creation, they are valid \CC\ pointers to the start of the arrays.  
After packing, they become offsets from the address of the pointer variable to 
the start of the pointed-to data.  Unpacking restores them to pointers. 


\subsubsection{Prioritized Execution}
\label{prioritized message passing}
\index{prioritized execution}
\index{prioritized message passing}
\index{priorities}

By default, \charmpp\ will process the messages you send in roughly
FIFO\index{message delivery order} order.  For most programs, this
behavior is fine.  However, some programs need more explicit control
over the order in which messages are processed.  \charmpp\ allows you
to control queueing behavior on a per-message basis.

The simplest call available to change the order in which messages
are processed is \kw{CkSetQueueing}\index{CkSetQueueing}.

\begin{tabbing}
~~~~ \=~~~~ \=~~~~ \=~~~~ \=~~~~ \=~~~~ \=~~~~ \=~~~~ \=~~~~ \=~~~~ \kill
\> \kw{void CkSetQueueing}(\uw{MsgType} {\it message}, \kw{int} {\it
queueingtype}) \\
\end{tabbing}

where {\it queueingtype}\index{queueing types} is one of the following
constants:

\begin{tabbing}
~~~~ \=~~~~ \=~~~~ \=~~~~ \=~~~~ \=~~~~ \=~~~~ \=~~~~ \=~~~~ \=~~~~ \kill
\> \kw{CK\_QUEUEING\_FIFO}\\
\> \kw{CK\_QUEUEING\_LIFO}\\
\> \kw{CK\_QUEUEING\_IFIFO}\\
\> \kw{CK\_QUEUEING\_ILIFO}\\
\> \kw{CK\_QUEUEING\_BFIFO}\\
\> \kw{CK\_QUEUEING\_BLIFO}\\
\index{CK\_QUEUEING\_FIFO}
\index{CK\_QUEUEING\_LIFO}
\index{CK\_QUEUEING\_IFIFO}
\index{CK\_QUEUEING\_ILIFO}
\index{CK\_QUEUEING\_BFIFO}
\index{CK\_QUEUEING\_BLIFO}
\end{tabbing}

The first two options,  \kw{CK\_QUEUEING\_FIFO} and
\kw{CK\_QUEUEING\_LIFO}, are used as follows: 

\begin{verbatim}
  MsgType *msg1 = new MsgType ;
  CkSetQueueing(msg1, CK_QUEUEING_FIFO);

  MsgType *msg2 = new MsgType ;
  CkSetQueueing(msg2, CK_QUEUEING_LIFO);
\end{verbatim}

When message {\tt msg1} arrives at its destination, it will be pushed
onto the end of the message queue as usual.  However, when {\tt msg2}
arrives, it will be pushed onto the {\it front} of the message queue.

The other four options involve the use of
priorities\index{priorities}.  To attach a priority field to a
message, one needs to set aside space in the message's buffer while
allocating the message\index{message priority}.  To achieve this, the
size of the priority field\index{priority field} in bits should be
specified as a placement argument to the \kw{new} operator, as
described in Section \ref{memory allocation}.  Although the size of
the priority field is specified in bits, it is always padded to an
integral number of {\tt int}s.  A pointer to the priority part of the
message buffer can be obtained with these calls:

\begin{tabbing}
~~~~ \=~~~~ \=~~~~ \=~~~~ \=~~~~ \=~~~~ \=~~~~ \=~~~~ \=~~~~ \=~~~~ \kill
\> \kw{unsigned int} *\kw{CkPriorityPtr}(\uw{MsgType} {\it msg}) \\
\end{tabbing}
\index{CkPriorityPtr}
\index{priority pointer}

There are two kinds of priorities which can be attached to a message:
{\sl integer priorities}\index{integer priorities} and {\sl bitvector
priorities}\index{bitvector priorities}.  Integer priorities are quite
straightforward.  One allocates a message, setting aside enough space
(in bits) in the message to hold the priority, which is an integer.
One then stores the priority in the message.  Finally, one informs the
system that the message contains an integer priority using
\kw{CkSetQueueing}:

\begin{verbatim}
  MsgType *msg = new (8*sizeof(int)) MsgType;
  *CkPriorityPtr(msg) = prio;
  CkSetQueueing(msg, CK_QUEUEING_IFIFO);
\end{verbatim}
\index{CkSetQueueing}

The predefined constant  \kw{CK\_QUEUEING\_IFIFO} indicates that the
message contains an integer priority, and that if there are other
messages of the same priority, they should be sequenced in FIFO order
(relative to each other).  Similarly, a  \kw{CK\_QUEUEING\_ILIFO} is
available.  Note that  \kw{MAXINT} is the lowest priority, and {\bf
NEGATIVE\_MAXINT} is the highest priority\index{integer priority range}.

Bitvector priorities are somewhat more complicated.  Bitvector
priorities are arbitrary-length bit-strings representing fixed-point
numbers in the range 0 to 1.  For example, the bit-string ``001001''
represents the number .001001\raisebox{-.5ex}{\scriptsize binary}.  As
with the simpler kind of priority, higher numbers represent lower
priorities.  Unlike the simpler kind of priority, bitvectors can be of
arbitrary length, therefore, the priority numbers they represent can
be of arbitrary precision.

Arbitrary-precision priorities\index{arbitrary-precision priorities}
are often useful in AI search-tree applications.  Suppose we have a
heuristic suggesting that tree node $N_1$ should be searched before
tree node $N_2$.  We therefore designate that node $N_1$ and its
descendants will use high priorities, and that node $N_2$ and its
descendants will use lower priorities.  We have effectively split the
range of possible priorities in two.  If several such heuristics fire
in sequence, we can easily split the priority range \index{priority
range splitting} in two enough times that no significant bits remain,
and the search begins to fail for lack of meaningful priorities to
assign.  The solution is to use arbitrary-precision priorities,
i.e. bitvector priorities.

To assign a bitvector priority, two methods are available.  The
first is to obtain a pointer to the priority field using  \kw{CkPriorityPtr},
and to then manually set the bits using the bit-setting operations
inherent to C.  To achieve this, one must know the format
\index{bitvector format} of the
bitvector, which is as follows: the bitvector is represented as an
array of unsigned integers.  The most significant bit of the first
integer contains the first bit of the bitvector.  The remaining bits
of the first integer contain the next 31 bits of the bitvector.
Subsequent integers contain 32 bits each.  If the size of the
bitvector is not a multiple of 32, then the last integer contains 0
bits for padding in the least-significant bits of the integer.

The second way to assign priorities is only useful for those who are
using the priority range-splitting\index{priority range splitting}
described above.  The root of the tree is assigned the null
priority-string.  Each child is assigned its parent's priority with
some number of bits concatenated.  The net effect is that the entire
priority of a branch is within a small epsilon of the priority of its
root.

It is possible to utilize unprioritized messages, integer priorities,
and bitvector priorities in the same program.  The messages will be
processed in roughly the following order\index{multiple priority types}:

\begin{itemize}

\item Among messages enqueued with bitvector priorities, the
messages are dequeued according to their priority.  The
priority ``0000...'' is dequeued first, and ``1111...'' is
dequeued last.

\item Unprioritized messages are treated as if they had the
priority ``1000...'' (which is the ``middle'' priority, it
lies exactly halfway between ``0000...'' and ``1111...'').
 
\item Integer priorities are converted to bitvector priorities.  They
are normalized so that the integer priority of zero is converted to
``1000...'' (the ``middle'' priority).  To be more specific, the
conversion is performed by adding 0x80000000 to the integer, and then
treating the resulting 32-bit quantity as a 32-bit bitvector priority.

\item Among messages with the same priority, messages are
dequeued in FIFO order or LIFO order, depending upon which
queuing strategy was used.

\end{itemize} 

A final warning about prioritized execution: \charmpp\ always processes
messages in {\it roughly} the order you specify; it never guarantees to
deliver the messages in {\it precisely} the order\index{message
delivery order} you specify.
However, it makes a serious attempt to be ``close'', so priorities
can strongly affect the efficiency of your program.

\subsubsection{Entry Method Attributes}
\label{attributes}

Charm++ provides a handful of special attributes that \index{entry method}entry
methods may have.  In order to give a particular \index{entry method}entry
method an attribute, you must specify the keyword for the desired attribute in
the attribute list of that entry method's .ci file declaration.  The syntax for
this is as follows:

\small
\keyword{entry} \userword{[attribute1, ..., attributeN] void EntryMethod(MessageType *)};
\normalsize

Charm++ currently offers four attributes that one may give an entry method:
threaded, sync, exclusive, and virtual.
\index{threaded}\index{sync}\index{exclusive}\index{virtual}

\index{threaded}Threaded \index{entry method}entry methods are simply entry
methods which are run in their own nonpremptible threads.  To make an
\index{entry method}entry method threaded, one simply adds the keyword
\keyword{threaded} to the attribute list of that entry method.

\index{sync}Sync \index{entry method}entry methods are special in that calls to
sync entry methods are blocking - they do not return control to the caller
until the method is finished executing completely.  Sync methods may have
return values; however, they may only return messages.  To make an \index{entry
method}entry method a sync entry method, add the keyword \keyword{sync} to the
attribute list of that entry method.

\index{exclusive}Exclusive entry methods, which exist only on node groups, are
\index{entry method}entry methods that do not execute while other exclusive
\index{entry method}entry methods of its node group are executing in the same
node.  If one exclusive method of a node group is executing on node 0, and
another one is scheduled to run on that same node, the second exclusive method
will wait for the first to finish before it executes.  To make an \index{entry
method}entry method exclusive, add the keyword \keyword{exclusive} to that
entry method's attribute list.

\index{virtual}Virtual \index{entry method}entry methods are inherited in the
same manner as virtual methods of sequential C++ objects.  For a detailed
discussion of inheritance in Charm++, refer to section \ref{inheritance and
templates}.  To make an entry method virtual, just add the keyword
\keyword{virtual} to that method's attribute list.  Additionally, one may make
a virtual \index{entry method}entry method a pure virtual entry method.  They
behave in the same way as pure virtual methods in C++ and are declared in a
very similar fashion.  To make a virtual entry method pure, add a ``= 0'' after
that entry method's .ci file declaration.  This looks like the following:

\small
\keyword{entry} \userword{[virtual] void PureVirtualEntry(MessageType *) = 0};
\normalsize




  \subsection{Sequential Objects}
\index{sequential objects} 

These are the same as \CC{} classes and functions.  All \CC{} features can
be used.  However, care needs to be taken when sequential objects
interact with \charmpp\ objects.

  \section{Chare Objects}

\index{chare}Chares are concurrent objects with methods that can be invoked
remotely. These methods are known as \index{entry method}entry methods. All
chares must have a constructor that is an entry method, and may have any
number of other entry methods. All chare classes and their entry methods are
declared in the interface (\texttt{.ci}) file:

\begin{alltt}
chare ChareType
\{
    entry ChareType(\uw{parameters1});
    entry void EntryMethodName(\uw{parameters2});
\};
\end{alltt}

Although it is {\em declared} in an interface file, a chare is a \CC{} object and must
have a normal \CC{} {\em implementation} (definition) in addition. A chare
class {\tt ChareType} must inherit from the class {\tt CBase\_ChareType}, which
is a special class that is generated by the \charmpp translator from the
interface file.

To be concrete, the \CC{} definition of the \index{chare}chare above might have 
the following definition in a \texttt{.h} file:

\begin{alltt}
   class ChareType : public CBase\_ChareType \{
       // Data and member functions as in C++
       // One or more entry methods definitions of the form:
       public:
           ChareType(\uw{parameters2});
           void EntryMethodName2(\uw{parameters2});
   \};
\end{alltt}

\index{chare}
Each chare encapsulates data associated with medium-grained units of work in a
parallel application.
Chares can be dynamically created on any processor; there may
be thousands of chares on a processor. The location of a chare is
usually determined by the dynamic load balancing strategy. However,
once a chare commences execution on a processor, it does not migrate
to other processors\footnote{Except when it is part of an array.}.  
Chares do not have a default ``thread of
control'': the entry methods \index{entry methods} in a
chare execute in a message driven fashion upon the arrival of a 
message\footnote{Threaded methods augment this behavior since they execute in
a separate user-level thread, and thus can block to wait for data.}.

The entry method definition specifies a function that is executed {\em without
interruption} when a message is received and scheduled for processing. Only one
message per chare is processed at a time.  Entry methods are defined exactly as
normal \CC{} function members, except that they must have the return value
\kw{void} (except for the constructor entry method which may not have a return
value, and for a {\em synchronous} entry method, which is invoked by a {\em
threaded} method in a remote chare). Each entry method can either take no
arguments, take a list of arguments that the runtime system can automatically
pack into a message and send (see section~\ref{marshalling}), or take a single
argument that is a pointer to a \charmpp message (see section~\ref{messages}).

A chare's entry methods can be invoked via {\it proxies} (see
section~\ref{proxies}). Proxies to a chare of type {\tt chareType} have type
{\tt CProxy\_chareType}. By inheriting from the CBase parent class, each chare
gets a {\tt thisProxy} member variable, which holds a proxy to itself. This
proxy can be sent to other chares, allowing them to invoke entry methods on this
chare.

\zap{
Each chare instance is identified by a {\em handle} \index{handle}
which is essentially a global pointer, and is unique across all
processors.  The handle of a chare has type \kw{CkChareID}.  The
variable \kw{thishandle} holds the handle of the
chare whose entry function or public function is currently executing.
\kw{thishandle} is a public instance variable of the chare object
which is inherited from the system-defined superclass
\kw{CBase}\_\uw{ClassType}.
Following the older syntax, chares are also allowed to inherit directly
for the superclass \kw{Chare} instead of \kw{CBase}\_\uw{ClassType}, although
this form is not suggested.
\kw{thishandle} can be used to set fields in a message. This  
mechanism allows chares to send their handles to other chares.
}

\subsection{Chare Creation}

\label{chare creation}

Once you have declared and defined a chare class, you will want to create some
chare objects to use. Chares are created by the {\tt ckNew} method, which is a
static method of the chare's proxy class:

\begin{alltt}
   CProxy_chareType::ckNew(\uw{parameters}, int destPE);
\end{alltt}

The {\tt parameters} correspond to the parameters of the chare's constructor.
Even if the constructor takes several arguments, all of the arguments should be
passed in order to {\tt ckNew}. If the constructor takes no arguments, the
parameters are omitted. By default, the new chare's location is determined by
the runtime system. However, this can be overridden by passing a value for
{\tt destPE}, which specifies the PE where the chare will be created.

The \index{chare}chare creation method deposits the \index{seed}{\em seed} for
a chare in a pool of seeds and returns immediately. The \index{chare}chare will
be created later on some processor, as determined by the dynamic \index{load
balancing}load balancing strategy (or by {\tt destPE}).
When a \index{chare}chare is created, it is
initialized by calling its \index{constructor}constructor \index{entry
method}entry method with the parameters specified by {\tt ckNew}.

Suppose we have declared a chare class {\tt C} with a constructor that takes two
arguments, an {\tt int} and a {\tt double}.

\begin{enumerate}
\item{This will create a new \index{chare}chare of type \uw{C} on {\em any}
processor and return a proxy to that chare:}

\begin{alltt}
   CProxy_C chareProxy = CProxy_C::ckNew(1, 10.0);
\end{alltt} 

\item{This will create a new \index{chare}chare of type \uw{C} on processor
\kw{destPE} and return a proxy to that chare:}

\begin{alltt}
   CProxy_C chareProxy = CProxy_C::ckNew(1, 10.0, destPE);
\end{alltt}

\end{enumerate}

For an example of chare creation in a full application, see
{\tt examples/charm++/fib} in the \charmpp software distribution, which
calculates fibonacci numbers in parallel.

\subsection{Method Invocation on Chares}

A message \index{message} may be sent to a \index{chare}chare through a proxy
object using the notation:

\begin{tabbing}
chareProxy.EntryMethod(\uw{parameters})
\end{tabbing}

This invokes the entry method \uw{EntryMethod} on the chare referred
to by the proxy \uw{chareProxy}. This call
is asynchronous and non-blocking; it returns immediately after sending the
message. 


\subsection{Local Access}

You can get direct access to a local chare using the
proxy's \kw{ckLocal} method, which returns an ordinary \CC\ pointer
to the chare if it exists on the local processor, and NULL otherwise.

\begin{alltt}
C *c=chareProxy.ckLocal();
if (c==NULL) //...is remote-- send message
else //...is local-- directly use members and methods of c
\end{alltt}


  \section{Group Objects}
\label{sec:group}

So far, we have discussed chares separately from the underlying hardware resources 
to which they are mapped. However, while writing lower-level libraries it is sometimes
useful to be able to refer to the PE on which a chare's entry method is being executed.
The \kw{group} \footnote{Originally called {\em Branch Office Chare} or 
{\em Branched Chare}}construct provides this facility by creating a 
%A \kw{group}
% \index{group}is a 
collection of chares, such that 
there exists \index{chare}a single chare (or {\sl branch} of the group) on each
PE.   Each branch has its own data members.  Groups have
a definition syntax similar to normal chares,
and they have to inherit from the system-defined class \kw{CBase}\_\uw{ClassName}, 
where \uw{ClassName} is the name of group's \CC{} class
\footnote{Older, deprecated syntax allows groups to inherit directly from the
system-defined class \kw{Group}}.

\subsection{Group Definition}

In the interface ({\tt .ci}) file, we declare

\begin{alltt}
group Foo \{
  // Interface specifications as for normal chares

  // For instance, the constructor ...
  entry Foo(\uw{parameters1});

  // ... and an entry method
  entry void someEntryMethod(\uw{parameters2});
\};
\end{alltt}

The definition of the {\tt Foo} class is given in the \texttt{.h} file, as follows:

\begin{alltt}
class Foo : public CBase\_Foo \{
  // Data and member functions as in C++
  // Entry functions as for normal chares

  public:
    Foo(\uw{parameters1});
    void someEntryMethod(\uw{parameters2});
\};
\end{alltt}

\subsection{Group Creation}

Groups are created in a manner similar to chares and chare arrays, i.e. 
through \kw{ckNew}. Given the declarations and definitions of group {\tt Foo}
from above, we can create a group in the following manner:

\begin{alltt}
CkGroupID fooGroupID = CProxy_Foo::ckNew(\uw{parameters1});
\end{alltt}


In the above, \kw{ckNew} returns an object of type \kw{CkGroupID}, which is
the globally unique identifier of the corresponding instance of the group.
This identifier is common to all of the group's branches and
can be obtained in the group's methods from the variable \kw{thisgroup}, which is a public data
member of the \kw{Group} superclass.

A group can also be identified through its proxy, which can be obtained in one of three ways:
(a) as the inherited {\tt thisProxy} data member of the class; (b) from a call to \kw{ckNew} 
as shown below:

\begin{alltt}
CkGroupID fooGroupID = CProxy_Foo::ckNew(\uw{parameters1});
\end{alltt}

or (c) by using a group identifier to create a proxy, as shown below:

\begin{alltt}
  // We have `fooGroupID' from the above `ckNew' invocation

  // Obtain a proxy to the group from its group ID
  CProxy_Foo anotherFooProxy = CProxy_Foo(fooGroupID);
\end{alltt}

It is possible to specify the dependence of group creations using
\uw{CkEntryOptions}. For example, in the following code, the creation of group
{\tt GroupB} on each PE depends on the creation of {\tt GroupA} on that PE.

\begin{alltt}
// Create GroupA
CkGroupID groupAID = CProxy_GroupA::ckNew(\uw{parameters1});

// Create GroupB. However, for each PE, do this only 
// after GroupA has been created on it

// Specify the dependency through a `CkEntryOptions' object
CkEntryOptions opts;
opts.setGroupDepID(groupAId);

// The last argument to `ckNew' is the `CkEntryOptions' object from above
CkGroupID groupBID = CProxy_GroupB::ckNew(\uw{parameters2}, opts);
\end{alltt}

%For groups, \kw{thishandle} is the
%handle of the particular branch in which the function is executing: it is a
%normal chare handle.

%Groups can be used to implement data-parallel operations easily.  In addition
%to sending messages to a particular branch of a group, one can broadcast
%messages to all branches of a group.  
Note that there can be several instances of each group type.
In such a case, each instance has a unique group identifier, and its own set
of branches.

\subsection{Method Invocation on Groups}

An asynchronous entry method can be invoked on a particular branch of a
group through a proxy of that group. If we have a group with a proxy
{\tt fooProxy} and we wish to invoke entry method {\tt someEntryMethod} on
that branch of the group which resides on PE {\tt somePE}, we would accomplish
this with the following syntax:

\begin{alltt}
 fooProxy[somePE].someEntryMethod(\uw{parameters});
\end{alltt}

%This sends the given parameters to the \index{branch}branch of
%the group referred to by \uw{groupProxy} which is on processor number
%\uw{Processor} at the entry method \uw{EntryMethod}, which must be a valid
%entry method of that group type. 
This call is asynchronous and non-blocking; it returns immediately after sending the message.
A message may be broadcast \index{broadcast} to all branches of a group
(i.e., to all PEs) using the notation :

\begin{alltt}
 fooProxy.anotherEntryMethod(\uw{parameters});
\end{alltt}

This invokes entry method \uw{anotherEntryMethod} with the given \uw{parameters} on 
all branches of the group. This call is asynchronous and non-blocking; it returns immediately
after sending the message.

Recall that each PE hosts a branch of every instantiated group. 
Sequential objects, chares and other groups can gain access to this {\em PE-local}
branch using \kw{ckLocalBranch()}:

\begin{alltt}
GroupType *g=fooProxy.ckLocalBranch();
\end{alltt}

This call returns a regular \CC\ pointer to the actual object (not a proxy)
referred to by the proxy \uw{groupProxy}.  Once a proxy to the
local branch of a group is obtained, that branch can be accessed as a regular
\CC\ object.  Its public methods can return values, and its public data is 
readily accessible.

Thus a dynamically created \index{chare}chare can invoke a public method of a
group without knowining the PE on which it actually resides. 
%the method
%executes in the local \index{branch}branch of the group.

Let us end with an example use-case for groups.
%One very nice use of Groups is to reduce the number of messages sent between
%processors by collecting the data from all the chares on a single processor
Suppose that we have a task-parallel program in which we dynamically spawn
new chares. Furthermore, assume that each one of these chares has some data
to send to the mainchare.  Instead of creating a separate message for each 
chare's data, we create a group. When a particular chare
finishes its work, it reports its findings to the local branch of the group.
When all the chares on a PE have finished their work, the local branch
can send single a message to the main chare.  This reduces the number of messages
sent to the mainchare from the number of chares created, to the number of processors. 


  \subsection{NodeGroup Objects}

The {\em node group} construct \index{node groups} \index{nodegroup} \index{Nodegroup} is 
similar to the group construct discussed above. 
That is, a node group is a collection of chares that can be addressed via globally unique
identifier. 
%already discussed in that node groups are
%collections of chares as well.  
However, a node group has one chare per {\em process}, or {\em logical node}, rather than one chare per PE.
%rather than one chare per processor.  
Therefore, each logical node hosts a single branch of the
node group.  When an entry method of a
node group is executed on one of its branches, it executes on {\em some} PE within the node.

\subsubsection{NodeGroup Declaration} 

Node groups are defined in a similar way to groups.  \footnote{As with groups,
older syntax allows node groups to inherit from \kw{NodeGroup} instead of a
specific, generated ``\uw{CBase\_}'' class.} For example, in the interface file, we declare:

\begin{alltt}
 nodegroup NodeGroupType \{
  // Interface specifications as for normal chares
 \};
\end{alltt}

In the {\tt .h} file, we define \uw{NodeGroupType} as follows:

\begin{alltt}
 class NodeGroupType : public CBase_NodeGroupType \{
  // Data and member functions as in \CC{}
  // Entry functions as for normal chares
 \};
\end{alltt}

Like groups, NodeGroups are identified by a globally unique identifier of type
\index{CkGroupID}\kw{CkGroupID}.  Just as with groups, this identifier is
common to all branches of the NodeGroup, and can be obtained from the inherited
data member \index{thisgroup}\kw{thisgroup}.
There can be many instances corresponding to a single NodeGroup
type, and each instance has a different identifier, and its own set of
branches.


%, and once again, \index{thishandle}
%\kw{thishandle} is the handle of the particular branch in which the function is
%executing.


\subsubsection{Method Invocation on NodeGroups}

As with chares, chare arrays and groups, entry methods are invoked on
NodeGroup branches via proxy objects. 
An entry method may be invoked on a {\em particular} \index{branch}branch of a
\index{nodegroup}nodegroup by specifying a {\em logical node number} argument
to the square bracket operator of the proxy object. A broadcast is expressed
by omitting the square bracket notation. For completeness, example syntax for these
two cases is shown below:

\begin{alltt}
 // Invoke `someEntryMethod' on the i-th logical node of
 // a NodeGroup whose proxy is `myNodeGroupProxy':
 myNodeGroupProxy[i].someEntryMethod(\uw{parameters});

 // Invoke `someEntryMethod' on all logical nodes of
 // a NodeGroup whose proxy is `myNodeGroupProxy':
 myNodeGroupProxy[i].someEntryMethod(\uw{parameters});
\end{alltt}

%In the absence of such a parameter, the call is treated as a broadcast
%to all branches of the NodeGroup of the a \index{nodegroup}nodegroup, i.e. executed by all nodes. 
It is worth restating that when an entry method is
invoked on a particular \index{branch}branch of a \index{nodegroup}nodegroup,
it may be executed by {\em any} PE in that logical node. Thus two invocations of
a single entry method on a particular \index{branch}branch of a
\index{nodegroup}NodeGroup may be executed {\em concurrently} by two
different PEs in the logical node. If this may cause data races in your
program, please consult \S~\ref{sec:nodegroups/exclusive} (below).

%If that method contains code that should be
%executed by only one processor at a time, the method should be flagged
%\index{exclusive}\kw{exclusive} in the interface file. 

\subsubsection{NodeGroups and \kw{exclusive} Entry Methods}
\ref{sec:nodegroups/exclusive}

Node groups may have \index{exclusive}\kw{exclusive} entry methods.  The execution of an \kw{exclusive}
entry method invocation is mutually exclusive with those of all other \kw{exclusive} entry methods invocations.
That is, an \kw{exclusive}
entry method invocation is not executed on a logical node as long as another \kw{exclusive} entry 
method is executing on it.
More explicitly, if a method \uw{M} of a
nodegroup \uw{NG} is marked exclusive, it means that while an instance of that method is being
executed by a PE within a logical node, no other PE within that
logical node will execute any other {\em exclusive} methods.
%of that
%\index{nodegroup}nodegroup \index{branch}branch.  
However, PEs in the logical node may still 
execute {\em non-exclusive} entry method invocations.
%on that l
%\index{branch}branch, however.
%of that node group are running on the same node.  
An entry method can be marked exclusive by tagging it with the \kw{exclusive} attribute,
as explained in \S~\ref{attributes}.


\subsubsection{Accessing the Local Branch of a NodeGroup}

The local \index{branch}branch of a \kw{NodeGroup} \uw{NG}, and hence its
member fields and methods, can be accessed through the method \kw{NG*
CProxy\_NG::ckLocalBranch()} of its proxy. Note that accessing data members of
a NodeGroup branch in this manner is {\em not} thread-safe by default, although
you may implement your own mutual exclusion schemes to ensure safety.
%accesses are {\em not} thread-safe by default.  Concurrent invocation of a
%method on a \index{nodegroup}nodegroup by different processors within a node
%may result in unpredictable runtime behavior.  
One way to ensure safety is to use node-level locks, which are described in the
Converse manual.

%For certain applications, node groups can be used in the place of regular
%groups to mitigate messaging overhead when sharing of address spaces between 
%PEs is possible.
%For example, consider a parallel program that does one calculation that can be
%decomposed into several mutually exclusive subcalculations.  The program
%distributes the work amongst all of the processors, the subresults are all
%stored in the local branch of a group, and when the local branch has recieved
%all of its results, it relays everything to one particular processor where the
%subresults are put together into the final result.  When normal groups are
%used, the number of messages sent is $O$(\# of processors).  However, if node
%groups are used, a number of message sends will be replaced by local memory
%accesses if there is more than one processor per node.  Instead, the number of
%messages sent is $O$(\# of nodes).
NodeGroups can be used in a similar way to groups so as to implement lower-level
optimizations such as data sharing and message reduction.



  \subsection{Basic Arrays}

Arrays \index{arrays} are arbitrarily-sized collections of chares.  The
entire array has a globally unique identifier of type \kw{CkArrayID}, and
each element has a unique index of type \kw{CkArrayIndex}.  A \kw{CkArrayIndex}
can be a single integer (i.e. 1D array), several integers (i.e. a
multidimensional array), or an arbitrary string of bytes (e.g. a binary tree
index).

Array elements can be dynamically created and destroyed on any processor,
and messages for the elements will still arrive properly.
Array elements can be migrated at any time, allowing arrays to be efficiently
load balanced.  Array elements can also receive array broadcasts and
contribute to array reductions.

\subsubsection{Declaring a 1D Array}

You can declare a one-dimensional \index{array}\index{chare array}chare array
as:

\begin{alltt}
//In the .ci file:
array [1D] A \{
  entry A(\uw{parameters1});
  entry void someEntry(\uw{parameters2});
\};
\end{alltt}

Just as every Chare inherits from the system class \kw{Chare}, every 
array element inherits from the system class \kw{ArrayElement} (or one
of its subclasses, \kw{ArrayElement1D}, \kw{ArrayElement2D}, 
\kw{ArrayElement3D}, \kw{ArrayElement4D}, \kw{ArrayElement5D}, or
\kw{ArrayElement6D}). Just as a Chare inherits ``thishandle'', each
array element inherits ``thisArrayID'', the \kw{CkArrayID} of its array,
and ``thisIndex'', the element's array index.

\begin{alltt}
class A : public ArrayElement1D \{
  public:
    A(\uw{parameters1});
    A(CkMigrateMessage *);

    void someEntry(\uw{parameters2});
\};
\end{alltt}

Note \uw{A}'s odd migration constructor, which is normally empty:

\begin{alltt}
//In the .C file:
A::A(void)
\{
  //...your constructor code...
\}
A::A(CkMigrateMessage *m) \{ \}
\end{alltt}

Read the section ``Migratable Array Elements'' for more
information on the \kw{CkMigrateMessage} constructor. 


\subsubsection{Creating a Simple Array}

You always create an array using the \kw{CProxy\_Array::ckNew}
routine.  This returns a proxy object, which can
be kept, copied, or sent in messages.
To create a 1D \index{array}array containing elements indexed 
(0, 1, ..., \uw{num\_elements}-1), use:

\begin{alltt}
CProxy_A1 a1 = CProxy_A1::ckNew(\uw{parameters},num_elements);
\end{alltt}

The constructor is invoked on each array element.
For creating higher-dimensional arrays, or for more options
when creating the array, see section~\ref{advanced array create}.


\subsubsection{Messages}

An array proxy responds to the appropriate index call--
for 1D arrays, use [i] or (i); for 2D use (x,y); for 3D
use (x,y,z); and for user-defined types use [f] or (f).

To send a \index{Array message} message to an array element, index the proxy 
and call the method name:

\begin{alltt}
a1[i].doSomething(\uw{parameters});
a3(x,y,z).doAnother(\uw{parameters});
aF[CkArrayIndexFoo(...)].doAgain(\uw{parameters});
\end{alltt}

You may invoke methods on array elements that have not yet
been created-- by default, the system will buffer the message until the
element is created\footnote{However, the element must eventually be 
created-- i.e., within a 3-minute buffering period.}.

Messages are not guarenteed to be delivered in order.
For example, if I invoke a method A, then method B;
it is possible for B to be executed before A.

\begin{alltt}
a1[i].A();
a1[i].B();
\end{alltt}

Messages sent to migrating elements will be delivered after
the migrating element arrives.  It is an error to send 
a message to a deleted array element.


\subsubsection{Broadcasts}
To \index{Array broadcast} broadcast a message to all the current elements of an array, 
simply omit the index, as:

\begin{alltt}
a1.doIt(\uw{parameters}); //<- invokes doIt on each array element
\end{alltt}

The broadcast message will be delivered to every existing array 
element exactly once.  Broadcasts work properly even with ongoing
migrations, insertions, and deletions.


\subsubsection{Reductions on Chare Arrays}
A \index{array reduction}reduction applies a single operation (e.g. add,
max, min, ...) to data items scattered across many processors and
collects the result in one place.  \charmpp{} supports reductions on the
elements of a Chare array.

The data to be reduced comes from each array element, 
which must call the \kw{contribute} method:

\begin{alltt}
ArrayElement::contribute(int nBytes,const void *data,CkReduction::reducerType type);
\end{alltt}

Reductions are described in more detail in Section~\ref{reductions}.


\subsubsection{Destroying Arrays}
To destroy an array element-- detach it from the array,
call its destructor, and release its memory--invoke its 
\kw{Array destroy} method, as:

\begin{alltt}
a1[i].ckDestroy();
\end{alltt}

You must ensure that no messages are sent to a deleted element. 
After destroying an element, you may insert a new element at
its index.




\subsection{Advanced Arrays}
\label{advanced arrays}

The basic array features described above (creation, messaging,
broadcasts, and reductions) are needed in almost every
\charmpp{} program.  The more advanced techniques that follow
are not universally needed; but are still often useful.


\subsubsection{Declaring Multidimensional, or User-defined Index Arrays}

\charmpp{} contains direct support for multidimensional and
even user-defined index arrays.  These arrays can be declared as:

\begin{alltt}
//In the .ci file:
message MyMsg;
array [1D] A1 \{ entry A1(); entry void e(\uw{parameters});\}
array [2D] A2 \{ entry A2(); entry void e(\uw{parameters});\}
array [3D] A3 \{ entry A3(); entry void e(\uw{parameters});\}
array [4D] A4 \{ entry A4(); entry void e(\uw{parameters});\}
array [5D] A5 \{ entry A5(); entry void e(\uw{parameters});\}
array [6D] A6 \{ entry A6(); entry void e(\uw{parameters});\}
array [Foo] AF \{ entry AF(); entry void e(\uw{parameters});\}
\end{alltt}

The last declaration expects an array index of type \kw{CkArrayIndex}\uw{Foo},
which must be defined before including the \texttt{.decl.h} file 
(see ``User-defined array index type'' below).  

\begin{alltt}
//In the .h file:
class A1:public ArrayElement1D \{ public: A1()\{\} ...\};
class A2:public ArrayElement2D \{ public: A2()\{\} ...\};
class A3:public ArrayElement3D \{ public: A3()\{\} ...\};
class A4:public ArrayElement4D \{ public: A4()\{\} ...\};
class A5:public ArrayElement5D \{ public: A5()\{\} ...\};
class A6:public ArrayElement6D \{ public: A6()\{\} ...\};
class AF:public ArrayElementT<Foo> \{ public: AF()\{\} ...\};
\end{alltt}

A 1D array element can access its index via its inherited ``thisIndex''
field; a 2D via ``thisIndex.x'' and ``thisIndex.y'', and a 3D via
``thisIndex.x'', ``thisIndex.y'', and ``thisIndex.z''. The subfields
of 4D, 5D, and 6D are respectively \{w,x,y,z\}, \{v,w,x,y,z\}, and 
\{x1,y1,z1,x2,y2,z2\}.
A user-defined index array can access its index as ``thisIndex''.


\subsubsection{Advanced Array Creation}
\label{advanced array create}
There are several ways to control the array creation process.
You can adjust the map and bindings before creation, change
the way the initial array elements are created, create elements
explicitly during the computation, and create elements implicitly,
``on demand''.  

You can create all your elements using any one of these methods,
or create different elements using different methods.  
An array element has the same syntax and semantics no matter
how it was created.


\subsubsection{Advanced Array Creation: CkArrayOptions}
\index{CkArrayOptions}
\label{CkArrayOptions}

The array creation method \kw{ckNew} actually takes a parameter
of type \kw{CkArrayOptions}.  This object describes several
optional attributes of the new array.

The most common form of \kw{CkArrayOptions} is to set the number
of initial array elements.  A \kw{CkArrayOptions} object will be 
constructed automatically in this special common case.  Thus
the following code segments all do exactly the same thing:

\begin{alltt}
//Implicit CkArrayOptions
  a1=CProxy_A1::ckNew(\uw{parameters},nElements);

//Explicit CkArrayOptions
  a1=CProxy_A1::ckNew(\uw{parameters},CkArrayOptions(nElements));

//Separate CkArrayOptions
  CkArrayOptions opts(nElements);
  a1=CProxy_A1::ckNew(\uw{parameters},opts);
\end{alltt}

Note that the ``numElements'' in an array element is simply
the numElements passed in when the array was created.
The true number of array elements may grow or shrink during 
the course of the computation, so numElements can become 
out of date. 


\subsubsection{Advanced Array Creation: Map Object}
\index{array map}
\label{array map}

You can use \kw{CkArrayOptions} to specify a ``map object''
for an array.  The map object is used by the array manager
to determine the ``home'' processor of each element.  The
home processor is the processor responsible for maintaining
the location of the the element.

There is a default map object, which maps 1D array indices
in a round-robin fashion to processors, and maps other array
indices based on a hash function.

A custom map object is implemented as a group which inherits from
\kw{CkArrayMap} and defines these virtual methods:

\begin{alltt}
class CkArrayMap : public Group
\{
public:
  //...
  
  //Return an ``arrayHdl'', given some information about the array
  virtual int registerArray(int numInitialElements,CkArrayID aid);
  //Return the home processor number for this element of this array
  virtual int procNum(int arrayHdl,const CkArrayIndex &element);
\}
\end{alltt}

Once you've instantiated a custom map object, you can use it to
control the location of a new array's elements using the
\kw{setMap} method of the \kw{CkArrayOptions} object described above.
For example, if you've declared a map object named ``blockMap'':

\begin{alltt}
//Create the map group
  CProxy_blockMap myMap=CProxy_blockMap::ckNew();
//Make a new array using that map
  CkArrayOptions opts(nElements);
  opts.setMap(myMap);
  a1=CProxy_A1::ckNew(\uw{parameters},opts);
\end{alltt}



\subsubsection{Advanced Array Creation: Initial Elements}
\index{array initial}
\label{array initial}

The map object described above can also be used to create
the initial set of array elements in a distributed fashion.
An array's initial elements are created by its map object,
by making a call to \kw{populateInitial} on each processor.

You can create your own set of elements by creating your
own map object and overriding this virtual function of \kw{CkArrayMap}:

\begin{alltt}
  virtual void populateInitial(int arrayHdl,int numInitial,
	void *msg,CkArrMgr *mgr)
\end{alltt}

In this call, \kw{arrayHdl} is the value returned by \kw{registerArray},
\kw{numInitial} is the number of elements passed to \kw{CkArrayOptions},
\kw{msg} is the constructor message to pass, and \kw{mgr} is the
array to create.

\kw{populateInitial} creates new array elements using the method
\kw{void CkArrMgr::insertInitial(CkArrayIndex idx,void *ctorMsg)}.
For example, to create one row of 2D array elements on each processor,
you would write:

\begin{alltt}
void xyElementMap::populateInitial(int arrayHdl,int numInitial,
	void *msg,CkArrMgr *mgr)
\{
  if (numInitial==0) return; //No initial elements requested
	
  //Create each local element
  int y=CkMyPe();
  for (int x=0;x<numInitial;x++) \{
    mgr->insertInitial(CkArrayIndex2D(x,y),CkCopyMsg(&msg));
  \}
  mgr->doneInserting();
  CkFreeMsg(msg);
\}
\end{alltt}

Thus calling \kw{ckNew(10)} on a 3-processor machine would result in
30 elements being created.


\subsubsection{Advanced Array Creation: Bound Arrays}
\experimental{}
\index{bound arrays} \index{bindTo}
\label{bound arrays}
You can ``bind'' a new array to an existing array
using the \kw{bindTo} method of \kw{CkArrayOptions}.  Bound arrays
act like separate arrays in all ways except for migration--
corresponding elements of bound arrays always migrate together.
For example, this code creates two arrays A and B which are
bound together-- A[i] and B[i] will always be on the same processor.

\begin{alltt}
//Create the first array normally
  aProxy=CProxy_A::ckNew(\uw{parameters},nElements);
//Create the second array bound to the first
  CkArrayOptions opts(nElements);
  opts.bindTo(aProxy);
  bProxy=CProxy_B::ckNew(\uw{parameters},opts);
\end{alltt}

Bound arrays are often useful if A[i] and B[i] perform different 
aspects of the same computation, and thus will run most efficiently 
if they lie on the same processor.  Bound array elements are guaranteed
to always be able to interact using \kw{ckLocal} (see 
section~\ref{ckLocal for arrays}), although the local pointer must
be refreshed after any migration. This should be done during the \kw{pup}
routine. When migrated, all elements that are bound together will be created
at the new processor before \kw{pup} is called on any of them, ensuring that
a valid local pointer to any of the bound objects can be obtained during the
\kw{pup} routine of any of the others.

An arbitrary number of arrays can be bound together--
in the example above, we could create yet another array
C and bind it to A or B.  The result would be the same
in either case-- A[i], B[i], and C[i] will always be
on the same processor.

There is no relationship between the types of bound arrays--
it is permissible to bind arrays of different types or of the
same type.  It is also permissible to have different numbers
of elements in the arrays, although elements of A which have
no corresponding element in B obey no special semantics.
Any method may be used to create the elements of any bound
array.


\subsubsection{Advanced Array Creation: Dynamic Insertion}

In addition to creating initial array elements using ckNew,
you can also
create array elements during the computation.

You insert elements into the array by indexing the proxy
and calling insert.  The insert call optionally takes 
parameters, which are passed to the constructor; and a
processor number, where the element will be created.
Array elements can be inserted in any order from 
any processor at any time.  Array elements need not 
be contiguous.

If using \kw{insert} to create all the elements of the array,
you must call \kw{CProxy\_Array::doneInserting} before using
the array.

\begin{alltt}
//In the .C file:
int x,y,z;
CProxy_A1 a1=CProxy_A1::ckNew();  //Creates a new, empty 1D array
for (x=...) \{
   a1[x  ].insert(\uw{parameters});  //Bracket syntax
   a1(x+1).insert(\uw{parameters});  // or equivalent parenthesis syntax
\}
a1.doneInserting();

CProxy_A2 a2=CProxy_A2::ckNew();   //Creates 2D array
for (x=...) for (y=...)
   a2(x,y).insert(\uw{parameters});  //Can't use brackets!
a2.doneInserting();

CProxy_A3 a3=CProxy_A3::ckNew();   //Creates 3D array
for (x=...) for (y=...) for (z=...)
   a3(x,y,z).insert(\uw{parameters});
a3.doneInserting();

CProxy_AF aF=CProxy_AF::ckNew();   //Creates user-defined index array
for (...) \{
   aF[CkArrayIndexFoo(...)].insert(\uw{parameters}); //Use brackets...
   aF(CkArrayIndexFoo(...)).insert(\uw{parameters}); //  ...or parenthesis
\}
aF.doneInserting();

\end{alltt}

The \kw{doneInserting} call starts the the reduction manager (see ``Array
Reductions'') and load balancer (see ~\ref{lbFramework})-- since
these objects need to know about all the array's elements, they
must be started after the initial elements are inserted.
You may call \kw{doneInserting} multiple times, but only the first
call actually does anything.  You may even \kw{insert} or \kw{destroy}
elements after a call to \kw{doneInserting}, with different semantics-- 
see the reduction manager and load balancer sections for details.

If you do not specify one, the system will choose a procesor to 
create an array element on based on the current map object.



\subsubsection{Advanced Array Creation: Demand Creation}

Normally, invoking an entry method on a nonexistant array
element is an error.  But if you add the attribute
\index{createhere} \index{createhome}
\kw{[createhere]} or \kw{[createhome]} to an entry method,
 the array manager will 
``demand create'' a new element to handle the message.  

With \kw{[createhome]}, the new element
will be created on the home processor, which is most efficient when messages for
the element may arrive from anywhere in the machine. With \kw{[createhere]},
the new element is created on the sending processor, which is most efficient
if when messages will often be sent from that same processor.

The new element is created by calling its default (taking no
paramters) constructor, which must exist and be listed in the .ci file.
A single array can have a mix of demand-creation and
classic entry methods; and demand-created and normally 
created elements.



\subsubsection{User-defined array index type}

\index{Array index type, user-defined}
\charmpp{} array indices are arbitrary collections of integers.
To define a new array index, you create an ordinary C++ class 
which inherits from \kw{CkArrayIndex} and sets the ``nInts'' member
to the length, in integers, of the array index.

For example, if you have a structure or class named ``Foo'', you 
can use a \uw{Foo} object as an array index by defining the class:

\begin{alltt}
#include <charm++.h>
class CkArrayIndexFoo:public CkArrayIndex \{
    Foo f;
public:
    CkArrayIndexFoo(const Foo \&in) 
    \{
        f=in;
        nInts=sizeof(f)/sizeof(int);
    \}
    //Not required, but convenient: cast-to-foo operators
    operator Foo &() \{return f;\}
    operator const Foo &() const \{return f;\}
\};
\end{alltt}

Note that \uw{Foo}'s size must be an integral number of integers--
you must pad it with zero bytes if this is not the case.
Also, \uw{Foo} must be a simple class-- it cannot contain 
pointers, have virtual functions, or require a destructor.
Finally, there is a \charmpp\ configuration-time option called
CK\_ARRAYINDEX\_MAXLEN \index{CK\_ARRAYINDEX\_MAXLEN} 
which is the largest allowable number of 
integers in an array index.  The default is 3; but you may 
override this to any value by passing ``-DCK\_ARRAYINDEX\_MAXLEN=n'' 
to the \charmpp\ build script as well as all user code. Larger 
values will increase the size of each message.

You can then declare an array indexed by \uw{Foo} objects with

\begin{alltt}
//in the .ci file:
array [Foo] AF \{ entry AF(); ... \}

//in the .h file:
class AF:public ArrayElementT<Foo>
\{ public: AF() \{\} ... \}

//in the .C file:
    Foo f;
    CProxy_AF a=CProxy_AF::ckNew();
    a[CkArrayIndexFoo(f)].insert();
    ...
\end{alltt}

Note that since our CkArrayIndexFoo constructor is not declared
with the explicit keyword, we can equivalently write the last line as:

\begin{alltt}
    a[f].insert();
\end{alltt}

When you implement your array element class, as shown above you 
can inherit from
\kw{ArrayElementT}, a class templated by the index type \uw{Foo}.
The array index (an object of type \uw{Foo}) is then accessible as 
``thisIndex''. For example:

\begin{alltt}

//in the .C file:
AF::AF()
\{
    Foo myF=thisIndex;
    functionTakingFoo(myF);
\}
\end{alltt}


\subsubsection{Migratable Array Elements}
\label{arraymigratable}
Array objects can \index{migrate}migrate from one PE to another.
For example, the load balancer (see section~\ref{lbFramework})
might migrate array elements to better balance the load between
processors.  For an array element to migrate, it must implement
a pack/unpack or ``pup'' method:

\begin{alltt}
//In the .h file:
class A2:public ArrayElement2D \{
private: //My data members:
    int nt;
    unsigned char chr;
    float flt[7];
    int numDbl;
    double *dbl;
public:	
    //...other declarations

    virtual void pup(PUP::er \&p);
\};

//In the .C file:
void A2::pup(PUP::er \&p)
\{
    ArrayElement2D::pup(p); //<- MUST call superclass's pup routine
    p|nt;
    p|chr;
    p(flt,7);
    p|numDbl;
    if (p.isUnpacking()) dbl=new double[numDbl];
    p(dbl,numDbl);
\}
\end{alltt}

Please note that if your object contains Structured Dagger code (see section ``Structured Dagger'') you must use the following syntax to correctly pup the object:

\begin{alltt}
class bar: public ArrayElement3D \{
 private:
    int a,b;
 public:
    bar_SDAG_CODE 
    ...other methods...

    virtual void pup(PUP::er& p) \{
      __sdag_pup(p);
      ...pup other data here...
    \}
\};
\end{alltt}

See the \index{PUP} section ``PUP'' for more details on pup routines
and the \kw{PUP::er} type.

The system uses one pup routine to do both packing and unpacking by
passing different types of \kw{PUP::er}s to it.  You can determine
what type of \kw{PUP::er} has been passed to you with the
\kw{isPacking()}, \kw{isUnpacking()}, and \kw{isSizing()} calls.

An array element can migrate by calling the \kw{migrateMe}(\uw{destination
processor}) member function-- this call must be the last action
in an element entry point.  The system can also migrate array elements
for load balancing (see the section~\ref{lbarray}).

To migrate your array element to another processor, the \charmpp{}
runtime will:

\begin{itemize}
\item Call your \kw{ckAboutToMigrate} method
\item Call your \uw{pup} method with a sizing \kw{PUP::er} to determine how 
big a message it needs to hold your element.
\item Call your \uw{pup} method again with a packing \kw{PUP::er} to pack 
your element into a message.
\item Call your element's destructor (killing off the old copy).
\item Send the message (containing your element) across the network.
\item Call your element's migration constructor on the new processor.
\item Call your \uw{pup} method on with an unpacking \kw{PUP::er} to unpack 
the element.
\item Call your \kw{ckJustMigrated} method
\end{itemize}

Migration constructors, then, are normally empty-- all the unpacking
and allocation of the data items is done in the element's \uw{pup} routine.
Deallocation is done in the element destructor as usual.


\subsubsection{Load Balancing Chare Arrays}
see section~\ref{lbFramework}


\subsubsection{Local Access}
\experimental{}
\index{ckLocal for arrays}
\label{ckLocal for arrays}
You can get direct access to a local array element using the
proxy's \kw{ckLocal} method, which returns an ordinary \CC\ pointer
to the element if it exists on the local processor; and NULL if
the element does not exist or is on another processor.

\begin{alltt}
A1 *a=a1[i].ckLocal();
if (a==NULL) //...is remote-- send message
else //...is local-- directly use members and methods of a
\end{alltt}

Note that if the element migrates or is deleted, any pointers 
obtained with \kw{ckLocal} are no longer valid.  It is best,
then, to either avoid \kw{ckLocal} or else call \kw{ckLocal} 
each time the element may have migrated; e.g., at the start 
of each entry method.


\subsubsection{Array Section}
\experimental{}

\charmpp{} supports array section which is a subset of array 
elements in a chare array.
A special proxy for an array section can be created given a list of array
indexes of elements.
Multicast operations are directly supported in array section proxy with
an unoptimized direct-sending implementation.
Section reduction is not directly supported by the section proxy. 
However, an optimized section multicast/reduction 
library called ''CkMulticast'' is provided as a separate library module,
which can be plugged in as a delegation of a section proxy for performing
section-based multicasts and reductions. 

For each chare array "A" declared in a ci file, a section proxy 
of type "CProxySection\_A" is automatically generated in the decl and def 
header files. 
In order to create an array section, a user needs to provide array indexes 
of all the array section members.
You can create an array section proxy in your application by 
invoking ckNew() function of the CProxySection.
For example, for a 3D array:

\begin{alltt}
  CkVec<CkArrayIndex3D> elems;    // add array indices
  for (int i=0; i<10; i++)
    for (int j=0; j<20; j+=2)
      for (int k=0; k<30; k+=2)
         elems.push_back(CkArrayIndex3D(i, j, k));
  CProxySection_Hello proxy = CProxySection_Hello::ckNew(helloArrayID, elems.getVec(), elems.size());
\end{alltt}

Alternatively, one can do the same thing by providing (lbound:ubound:stride) 
for each dimension:

\begin{alltt}
  CProxySection_Hello proxy = CProxySection_Hello::ckNew(helloArrayID, 0, 9, 1, 0, 19, 2, 0, 29, 2);
\end{alltt}

The above codes create a section proxy that contains array elements of 
[0:9, 0:19:2, 0:29:2).

For user-defined array index other than CkArrayIndex1D to CkArrayIndex6D,
one needs to use the generic array index type: CkArrayIndexMax.

\begin{alltt}
  CkArrayIndexMax *elems;    // add array indices
  int numElems;
  CProxySection_Hello proxy = CProxySection_Hello::ckNew(helloArrayID, elems, numElems);
\end{alltt}

Once you have the array section proxy, you can do multicast to all the 
section members, or send messages to one member using its index that
is local to the section, like these:

\begin{alltt}
  CProxySection_Hello proxy;
  proxy.someEntry(...)          // multicast
  proxy[0].someEntry(...)       // send to the first element in the section.
\end{alltt}

You can move the section proxy in a message to another processor, and still 
safely invoke the entry functions to the section proxy.

In the multicast example above, for a section with k members, total number 
of k messages will be sent to all the memebers, which is considered 
inefficient when several members are on a same processor, in which 
case only one message needs to be sent to that processor and delivered to
all section members on that processor locally. To support this optimization,
a separate library called CkMulticast is provided. This library also supports
section based reduction.

\label {array_section_multicast}

To use the library, you need to compile and install CkMulticast library and 
link your applications against the library using -module:

\begin{alltt}
  # compile and install the CkMulticast library, do this only once
  cd charm/net-linux/tmp
  make multicast

  # link CkMulticast library using -module when compiling application
  charmc  -o hello hello.o -module CkMulticast -language charm++ 
\end{alltt}

CkMulticast library is implemented using delegation(Sec. ~\ref{delegation}). 
A special ''CkMulticastMgr'' Chare Group is created as a 
deletegation for section multicast/reduction - all the messages sent
by the section proxy will be passed to the local delegation branch.

To use the CkMulticast delegation, you need to create the CkMulticastMgr Group 
first, and setup the delegation relationship between the section proxy and 
CkMulticastMgr Group. You only need to create one CkMulticastMgr Group though,
it can serve as multicast/reduction delegation for all array sections:

\begin{alltt}
  CProxySection_Hello sectProxy = CProxySection_Hello::ckNew(...);
  CkGroupID mCastGrpId = CProxy_CkMulticastMgr::ckNew();
  CkMulticastMgr *mcastGrp = CProxy_CkMulticastMgr(mCastGrpId).ckLocalBranch();

  sectProxy.ckSectionDelegate(mCastGrpId);  // initialize section proxy

  sectProxy.someEntry(...)           //multicast via delegation library as before
\end{alltt}

Note, to use CkMulticast library, all multicast messages must inherit from 
CkMcastBaseMsg, as following:

\begin{alltt}
class HiMsg : public CkMcastBaseMsg, public CMessage_HiMsg
\{
public:
  int *data;
\};
\end{alltt}

Due to this restriction, you need to define message explicitly for multicast 
entry functions and no parameter marshalling can be used for multicast with 
CkMulticast library.

\paragraph{Array Section Reduction} 

Since an array element can be members for multiple array sections, 
there has to be a way for each array element to tell for which array
section it wants to contribute. For this purpose, a data structure 
called ''CkSectionInfo'' is created by CkMulticastMgr for each 
array section that the array element belongs to.
When doing section reduction, the array element needs to pass the 
\kw{CkSectionInfo} as a parameter in the \kw{contribute()}. 
The \kw{CkSectionInfo} can be retrieved
from a message in a multicast entry function using function call 
\kw{CkGetSectionInfo}:

\begin{alltt}
  CkSectionInfo cookie;

  void SayHi(HiMsg *msg)
  \{
    CkGetSectionInfo(cookie, msg);     // update section cookie every time
    int data = thisIndex;
    mcastGrp->contribute(sizeof(int), &data, CkReduction::sum_int, cookie);
  \}
\end{alltt}

Note that the cookie cannot be used as a one-time local variable in the 
function, the same cookie is needed for the next contribute. This is 
because cookie includes some context sensive information for example the 
reduction counter. Function \kw{CkGetSectionInfo()} only update some part 
of the data in cookie, not creating a brand new one.

Similar to array reduction, to use section based reduction, a reduction
client CkCallback object need to be created. You may pass the client callback 
as an additional parameter to \kw{contribute}. If different contribute calls 
pass different callbacks, some (unspecified, unreliable) callback will be 
chosen for use. See the followin example:

\begin{alltt}
    CkCallback cb(CkIndex_myArrayType::myReductionEntry(NULL),thisProxy); 
    mcastGrp->contribute(sizeof(int), &data, CkReduction::sum_int, cookie, cb);
\end{alltt}

If no member passes a callback to contribute, the reduction will use the 
default callback. You set the default callback for an array section using the 
\kw{setReductionClient} call by the section root member. A 
{\bf CkReductionMsg} message will be passed to this callback, which 
must delete the message when done.

\begin{alltt}
  CProxySection_Hello sectProxy;
  CkMulticastMgr *mcastGrp = CProxy_CkMulticastMgr(mCastGrpId).ckLocalBranch();
  mcastGrp->setReductionClient(sectProxy, new CkCallback(...));
\end{alltt}

Same as in array reduction, users can use built-in reduction 
types(Section~\ref{builtin_reduction}) or define his/her own reducer functions
(Section~\ref{new_type_reduction}).

\paragraph{Array section multicast/reduction when migration happens}
Using multicast/reduction, you don't need to worry about array migrations.
When migration happens, you can still use the \kw{CkSectionInfo} for
reduction. Reduction messages will be correctly delivered but 
may not be as efficient until a new multicast spanning tree is updated
internally in \kw{CkMulticastMgr}. A new updated \kw{CkSectionInfo} is always
contained in a multicast message, so it is recommended that 
\kw{CkGetSectionInfo()} function is called everytime time when a multicast 
message arrives(as shown in the above SayHi example).




  \section{Read-only Variables, Messages and Arrays}

Since \charmpp\ does not allow global variables for keeping
programs portable across a wide range of machines, it provides a special
mechanism for sharing data amongst all objects. {\it Read-only}
variables, messages and arrays are used to share information that 
is obtained only after the program begins execution and does not
change after they are initialized in the dynamic scope of 
{\tt main::main()} function. They
can be accessed from any \index{chare}chare on any processor as ``global''
variables. Large data structures containing pointers can be made
available as read-only variables using read-only messages or
read-only arrays. Read-only variables, messages and arrays can
be used just like local variables for each processor, but the user has
to allocate space for read-only messages using \kw{new} to create
the message in the {\tt main} function of the \kw{mainchare}. 

Read-only variables, messages, and arrays are declared by using the type
modifier \kw{readonly}, which is similar to \kw{const} in
\CC. Read-only data is specified in the {\tt .ci} file (the interface
file) as: 

\begin{tabbing}
~~~~ \=~~~~ \=~~~~ \=~~~~ \=~~~~ \=~~~~ \=~~~~ \=~~~~ \=~~~~ \=~~~~ \kill
\> \kw{readonly} \uw{Type} {\it ReadonlyVarName};
\end{tabbing}

The variable {\it ReadonlyVarName} is declared to be a read-only
variable of type \uw{Type}. \uw{Type} must be a single token and not a
type expression.

\begin{tabbing}
~~~~ \=~~~~ \=~~~~ \=~~~~ \=~~~~ \=~~~~ \=~~~~ \=~~~~ \=~~~~ \=~~~~ \kill
\> \kw{readonly} \uw{MessageType} *{\it ReadonlyMsgName};
\end{tabbing}

The variable {\it ReadonlyMsgName} is declared to be a read-only
message of type \uw{MessageType}. Pointers are not allowed to be
readonly variables unless they are pointers to message types. In this
case, the message will be initialized on every processor.

\begin{tabbing}
~~~~ \=~~~~ \=~~~~ \=~~~~ \=~~~~ \=~~~~ \=~~~~ \=~~~~ \=~~~~ \=~~~~ \kill
\> \kw{readonly} \uw{Type} {\it ReadonlyArrayName} [{\it arraysize}];
\end{tabbing}

The variable {\it ReadonlyArrayName} is declared to be a read-only
array of type \uw{Type}. \uw{Type} must be a single token and not a
type expression.

Read-only variables, messages and arrays must be declared either as
global or as public class static data, and these declarations have the
usual form:

\begin{tabbing}
~~~~ \=~~~~ \=~~~~ \=~~~~ \=~~~~ \=~~~~ \=~~~~ \=~~~~ \=~~~~ \=~~~~ \kill
\> \uw{Type} {\it ReadonlyVarName}; \\
\> \uw{MessageType} *{\it ReadonlyMsgName}; \\
\> \uw{Type} {\it ReadonlyArrayName} [{\it arraysize}];
\end{tabbing}

Similar declarations preceded by \kw{extern} would appear in the {\tt
.h} file. 

{\it Note:}  The current \charmpp\ compiler cannot prevent
assignments to read-only variables.  The user must make sure that no
assignments occur in the program.




    
  \section{Completion Detection}

Completion detection is a method for automatically detecting completion of a
distributed process within an application. This functionality is helpful when
the exact number of messages expected by individual objects is not known. In
such cases, the process must achieve global consensus as to the number of
messages produced and the number of messages consumed.  Completion is reached
within a distributed process when the participating objects have produced and
consumed an equal number of events globally. The number of global events that
will be produced and consumed does not need to be known, just the number of
producers is required.


The completion detection feature is implemented in \charmpp{} as a
module, and therefore is only included when ``{\tt -module completion}'' is
specified when linking your application.

First, the detector should be constructed. This call would typically
belong in application startup code (it initializes the group that
keeps track of completion):

\begin{alltt}
CProxy_CompletionDetector detector = CProxy_CompletionDetector::ckNew();
\end{alltt}

When it is time to start completion detection, invoke the following method of the
library on {\em all} branches of the completion detection group:

\begin{alltt}
void start_detection(int num_producers, CkCallback start, CkCallback allProduced, CkCallback finish, int prio_);
\end{alltt}

The \verb|num_producers| parameter is the number of objects (chares)
that will produce elements. So if every chare array element will produce one
event, then it would be the size of the array.

The \verb|start| callback notifies your program that it is safe to
begin producing and consuming (this state is reached when the module
has finished its internal initialization).

The \verb|allProduced| callback notifies your program when the client has
called \verb|done| with arguments summing to \verb|num_producers|.

The \verb|finish| callback is invoked when completion has been
detected (all objects participating have produced and consumed an
equal number of elements globally).

The \verb|prio| parameter is the priority with which the completion detector will run. 
This feature is still under development, but it should be set below the
application's priority if possible.

For example, the call

\begin{alltt}
detector.start_detection(10, CkCallback(CkIndex_chare1::start_test(0), thisProxy),
                             CkCallback(CkIndex_chare2::finish_test(0), thisProxy), 0);
\end{alltt}

sets up completion detection for 10 producers. Once initialization is done, the callback 
associated with the {\tt start\_test} method will be invoked. Furthermore, when the system
detects completion, the callback associated with {\tt finish\_test} will be invoked. Finally,
the priority given to the completion detection library is set to 0 in this case.

Once initialization is complete (the ``start'' callback is triggered),
make the following call to the library:

\begin{alltt}
void CompletionDetector::produce(int events_produced)
void CompletionDetector::produce() // 1 by default
\end{alltt}

For example, within the code for a chare array object, you might make the following call:
\begin{alltt}
detector.ckLocalBranch()->produce(4);
\end{alltt}

Once all the ``events'' that this chare is going to produce have been sent out,
make the following call:

\begin{alltt}
void CompletionDetector::done(int producers_done)
void CompletionDetector::done() // 1 by default
\end{alltt}

\begin{alltt}
detector.ckLocalBranch()->done();
\end{alltt}

At the same time, objects can also consume produced elements, using the following calls:

\begin{alltt}
void CompletionDetector::consume(int events_consumed)
void CompletionDetector::consume() // 1 by default
\end{alltt}

\begin{alltt}
detector.ckLocalBranch()->consume();
\end{alltt}

Note that an object may interleave calls to {\tt produce()} and {\tt consume()}, i.e.
it could produce a few elements, consume a few, etc. When it is done producing its elements,
it should call {\tt done()}, after which cannot {\tt produce()} any more elements. However,
it can continue to {\tt consume()} elements even after calling {\tt done()}. 
When the library detects that, globally, the number of produced elements equals
the number of consumed elements, and all producers have finished producing
(i.e. called {\tt done()}), it will invoke the \verb|finish| callback.
Thereafter, \verb|start_detection| can be called again to restart the process.

\section{Quiescence Detection}
\label{sec:qd}

In \charmpp, \index{quiescence}quiescence is defined as the state in which no
processor is executing an entry point, no messages are awaiting processing, and
there are no messages in-flight.  \charmpp\ provides two facilities for
detecting quiescence: \kw{CkStartQD} and \kw{CkWaitQD}.  \kw{CkStartQD}
registers with the system a callback that is to be invoked the next time
\index{quiescence}quiescence is detected. Note that if immediate messages are
used, QD cannot be used.  \kw{CkStartQD} has two variants
which expect the following arguments: 

\begin{enumerate}
\item 
A \uw{CkCallback} object. The syntax of this call looks like:
\begin{alltt}
  CkStartQD(const CkCallback& cb);
\end{alltt}

Upon quiescence detection, the specified callback is called with no parameters. Note that
using this variant, you could have your program terminate after quiescence is detected, by
supplying the above method with a CkExit callback (\S~\ref{sec:callbacks/creating}).

\item An index corresponding to the entry function that is to be called,
and a handle to the chare on which that entry function should be called.  The
syntax of this call looks like this:

\begin{alltt}
 CkStartQD(int Index,const CkChareID* chareID);
\end{alltt}

To retrieve the corresponding index of a particular \index{entry method}entry
method, you must use a static method contained within the
(\kw{charmc}-generated) \uw{CkIndex} object corresponding to the
\index{chare}chare containing that entry method.  The syntax of this call is as
follows:

\begin{alltt}
\kw{myIdx}=CkIndex_\uw{ChareClass}::\uw{entryMethod}(\uw{parameters});
\end{alltt}

where \uw{ChareClass} is the \CC{} class of the chare containing
the desired entry method, \uw{entryMethod} is the name of that entry method,
and \uw{parameters} are the parameters taken by the method.
These parameters are only used to resolve the proper \uw{entryMethod};
they are otherwise ignored.

\end{enumerate}

\kw{CkWaitQD}, by contrast, does not register a callback.  Rather,
\kw{CkWaitQD} {\em blocks} and does not return until \index{quiescence}quiescence is
detected.  It takes no parameters and returns no value.  A call to
\kw{CkWaitQD} simply looks like this: 

\begin{alltt}
  CkWaitQD();
\end{alltt}

Note that \kw{CkWaitQD} should only be called from a threaded
\index{entry method}entry method because a call to \kw{CkWaitQD} suspends the
current thread of execution ({\em cf.} \S~\ref{threaded}). 
%If it were called from outside a threaded entry
%method it would suspend the main thread of execution of the processor from
%which \kw{CkWaitQD} was called, and the entire program would come to a grinding
%halt on that processor.

%\function{void CkExitAfterQuiescence()} \index{CkExitAfterQuiescence}
%\desc{This call informs the Charm RTS that computation on all processors
%should terminate as soon as the machine becomes completely idle--that is,
%after all messages and entry methods are finished.  This is the state of 
%quiescence, as described further in Section~\ref{sec:qd}.
%This routine returns immediately.}

    
  \subsection{Terminal I/O}

\index{input/output}
\charmpp\ provides both C and \CC\ style methods of doing terminal I/O.  

In place of C-style printf and scanf, \charmpp\ provides
\kw{CkPrintf} and \kw{CkScanf}.  These functions have
interfaces that are identical to their C counterparts, but there are some
differences in their behavior that should be mentioned.

\function{int CkPrintf(format [, arg]*)} \index{CkPrintf} \index{input/output}
\desc{This call is used for atomic terminal output. Its usage is similar to
\texttt{printf} in C.  However, \kw{CkPrintf} has some special properties
that make it more suited for parallel programming on networks of
workstations.  \kw{CkPrintf} routes all terminal output to the \kw{charmrun},
which is running on the host computer.  So, if a
\index{chare}chare on processor 3 makes a call to \kw{CkPrintf}, that call
puts the output in a TCP message and sends it to host
computer where it will be displayed.  This message passing is an asynchronous
send, meaning that the call to \kw{CkPrintf} returns immediately after the
message has been sent, and most likely before the message has actually
been received, processed, and displayed. \footnote{Because of
communication latencies, the following scenario is actually possible:
Chare 1 does a \kw{Ckprintf} from processor 1, then creates chare 2 on
processor 2.  After chare 2's creation, it calls \kw{CkPrintf}, and the
message from chare 2 is displayed before the one from chare 1.}
}

\function{void CkError(format [, arg]*))} \index{CkError} \index{input/output} 
\desc{Like \kw{CkPrintf}, but used to print error messages on \texttt{stderr}.}

\function{int CkScanf(format [, arg]*)} \index{CkScanf} \index{input/output}
\desc{This call is used for atomic terminal input. Its usage is similar to
{\tt scanf} in C.  A call to \kw{CkScanf}, unlike \kw{CkPrintf},
blocks all execution on the processor it is called from, and returns
only after all input has been retrieved.
}

For \CC\ style stream-based I/O, \charmpp\ offers \kw{ckin},
\kw{ckout}, and \kw{ckerr} in the place of cin, cout, and cerr.  The
\CC streams and their \charmpp\ equivalents are related in the same
manner as printf and scanf are to \kw{CkPrintf} and \kw{CkScanf}.  The
\charmpp\ streams are all used through the same interface as the \CC\ 
streams, and all behave in a slightly different way, just like C-style
I/O. \kw{ckout} and \kw{ckerr} both have the same idiosyncratic
behavior as \kw{CkPrintf}, and \kw{ckin} behaves in the same way as
\kw{CkScanf}.

  \subsection{Other Calls}

\label{other Charm++ calls}

The following calls provide information about the machines upon which the
parallel program is executing.  Processing Element refers to a single CPU.
Node refers to a single machine-- a set of processing elements which share
memory (i.e. an address space).  Processing Elements and Nodes are numbered,
starting from zero.

Thus if a parallel program is executing on one 4-processor workstation and one
2-processor workstation, there would be 6 processing elements (0, 1 ,2, 3, 4,
and 5) but only 2 nodes (0 and 1).  A given node's processing elements are
numbered sequentially.

{\bf int CkNumPes()} \index{CkNumPes} \\
returns the total number of processors, across all nodes.

{\bf int CkMyPe()} \index{CkMyPe} \\
returns the processor number on which the call was made.

{\bf int CkMyRank()} \index{CkMyRank} \\
returns the rank number of the processor on which the call was made.
Processing elements within a node are ranked starting from zero.

{\bf int CkMyNode()} \index{CkMyNode} \\
returns the address space number (node number) on which the call was made.

{\bf int CkNumNodes()} \index{CkMyNodes} \\
returns the total number of address spaces.

{\bf int CkNodeFirst(int node)} \index{CkNodeFirst} \\
returns the processor number of the first processor in this address space.

{\bf int CkNodeSize(int node)} \index{CkNodeSize} \\
returns the number of processors in the address space on which the call was made.

{\bf int CkNodeOf(int pe)} \index{CkNodeOf} \\
returns the node number on which the call was made.

{\bf int CkRankOf(int pe)} \index{CkRankOf} \\
returns the rank of the given processor within its node.

The following calls provide commonly needed functions.

{\bf void CkAbort(const char *message)} \index{CkAbort} \\
Cause the program to abort, printing the given error message.

{\bf void CkExit()} \index{CkExit} \\
This call informs the Charm kernel that computation on all processors should 
terminate.  After the currently executing entry method completes, no more 
messages or entry methods will be called.  \keyword{CkExit()} should be the 
last statement of the entry method from which it was called. 

{\bf int CkTimer()} \index{CkTimer} \index{timers} \\
returns the current value of the system timer in milliseconds. The system
timer is started when \\
the program begins execution. This timer measures
process time (user and system).

{\bf double CkWallTimer()} \index{CkWallTimer} \index{timers} \\
returns the elapsed time since the program has started from the wall clock 
timer.

    

\newpage
\section{Inheritance and Templates in Charm++}
\label{inheritance and templates}

\charmpp\ supports inheritance among \charmpp\ objects such as
chares, groups, and messages. This, along with facilities for generic
programming using \CC\ style templates for \charmpp\ objects, is a
major enhancement over the previous versions of \charmpp.

\subsection{Chare Inheritance}
\index{inheritance}

Chare inheritance makes it possible to remotely invoke methods of a base
chare \index{base chare} from a proxy of a derived
chare.\index{derived chare} Suppose a base chare is of type 
\uw{BaseChare}, then the derived chare of type \uw{DerivedChare} needs to be
declared in the \charmpp\ interface file to be explicitly derived from
\uw{BaseChare}. Thus, the constructs in the \texttt{.ci} file should look like:

\begin{alltt}
  chare BaseChare \{
    entry BaseChare(someMessage *);
    entry void baseMethod(void);
    ...
  \}
  chare DerivedChare : BaseChare \{
    entry DerivedChare(otherMessage *);
    entry void derivedMethod(void);
    ...
  \}
\end{alltt}

Note that the access specifier \kw{public} is omitted, because \charmpp\
interface translator only needs to know about the public inheritance,
and thus \kw{public} is implicit. A Chare can inherit privately from other
classes too, but the \charmpp\ interface translator does not need to know
about it, because it generates support classes ({\em proxies}) to remotely
invoke only \kw{public} methods.

The class definitions of both these chares should look like:

\begin{alltt}
  class BaseChare : public Chare \{
    // private or protected data
    public:
      BaseChare(someMessage *);
      void baseMethod(void);
  \};
  class DerivedChare : public BaseChare \{
    // private or protected data
    public:
      DerivedChare(otherMessage *);
      void derivedMethod(void);
  \};
\end{alltt}

Now, it is possible to create a derived chare, and invoke methods of base
chare from it, or to assign a derived chare proxy to a base chare proxy
as shown below:

\begin{alltt}
  ...
  otherMessage *msg = new otherMessage();
  CProxy_DerivedChare *pd = new CProxy_DerivedChare(msg);
  pd->baseMethod();     // OK
  pd->derivedMethod();  // OK
  ...
  Cproxy_BaseChare *pb = pd;
  pb->baseMethod();    // OK
  pb->derivedMethod(); // COMPILE ERROR
\end{alltt}

Note that \CC\ calls the default constructor \index{default constructor} of the
base class from any constructor for the derived class where base class
constructor is not called explicitly. Therefore, one should always provide a
default constructor for the base class, or explicitly call another base
class constructor.

Multiple inheritance \index{multiple inheritance} is also allowed for Chares
and Groups. Often, one should make each of the base classes inherit
``virtually'' from \kw{Chare} or \kw{Group}, so that a single copy of
\kw{Chare} or \kw{Group} exists for each multiply derived class.

Entry methods are inherited in the
same manner as methods of sequential \CC{} objects.  
To make an entry method virtual, just add the keyword \kw{virtual}
to the corresponding chare method-- no change is needed in the interface file.
Pure virtual entry methods also require no special description
in the interface file.


\subsection{Inheritance for Messages}
\index{message inheritance}

Messages cannot inherit from other messages.  A message can, however,
inherit from a regular \CC\ class.  For example:

\begin{alltt}
//In the .ci file:
  message BaseMessage1;
  message BaseMessage2;

//In the .h file:
  class Base \{
    // ...
  \};
  class BaseMessage1 : public Base, public CMessage_BaseMessage1 \{
    // ...
  \};
  class BaseMessage2 : public Base, public CMessage_BaseMessage2 \{
    // ...
  \};
\end{alltt}

Messages cannot contain virtual methods
or virtual base classes unless you use a packed message.
Parameter marshalling has complete support for inheritance, virtual
methods, and virtual base classes via the PUP::able framework.


% ( I think the following is now a lie  OSL 7/5/2001 )  
%Similar to Chares, messages can also be derived from base messages. One needs
%to specify this in the \charmpp\ interface file similar to the Chare
%inheritance specification (that is, without the \kw{public} access specifier.)
%Message inheritance makes it possible to send a message of derived type to the
%method expecting a base class message.


\subsection{Generic Programming Using Templates}
\index{templates}

One can write ``templated'' code for Chares, Groups, Messages and other
\charmpp\  entities using familiar \CC\ template syntax (almost). The \charmpp\
interface translator now recognizes most of the \CC\ templates syntax,
including a variety of formal parameters, default parameters, etc. However, not
all \CC\ compilers currently recognize templates in ANSI drafts, therefore the
code generated by \charmpp\ for templates may not be acceptable to some current
\CC\ compilers\footnote{ Most modern \CC\ compilers belong to one of the two
camps. One that supports Borland style template instantiation, and the other
that supports AT\&T Cfront style template instantiation. In the first, code is
generated for the source file where the instantiation is seen.  GNU \CC\ falls
in this category.  In the second, which template is to be instantiated, and
where the templated code is seen is noted in a separate area (typically a local
directory), and then just before linking all the template instantiations are
generated. Solaris CC 5.0 belongs to this category. For templates to work for
compilers in the first category such as for GNU \CC\ all the templated code
needs to be visible to the compiler at the point of instantiation, that is,
while compiling the source file containing the template instantiation. For a
variety of reasons, \charmpp\ interface translator cannot generate all the
templated code in the declarations file {\tt *.decl.h}, which is included in
the source file where templates are instantiated. Thus, for \charmpp\ generated
templates to work for GNU \CC\ even the definitions file {\tt *.def.h} should
be included in the \CC\ source file. However, this file may contain other
definitions apart from templates that will be duplicated if the same file is
included in more than one source files. To alleviate this problem, we have to
do a little trick. Fortunately, this trick works for compilers supporting both
Borland-style and Cfront-style template instantiation, therefore, code using
this trick will be portable. The trick is to include {\tt *.def.h} with a
preprocessor symbol \kw{CK\_TEMPLATES\_ONLY} defined, whenever templates
defined in an \kw{extern} module are instantiated. If your interface file does
not contain template declarations or definitions, you need not bother about
including {\tt *.def.h} for \kw{extern} modules.  For example, if module {\tt
stlib} contains template definitions, that you may want to instantiate in
another module called {\tt pgm}, then {\tt pgm.C} should include {\tt
stlib.def.h} with \kw{CK\_TEMPLATES\_ONLY} defined. Of course, {\tt
stlib.decl.h} needs to be included at the top of {\tt pgm.C}.  }. 

The \charmpp\ interface file should contain the template
definitions as well as the instantiation. For example, if a message
class \uw{TMessage} is templated with a formal type parameter 
\uw{DType}, then every instantiation of \uw{TMessage} should be specified
in the \charmpp\ interface file. An example will illustrate this better:
\index{template}

\begin{alltt}
  template <class DType=int, int N=3> message TMessage;
  message TMessage<>; // same as TMessage<int,3>
  message TMessage<double>; // same as TMessage<double, 3>
  message TMessage<UserType, 1>;
\end{alltt}

Note the use of default template parameters. It is not necessary for
template definitions and template instantiations to be part of the
same module.  Thus, templates could be defined in one module, and
could be instantiated in another module \index{module}, as long as the
module defining a template is imported into the other module using the
\kw{extern module} construct. Thus it is possible to build a standard
\charmpp\ template library. Here we give a flavor of possibilities:

\begin{alltt}
module SCTL \{
  template <class dtype> message  Singleton;
  template <class dtype> group Reducer \{
    entry Reducer(void);
    entry void submit(Singleton<dtype> *);
  \}
  template <class dtype> chare ReductionClient \{
    entry void recvResult(Singleton<dtype> *);
  \}
\};

module User \{
  extern module SCTL;
  message Singleton<int>;
  group Reducer<int>;
  chare RedcutionClient<int>;
  chare UserClient : ReductionClient<int> \{
    entry UserClient(void);
  \}
\};
\end{alltt}

The \uw{Singleton} message is a template for storing one element of any
\uw{dtype}. The \uw{Reducer} is a group template for a spanning-tree reduction,
which is started by submitting data to the local branch. It also contains a
public method to register the \uw{ReductionClient} (or any of its derived
types), which acts as a callback to receive results of a reduction.


\newpage
\appendix

\section{Compiling, Running and Debugging Charm++/Converse Programs}

 In this section, we give a brief description on how to compile, run and debug a charm++/Converse programs. This is a quick start introduction, only most commonly used options will be mentioned. For more information, refer to Charm/Charm++/Converse Installation and Usage Manual.

\subsection{Compiling Converse, Charm, and Charm++ Programs}

The {\tt charmc} program standardizes compiling and linking procedures
among various machines and operating systems.  

Charmc can perform the following tasks.  The (simplified) syntax for
each of these modes is shown. The options are described next.

\begin{verbatim}
 * Compile C                        charmc -o pgm.o pgm.c
 * Compile C++                      charmc -o pgm.o pgm.C
 * Link                             charmc -o pgm   obj1.o obj2.o obj3.o...
 * Compile + Link                   charmc -o pgm   src1.c src2.ci src3.C
 * Create Library                   charmc -o lib.a obj1.o obj2.o obj3.o...
 * CPM preprocessing                charmc -gen-cpm file.c
 * Translate Charm++ Interface File charmc file.ci
\end{verbatim}

The following command-line options are most useful to users of charmc:

\begin{description}

\item[{\tt -o} {\em output-file}:]

Output file name.  Note: charmc only ever produces one output file at
a time.  Because of this, you cannot compile multiple source files at
once, unless you then link or archive them into a single output-file.
If exactly one source-file is specified, then an output file will be
selected by default using the obvious rule (eg, if the input file if
pgm.c, the output file is pgm.o).  If multiple input files are
specified, you must manually specify the name of the output file,
which must be a library or executable.

\item[{\tt -c}:]

Ignored.  There for compatibility with {\tt cc}.

\item[{\tt -D*}:]

Defines preprocessor variables from the command line at compile time.

\item[{\tt -I}:]

Add a directory to the search path for preprocessor include files.

\item[{\tt -g}:]

Causes compiled files to include debugging information.

\item[{\tt -L*}:]

Add a directory to the search path for libraries selected by
the {\tt -l} command.

\item[{\tt -l*}:]

Specifies libraries to link in.

\item[{\tt -O}:]

Causes files to be compiled with maximum optimization.

\item[{\tt -NO}:]

If this follows -O on the command line, it turns optimization back off.
This is just a convenience for simple-minded makefiles.

\item[{\tt -verbose}:]

All commands executed by charmc are echoed to stdout.

\item[{\tt -language \{converse|charm++|sdag|idl\}}:]

When linking with charmc, one must specify the ``language''.  This
is just a way to help charmc include the right libraries.  Pick the
``language'' according to this table:

\begin{itemize}
\item{{\bf Charm++} if your program includes Charm++, Charm, C++, and C.}
\item{{\bf Converse} if your program includes C or C++.}
\item{{\bf sdag} if your program includes structured dagger.}
\item{{\bf idl} if your program includes IDL bindings for Charm++.}
\end{itemize}

\item[{\tt -tracemode} {\em tracing-mode}:]

Selects the desired degree of tracing for Charm and Charm++ programs.
See the Charm manual and the Projections and SummaryTool manuals for
more information.  Currently supported modes are {\tt none}, {\tt
summary}, and {\tt projections}. Default is {\tt -tracemode none}.


\end{description}


\subsection[Executing Converse/Charm/Charm++ Programs]
	{Executing Converse/Charm/Charm++ Programs}
\label{executing charm programs}

The Charm linker produces one executable file.  On machines with a host
(such as a network of workstations), a link to the proper host program
{\fexec conv-host} is created in the user program directory.  Sample
execution examples are given below (the executable is called {\fparm
pgm}). Exact details will differ from site to site.  The list of Charm
command line options is in Section~\ref{command line options}.

\begin{itemize}

\item \underline{\bf ASCI Red:} 
	\begin{tabbing}
	{\fexec yod -sz 4 pgm}
	\end{tabbing}
	runs {\fparm pgm} on four processors.

\item \underline{\bf Cray T3E:} 
	\begin{tabbing}
	{\fexec mpprun -n 4 pgm}
	\end{tabbing}
	runs {\fparm pgm} on four processors.

\item \underline{\bf SGI Origin2000 (origin-mpi):} 
	\begin{tabbing}
	{\fexec mpirun -np 4 pgm}
	\end{tabbing}
	runs {\fparm pgm} on four processors.

\item \underline{\bf SGI Origin2000 (origin2000 or origin-pthreads):} 
	\begin{tabbing}
	{\fexec pgm +p4}
	\end{tabbing}
	runs {\fparm pgm} on four processors.

\item \underline{\bf Network of workstations:} 
	\begin{tabbing}
	{\fexec conv-host pgm +p4}
	\end{tabbing}
	executes {\fparm pgm} on 4 nodes.  In a network environment, Charm must
	be able to locate the directory of the executable.  If all workstations
	share a common file name space this is trivial.  If they don't, Charm
	will attempt to find the executable in a directory with the same path
	from the {\bf \$HOME} directory.  Pathname resolution is performed as 
	follows:
	\begin{enumerate}
		\item The system computes the absolute path of {\fexec pgm}.
		\item If the absolute path starts with the equivalent of {\bf \$HOME} 
			or the current working directory, the beginning part of the path 
			is replaced with the environment variable {\bf \$HOME} or the 
			current working directory. However, if {\fparm exec\_home} is 
            specified in the nodes file (see below), the beginning part of
            the path is replaced with {\fparm exec\_home}.
		\item The system tries to locate this program (with modified 
			pathname and appended extension if specified) on all nodes.
	\end{enumerate}

The list of nodes must be specified in a file.  The format of this file
allows you to define groups of machines, giving each group a name.
Each line of the nodes file is a command.  The most important command
is:

\begin{verbatim}
host <hostname> <qualifiers>
\end{verbatim}

which specifies a host.  The other commands are qualifiers: they modify
the properties of all hosts that follow them.  The qualifiers are:


\begin{tabbing}
{\tt group <groupname>}~~~\= - subsequent hosts are members of specified group\\
{\tt login <login>  }     \> - subsequent hosts use the specified login\\
{\tt shell <shell>  }     \> - subsequent hosts use the specified remote 
shell\\
%{\tt passwd <passwd>}     \> - subsequent hosts use the specified password\\
{\tt setup <cmd>  }       \> - subsequent hosts should execute cmd\\
{\tt home <dir> }         \> - subsequent hosts should find programs under dir\\
{\tt cpus <n>}            \> - subsequent hosts should use N light-weight processes\\
{\tt speed <s>}           \> - subsequent hosts have relative speed rating\\
{\tt ext <extn>}          \> - subsequent hosts should append extn to the pgm name\\
\end{tabbing}

{\bf Note:}
By default, conv-host uses a remote shell ``rsh'' to spawn node processes
on the remote hosts. The {\tt shell} qualifier can be used to override
it with say, ``ssh''. One can set the {\tt CONV\_RSH} environment variable
or use conv-host option {\tt ++remote-shell} to override the default remote 
shell for all hosts with unspecified {\tt shell} qualifier.

All qualifiers accept ``*'' as an argument, this resets the modifier to
its default value.  Note that currently, the passwd, cpus, and speed
factors are ignored.  Inline qualifiers are also allowed:

\begin{verbatim}
host beauty ++cpus 2 ++shell ssh
\end{verbatim}

Except for ``group'', every other qualifier can be inlined, with the
restriction that if the ``setup'' qualifier is inlined, it should be
the last qualifier on the ``host'' or ``group'' statement line.

Here is a simple nodes file:

\begin{verbatim}
        group kale-sun ++cpus 1
          host charm.cs.uiuc.edu ++shell ssh
          host dp.cs.uiuc.edu
          host grace.cs.uiuc.edu
          host dagger.cs.uiuc.edu
        group kale-sol
          host beauty.cs.uiuc.edu ++cpus 2
        group main
          host localhost
\end{verbatim}

This defines three groups of machines: group kale-sun, group kale-sol,
and group main.  The ++nodegroup option is used to specify which group
of machines to use.  Note that there is wraparound: if you specify
more nodes than there are hosts in the group, it will reuse
hosts. Thus,

\begin{verbatim}
        conv-host pgm ++nodegroup kale-sun +p6
\end{verbatim}

uses hosts (charm, dp, grace, dagger, charm, dp) respectively as
nodes (0, 1, 2, 3, 4, 5).

If you don't specify a ++nodegroup, the default is ++nodegroup main.
Thus, if one specifies

\begin{verbatim}
        conv-host pgm +p4
\end{verbatim}

it will use ``localhost'' four times.  ``localhost'' is a Unix
trick; it always find a name for whatever machine you're on.

Since the new nodes file is incompatible with the old nodes file, it has
been renamed.  It now is called ``.nodelist'', and all the options and
environment variables pertaining to it have also been renamed {\tt NODELIST}.

The user is required to set up remote login permissions on all nodes
using the ``.rhosts'' file in the home directory if ``rsh'' is used for remote
login into the hosts. If ``ssh'' is used, the user will have to setup
password-less login to remote hosts either using ``.shosts'' file, or using
RSA authentication based on a key-pair and adding public keys to ``.ssh/authorized\_keys'' file. See ``ssh'' documentation for more information.

\end{itemize}

Note that the Charm linker will provide the correct 
executable. The user, however, needs to know how programs are run for
the particular machine.

\subsubsection[Command Line Options]{Command Line Options}
\label{command line options}
\index{command line options}

A Charm program accepts the following command line options:
\begin{description}

\item[{\fexec +pN}] Run the program with N processors. The default is 1.
Note that on some nonshared memory machines, e.g., nCUBE/2, the user must
specify the number of processors using the command provided for that
machine (e.g. {\fexec xnc} on the nCUBE/2).
In such cases the {\fexec +p} option is ignored.

%\item[{\fexec +mM}] Run the program with M Kwords of memory per
%processor. The default is 50 Kwords of memory per processor.

%\item[{\fexec +mmM}] Run the program with M Kwords of memory for
%processor 0.

\item[{\fexec +ss}] Print summary statistics about chare creation.  This option
prints the total number of chare creation requests, and the total number of
chare creation requests processed across all processors.

\item[{\fexec +cs}] Print statistics about the number of create chare messages
requested and processed, the number of messages for chares requested and 
processed, and the number of messages for branch office chares requested and
processed, on a per processor basis.  Note that the number of messages 
created and processed for a particular type of message on a given node 
may not be the same, since a message may be processed by a different
processor from the one originating the request.

%\item[{\fexec +mems}] Print the Memory Usage Statistics at the end, including
%the number of memory allocation requests and memory free requests, based on
%the size of the memory allocated or freed.

\item[{\fexec user\_options}] Options that are be interpreted by the user
program may be included after all the system options. 
However, {\fexec user\_options} cannot start with +.
The {\fexec user\_options} will be passed as arguments to the user program 
via the usual {\fcmd argc/argv} construct to the {\fcmd main}\index{main}
entry point of the main chare. 
Charm system options will not appear in {\fcmd argc/argv}.

\end{description}

\subsubsection[Additional Uniprocessor Command Line Options]
{Additional Uniprocessor Command Line Options}
\label{uniprocessor command line options}

The uniprocessor versions can be used to simulate multiple
processors on a single workstation\index{uniprocessor command line
options}.  Any number of processors between 1 and 32 can be simulated by
using the {\fexec +p} option, limited only by the available memory on the
uniprocessor workstation.  By default, the uniprocessor versions handle
a single message from each processor, going in order from processor 0
thru $P-1$ (where $P$ is the number of processors) repeatedly.  
%If the
%user supplies the {\fexec +seed} \index{+seed} command line option
%followed by an 
%integer value, the processors will be accessed in a random (but
%deterministic) order.  {\fexec +seed} is only recognized by the
%uniprocessor version.

\subsubsection[Additional Network Command Line Options]
{Additional Network Command Line Options}
\label{network command line options}

The following {\fexec ++} command line options are available in
the network version\index{network command line options}:
\begin{description}

\item[{\fexec ++debug}] Run each node under gdb in an xterm window, prompting
the user to begin execution.
\index{++debug}

\item[{\fexec ++debug-no-pause}] Run each node under gdb in an xterm window
immediately (i.e. without prompting the user to begin execution).
\index{++debug-no-pause}

\item[{\fexec ++maxrsh}] Maximum number of {\fcmd rsh}'s to run at a
time.
\index{++maxrsh}

\item[{\fexec ++resend-wait}] Timeout before retransmitting datagrams
(in msec).
\index{++resend-wait}

\item[{\fexec ++resend-fail}] Timeout before retransmission fails (in
msec).\index{++resend-fail}
This parameter can help the user kill ``runaway'' processes, which may not
be killed otherwise when the user interrupts the program before it 
completes execution.
Currently a bug exists in the network version that may cause programs to
terminate prematurely if this value is set too low and {\fexec scanf} 
operations are being performed.

\item[{\fexec ++nodelist}] File containing list of nodes.
\index{++nodelist}\index{.nodes}\index{nodes file}

\end{description}

If using the {\fexec ++debug} option, the user must ensure the
following:
\index{++debug}
\begin{enumerate}

\item {\fexec xterm}, {\fexec xdpyinfo},  and {\fexec gdb} must be in
the user's path.

\item The path must be set in the {\fexec .cshrc} file, not the {\fexec .login}
file, because {\fexec rsh} does not run the {\fexec .login} file. 

\item The nodes must be authorized to create windows on the host machine (see
man pages for {\fexec xhost} and {\fexec xauth}).

\end{enumerate}


The following words are reserved for the \charm interface translator, and
cannot appear as variable or entry method names in a {\tt .ci} file:

% NOTE:
% This list is also duplicated (albeit, not exactly) in /www/codemirror/mode/clike/clike.js
% This helps the html version of the charm manual display syntax highlighting correctly.
%
% Any additions or modifications to this list should also be reflected there.
% In that file, it should be sufficient to only list ci keywords that are NOT
% already C++ keywords For eg, class and namespace are already cpp reserved
% words and need not be included in the list of permitted ci file reserved
% words.

\begin{itemize}
\item module
\item mainmodule
\item chare
\item mainchare
\item group
\item nodegroup
\item namespace
\item array
\item message
\item conditional
\item extern
\item initcall
\item initnode
\item initproc
\item readonly
\item PUPable
\item pupable
\item template
\item class
\item include
\item virtual
\item packed
\item varsize
\item entry
\item using
\item rdma
\item Entry method attributes
\begin{itemize}
\item stacksize
\item threaded
\item migratable
\item createhere
\item createhome
\item sync
\item iget
\item exclusive
\item immediate
\item expedited
\item inline
\item local
\item aggregate
\item nokeep
\item notrace
\item python
\item accel
\item readwrite
\item writeonly
\item accelblock
\item memcritical
\item reductiontarget
\end{itemize}
\item Basic C++ types
\begin{itemize}
\item int
\item short
\item long
\item char
\item float
\item double
\item unsigned
\item void
\item const
\end{itemize}
\item SDAG constructs
\begin{itemize}
\item atomic
\item serial
\item forward
\item when
\item while
\item for
\item forall
\item if
\item else
\item overlap
\item connect
\item publishes
\end{itemize}
\end{itemize}


\section{Syntax Changes from \charmpp\ 4.9}

The following changes are required to make older \charmpp\ 4.9
programs run with the new translator and runtime system in
\charmpp\ 5.0.

\begin{itemize}

\item Replace all references to {\tt *.top.h} and {\tt *.bot.h} to
{\tt *.decl.h} and {\tt *.def.h} respectively. This should be done in
Makefile, and everywhere these two types of files are included.

\item Change all X.ci files to include a top-level enclosure of module X {...}.

\item Change \kw{chare} \uw{main} to \kw{mainchare} \uw{main}.

\item Replace \kw{chare\_object} by \kw{Chare}.

\item Replace \kw{groupmember} by \kw{Group}.

\item Replace \kw{comm\_object} by \uw{CMessage\_<msgName>} in all
message declarations in {\tt *.h} files. 

\item Remove all \kw{operator new} methods of messages.

\item Remove \kw{MsgIndex(..)} parameter to \kw{new} for message allocation.

\item Remove all the empty messages (used only for triggering
computations.) from {\tt *.h}, {\tt *.ci}, {\tt *.C} files. All the
entry methods that take these empty messages as parameters should be
made methods with \kw{void} parameter. This should be done in all {\tt
*.ci}, {\tt *.h}, and {\tt *.C} files.  In these methods, there may be
a {\tt delete msg}. Remove that. 

\item Check for \kw{mainhandle} in the source. If it is there, declare
it as a \kw{readonly} variable in {\tt .ci} file, and initialize it in
the \kw{mainchare}'s constructor, so that it is available to all the
processors during the run.

\item All the \kw{packmessage} declarations in {\tt *.ci} files should
be changed to \kw{message [packed]}. 

\item All \kw{CPrintf}, \kw{CScanf}, and \kw{CError} should be changed
to \kw{CkPrintf}, \kw{CkScanf}, and \kw{CkError}. 

\item All \kw{CharmExit} should be changed to \kw{CkExit}.

\item Replace \kw{CMyPe} by \kw{CkMyPe}. Replace \kw{CNumPes} by \kw{CkNumPes}.

\item Change \kw{ChareIDType} to \kw{CkChareID}.

\item Change signature of \uw{M::pack} and \uw{M::unpack} for all
messages \uw{M} in {\tt *.h}, {\tt *.C} to\\
\verb+static void *pack(M* msg)+\\
and\\
\verb+static M* unpack(void *buf)+\\
and change the code
accordingly. This is a significant change because {\tt pack} and {\tt unpack}
used to be instance methods, and now they are class static
methods.  Avenues of optimizations open with this change, but one need
not explore those in the interest of time immediately. Further, one
should make sure that the performance of the new scheme is at least as
good as the old one. 

\item Replace all \kw{new\_group} by \uw{CProxy\_<grpName>::ckNew}.

\item Replace \kw{new\_chare2} by proxy creation.

\item Replace \kw{CSendMsg} by temporary proxy creation based on
ChareID and invoking appropriate method on it. One optimization is to
create a proxy immediately after a ChareID is received, and reusing it
everytime.

\item Similarly replace \kw{CSendMsgBranch} by temporary proxy
creation based on the groupID, and invoking appropriate method on it,
with processor number as the second parameter. Once again, there is an
opportunity for optimization here. 

\item Similarly replace \kw{CBroadcastMsgBranch} by temporary proxy
creation based on the groupID, and invoking appropriate method on it,
without any second parameter. Once again, there is an opportunity for
optimization here. 

\item Replace \kw{CStartQuiescence} by \kw{CkStartQD}.

\item Replace \kw{GetEntryPtr} by appropriate static method
(\kw{ckIdx\_*}) calls. 

\item Replace \kw{CLocalBranch} macros by \kw{ckLocalBranch} instance
method on temporarily created proxy. 

\item Change \kw{CPriorityPtr} to \kw{CkPriorityPtr}, also cast it
explicitly to {\tt (int *)}. 

\item Replace \kw{QuiescenceMessage} by \kw{CkQDMsg}. Remove all
extern declarations of \kw{CkQDMsg} from {\tt *.ci} files.

\end{itemize}


\section{Structured Dagger}
\label{sec:sdag}

\charmpp\ is based on the Message-Driven parallel programming paradigm.  The
message-driven programming style avoids the use of blocking receives and
allows overlap of computation and communication by scheduling computations
depending on availability of data.  This programing style enables \charmpp\
programs to tolerate communication latencies adaptively. Threads suffer from
loss of performance due to context-switching overheads and limited scalability
due to large and unpredictable stack memory requirements, when used in a
data-driven manner to coordinate a sequence of remotely triggered actions.

The need to sequence remotely triggered actions
arises in many situations. Let us consider an example:

%\begin{figure}[ht]
\begin{center}
\begin{alltt}
      class compute_object : public Chare \{
      private:
      int         count;
      Patch       *first, *second;
      public:
      compute_object(MSG *msg) \{
      count = 2; MyChareID(\&chareid);
      PatchManager->Get(msg->first_index, recv_first, \&thishandle,NOWAIT);
      PatchManager->Get(msg->second_index, recv_second, \&thishandle,NOWAIT);
      \}
      void recv_first(PATCH_MSG *msg) \{
       first = msg->patch;
       filter(first);
       if (--count == 0 ) computeInteractions(first,second);
      \} 
      void recv_second(PATCH_MSG *msg)\{
       second = msg->patch;
       filter(second);
       if (--count == 0) computeInteractions(first,second);
      \}
     \}
\end{alltt}
\end{center}
%\caption{Compute Object in a Molecular Dynamics Application}
%\label{figchareexample}
%\end{figure}


Consider an algorithm for computing cutoff-based pairwise interactions
between atoms in a molecular dynamics application, where interaction
between atoms is considered only when they are within some cutoff
distance of each other.  This algorithm is based on a combination of
task and spatial decompositions of the molecular system. The bounding
box for the molecule is divided into a number of cubes ({\em Patches})
each containing some number of atoms.  Since each patch contains a
different number of atoms and these atoms migrate between patches as
simulation progresses, a dynamic load balancing scheme is used. In
this scheme, the task of computing the pairwise interactions between
atoms of all pairs of patches is divided among a number of {\em
Compute Objects}. These compute objects are assigned at runtime to
different processors. The initialization message for each compute
object contains the indices of the patches. The patches themselves are
distributed across processors. Mapping information of patches to
processors is maintained by a replicated object called {\em
PatchManager}.  Figure~\ref{figchareexample} illustrates the \charmpp\
implementation of the compute object. Each compute object requests
information about both patches assigned to it from the
PatchManager. PatchManager then contacts the appropriate processors
and delivers the patch information to the requesting compute
object. The compute object, after receiving information about each
patch, determines which atoms in a patch do not interact with atoms in
another patch since they are separated by more than the cut-off
distance. This is done in method {\tt filter}.  Filtering could be
done after both patches arrive. However, in order to increase
processor utilization, we do it immediately after any patch
arrives. Since the patches can arrive at the requesting compute object
in any order, the compute object has to buffer the received patches,
and maintain state information using counters or flags.  This example
has been chosen for simplicity in order to demonstrate the necessity
of counters and buffers.  In general, a parallel algorithm may have
more interactions leading to the use of many counters, flags, and
message buffers, which complicates program development significantly.

Threads are typically used to perform the abovementioned sequencing.
Lets us code our previous example using threads.

%\begin{figure}[ht]
\begin{center}
\begin{alltt}
void compute_thread(int first_index, int second_index)
\{
    getPatch(first_index);
    getPatch(second_index);
    threadId[0] = createThread(recvFirst);
    threadId[1] = createThread(recvSecond);
    threadJoin(2, threadId);
    computeInteractions(first, second);
  \}
  void recvFirst(void)
  \{
    recv(first, sizeof(Patch), ANY_PE, FIRST_TAG);
    filter(first);
  \}
  void recvSecond(void)
  \{
    recv(second, sizeof(Patch), ANY_PE, SECOND_TAG);
    filter(second);
  \}
\end{alltt}
\end{center}
%\caption{Compute Thread in a Molecular Dynamics Application}
%\label{figthreadexample}
%\end{figure}

Contrast the compute chare-object example in figure~\ref{figchareexample} with
a thread-based implementation of the same scheme in
figure~\ref{figthreadexample}. Functions \uw{getFirst}, and \uw{getSecond} send
messages asynchronously to the PatchManager, requesting that the specified
patches be sent to them, and return immediately. Since these messages with
patches could arrive in any order, two threads, \uw{recvFirst} and
\uw{recvSecond}, are created. These threads block, waiting for messages to
arrive. After each message arrives, each thread performs the filtering
operation. The main thread waits for these two threads to complete, and then
computes the pairwise interactions. Though the programming complexity of
buffering the messages and maintaining the counters has been eliminated in this
implementation, considerable overhead in the form of thread creation, and
synchronization in the form of {\em join} has been added. Let us now code the
same example in \sdag. It reduces the parallel programming complexity without
adding any significant overhead.

%\begin{figure}[ht]
\begin{center}
\begin{alltt}
  array[1D] compute_object \{
    entry void recv_first(Patch *first);
    entry void recv_second(Patch *first);
    entry void compute_object(MSG *msg)\{
      atomic \{
         PatchManager->Get(msg->first_index,\dots);
         PatchManager->Get(msg->second_index,\dots);
      \}
      overlap \{
        when recv_first(Patch *first) atomic \{ filter(first); \}
        when recv_second(Patch *second) atomic \{ filter(second); \}
      \}
      atomic \{ computeInteractions(first, second); \}
    \}
  \}
\end{alltt}
\end{center}
%\caption{\sdag\ Implementation of the Compute Object}
%\label{figsdagexample}
%\end{figure}

\sdag\ is a coordination language built on top of \charmpp\ that supports the
sequencing mentioned above, while overcoming limitations of thread-based
languages, and facilitating a clear expression of flow of control within the
object without losing the performance benefits of adaptive message-driven
execution.  In other words, \sdag\ is a structured notation for specifying
intra-process control dependences in message-driven programs. It combines the
efficiency of message-driven execution with the explicitness of control
specification. \sdag\ allows easy expression of dependences among messages and
computations and also among computations within the same object using
when-blocks and various structured constructs.  \sdag\ is adequate for
expressing control-dependencies that form a series-parallel control-flow graph.
\sdag\ has been developed on top of \charmpp\. \sdag\ allows \charmpp\ entry
methods (in chares, groups or arrays) to specify code (a when-block body) to be
executed upon occurrence of certain events.  These events (or guards of a
when-block) are entry methods of the object that can be invoked remotely. While
writing a \sdag\ program, one has to declare these entries in \charmpp\
interface file. The implementation of the entry methods that contain the
when-block is written using the \sdag\ language. Grammar of \sdag\ is given in
the EBNF form below.

\subsection{Usage}

\sdag{} code can be inserted into the .ci file for any array, group, or chare's entry methods.

If you've added \sdag\ code to your class, you must link in the code by:
\begin{itemize}
  \item Adding ``{\it className}\_SDAG\_CODE'' inside the class declaration
     in the .h file.  This macro defines the entry points and support
     code used by \sdag{}.  Forgetting this results in a compile error
     (undefined sdag entry methods referenced from the .def file).
  \item Adding a call to the routine ``\_\_sdag\_init();'' from every constructor,
     including the migration constructor.  Forgetting this results in
     using uninitalized data, and a horrible runtime crash.
  \item Adding a call to the pup routine ``\_\_sdag\_pup(p);'' from your pup routine.
     Forgetting this results in failure after migration.
\end{itemize}

For example, an array named ``Foo'' that uses sdag code might contain:

\begin{alltt}
class Foo : public CBase_Foo \{
public:
    Foo_SDAG_CODE
    Foo(...) \{
       __sdag_init();
       ...
    \}
    Foo(CkMigrateMessage *m) \{
       __sdag_init();
    \}
    
    void pup(PUP::er &p) \{
       CBase_Foo::pup(p);
       __sdag_pup(p);
    \}
\};
\end{alltt}

For more details regarding \sdag{}, look at the example located in the 
{\tt examples/charm++/hello/sdag} directory in the \charmpp\ distribution.


\subsection{Grammar}

\subsubsection{Tokens}

\begin{alltt}
  <ident> = Valid \CC{} identifier 
  <int-expr> = Valid \CC{} integer expression 
  <\CC{}-code> = Valid \CC{} code 
\end{alltt}

\subsubsection{Grammar in EBNF Form}

\begin{alltt}
<sdag> := <class-decl> <sdagentry>+ 

<class-decl> := "class" <ident> 

<sdagentry> := "sdagentry" <ident> "(" <ident> "*" <ident> ")" <body> 

<body> := <stmt> 
        | "\{" <stmt>+ "\}" 

<stmt> := <overlap-stmt> 
        | <when-stmt> 
        | <atomic-stmt> 
        | <if-stmt> 
        | <while-stmt> 
        | <for-stmt> 
        | <forall-stmt> 

<overlap-stmt> := "overlap" <body> 

<atomic-stmt> := "atomic" "\{" <\CC-code> "\}" 

<if-stmt> := "if" "(" <int-expr> ")" <body> [<else-stmt>] 

<else-stmt> := "else" <body> 

<while-stmt> := "while" "(" <int-expr> ")" <body> 

<for-stmt> := "for" "(" <c++-code> ";" <int-expr> ";" <c++-code> ")" <body> 

<forall-stmt> := "forall" "[" <ident> "]" "(" <range-stride> ")" <body> 

<range-stride> := <int-expr> ":" <int-expr> "," <int-expr> 

<when-stmt> := "when" <entry-list>  <body> 

<entry-list> := <entry> 
              | <entry> [ "," <entry-list> ] 

<entry> := <ident> [ "[" <int-expr> "]" ] "(" <ident> "*" <ident> ")" 
  
\end{alltt}



\section{Interface Description Language (IDL)}

The Interface Description Language(IDL) is the language used to
describe the interfaces that client objects call and server object
implementations provide.

A typical simple IDL interface file follows:

\begin{verbatim}
module myModule {
  interface myClass1 {
    attribute long myAttr1;
    attribute int myAttr2;
    oneway void myMethod1(
        in char c,
        in short s,
        in long l,
        in unsigned short us,
        in unsigned long ul,
        out float f,
        inout double d,
 	out int a[10]);
  };
};
\end{verbatim}

An interface can have attributes as well as operations.  Array
parameters in operation declaration are supported. Templates and String
types are currently not supported.

\noindent {\bf Parameter declaration}:

Parameter declaration attributes include:
\begin{itemize}
\item {\bf in} - the parameter is passed from client to server
\item {\bf out} - the parameter is passed from server to client
\item {\bf inout} - the parameter is passed in both directions
\end{itemize}

\noindent {\bf How to write client/server side code }:

To define the classes declared in the {\tt .idl} file, one must write
{\tt .h} and {\tt .C} file for the classes as usual. For example, in the
{\tt .idl} file, we have:

\begin{verbatim}
module myModule {
  interface myClass1 {
    oneway void myMethod1( in char c);
  };
};
\end{verbatim}

In the  {\tt .h} file:

\begin{verbatim}
class myClass1 {
  public:
    void myMethod1( char );
    myClass1();
    ~myClass1();
};
\end{verbatim}

We implement these methods in the {\tt .C} file.

Instead of using {\tt myClass1} directly, one should declare and use the
class by using the name {\tt CImyClass1}. For example, in {\tt main.C}: 

\begin{verbatim}
main::main(CkArgMsg* m)
{
  CImyClass1 a;
  // create the object
  a.ciSetProc(CI_PE_ANY).ciCreate();
  // call method
  a.myMethod1('a');
  // delete the object
  a.ciDelete();
}
\end{verbatim}



\section{Related Publications}
\label{publications}

For starters, see the publications, reports, and manuals 
on the Parallel Programming Laboratory website: \texttt{http://charm.cs.uiuc.edu/}. 

\section{Associated Tools and Libraries}

Several tools and libraries are provided for \charmpp{}. \projections{} 
is an automatic performance analysis tool which provides
the user with information about the parallel behavior of \charmpp\ programs. 
The purpose of implementing \charmpp{} standard
libraries is to reduce the time needed to develop parallel
applications with the help of a set of efficient and re-usable modules.
Most of the libraries have been described in a separate manual.

\section{\projections}

\projections{} is a performance visualization and feedback tool. The system has
a much more refined understanding of user computation than is possible in
traditional tools.

\projections{} displays information about the request for creation and the
actual creation of tasks in \charmpp\ programs. Projections also provides the
function of post-mortem clock synchronization. Additionally, it can also
automatically partition the execution of the running program into logically
separate units, and automatically analyzes each individual partition. 

Future versions will be able to provide recommendations/suggestions for
improving performance as well.

\section{Contacts}
\label{Distribution}

While we can promise neither bug-free software nor immediate solutions   
to all problems, \charmpp\ is a stable system and it is our intention to
keep it as up-to-date and usable as our resources will allow
by responding quickly to questions and bug reports.  To that
end, there are mechanisms in place for contacting Charm users
and developers. 

Our software is made available for research use and evaluation.
For the latest software distribution, further information about
\converse{}/\charmpp\ and information on how to contact the Parallel
Programming laboratory, see our website at \texttt{http://charm.cs.uiuc.edu/}.

If retrieval of a publication via these channels is not possible,
please send electronic mail to \texttt{kale@cs.uiuc.edu} or postal mail to:

\begin{alltt}
   Laxmikant Kale
   Department of Computer Science 
   University of Illinois 
   201 N. Goodwin Ave.
   Urbana, IL 61801 
\end{alltt}



\newpage
\addtocontents{toc}{\contentsline {section}{\numberline {}References}{46}}
\bibliographystyle{plain}
\bibliography{group}

\newpage
\addtocontents{toc}{\contentsline {section}{\numberline {}Index}{47}}
\include{index}

\end{document}
