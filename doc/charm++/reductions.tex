\subsection{Reductions}

\label{reductions}

A \index{array reduction}reduction applies a single operation (e.g. add,
max, min, ...) to data items scattered across many processors and
collects the result in one place.  \charmpp{} supports reductions
over the members of an array or group.

The data to be reduced comes from a call to the member \kw{contribute} 
method:
\begin{alltt}
void contribute(int nBytes,const void *data,CkReduction::reducerType type);
\end{alltt}

This call contributes \kw{nBytes} bytes starting at \kw{data} to the
reduction \kw{type} (see reduction types, below).  Unlike sending a
message, you may use \kw{data} after the call to \kw{contribute}.  All
members must call \kw{contribute}, and all must use the same
reduction type.  

When you create a new member, it is expected
to contribute to the next reduction not already in progress on that
processor.  The
reduction will complete properly even if members are migrated
or deleted during the reduction. 

For example, if we want to sum each member's single integer myInt, 
we would use:

\begin{alltt}
    //Inside any member method
    int myInt=get_myInt();
    contribute(sizeof(int),\&myInt,CkReduction::sum_int);
\end{alltt}

The built-in reduction types (see below) can also handle arrays of
numbers.  For example, if each element of an array has a pair of
doubles \uw{forces}[2] which need to be summed up (separately) across
every element, from each element call:

\begin{alltt}
    double forces[2]=get_my_forces();
    contribute(2*sizeof(double),forces,CkReduction::sum_double);
\end{alltt}

Note that since C++ arrays (like \uw{forces}[2]) are already pointers, we 
don't use \&\uw{forces}.

Sometimes it is not important the data to be reduced, but only the fact that all
elements have reached a synchronization point. In this case a simpler version of
contribute can be used:

\begin{alltt}
    contribute();
\end{alltt}

In all cases, the result of the reduction operation is passed to the {\em reduction
client}.  Many different kinds of reduction clients can be used, as
explained below (Section~\ref{reductionClients}).



\subsubsection{Reduction Clients}

\label{reductionClients}

After the data is reduced, it is passed to a you via a callback object,
as described in section~\ref{callbacks}.  The message passed to
the callback is of type \kw{CkReductionMsg}.
The important members of \kw{CkReductionMsg} are
\kw{getSize()}, which returns the number of bytes of reduction data; and
\kw{getData()}, which returns a ``void *'' to the actual reduced data.

You may pass the client callback as an additional parameter to \kw{contribute}.
If different \kw{contribute} calls pass different callbacks, some (unspecified,
unreliable) callback will be chosen for use.
\begin{alltt}
    double forces[2]=get_my_forces();
    //When done, broadcast the CkReductionMsg to ``myReductionEntry''
    CkCallback cb(CkIndex_myArrayType::myReductionEntry(NULL), thisProxy);
    contribute(2*sizeof(double), forces,CkReduction::sum_double, cb);
\end{alltt}

In the case of the reduced version used for synchronization purposes, the
callback parameter will be the only input parameter:
\begin{alltt}
    contribute( CkCallback(CkIndex_myArrayType::myReductionEntry(NULL), thisProxy) );
\end{alltt}

If no member passes a callback to \kw{contribute}, the reduction will use
the default callback. Programmers can set the default callback for an array or group
using the \kw{ckSetReductionClient} proxy call on processor zero, or
by passing the callback to {\tt CkArrayOptions::setReductionClient()}
before creating the array, as described in section~\ref{CkArrayOptions}.
Again, a \kw{CkReductionMsg} message will be passed to this callback,
which must delete the message when done.

\begin{alltt}
    //Somewhere on processor zero:
    myProxy.ckSetReductionClient(new CkCallback(...));
\end{alltt}

So, for the previous reduction on chare array arr:
\begin{alltt}
    CkCallback *cb = new CkCallback(CkIndex_main::reportIn(NULL),  mainProxy);
    arr.ckSetReductionClient(cb);
\end{alltt}

and the actual entry point:

\begin{alltt}
void myReductionEntry(CkReductionMsg *msg)
\{
  int reducedArrSize=msg->getSize() / sizeof(double);
  double *output=(double *) msg->getData();
  for(int i=0 ; i<reducedArrSize ; i++)
  \{
   // do something with the reduction results in each output[i] array element
   .
   .
   .
  \}
  delete msg;
\}
\end{alltt}

(See \kw{pgms/charm++/RedExample} for a complete example).

For backward compatability, rather than a general callback you can
specify a peculiar kind of C function using \kw{ckSetReductionClient}
or \kw{setReductionClient}.  This C function takes a user-defined
parameter (passed to \kw{setReductionClient}) and the actual reduction data,
which it must not deallocate.

\begin{alltt}
  //Somewhere on processor zero:
  myProxy.setReductionClient(myClient,(void *)NULL);

void myClient(void *param,int dataSize,void *data)
\{
  double *forceSum=(double *)data;
  cout<<``First force sum is ``<<forceSum[0]<<endl;
  cout<<``Second force sum is ``<<forceSum[1]<<endl;
\}
\end{alltt}

\subsubsection{Typed Reductions}

\label{typed_reductions}

Typically the entry method that is invoked when a reduction is completed
takes a single argument of type CkReductionMsg. However, by giving an entry
method the \kw{reductiontarget} attribute, you can instead use entry methods
that take arguments of the same type that the reduction function yields.
For example, for a reduction that yields an int, a typed reduction might look
like this:

\begin{alltt}
// in the .ci file...
entry [reductiontarget] void done(int result);

// in the .cc file...
void Driver::done(int result) {
   CkPrintf("Reduction value: %d\n", result);
}
\end{alltt}

This will also work for arrays, and for any user-defined type with a PUP method
(see ~\ref{sec:pup}). If you know that the reduction will yield a particular
number of elements, say 3 ints, you can also specify a reduction target which
takes 3 ints and it will be invoked correctly. When creating a callback to the
reduction target, the entry method index is generated by {\tt
CkReductionTarget(chare_name, method_name)}.

\subsubsection{Built-in Reduction Types}

\label{builtin_reduction}

\charmpp{} includes several built-in reduction types, used to combine 
the separate contributions.  Any of them may be passed as an
\kw{CkReduction::reducerType} type to \kw{contribute}.

The first four reductions (sum, product, max, and min) work on int,
float, or double data as indicated by the suffix.  The logical
reductions (and, or) only work on integer data.  All the built-in
reductions work on either single numbers (pass a pointer) or arrays-- just
pass the correct number of bytes to \kw{contribute}.

\begin{enumerate}

\item \kw{CkReduction::sum\_int}, \kw{sum\_float}, \kw{sum\_double}-- the
result will be the sum of the given numbers.

\item \kw{CkReduction::product\_int}, \kw{product\_float},
\kw{product\_double}-- the result will be the product of the given numbers.

\item \kw{CkReduction::max\_int}, \kw{max\_float}, \kw{max\_double}-- the
result will be the largest of the given numbers.

\item \kw{CkReduction::min\_int}, \kw{min\_float}, \kw{min\_double}-- the
result will be the smallest of the given numbers.

\item \kw{CkReduction::logical\_and}-- the result will be the logical AND of the given
integers.  0 is false, nonzero is true.

\item \kw{CkReduction::logical\_or}-- the result will be the logical OR of the given
integers.

\item \kw{CkReduction::bitvec\_and}-- the result will be the bitvector AND of the given numbers (represented as integers).

\item \kw{CkReduction::bitvec\_or}-- the result will be the bitvector OR of the given numbers (represented as integers).

\item \kw{CkReduction::set}-- the result will be a verbatim concatenation of
all the contributed data, separated into \kw{CkReduction::setElement} records.
The data contributed can be of any length, and can vary across array elements
or reductions.  To extract the data from each element, see the description
below.

\item \kw{CkReduction::concat}-- the result will be a byte-by-byte
concatentation of all the contributed data.  There is no separation
added between different contributions.

\end{enumerate}


\kw{CkReduction::set} returns a collection of \kw{CkReduction::setElement}
objects, one per contribution.  This class has definition:

\begin{alltt}
class CkReduction::setElement 
\{
public:
  int dataSize;//The length of the data array below
  char data[];//The (dataSize-long) array of data
  CkReduction::setElement *next(void);
\};
\end{alltt}

To extract the contribution of each array element from a reduction set, use the
\uw{next} routine repeatedly:

\begin{alltt}
  //Inside a reduction handler-- 
  //  data is our reduced data from CkReduction_set
  CkReduction::setElement *cur=(CkReduction::setElement *)data;
  while (cur!=NULL)
  \{
    ... //Use cur->dataSize and cur->data
    //Now advance to the next element's contribution
    cur=cur->next();
  \}
\end{alltt}

The reduction set order is undefined.  Add a source field to your
contribution if you need to know which array element gave a particular
contribution.  This will require you to do your own
serialize/unserialize operation on your element structure if your
reduction element data is complex.  Consider using the \kw{PUP}
interface see ~\ref{sec:pup} to simplify your object serialization
needs.

If your data is order dependant, or if your data is just too
heterogenous to be handled elegantly by the predefined types and you
don't want to undertake multiple reductions, it may be best to define
your own reduction type.  See the next section
(Section~\ref{new_type_reduction}) for details.


\subsubsection{Defining a New Reduction Type}

\label{new_type_reduction}

It is possible to define a new type of reduction, performing a 
user-defined operation on user-defined data.  A reduction function
combines separate contributions (from this or other processors)
into a single combined value.

The input to a reduction function is a list of \kw{CkReductionMsg}s.
A \kw{CkReductionMsg} is a thin wrapper around a buffer of untyped data
to be reduced.  
The output of a reduction function is a single CkReductionMsg
containing the reduced data, which you should create using the
\kw{CkReductionMsg::buildNew(int nBytes,const void *data)} method.  

Thus every reduction function has the prototype:
\begin{alltt}
CkReductionMsg *\uw{reductionFn}(int nMsg,CkReductionMsg **msgs);
\end{alltt}

For example, a reduction function to add up contributions 
consisting of two machine short integers would be:

\begin{alltt}
CkReductionMsg *sumTwoShorts(int nMsg,CkReductionMsg **msgs)
\{
  //Sum starts off at zero
  short ret[2]={0,0};
  for (int i=0;i<nMsg;i++) \{
    //Sanity check:
    CkAssert(msgs[i]->getSize()==2*sizeof(short));
    //Extract this message's data
    short *m=(short *)msgs[i]->getData();
    ret[0]+=m[0];
    ret[1]+=m[1];
  \}
  return CkReductionMsg::buildNew(2*sizeof(short),ret);
\}
\end{alltt}

You must register your reduction function with \charmpp{} 
using \kw{CkReduction::addReducer} from
an \kw{initcall} routine (see section~\ref{initcall} for details
on the \kw{initcall} mechanism).   \kw{CkReduction::addReducer}
returns a \kw{CkReduction::reducerType} which you can later 
pass to \kw{contribute}.  Since \kw{initcall} routines are executed
once on every node, you can safely store the \kw{CkReduction::reducerType}
in a global or class-static variable.  For the example above:

\begin{alltt}
//In the .ci file:
  initcall void registerSumTwoShorts(void);

//In some .C file:
/*global*/ CkReduction::reducerType sumTwoShortsType;
/*initcall*/ void registerSumTwoShorts(void)
\{
  sumTwoShortsType=CkReduction::addReducer(sumTwoShorts);
\}

//In some member:
  short data[2]=...;
  contribute(2*sizeof(short),data,sumTwoShortsType);
\end{alltt}

Note that you cannot call \kw{CkReduction::addReducer}
from anywhere but in an \kw{initcall} routine.


