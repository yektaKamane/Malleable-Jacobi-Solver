\subsection{Terminal I/O}

\index{input/output}
Charm++ provides both C and C++ style methods of doing terminal I/O.  

In the place of C-style printf and scanf, charm++ provides
\index{CkPrintf}CkPrintf and \index{CkScanf}CkScanf.  These functions have
interfaces that are identical to their C counterparts, but there are some
differences in their behavior that should be mentioned.

{\bf int CkPrintf(format [, arg]*)} \index{CkPrintf} \index{input/output} \\
This call is used for atomic terminal output. Its usage is similar to {\sf
printf} in C.  However, CkPrintf has some special properties that make it more
suited for parallel programming on networks of workstations.  CkPrintf routes
all terminal output to the processor 0, which is the one running the host
program.  So, if a \index{chare}chare on processor 3 makes a call to CkPrintf,
that call puts the output in a \index{message}message and sends it to processor
0 where it will be displayed.  This message passing is an asynchronous send,
meaning that the call to CkPrintf returns immediately after the message has
been sent, and most likely before the message has actually been received,
processed, and displayed. \footnote{Because of communication latencies, the
following scenario is actually possible: Chare 1 does a Ckprintf from processor
1, then creates chare 2 on processor 2.  After chare 2's creation, it calls
CkPrintf, and the message from chare 2 is displayed before the one from chare
1.}   

{\bf void CkError(format [, arg]*))} \index{CkError} \index{input/output} \\
Like CkPrintf, but used to print error messages.

{\bf int CkScanf(format [, arg]*)} \index{CkScanf} \index{input/output} \\
This call is used for atomic terminal input. Its usage is similar to
{\sf scanf} in C.  A call to CkScanf, unlike CkPrintf, blocks all execution on
the processor it is called from, and returns only after all input has been
retrieved.

For C++ style stream-based I/O, Charm++ offers \index{ckin}\keyword{ckin},
\index{ckout}\keyword{ckout}, and \index{ckerr}\keyword{ckerr} in the place of
cin, cout, and cerr.  The C++ streams and their Charm++ equivalents are related
in the same manner as printf and scanf are to CkPrintf and CkScanf.  The
Charm++ streams are all used through the same interface as the C++ streams, and
all behave in a slightly different way, just like C-style I/O.  \keyword{ckout}
and \keyword{ckerr} both have the same idiosyncratic behavior as CkPrintf, and
\keyword{ckin} behaves in the same way as CkScanf.

