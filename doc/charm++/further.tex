\section{Further Information}

\subsection{Related Publications}
\label{publications}

For starters, see the publications and reports as well
as related manuals that can be found on the Parallel Programming
Laboratory website: {\tt http://charm.cs.uiuc.edu/}. 

Some \charmpp\ related papers:
\cite{CharmppPPWCPP96},\cite{CharmppOOPSLA93}.  Papers on {\sc Converse}:
\cite{ConverseRTSPP98}, \cite{InterOpIPPS96}. Some
{\sc Charm} related papers that formed the early foundation for \charmpp:
\cite{CharmSys1TPDS94}, \cite{CharmSys2TPDS94},
\cite{CharmOverviewINTL93}.  Papers on Structured Dagger and Dagger: 
\cite{DaggerSyncIPPS94}, \cite{StructDaggerEURO96}.  Projections
papers: \cite{ProjectionsIPPS93}, \cite{Projections}.

\subsection{Associated Tools and Libraries}

Several tools and libraries are provided for \charmpp. {\bf
Projections} is an automatic performance analysis tool which provides
the user with information about the parallel behavior of \charmpp\ programs. The purpose of implementing \charmpp standard
libraries is to reduce the time needed to develop parallel
applications with the help of a set of efficient and re-usable modules.

\subsubsection{Projections}
{\bf Projections} \cite{ProjectionsIPPS93}, \cite{Projections} is a
performance visualization and feedback tool. The system has a much
more refined understanding of user computation than is possible in
traditional tools.

Projections displays information about the request for creation and
the actual creation of tasks in \charmpp\ programs. Projections also
provides the function of post-mortem clock
synchronization. Additionally, it can also automatically partition
the execution of the running program into logically separate units,
and automatically analyzes each individual partition. 

Future versions will be able to provide recommendations/suggestions
for improving performance as well.

\subsubsection{Communication}
Communication optimizations tend to be specific to a particular
architecture or an application. To improve portability and to reduce the
cost of developing parallel applications a mechanism to integrate these
different optimizations should exist. Moreover, it should be possible to
automatically adapt the strategy to the situation at hand. The
communication library integrates the different strategies to perform
each-to-many multicast, including tree-based multicast, grid -based
multicast and hypercube-based (dimensional exchange) schemes. The
framework provided is flexible enough to absorb new strategies and
communication patterns. It also provides the capability to do dynamic
switching of strategies. This helps the library to adapt itself to the
existing environment.

\subsection{Contacts}
\label{Distribution}

While we can promise neither bug-free software nor immediate solutions   
to all problems, \charmpp\ is a stable system and it is our intention to
keep it as up-to-date and usable as our resources will allow
by responding quickly to questions and bug reports.  To that
end, there are mechanisms in place for contacting Charm users
and developers. 

Our software is made available for research use and evaluation.
For the latest software distribution, further information about {\sc
Charm}/\charmpp\ and information on how to contact the Parallel
Programming laboratory, see our website at {\it
http://charm.cs.uiuc.edu/}.  The software is also available by
anonymous ftp, from a.cs.uiuc.edu, under the directory
pub/research-groups/CHARM.  

If retrieval of a publication via these channels is not possible,
please send electronic mail to {\tt kale@cs.uiuc.edu} or postal mail to:

{\bf 
\begin{tabbing}
\hspace{0.5in}\=\hspace{0.3in}\=\hspace{0.3in}\=\hspace{0.3in}\= \kill
\> Laxmikant Kale \\
\> Department of Computer Science \\
\> University of Illinois \\
\> 1304 West Springfield Avenue \\
\> Urbana, IL 61801 \\
\end{tabbing}
}

A mailing list exists for announcements about software releases and
updates relating to \charmpp/{\sc Converse}.  To subscribe, send
e-mail to: ???????.
