\subsection{Quiescence Detection}

In \charmpp, \index{quiescence}quiescence is defined as the state in which no
processor is executing an entry point, and no messages are awaiting processing.

\charmpp\ provides two facilities for detecting quiescence: \kw{CkStartQd} and
\kw{CkWaitQd}. \index{CkStartQd} \index{CkWaitQd}

\kw{CkStartQd} registers with the system a callback that should be made the
next time \index{quiescence}quiescence is detected.  \kw{CkStartQd} takes two
parameters: an index corresponding to the entry function that is to be called,
and a handle to the chare on which that entry function should be called.  The
syntax of this call looks like this:

\begin{tabbing}
~~~~ \=~~~~ \=~~~~ \=~~~~ \=~~~~ \=~~~~ \=~~~~ \=~~~~ \=~~~~ \=~~~~ \kill
\> \kw{CkStartQd}(\kw{int} {\it Index}, \kw{CkChareID} {\it chareID});
\end{tabbing}

To retrieve the corresponding index of a particular \index{entry method}entry
method, you must use a static method contained within the
\index{CProxy}\kw{CProxy} object corresponding to the \index{chare}chare
containing that entry method.  The syntax of this call is as follows:

\begin{tabbing}
~~~~ \=~~~~ \=~~~~ \=~~~~ \=~~~~ \=~~~~ \=~~~~ \=~~~~ \=~~~~ \=~~~~ \kill
\kw{CProxy}\_\uw{ChareName}::\kw{ckIdx}\_\uw{EntryMethod}(\uw{Msg}
*{\it Message});
\end{tabbing}

where {\it chareID} is the name of the chare identifier of the chare containing
the desired entry method, \uw{EntryMethod} is the name of that entry method,
and {\it Message} is a pointer to the kind of message that the desired entry
method takes as a parameter. To make this look a little cleaner, we have
provided a simple macro called \kw{EntryIndex}, which can be used in the place
of this convoluted looking static method call.
\kw{EntryIndex}\index{EntryIndex} takes as parameters the type of chare in
which the entry method is located, the name of the entry method itself, and the
type of message that the entry method takes as a parameter. For example:

\begin{tabbing}
~~~~ \=~~~~ \=~~~~ \=~~~~ \=~~~~ \=~~~~ \=~~~~ \=~~~~ \=~~~~ \=~~~~ \kill
\> \kw{EntryIndex}(\uw{ChareName}, \uw{EntryName}, \uw{MsgName});
\end{tabbing}

Note that ChareName, EntryName, and MsgName are {\bf NOT} variables or
constants. This is text that the preprocessor uses to fill in portions of the
previously mentioned static method call.  Additionally, this macro method will
not work with templated chares (refer to Section ~\ref{inheritance and
templates} for details on templated chares).

\index{CkWaitQd}\kw{CkWaitQd}, however, does not register a callback.  Rather,
\kw{CkWaitQd} blocks and does not return until \index{quiescence}quiescence is
detected.  It takes no parameters and returns no value.  A call to
\kw{CkWaitQd} simply looks like this: 

\begin{tabbing}
~~~~ \=~~~~ \=~~~~ \=~~~~ \=~~~~ \=~~~~ \=~~~~ \=~~~~ \=~~~~ \=~~~~ \kill
\> \kw{CkWaitQd}();
\end{tabbing}

Keep in mind that \kw{CkWaitQd} should only be called from threaded
\index{entry method}entry methods because a call to \kw{CkWaitQd} suspends the
current thread of execution, and if it were called outside of a threaded entry
method it would suspend the main thread of execution of the processor from
which \kw{CkWaitQd} was called and the entire program would come to a grinding
halt on that processor.
