\subsection{Quiescence Detection}

In \charmpp, \index{quiescence}quiescence is defined as the state in which no
processor is executing an entry point, and no messages are awaiting processing.

\charmpp\ provides two facilities for detecting quiescence: \kw{CkStartQD} and
\kw{CkWaitQD}.

\kw{CkStartQD} registers with the system a callback that should be made the
next time \index{quiescence}quiescence is detected.  \kw{CkStartQD} takes two
parameters: an index corresponding to the entry function that is to be called,
and a handle to the chare on which that entry function should be called.  The
syntax of this call looks like this:

\begin{alltt}
 CkStartQD(int Index,const CkChareID* chareID);
\end{alltt}

To retrieve the corresponding index of a particular \index{entry method}entry
method, you must use a static method contained within the \uw{CkIndex} object
corresponding to the \index{chare}chare containing that entry method.  The
syntax of this call is as follows:

\begin{alltt}
\kw{myIdx}=CkIndex_\uw{ChareName}::\uw{EntryMethod}(\uw{parameters});
\end{alltt}

where \uw{ChareName} is the name of the chare containing
the desired entry method, \uw{EntryMethod} is the name of that entry method,
and \uw{parameters} are the parameters taken by the method.
These parameters are only used to resolve the proper \uw{EntryMethod};
they are otherwise ignored.

\kw{CkWaitQD}, by contrast, does not register a callback.  Rather,
\kw{CkWaitQD} blocks and does not return until \index{quiescence}quiescence is
detected.  It takes no parameters and returns no value.  A call to
\kw{CkWaitQD} simply looks like this: 

\begin{alltt}
  CkWaitQD();
\end{alltt}

Keep in mind that \kw{CkWaitQD} should only be called from threaded
\index{entry method}entry methods because a call to \kw{CkWaitQD} suspends the
current thread of execution, and if it were called outside of a threaded entry
method it would suspend the main thread of execution of the processor from
which \kw{CkWaitQD} was called and the entire program would come to a grinding
halt on that processor.
