\subsection{\kw{initnode} and \kw{initproc} routines}

\index{initcall}
\label{initcall}
Some registration routines need be executed exactly once
before the computation begins. You may choose to 
declare a regular  \CC\ subroutine \kw{initnode} in the .ci file
to ask \charmpp to execute the routine exactly once on {\em every node} 
before the computation begins, or to declare a regular  \CC\ subroutine 
\kw{initproc} to be executed exactly once on {\em every processor}.

\begin{alltt}
module foo \{
    initnode void fooNodeInit(void);
    initproc void fooProcInit(void);
    chare bar \{
        ...
        initnode void barNodeInit(void);
        initproc void barProcInit(void);
    \};
\};
\end{alltt}

This code will execute the routines \uw{fooNodeInit} and static 
\uw{bar::barNodeInit} once on every node and \uw{fooProcInit}
and \uw{bar::barProcInit} on every processor before the main computation 
starts.
Initnode calls are always executed before initproc calls.
Both init calls (declared as static member function) can be used in chare, 
group or chare arrays.

Note that these routines should only do registration, not computation since
Charm run-time initialization does not start yet ---
use a mainchare instead, which gets executed on only processor 0,
to begin the computation.  Initcall routines are typically
used to do special registrations and global variable setup
before the computation actually begins.


\subsection{Other Calls}

\label{other Charm++ calls}

The following calls provide information about the machines upon which the
parallel program is executing.  Processing Element refers to a single CPU.
Node refers to a single machine-- a set of processing elements which share
memory (i.e. an address space).  Processing Elements and Nodes are numbered,
starting from zero.

Thus if a parallel program is executing on one 4-processor workstation and one
2-processor workstation, there would be 6 processing elements (0, 1 ,2, 3, 4,
and 5) but only 2 nodes (0 and 1).  A given node's processing elements are
numbered sequentially.

\function{int CkNumPes()} \index{CkNumPes}
\desc{returns the total number of processors, across all nodes.}

\function{int CkMyPe()} \index{CkMyPe}
\desc{returns the processor number on which the call was made.}

\function{int CkMyRank()} \index{CkMyRank}
\desc{returns the rank number of the processor on which the call was made.
Processing elements within a node are ranked starting from zero.}

\function{int CkMyNode()} \index{CkMyNode}
\desc{returns the address space number (node number) on which the call was made.}

\function{int CkNumNodes()} \index{CkMyNodes}
\desc{returns the total number of address spaces.}

\function{int CkNodeFirst(int node)} \index{CkNodeFirst}
\desc{returns the processor number of the first processor in this address space.}

\function{int CkNodeSize(int node)} \index{CkNodeSize}
\desc{returns the number of processors in the address space on which the call was made.}

\function{int CkNodeOf(int pe)} \index{CkNodeOf}
\desc{returns the node number on which the call was made.}

\function{int CkRankOf(int pe)} \index{CkRankOf}
\desc{returns the rank of the given processor within its node.}

The following calls provide commonly needed functions.

\function{void CkAbort(const char *message)} \index{CkAbort}
\desc{Cause the program to abort, printing the given error message.
This routine never returns.}

\function{void CkExit()} \index{CkExit}
\desc{This call informs the Charm kernel that computation on all processors
should terminate.  After the currently executing entry method completes, no
more messages or entry methods will be called on any other processor. 
This routine never returns.}

\function{void CkExitAfterQuiescence()} \index{CkExitAfterQuiescence}
\desc{This call informs the Charm kernel that computation on all processors
should terminate as soon as the machine becomes completely idle--that is,
after all messages and entry methods are finished.  This is the state of 
quiescence, as described further in Section~\ref{sec:qd}.
This routine returns immediately.}

\function{double CkCpuTimer()} \index{CkCpuTimer} \index{timers}
\desc{Returns the current value of the system timer in seconds. The system
timer is started when the program begins execution. This timer measures process
time (user and system).}

\function{double CkWallTimer()} \index{CkWallTimer} \index{timers}
\desc{Returns the elapsed time in seconds since the program has started from the wall
clock timer.}

\function{double CkTimer()} \index{CkTimer} \index{timers}
\desc{This is an alias for either \kw{CkWallTimer} on dedicated machines (such as ASCI Red) or 
\kw{CkCpuTimer} for machines with multiple user processes per CPU (such as workstation cluster.)}

