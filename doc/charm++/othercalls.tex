\subsection{\kw{initnode} and \kw{initproc} routines}

\index{initcall}
\label{initcall}
Some registration routines need be executed exactly once
before the computation begins. You may choose to 
declare a regular  \CC\ subroutine \kw{initnode} in the .ci file
to ask \charmpp to execute the routine exactly once on {\em every node} 
before the computation begins, or to declare a regular  \CC\ subroutine 
\kw{initproc} to be executed exactly once on {\em every processor}.

\begin{alltt}
module foo \{
    initnode void fooNodeInit(void);
    initproc void fooProcInit(void);
    chare bar \{
        ...
        initnode void barNodeInit(void);
        initproc void barProcInit(void);
    \};
\};
\end{alltt}

This code will execute the routines \uw{fooNodeInit} and static 
\uw{bar::barNodeInit} once on every node and \uw{fooProcInit}
and \uw{bar::barProcInit} on every processor before the main computation 
starts.
Initnode calls are always executed before initproc calls.
Both init calls (declared as static member function) can be used in chare, 
group or chare arrays.

Note that these routines should only do registration, not computation since
Charm run-time initialization does not start yet ---
use a mainchare instead, which gets executed on only processor 0,
to begin the computation.  Initcall routines are typically
used to do special registrations and global variable setup
before the computation actually begins.


\subsection{Other Calls}

\label{other Charm++ calls}

The following calls provide commonly needed functions.

\function{double CkCpuTimer()} \index{CkCpuTimer} \index{timers}
\desc{Returns the current value of the system timer in seconds. The system
timer is started when the program begins execution. This timer measures process
time (user and system).}

\function{double CkTimer()} \index{CkTimer} \index{timers}
\desc{This is an alias for either \kw{CkWallTimer} on dedicated machines (such as ASCI Red) or 
\kw{CkCpuTimer} for machines with multiple user processes per CPU (such as workstation cluster.)}

