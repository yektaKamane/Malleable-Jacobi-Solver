\section{Group Objects}
\label{sec:group}

So far, we have discussed chares separately from the underlying hardware resources 
to which they are mapped. However, while writing lower-level libraries it is sometimes
useful to be able to refer to the PE on which a chare's entry method is being executed.
The \kw{group} \footnote{Originally called {\em Branch Office Chare} or 
{\em Branched Chare}}construct provides this facility by creating a 
%A \kw{group}
% \index{group}is a 
collection of chares, such that 
there exists \index{chare}a single chare (or {\sl branch} of the group) on each
PE.   Each branch has its own data members.  Groups have
a definition syntax similar to normal chares,
and they have to inherit from the system-defined class \kw{CBase}\_\uw{ClassName}, 
where \uw{ClassName} is the name of group's \CC{} class
\footnote{Older, deprecated syntax allows groups to inherit directly from the
system-defined class \kw{Group}}.

\subsection{Group Definition}

In the interface ({\tt .ci}) file, we declare

\begin{alltt}
group Foo \{
  // Interface specifications as for normal chares

  // For instance, the constructor ...
  entry Foo(\uw{parameters1});

  // ... and an entry method
  entry void someEntryMethod(\uw{parameters2});
\};
\end{alltt}

The definition of the {\tt Foo} class is given in the \texttt{.h} file, as follows:

\begin{alltt}
class Foo : public CBase\_Foo \{
  // Data and member functions as in C++
  // Entry functions as for normal chares

  public:
    Foo(\uw{parameters1});
    void someEntryMethod(\uw{parameters2});
\};
\end{alltt}

\subsection{Group Creation}

Groups are created in a manner similar to chares and chare arrays, i.e. 
through \kw{ckNew}. Given the declarations and definitions of group {\tt Foo}
from above, we can create a group in the following manner:

\begin{alltt}
CkGroupID fooGroupID = CProxy_Foo::ckNew(\uw{parameters1});
\end{alltt}


In the above, \kw{ckNew} returns an object of type \kw{CkGroupID}, which is
the globally unique identifier of the corresponding instance of the group.
This identifier is common to all of the group's branches and
can be obtained in the group's methods from the variable \kw{thisgroup}, which is a public data
member of the \kw{Group} superclass.

A group can also be identified through its proxy, which can be obtained in one of three ways:
(a) as the inherited {\tt thisProxy} data member of the class; (b) from a call to \kw{ckNew} 
as shown below:

\begin{alltt}
CkGroupID fooGroupID = CProxy_Foo::ckNew(\uw{parameters1});
\end{alltt}

or (c) by using a group identifier to create a proxy, as shown below:

\begin{alltt}
  // We have `fooGroupID' from the above `ckNew' invocation

  // Obtain a proxy to the group from its group ID
  CProxy_Foo anotherFooProxy = CProxy_Foo(fooGroupID);
\end{alltt}

It is possible to specify the dependence of group creations using
\uw{CkEntryOptions}. For example, in the following code, the creation of group
{\tt GroupB} on each PE depends on the creation of {\tt GroupA} on that PE.

\begin{alltt}
// Create GroupA
CkGroupID groupAID = CProxy_GroupA::ckNew(\uw{parameters1});

// Create GroupB. However, for each PE, do this only 
// after GroupA has been created on it

// Specify the dependency through a `CkEntryOptions' object
CkEntryOptions opts;
opts.setGroupDepID(groupAId);

// The last argument to `ckNew' is the `CkEntryOptions' object from above
CkGroupID groupBID = CProxy_GroupB::ckNew(\uw{parameters2}, opts);
\end{alltt}

%For groups, \kw{thishandle} is the
%handle of the particular branch in which the function is executing: it is a
%normal chare handle.

%Groups can be used to implement data-parallel operations easily.  In addition
%to sending messages to a particular branch of a group, one can broadcast
%messages to all branches of a group.  
Note that there can be several instances of each group type.
In such a case, each instance has a unique group identifier, and its own set
of branches.

\subsection{Method Invocation on Groups}

An asynchronous entry method can be invoked on a particular branch of a
group through a proxy of that group. If we have a group with a proxy
{\tt fooProxy} and we wish to invoke entry method {\tt someEntryMethod} on
that branch of the group which resides on PE {\tt somePE}, we would accomplish
this with the following syntax:

\begin{alltt}
 fooProxy[somePE].someEntryMethod(\uw{parameters});
\end{alltt}

%This sends the given parameters to the \index{branch}branch of
%the group referred to by \uw{groupProxy} which is on processor number
%\uw{Processor} at the entry method \uw{EntryMethod}, which must be a valid
%entry method of that group type. 
This call is asynchronous and non-blocking; it returns immediately after sending the message.
A message may be broadcast \index{broadcast} to all branches of a group
(i.e., to all PEs) using the notation :

\begin{alltt}
 fooProxy.anotherEntryMethod(\uw{parameters});
\end{alltt}

This invokes entry method \uw{anotherEntryMethod} with the given \uw{parameters} on 
all branches of the group. This call is asynchronous and non-blocking; it returns immediately
after sending the message.

Recall that each PE hosts a branch of every instantiated group. 
Sequential objects, chares and other groups can gain access to this {\em PE-local}
branch using \kw{ckLocalBranch()}:

\begin{alltt}
GroupType *g=fooProxy.ckLocalBranch();
\end{alltt}

This call returns a regular \CC\ pointer to the actual object (not a proxy)
referred to by the proxy \uw{groupProxy}.  Once a proxy to the
local branch of a group is obtained, that branch can be accessed as a regular
\CC\ object.  Its public methods can return values, and its public data is 
readily accessible.

Thus a dynamically created \index{chare}chare can invoke a public method of a
group without knowining the PE on which it actually resides. 
%the method
%executes in the local \index{branch}branch of the group.

Let us end with an example use-case for groups.
%One very nice use of Groups is to reduce the number of messages sent between
%processors by collecting the data from all the chares on a single processor
Suppose that we have a task-parallel program in which we dynamically spawn
new chares. Furthermore, assume that each one of these chares has some data
to send to the mainchare.  Instead of creating a separate message for each 
chare's data, we create a group. When a particular chare
finishes its work, it reports its findings to the local branch of the group.
When all the chares on a PE have finished their work, the local branch
can send single a message to the main chare.  This reduces the number of messages
sent to the mainchare from the number of chares created, to the number of processors. 

For a more concrete example on how to use groups, please refer to the {\tt
histogram\_group} example in the {\tt examples/charm++} directory in your
\charmpp{} distribution. It presents a parallel histogramming operation in
which chare array elements funnel their bin counts through a group, instead of
contributing directly to a reduction across all chares. 

