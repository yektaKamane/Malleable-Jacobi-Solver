
\section{Message Order Race Conditions}
While common \charm programs are data-race free due to a lack of
shared mutable state between threads, it is still possible to observe
race conditions resulting from variation in the order that messages
are delivered to each chare object. The \charm ecosystem offers a
variety of ways to attempt to reproduce these often non-deterministic
bugs, diagnose their causes, and test fixes.

\subsection{Randomized Message Queueing}
To facilitate debugging of applications and to identify race
conditions due to message order, the user can enable a randomized
message scheduler queue.  Using the build-time configuration
option \verb|--enable-randomized-msgq|, the charm message queue will
be randomized. Note that a randomized message queue is only available
when message priority type is not bit vector.  Therefore, the user
needs to specify prio-type to be a data type long enough to hold the
msg priorities in your application for
eg: \verb|--with-prio-type=int|.

\subsection{CharmDebug}
The CharmDebug interactive debugging tool can be used to inspect the
messages in the scheduling queue of each processing element, and to
manipulate the order in which they're delivered. More details on how
to use CharmDebug can be found in its manual.

\subsection{Deterministic Record-Replay}
\charm supports recording the order in which messages are processed from one
run, to deterministically replay the same order in subsequent
runs. This can be useful to capture the infrequent undesirable message
order cases that cause intermittent failures. Once an impacted run has
been recorded, various debugging methods can be more easily brought to
bear, without worrying that they will perturb execution to avoid the
bug.

Support for record-replay is enabled in common builds
of \charm. Builds with the \verb|--with-production| option disable
this support to reduce overhead. To record traces, simply run the
program with an additional command line-flag \verb|+record|. The
generated traces can be repeated with the command-line
flag \verb|+replay|. The full range of parallel and sequential
debugging techniques are available to apply during deterministic
replay.

The traces will work even if the application is modified and
recompiled, as long as entry method numbering and send/receive
sequences do not change. For instance, it is acceptable to add print
statements or assertions to aid in the debugging process.


\section{Memory Access Errors}


\subsection{Using Valgrind}

The popular Valgrind memory debugging tool can be used to
monitor \charm applications in both serial and parallel
executions. For single-process runs, it can be used directly:
\begin{verbatim}
valgrind ...valgrind options... ./application_name ...application arguments...
\end{verbatim}

When running in parallel, it is helpful to note a few useful
adaptations of the above incantation, for various kinds of process launchers:
\begin{verbatim}
./charmrun +p2 `which valgrind` --log-file=VG.out.%p --trace-children=yes ./application_name ...application arguments...
aprun -n 2 `which valgrind` --log-file=VG.out.%p --trace-children=yes ./application_name ...application arguments...
\end{verbatim}
The first adaptation is to use \verb+`which valgrind`+ to obtain a
full path to the valgrind binary, since parallel process launchers
typically do not search the environment \verb+$PATH+ directories for
the program to run. The second adaptation is found in the options
passed to valgrind. These will make sure that valgrind tracks the
spawned application process, and write its output to per-process logs
in the file system rather than standard error.
