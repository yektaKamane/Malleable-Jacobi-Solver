\subsection{Entry Methods}
\label{entry}

In \charmpp, \index{chare}chares, \index{group}groups and \index{nodegroup}
nodegroups communicate using remote method invocation.  These ``remote entry'' methods may either take marshalled parameters, described in the next section; or special objects called messages.  Messages are lower level, more efficient, more flexible, and more difficult to use than parameter marshalling.

An entry method is always a part of a chare--
there are no global entry methods in \charmpp{}.
Entry methods are declared in the the interface file as:

\begin{alltt}
entry void \uw{Entry1}(\uw{parameters});
\end{alltt}

\uw{Parameters} is either a list of marshalled parameters,
(e.g., ``int i, double x''), or a message description (e.g.,
``MyMessage *msg'').  See section~\ref{marshalling} and
section~\ref{messages} for details on these types of
parameters.

Entry methods typically do not return data-- in \CC, they have
return type ``void''.  An entry method with the same name
as its enclosing class is a constructor.  Constructors in \CC
have no return type.  Finally, sync methods, described below,
may return a message.

\subsubsection{Entry Method Attributes}
\label{attributes}

\charmpp{}  provides a handful of special attributes that \index{entry
method}entry methods may have.  In order to give a particular \index{entry
method}entry method an attribute, you must specify the keyword for the desired
attribute in the attribute list of that entry method's {\tt .ci} file
declaration.  The syntax for this is as follows:

\begin{alltt}
entry [\uw{attribute1}, ..., \uw{attributeN}] void \uw{EntryMethod}(\uw{parameters});
\end{alltt}

\charmpp{} currently offers four attributes that one may give an entry method:
\kw{threaded}, \kw{sync}, \kw{exclusive}, \kw{immediate}.

\index{threaded}Threaded \index{entry method}entry methods are simply entry
methods which are run in their own nonpremptible threads.  To make an
\index{entry method}entry method threaded, one simply adds the keyword
\kw{threaded} to the attribute list of that entry method.

\index{sync}Sync \index{entry method}entry methods are special in that calls to
sync entry methods are blocking - they do not return control to the caller
until the method is finished executing completely.  Sync methods may have
return values; however, they may only return messages.  To make an \index{entry
method}entry method a sync entry method, add the keyword \kw{sync} to the
attribute list of that entry method.

\index{exclusive}Exclusive entry methods, which exist only on node groups, are
\index{entry method}entry methods that do not execute while other exclusive
\index{entry method}entry methods of its node group are executing in the same
node.  If one exclusive method of a node group is executing on node 0, and
another one is scheduled to run on that same node, the second exclusive method
will wait for the first to finish before it executes.  To make an \index{entry
method}entry method exclusive, add the keyword \kw{exclusive} to that
entry method's attribute list.

\index{immediate}Immediate entry methods are entry functions in which short
messages can be executed in an "immediate" fashion when they are received 
either by an interrupt(Network version) or by a communication 
thread(SMP version). Such messages can be useful for 
implementing multicasts/reductions as well as data lookup, in which case 
processing of critical messages won't be blocked by some entry functions that
could take long time to finish. Immediate entry methods should be reentrant,
and it cannot depend on any processor private data. 
It is user's responsibility to use lock to protect critical data. 
Function \kw{CmiPollImmediateMsg()} can be called in users code to 
probe and process immediate messages periodically.




