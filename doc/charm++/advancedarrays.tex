The basic array features described previously (creation, messaging,
broadcasts, and reductions) are needed in almost every
\charmpp{} program.  The more advanced techniques that follow
are not universally needed, but represent many useful optimisations.

\section{Local Access}

\index{ckLocal for arrays}
\label{ckLocal for arrays}
Provides direct access to a local array element using the
proxy's \kw{ckLocal} method, which returns an ordinary \CC\ pointer
to the element if it exists on the local processor; and NULL if
the element does not exist or is on another processor.

\begin{alltt}
A1 *a=a1[i].ckLocal();
if (a==NULL) //...is remote-- send message
else //...is local-- directly use members and methods of a
\end{alltt}

Note that if the element migrates or is deleted, any pointers 
obtained with \kw{ckLocal} are no longer valid.  It is best,
then, to either avoid \kw{ckLocal} or else call \kw{ckLocal} 
each time the element may have migrated; e.g., at the start 
of each entry method.

An example of this usage is available in matmul3d:
\begin{alltt}
examples/charm++/topology/matmul3d
\end{alltt}

\section{Advanced Array Creation}
\label{advanced array create}

There are several ways to control the array creation process.
You can adjust the map and bindings before creation, change
the way the initial array elements are created, create elements
explicitly during the computation, and create elements implicitly,
``on demand''.  

You can create all your elements using any one of these methods,
or create different elements using different methods.  
An array element has the same syntax and semantics no matter
how it was created.  



\subsection{Configuring Array Characteristics Using CkArrayOptions}
\index{CkArrayOptions}
\label{CkArrayOptions}

The array creation method \kw{ckNew} actually takes a parameter
of type \kw{CkArrayOptions}.  This object describes several
optional attributes of the new array.

The most common form of \kw{CkArrayOptions} is to set the number
of initial array elements.  A \kw{CkArrayOptions} object will be 
constructed automatically in this special common case.  Thus
the following code segments all do exactly the same thing:

\begin{alltt}
//Implicit CkArrayOptions
  a1=CProxy_A1::ckNew(\uw{parameters},nElements);

//Explicit CkArrayOptions
  a1=CProxy_A1::ckNew(\uw{parameters},CkArrayOptions(nElements));

//Separate CkArrayOptions
  CkArrayOptions opts(nElements);
  a1=CProxy_A1::ckNew(\uw{parameters},opts);
\end{alltt}


Note that the ``numElements'' in an array element is simply the
numElements passed in when the array was created.  The true number of
array elements may grow or shrink during the course of the
computation, so numElements can become out of date.  This ``bulk''
constructor approach should be preferred where possible, especially
for large arrays.  Bulk construction is handled via a broadcast which
will be significantly more efficient in the number of messages
required than inserting each element individually, which will require
one message send per element.

Examples of bulk construction are commonplace, see:
\begin{alltt}
examples/charm++/jacobi3d-sdag
\end{alltt}
for a demonstration of the slightly more complicated case of
multidimensional chare array bulk construction.

\kw{CkArrayOptions} contains a few flags that the runtime can use to
optimize handling of a given array. If the array elements will only
migrate at controlled points (such as periodic load balancing with
{\tt AtASync()}), this is signalled to the runtime by calling {\tt
  opts.setAnytimeMigration(false)}\footnote{At present, this optimizes
broadcasts to not save old messages for immigrating chares.}. If all
array elements will be inserted by bulk creation or by {\tt
  fooArray[x].insert()} calls, signal this by calling {\tt
  opts.setStaticInsertion(true)} \footnote{This can enable a slightly
  faster default mapping scheme.}.

\subsection{Initial Placement Using Map Objects}
\index{array map}
\label{array map}

You can use \kw{CkArrayOptions} to specify a ``map object'' for an
array.  The map object is used by the array manager to determine the
``home'' PE of each element.  The home PE is the PE upon which it is
initially placed, which will retain responsibility for maintaining the
location of the element.

There is a default map object, which maps 1D array indices
in a block fashion to processors, and maps other array
indices based on a hash function. Some other mappings such as round-robin
(\kw{RRMap}) also exist, which can be used
similar to custom ones described below.

A custom map object is implemented as a group which inherits from
\kw{CkArrayMap} and defines these virtual methods:

\begin{alltt}
class CkArrayMap : public Group
\{
public:
  //...
  
  //Return an ``arrayHdl'', given some information about the array
  virtual int registerArray(CkArrayIndex& numElements,CkArrayID aid);
  //Return the home processor number for this element of this array
  virtual int procNum(int arrayHdl,const CkArrayIndex &element);
\}
\end{alltt}

For example, a simple 1D blockmapping scheme.  Actual mapping is
handled in the procNum function.

\begin{alltt}
class BlockMap : public CkArrayMap 
\{
 public:
  BlockMap(void) \{\}
  BlockMap(CkMigrateMessage *m)\{\}
  int registerArray(CkArrayIndex& numElements,CkArrayID aid) \{
    return 0;
  \}
  int procNum(int /*arrayHdl*/,const CkArrayIndex &idx) \{
    int elem=*(int *)idx.data();
    int penum =  (elem/(32/CkNumPes()));
    return penum;
  \}
\};

\end{alltt}
Once you've instantiated a custom map object, you can use it to
control the location of a new array's elements using the
\kw{setMap} method of the \kw{CkArrayOptions} object described above.
For example, if you've declared a map object named ``BlockMap'':

\begin{alltt}
//Create the map group
  CProxy_BlockMap myMap=CProxy_BlockMap::ckNew();
//Make a new array using that map
  CkArrayOptions opts(nElements);
  opts.setMap(myMap);
  a1=CProxy_A1::ckNew(\uw{parameters},opts);
\end{alltt}

An example which contructs one element per physical node may be found here: 
\begin{alltt}
examples/charm++/pupDisk
\end{alltt}

Other 3D Torus network oriented map examples are in:
\begin{alltt}
examples/charm++/topology
\end{alltt}

\subsection{Initial Elements}
\index{array initial}
\label{array initial}

The map object described above can also be used to create
the initial set of array elements in a distributed fashion.
An array's initial elements are created by its map object,
by making a call to \kw{populateInitial} on each processor.

You can create your own set of elements by creating your
own map object and overriding this virtual function of \kw{CkArrayMap}:

\begin{alltt}
  virtual void populateInitial(int arrayHdl,int numInitial,
	void *msg,CkArrMgr *mgr)
\end{alltt}

In this call, \kw{arrayHdl} is the value returned by \kw{registerArray},
\kw{numInitial} is the number of elements passed to \kw{CkArrayOptions},
\kw{msg} is the constructor message to pass, and \kw{mgr} is the
array to create.

\kw{populateInitial} creates new array elements using the method
\kw{void CkArrMgr::insertInitial(CkArrayIndex idx,void *ctorMsg)}.
For example, to create one row of 2D array elements on each processor,
you would write:

\begin{alltt}
void xyElementMap::populateInitial(int arrayHdl,int numInitial,
	void *msg,CkArrMgr *mgr)
\{
  if (numInitial==0) return; //No initial elements requested
	
  //Create each local element
  int y=CkMyPe();
  for (int x=0;x<numInitial;x++) \{
    mgr->insertInitial(CkArrayIndex2D(x,y),CkCopyMsg(&msg));
  \}
  mgr->doneInserting();
  CkFreeMsg(msg);
\}
\end{alltt}

Thus calling \kw{ckNew(10)} on a 3-processor machine would result in
30 elements being created.


\section{Bound Arrays}
\index{bound arrays} \index{bindTo}
\label{bound arrays}

You can ``bind'' a new array to an existing array
using the \kw{bindTo} method of \kw{CkArrayOptions}.  Bound arrays
act like separate arrays in all ways except for migration--
corresponding elements of bound arrays always migrate together.
For example, this code creates two arrays A and B which are
bound together-- A[i] and B[i] will always be on the same processor.

\begin{alltt}
//Create the first array normally
  aProxy=CProxy_A::ckNew(\uw{parameters},nElements);
//Create the second array bound to the first
  CkArrayOptions opts(nElements);
  opts.bindTo(aProxy);
  bProxy=CProxy_B::ckNew(\uw{parameters},opts);
\end{alltt}

An arbitrary number of arrays can be bound together--
in the example above, we could create yet another array
C and bind it to A or B.  The result would be the same
in either case-- A[i], B[i], and C[i] will always be
on the same processor.

There is no relationship between the types of bound arrays--
it is permissible to bind arrays of different types or of the
same type.  It is also permissible to have different numbers
of elements in the arrays, although elements of A which have
no corresponding element in B obey no special semantics.
Any method may be used to create the elements of any bound
array.

Bound arrays are often useful if A[i] and B[i] perform different 
aspects of the same computation, and thus will run most efficiently 
if they lie on the same processor.  Bound array elements are guaranteed
to always be able to interact using \kw{ckLocal} (see 
section~\ref{ckLocal for arrays}), although the local pointer must
be refreshed after any migration. This should be done during the \kw{pup}
routine. When migrated, all elements that are bound together will be created
at the new processor before \kw{pup} is called on any of them, ensuring that
a valid local pointer to any of the bound objects can be obtained during the
\kw{pup} routine of any of the others.

For example, an array {\it Alibrary} is implemented as a library module.
It implements a certain functionality by operating on a data array {\it dest}
which is just a pointer to some user provided data.
A user defined array {\it UserArray} is created and bound to 
the array {\it Alibrary} to take advanatage of the functionality provided 
by the library.
When bound array element migrated, the {\it data} pointer in {\it UserArray}
is re-allocated in {\it pup()}, thus {\it UserArray} is responsible to refresh
the pointer {\it dest} stored in {\it Alibrary}.

\begin{alltt}
class Alibrary: public CProxy_Alibrary \{
public:
  ...
  void set_ptr(double *ptr) \{ dest = ptr; \}
  virtual void pup(PUP::er &p);
private:
  double *dest;           // point to user data in user defined bound array
\};

class UserArray: public CProxy_UserArray \{
public:
  virtual void pup(PUP::er &p) \{
                p|len;
                if(p.isUnpacking()) \{ 
                  data = new double[len];
                  Alibrary *myfellow = AlibraryProxy(thisIndex).ckLocal();
                  myfellow->set_ptr(data);    // refresh data in bound array
                \}
                p(data, len);
  \}
private:
  CProxy_Alibrary  AlibraryProxy;   // proxy to my bound array
  double *data;          // user allocated data pointer
  int len;
\};
\end{alltt}

A demonstration of bound arrays can be found in startupTest
\begin{alltt}
test/charm++/startupTest
\end{alltt}


\section{Dynamic Insertion}
\label{dynamic_insertion}

In addition to creating initial array elements using ckNew,
you can also
create array elements during the computation.

You insert elements into the array by indexing the proxy
and calling insert.  The insert call optionally takes 
parameters, which are passed to the constructor; and a
processor number, where the element will be created.
Array elements can be inserted in any order from 
any processor at any time.  Array elements need not 
be contiguous.

If using \kw{insert} to create all the elements of the array,
you must call \kw{CProxy\_Array::doneInserting} before using
the array.

\begin{alltt}
//In the .C file:
int x,y,z;
CProxy_A1 a1=CProxy_A1::ckNew();  //Creates a new, empty 1D array
for (x=...) \{
   a1[x  ].insert(\uw{parameters});  //Bracket syntax
   a1(x+1).insert(\uw{parameters});  // or equivalent parenthesis syntax
\}
a1.doneInserting();

CProxy_A2 a2=CProxy_A2::ckNew();   //Creates 2D array
for (x=...) for (y=...)
   a2(x,y).insert(\uw{parameters});  //Can't use brackets!
a2.doneInserting();

CProxy_A3 a3=CProxy_A3::ckNew();   //Creates 3D array
for (x=...) for (y=...) for (z=...)
   a3(x,y,z).insert(\uw{parameters});
a3.doneInserting();

CProxy_AF aF=CProxy_AF::ckNew();   //Creates user-defined index array
for (...) \{
   aF[CkArrayIndexFoo(...)].insert(\uw{parameters}); //Use brackets...
   aF(CkArrayIndexFoo(...)).insert(\uw{parameters}); //  ...or parenthesis
\}
aF.doneInserting();

\end{alltt}

The \kw{doneInserting} call starts the reduction manager (see ``Array
Reductions'') and load balancer (see ~\ref{lbFramework})-- since
these objects need to know about all the array's elements, they
must be started after the initial elements are inserted.
You may call \kw{doneInserting} multiple times, but only the first
call actually does anything.  You may even \kw{insert} or \kw{destroy}
elements after a call to \kw{doneInserting}, with different semantics-- 
see the reduction manager and load balancer sections for details.

If you do not specify one, the system will choose a processor to 
create an array element on based on the current map object.

A demonstration of dynamic insertion is available:
\begin{alltt}
examples/charm++/hello/fancyarray
\end{alltt}

\section{Demand Creation}

Normally, invoking an entry method on a nonexistant array
element is an error.  But if you add the attribute
\index{createhere} \index{createhome}
\kw{[createhere]} or \kw{[createhome]} to an entry method,
 the array manager will 
``demand create'' a new element to handle the message.  

With \kw{[createhome]}, the new element
will be created on the home processor, which is most efficient when messages for
the element may arrive from anywhere in the machine. With \kw{[createhere]},
the new element is created on the sending processor, which is most efficient
if when messages will often be sent from that same processor.

The new element is created by calling its default (taking no
parameters) constructor, which must exist and be listed in the .ci file.
A single array can have a mix of demand-creation and
classic entry methods; and demand-created and normally 
created elements.

A simple example of demand creation
\begin{alltt}
tests/charm++/demand\_creation
\end{alltt}

\section{User-defined Array Indices}
\label{user-defined array index type}
\index{Array index type, user-defined}

\charmpp{} array indices are arbitrary collections of integers.
To define a new array index, you create an ordinary C++ class 
which inherits from \kw{CkArrayIndex} and sets the ``nInts'' member
to the length, in integers, of the array index.

For example, if you have a structure or class named ``Foo'', you 
can use a \uw{Foo} object as an array index by defining the class:

\begin{alltt}
#include <charm++.h>
class CkArrayIndexFoo:public CkArrayIndex \{
    Foo f;
public:
    CkArrayIndexFoo(const Foo \&in) 
    \{
        f=in;
        nInts=sizeof(f)/sizeof(int);
    \}
    //Not required, but convenient: cast-to-foo operators
    operator Foo &() \{return f;\}
    operator const Foo &() const \{return f;\}
\};
\end{alltt}

Note that \uw{Foo}'s size must be an integral number of integers--
you must pad it with zero bytes if this is not the case.
Also, \uw{Foo} must be a simple class-- it cannot contain 
pointers, have virtual functions, or require a destructor.
Finally, there is a \charmpp\ configuration-time option called
CK\_ARRAYINDEX\_MAXLEN \index{CK\_ARRAYINDEX\_MAXLEN} 
which is the largest allowable number of 
integers in an array index.  The default is 3; but you may 
override this to any value by passing ``-DCK\_ARRAYINDEX\_MAXLEN=n'' 
to the \charmpp\ build script as well as all user code. Larger 
values will increase the size of each message.

You can then declare an array indexed by \uw{Foo} objects with

\begin{alltt}
//in the .ci file:
array [Foo] AF \{ entry AF(); ... \}

//in the .h file:
class AF : public CBase\_AF
\{ public: AF() \{\} ... \}

//in the .C file:
    Foo f;
    CProxy_AF a=CProxy_AF::ckNew();
    a[CkArrayIndexFoo(f)].insert();
    ...
\end{alltt}

Note that since our CkArrayIndexFoo constructor is not declared
with the explicit keyword, we can equivalently write the last line as:

\begin{alltt}
    a[f].insert();
\end{alltt}

When you implement your array element class, as shown above you 
can inherit from \kw{CBase}\_\uw{ClassName}, 
a class templated by the index type \uw{Foo}. In the old syntax,
you could also inherit directly from \kw{ArrayElementT}.
The array index (an object of type \uw{Foo}) is then accessible as 
``thisIndex''. For example:

\begin{alltt}

//in the .C file:
AF::AF()
\{
    Foo myF=thisIndex;
    functionTakingFoo(myF);
\}
\end{alltt}

A demonstration of user defined indices available:
\begin{alltt}
examples/charm++/hello/fancyarray
\end{alltt}

%\section{Load Balancing Chare Arrays}
%see section~\ref{lbFramework}

