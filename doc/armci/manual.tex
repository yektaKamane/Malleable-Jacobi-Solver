%\documentclass[10pt,dvips]{article}
\documentclass[10pt]{article}
\usepackage{../pplmanual}
\usepackage[pdftex]{graphicx}
%\usepackage[dvips]{graphicx}
%\usepackage[usenames,dvipsnames]{color}
%\usepackage[pdftex]{hyperref}
\usepackage{epsfig}
\NeedsTeXFormat{LaTeX2e}
\typeout{^^J^^J
Parallel Programming Laboratory^^J
Manual Style^^J
Written by Milind A. Bhandarkar, 12/00^^J}

%%% Make it possible for both ps and pdf to be generated
\newif\ifpdf
\ifx\pdfoutput\undefined
  \pdffalse
\else
  \pdfoutput=1
  \pdftrue
\fi

\ifpdf
  \pdfcompresslevel=9
\fi

%%% Imported from fullpage.sty, since it is not always available
\topmargin 0pt
\advance \topmargin by -\headheight
\advance \topmargin by -\headsep

\textheight 8.9in

\oddsidemargin 0pt
\evensidemargin \oddsidemargin
\marginparwidth 1.0in

\textwidth 6.5in
%%% end import from fullpage

%%% Commonly Needed packages
\usepackage{graphicx,color,calc}
\usepackage{makeidx}
\usepackage{alltt}

%%% Commands for uniform looks of C++, Charm++, and Projections
\newcommand{\CC}{C\kern -0.0em\raise 0.5ex\hbox{\normalsize++}}
\newcommand{\emCC}{C\kern -0.0em\raise 0.4ex\hbox{\normalsize\em++}}
\newcommand{\charmpp}{\sc Charm++}
\newcommand{\projections}{\sc Projections}
\newcommand{\converse}{\sc Converse}
\newcommand{\ampi}{\sc AMPI}

%%% Commands to produce margin symbols
\newcommand{\new}{\marginpar{\fbox{\bf$\mathcal{NEW}$}}}
\newcommand{\important}{\marginpar{\fbox{\bf\Huge !}}}
\newcommand{\experimental}{\marginpar{\fbox{\bf\Huge $\beta$}}}

%%% Commands for manual elements
\newcommand{\zap}[1]{ }
\newcommand{\function}[1]{{\noindent{\textsf{#1}}\\}}
\newcommand{\cmd}[1]{{\noindent{\textsf{#1}}\\}}
\newcommand{\args}[1]{\hspace*{2em}{\texttt{#1}}\\}
\newcommand{\param}[1]{{\texttt{#1}}}
\newcommand{\kw}[1]{{\textsf{#1}}}
\newcommand{\uw}[1]{{\textsl{#1}}}
\newcommand{\desc}[1]{\indent{#1}}

%%% Commands needed for Maketitle
\newcommand{\@version}{}
\newcommand{\@credits}{}
\newcommand{\version}[1]{\renewcommand{\@version}{#1}}
\newcommand{\credits}[1]{\renewcommand{\@credits}{#1}}

%%% Print the License Page
\newcommand{\@license}{%
 \begin{center}
   {University of Illinois}\\
   {\charmpp/\converse\ Parallel Programming System Software}\\
   {Non-Exclusive, Non-Commercial Use License}\\
 \end{center}
 \rule{\textwidth}{1pt}
{\tiny
Upon execution of this Agreement by the party identified below (``Licensee''),
The Board of Trustees of the University of Illinois  (``Illinois''), on behalf
of The Parallel Programming Laboratory (``PPL'') in the Department of Computer
Science, will provide the \charmpp/\converse\ Parallel Programming System
software (``\charmpp'') in Binary Code and/or Source Code form (``Software'')
to Licensee, subject to the following terms and conditions. For purposes of
this Agreement, Binary Code is the compiled code, which is ready to run on
Licensee's computer.  Source code consists of a set of files which contain the
actual program commands that are compiled to form the Binary Code.

\begin{enumerate}
  \item
    The Software is intellectual property owned by Illinois, and all right,
title and interest, including copyright, remain with Illinois.  Illinois
grants, and Licensee hereby accepts, a restricted, non-exclusive,
non-transferable license to use the Software for academic, research and
internal business purposes only, e.g. not for commercial use (see Clause 7
below), without a fee.

  \item 
    Licensee may, at its own expense, create and freely distribute
complimentary works that interoperate with the Software, directing others to
the PPL server (\texttt{http://charm.cs.uiuc.edu}) to license and obtain the
Software itself. Licensee may, at its own expense, modify the Software to make
derivative works.  Except as explicitly provided below, this License shall
apply to any derivative work as it does to the original Software distributed by
Illinois.  Any derivative work should be clearly marked and renamed to notify
users that it is a modified version and not the original Software distributed
by Illinois.  Licensee agrees to reproduce the copyright notice and other
proprietary markings on any derivative work and to include in the documentation
of such work the acknowledgement:

\begin{quote}
``This software includes code developed by the Parallel Programming Laboratory
in the Department of Computer Science at the University of Illinois at
Urbana-Champaign.''
\end{quote}

Licensee may redistribute without restriction works with up to 1/2 of their
non-comment source code derived from at most 1/10 of the non-comment source
code developed by Illinois and contained in the Software, provided that the
above directions for notice and acknowledgement are observed.  Any other
distribution of the Software or any derivative work requires a separate license
with Illinois.  Licensee may contact Illinois (\texttt{kale@cs.uiuc.edu}) to
negotiate an appropriate license for such distribution.

  \item
    Except as expressly set forth in this Agreement, THIS SOFTWARE IS PROVIDED
``AS IS'' AND ILLINOIS MAKES NO REPRESENTATIONS AND EXTENDS NO WARRANTIES OF
ANY KIND, EITHER EXPRESS OR IMPLIED, INCLUDING BUT NOT LIMITED TO WARRANTIES OR
MERCHANTABILITY OR FITNESS FOR A PARTICULAR PURPOSE, OR THAT THE USE OF THE
SOFTWARE WILL NOT INFRINGE ANY PATENT, TRADEMARK, OR OTHER RIGHTS.  LICENSEE
ASSUMES THE ENTIRE RISK AS TO THE RESULTS AND PERFORMANCE OF THE SOFTWARE
AND/OR ASSOCIATED MATERIALS.  LICENSEE AGREES THAT UNIVERSITY SHALL NOT BE HELD
LIABLE FOR ANY DIRECT, INDIRECT, CONSEQUENTIAL, OR INCIDENTAL DAMAGES WITH
RESPECT TO ANY CLAIM BY LICENSEE OR ANY THIRD PARTY ON ACCOUNT OF OR ARISING
FROM THIS AGREEMENT OR USE OF THE SOFTWARE AND/OR ASSOCIATED MATERIALS.

  \item 
    Licensee understands the Software is proprietary to Illinois. Licensee
agrees to take all reasonable steps to insure that the Software is  protected
and secured from unauthorized disclosure, use, or release and  will treat it
with at least the same level of care as Licensee would use to  protect and
secure its own proprietary computer programs and/or information, but using no
less than a reasonable standard of care.  Licensee agrees to provide the
Software only to any other person or entity who has registered with Illinois.
If licensee is not registering as an individual but as an institution or
corporation each member of the institution or corporation who has access to or
uses Software must agree to and abide by the terms of this license. If Licensee
becomes aware of any unauthorized licensing, copying or use of the Software,
Licensee shall promptly notify Illinois in writing. Licensee expressly agrees
to use the Software only in the manner and for the specific uses authorized in
this Agreement.

  \item
    By using or copying this Software, Licensee agrees to abide by the
copyright law and all other applicable laws of the U.S. including, but not
limited to, export control laws and the terms of this license. Illinois  shall
have the right to terminate this license immediately by written  notice upon
Licensee's breach of, or non-compliance with, any terms of the license.
Licensee may be held legally responsible for any  copyright infringement that
is caused or encouraged by its failure to  abide by the terms of this license.
Upon termination, Licensee agrees to  destroy all copies of the Software in its
possession and to verify such  destruction in writing.

  \item
  The user agrees that any reports or published results obtained with  the
Software will acknowledge its use by the appropriate citation as  follows:

\begin{quote}
``\charmpp/\converse\ was developed by the Parallel Programming Laboratory in
the Department of Computer Science at the University of  Illinois at
Urbana-Champaign.''
\end{quote}

Any published work which utilizes \charmpp\ shall include the following
reference:

\begin{quote}
``L. V. Kale and S. Krishnan. \charmpp: Parallel Programming with Message-Driven
Objects. In 'Parallel Programming using \CC' (Eds. Gregory V. Wilson and Paul
Lu), pp 175-213, MIT Press, 1996.''
\end{quote}

Any published work which utilizes \converse\ shall include the following
reference:

\begin{quote}
``L. V. Kale, Milind Bhandarkar, Narain Jagathesan, Sanjeev Krishnan and Joshua
Yelon. \converse: An Interoperable Framework for Parallel Programming.
Proceedings of the 10th International Parallel Processing Symposium, pp
212-217, April 1996.''
\end{quote}

Electronic documents will include a direct link to the official \charmpp\ page
at \texttt{http://charm.cs.uiuc.edu/}

  \item
    Commercial use of the Software, or derivative works based thereon,
REQUIRES A COMMERCIAL LICENSE.  Should Licensee wish to make commercial use of
the Software, Licensee will contact Illinois (kale@cs.uiuc.edu) to negotiate an
appropriate license for such use. Commercial use includes: 

    \begin{enumerate}
      \item
	integration of all or part of the Software into a product for sale,
lease or license by or on behalf of Licensee to third parties, or 

      \item
	distribution of the Software to third parties that need it to
commercialize product sold or licensed by or on behalf of Licensee.
    \end{enumerate}

  \item
    Government Rights. Because substantial governmental funds have been  used
in the development of \charmpp/\converse, any possession, use or sublicense of
the Software by or to the United States government shall be subject to such
required restrictions.

  \item
    \charmpp/\converse\ is being distributed as a research and teaching tool
and as such, PPL encourages contributions from users of the code that might, at
Illinois' sole discretion, be used or incorporated to make the basic  operating
framework of the Software a more stable, flexible, and/or useful  product.
Licensees who contribute their code to become an internal  portion of the
Software agree that such code may be distributed by  Illinois under the terms
of this License and may be required to sign an  ``Agreement Regarding
Contributory Code for \charmpp/\converse\ Software'' before Illinois  can
accept it (contact \texttt{kale@cs.uiuc.edu} for a copy).
\end{enumerate}

UNDERSTOOD AND AGREED.

Contact Information:

The best contact path for licensing issues is by e-mail to
\texttt{kale@cs.uiuc.edu} or send correspondence to:

\begin{quote}
Prof. L. V. Kale\\
Dept. of Computer Science\\
University of Illinois\\
1304 W. Springfield Ave\\
Urbana, Illinois 61801 USA\\
FAX: (217) 333-3501
\end{quote}
}%tiny
 \newpage
}% end of license

\renewcommand{\maketitle}{\begin{titlepage}%
 \begin{flushright}
   {\Large
     Parallel Programming Laboratory\\
     University of Illinois at Urbana-Champaign\\
   }
 \end{flushright}
 \rule{\textwidth}{3pt}
 \vspace{\fill}
 \begin{flushright}
   \textsf{\Huge \@title \\}
 \end{flushright}
 \vspace{\fill}
 \@credits \\
 \rule{\textwidth}{3pt}
 \begin{flushright}
   {\large Version \@version}
 \end{flushright}
 \end{titlepage}
 \@license

 \tableofcontents
 \newpage
}% maketitle



\ifpdf
\DeclareGraphicsExtensions{.jpg,.pdf,.mps,.png}
\else
\DeclareGraphicsExtensions{.eps}
\fi

\title{ARMCI Interface under \charmpp{}}
\version{1.0}
\credits{
Chee Wai Lee and Chao Huang
}

\begin{document}
\maketitle

\section{Introduction}
\label{sec::introduction}

This manual describes the basic features and API of the Aggregate
Remote Memory Copy Interface (ARMCI) library implemented under
\charmpp{}. It is meant for developers using ARMCI who desire the
performance features of the \charmpp{} run-time system (e.g. dynamic
load balancing, fault tolerance and scalability) applied transparently
to their libraries written using the ARMCI API.

ARMCI is a library that supports remote memory copy functionality. It
focuses on non-contiguous data transfers and is meant to be used by
other libraries as opposed to application development. Libraries that
the original ARMCI developers have targeted include Global Arrays,
P++/Overture and the Adlib PCRC run-time system.

ARMCI remote copy operations are one-sided and complete, regardless of
the actions taken by the remote process. For performance reasons,
polling can be helpful but should not be necessary to ensure
progress. The operations are ordered when referencing the same remote
process. Operations issued to different processes can complete in an
arbitrary order. Both blocking and non-blocking APIs are supported.

ARMCI supports three classes of operations: data transfer using {\em
put}, {\em get} and {\em accumulate} operations; synchronization with
local and global {\em fence} operations and atomic read-modify-write;
and utility functions for memory management and error handling. {\em
Accumulate} and atomic read-modify-write operations are currently not
implemented for the charmpp{} port.

A {\em get} operation transfers data from the remote process memory
(source) to the calling processing local memory (destination). A {\em
put} operation transfers data from the local memory of the calling
process (source) to the memory of the remote process (destination).

This manual will include several useful \charmpp{}-specific extensions to
the ARMCI API. It will also list the functions that have not yet been
implemented but exists in the original ARMCI implementation. Readers
of this manual are advised to refer to the original ARMCI
documentation (See Section \ref{sec::related doc}) for more complete
information and motivation for the development of this library.

\subsection{Building ARMCI Support under The \charmpp{} Runtime System}
\label{sec::charm build}

Build charm target ARMCI (instead of charm or AMPI):
\begin{verbatim}
> cd charm
> ./build ARMCI net-linux -O3
\end{verbatim}

\subsection{Writing a Simple ARMCI Program}
\label{sec::simple program}

The following simple example has two processes place their own string
into the global array and then acquire the appropriate string from the
other's global address space in order to print ``hello world''. 

The main function has to be compliant to ANSI C:

\begin{verbatim}
#include <stdio.h>
#include <stdlib.h>
#include <string.h>

#include <armci.h>

#define MAX_PROCESSORS 2

int main(int argc, char * argv[]) {
  void *baseAddress[MAX_PROCESSORS];
  char *myBuffer;
  int thisImage;
  
  // initialize
  ARMCI_Init();
  ARMCI_Myid(&thisImage);

  // allocate data (collective operation)
  ARMCI_Malloc(baseAddress, strlen("hello")+1);
 
  if (thisImage == 0) {
    sprintf((char *)baseAddress[0], "%s", "hello");
  } else if (thisImage == 1) {
    sprintf((char *)baseAddress[1], "%s", "world");
  }

  // allocate space for local buffer
  myBuffer = (char *)AMRCI_Malloc_local(strlen("hello")+1);
  
  ARMCI_Barrier();

  if (thisImage == 0) {
    ARMCI_Get(baseAddress[1], myBuffer, strlen("hello")+1, 1);
    printf("[%d] %s %s\n",thisImage, baseAddress[0], myBuffer);
  } else if (thisImage == 1) {
    ARMCI_Get(baseAddress[0], myBuffer, strlen("hello")+1, 0);
    printf("[%d] %s %s\n",thisImage, myBuffer, baseAddress[1]);
  }

  // finalize
  ARMCI_Finalize();
  return 0;
}
\end{verbatim}

\subsection{Building an ARMCI Binary and Execution}
\label{sec::armci build}

Compiling the code with:
\begin{verbatim}
> charm/bin/charmc -c hello.c /$(OPTS)
\end{verbatim}

\noindent
Linking the program with:
\begin{verbatim}
> charm/bin/charmc hello.o -o hello -swapglobals -memory isomalloc -language armci $(OPTS)
\end{verbatim}

\noindent
Run the program:
\begin{verbatim}
> ./charmrun ./hello +p2 +vp8
\end{verbatim}

\section{ARMCI Data Structures}
\label{sec::data structures}

ARMCI provides two formats to describe non-contiguous layouts of data
in memory.

The {\em generalized I/O vector} is the most general format intended
for multiple sets of equally sized data segments to be moved between
arbitrary local and remote memory locations. It uses two arrays of
pointers: one for source and one for destination addresses. The length
of each array is equal to the number of segments.

\begin{verbatim}
typedef struct {
  void *src_ptr_ar;
  void *dst_ptr_ar;
  int bytes;
  int ptr_ar_len;
} armci_giov_t;
\end{verbatim}

Currently, there is no support for {\em generalized I/O vector}
operations in the charmpp{} implementation.

The {\em strided} format is an optimization of the generalized I/O
vector format. It is intended to minimize storage required to describe
sections of dense multi-dimensional arrays. Instead of including
addresses for all the segments, it specifies only an address of the
first segment in the set for source and destination. The addresses of
the other segments can be computed using the stride information.

\section{Application Programmer's Interface}
\label{sec::api}

The following is a list of functions supported on the \charmpp{} port
of ARMCI. The integer value returned by most ARMCI operations
represents the error code. The zero value is successful, other values
represent failure (See Section \ref{sec::error codes} for details).

\subsection{Startup, Cleanup and Status Functions}

\begin{verbatim}
int ARMCI_Init(void);
\end{verbatim}
Initializes the ARMCI library. This function must be called before any
ARMCI functions may be used.

\begin{verbatim}
int ARMCI_Finalize(void);
\end{verbatim}
Shuts down the ARMCI library. No ARMCI functions may be called after
this call is made. It must be used before terminating the program normally.

\begin{verbatim}
void ARMCI_Cleanup(void);
\end{verbatim}
Releases system resources that the ARMCI library might be holding. This is
intended to be used before terminating the program in case of error.

\begin{verbatim}
void ARMCI_Error(char *msg, int code);
\end{verbatim}
Combines the functionality of ARMCI\_Cleanup and \charmpp{}'s CkAbort
call. Prints to {\em stdout} and {\em stderr} {\tt msg} followed by an
integer {\tt code}.

\begin{verbatim}
int ARMCI_Procs(int *procs);
\end{verbatim}
The number of processes is stored in the address {\tt procs}.

\begin{verbatim}
int ARMCI_Myid(int *myid);
\end{verbatim}
The id of the process making this call is stored in the address {\tt myid}.

\subsection{ARMCI Memory Allocation}

\begin{verbatim}
int ARMCI_Malloc(void* ptr_arr[], int bytes);
\end{verbatim}
Collective operation to allocate memory that can be used in the
context of ARMCI copy operations. Memory of size {\tt bytes} is
allocated on each process. The pointer address of each process'
allocated memory is stored at {\tt ptr\_arr[]} indexed by the process'
id (see {\tt ARMCI\_Myid}). Each process gets a copy of {\tt ptr\_arr}.

\begin{verbatim}
int ARMCI_Free(void *ptr);
\end{verbatim}
Collective operation to free memory which was allocated by 
{\tt ARMCI\_Malloc}.

\begin{verbatim}
void *ARMCI_Malloc_local(int bytes);
\end{verbatim}
Local memory of size {\tt bytes} allocated. Essentially a wrapper for
{\tt malloc}.

\begin{verbatim}
int ARMCI_Free_local(void *ptr);
\end{verbatim}
Local memory address pointed to by {\tt ptr} is freed. Essentially a
wrapper for {\tt free}.

\subsection{Put and Get Communication}

\begin{verbatim}
int ARMCI_Put(void *src, void *dst, int bytes, int proc);
\end{verbatim}
Transfer contiguous data of size {\tt bytes} from the local process
memory (source) pointed to by {\tt src} into the remote memory of
process id {\tt proc} pointed to by {\tt dst} (remote memory pointer
at destination).

\begin{verbatim}
int ARMCI_NbPut(void *src, void* dst, int bytes, int proc, 
                armci_hdl_t *handle);
\end{verbatim}
The non-blocking version of {\tt ARMCI\_Put}. Passing a {\tt NULL}
value to {\tt handle} makes this function perform an implicit handle
non-blocking transfer.

\begin{verbatim}
int ARMCI_PutS(void *src_ptr, int src_stride_ar[],
               void *dst_ptr, int dst_stride_ar[],
               int count[], int stride_levels, int proc);
\end{verbatim}
Transfer strided data from the local process memory (source) into
remote memory of process id {\tt proc}. {\tt src\_ptr} points to the
first memory segment in local process memory. {\tt dst\_ptr} is a
remote memory address that points to the first memory segment in the
memory of process {\tt proc}. {\tt stride\_levels} represents the
number of additional dimensions of striding beyond 1. {\tt
src\_stride\_ar} is an array of size {\tt stride\_levels} whose values
indicate the number of bytes to skip on the local process memory
layout. {\tt dst\_stride\_ar} is an array of size {\tt stride\_levels}
whose values indicate the number of bytes to skip on process {\tt
proc}'s memory layout. {\tt count} is an array of size {\tt
stride\_levels + 1} whose values indicate the number of bytes to copy.

As an example, assume two 2-dimensional C arrays residing on different
processes.

\begin{verbatim}
          double A[10][20]; /* local process */
          double B[20][30]; /* remote process */
\end{verbatim}

To put a block of data of 3x6 doubles starting at location (1,2) in
{\tt A} into location (3,4) in {\tt B}, the arguments to {\tt
ARMCI\_PutS} will be as follows (assuming C/C++ memory layout):

\begin{verbatim}
          src_ptr = &A[0][0] + (1 * 20 + 2); /* location (1,2) */
          src_stride_ar[0] = 20 * sizeof(double);
          dst_ptr = &B[0][0] + (3 * 30 + 4); /* location (3,4) */
          dst_stride_ar[0] = 30 * sizeof(double);
          count[0] = 6; * sizeof(double); /* contiguous data */
          count[1] = 3; /* number of rows of contiguous data */
          stride_levels = 1;
          proc = <B's id>;
\end{verbatim}

\begin{verbatim}
int ARMCI_NbPutS(void *src_ptr, int src_stride_ar[],
                 void *dst_ptr, int dst_stride_ar[],
                 int count[], int stride_levels, int proc
                 armci_hdl_t *handle);
\end{verbatim}
The non-blocking version of {\tt ARMCI\_PutS}. Passing a {\tt NULL}
value to {\tt handle} makes this function perform an implicit handle
non-blocking transfer.

\begin{verbatim}
int ARMCI_Get(void *src, void *dst, int bytes, int proc);
\end{verbatim}
Transfer contiguous data of size {\tt bytes} from the remote process
memory at process {\tt proc} (source) pointed to by {\tt src} into the
local memory of the calling process pointed to by {\tt dst}.

\begin{verbatim}
int ARMCI_NbGet(void *src, void *dst, int bytes, int proc,
                armci_hdl_t *handle);
\end{verbatim}
The non-blocking version of {\tt ARMCI\_Get}. Passing a {\tt NULL}
value to {\tt handle} makes this function perform an implicit handle
non-blocking transfer.

\begin{verbatim}
int ARMCI_GetS(void *src_ptr, int src_stride_ar[],
               void* dst_ptr, int dst_stride_ar[],
               int count[], int stride_levels, int proc);
\end{verbatim}
Transfer strided data segments from remote process memory on process
{\tt proc} to the local memory of the calling process. The semantics
of the parameters to this function are the same as that for {\tt
ARMCI\_PutS}.

\begin{verbatim}
int ARMCI_NbGetS(void *src_ptr, int src_stride_ar[],
                 void* dst_ptr, int dst_stride_ar[],
                 int count[], int stride_levels, int proc,
                 armci_hdl_t *handle);
\end{verbatim}
The non-blocking version of {\tt ARMCI\_GetS}. Passing a {\tt NULL}
value to {\tt handle} makes this function perform an implicit handle
non-blocking transfer.

\subsection{Explicit Synchronization}

\begin{verbatim}
int ARMCI_Wait(armci_hdl_t *handle);
int ARMCI_WaitProc(int proc);
int ARMCI_WaitAll();
int ARMCI_Test(armci_hdl_t *handle);
int ARMCI_Barrier();
\end{verbatim}

\begin{verbatim}
int ARMCI_Fence(int proc);
\end{verbatim}
Blocks the calling process until all {\em put} or {\em accumulate}
operations the process issued to the remote process {\tt proc} are
completed at the destination.

\begin{verbatim}
int ARMCI_AllFence(void);
\end{verbatim}
Blocks the calling process until all outstanding {\em put} or {\em
accumulate} operations it issued are completed on all remote
destinations.

\subsection{Extensions to the Standard API}
\label{sec::extensions}

\begin{verbatim}
void ARMCI_Migrate(void);
void ARMCI_Async_Migrate(void);
void ARMCI_Checkpoint(char* dirname);
void ARMCI_MemCheckpoint(void);

int armci_notify(int proc);
int armci_notify_wait(int proc, int *pval);
\end{verbatim}

\section{List of Unimplemented Functions}

The following functions are supported on the standard ARMCI
implementation but not yet supported in the \charmpp{} port.

\begin{verbatim}
int ARMCI_GetV(...);
int ARMCI_NbGetV(...);
int ARMCI_PutV(...);
int ARMCI_NbPutV(...);
int ARMCI_AccV(...);
int ARMCI_NbAccV(...);

int ARMCI_Acc(...);
int ARMCI_NbAcc(...);
int ARMCI_AccS(...);
int ARMCI_NbAccS(...);

int ARMCI_PutValueLong(long src, void* dst, int proc);
int ARMCI_PutValueInt(int src, void* dst, int proc);
int ARMCI_PutValueFloat(float src, void* dst, int proc);
int ARMCI_PutValueDouble(double src, void* dst, int proc);
int ARMCI_NbPutValueLong(long src, void* dst, int proc, armci_hdl_t* handle);
int ARMCI_NbPutValueInt(int src, void* dst, int proc, armci_hdl_t* handle);
int ARMCI_NbPutValueFloat(float src, void* dst, int proc, armci_hdl_t* handle);
int ARMCI_NbPutValueDouble(double src, void* dst, int proc, armci_hdl_t* handle);
long ARMCI_GetValueLong(void *src, int proc);
int ARMCI_GetValueInt(void *src, int proc);
float ARMCI_GetValueFloat(void *src, int proc);
double ARMCI_GetValueDouble(void *src, int proc);

void ARMCI_SET_AGGREGATE_HANDLE (armci_hdl_t* handle);
void ARMCI_UNSET_AGGREGATE_HANDLE (armci_hdl_t* handle);

int ARMCI_Rmw(int op, int *ploc, int *prem, int extra, int proc);
int ARMCI_Create_mutexes(int num);
int ARMCI_Destroy_mutexes(void);
void ARMCI_Lock(int mutex, int proc);
void ARMCI_Unlock(int mutex, int proc);
\end{verbatim}

\section{Error Codes}
\label{sec::error codes}

As of this writing, attempts to locate the documented error codes have
failed because the release notes have not been found. Attempts are
being made to derive these from the ARMCI source directly. Currently
\charmpp{} implementation does not implement any error codes.

\section{Related Manuals and Documents}
\label{sec::related doc}

\noindent
ARMCI website:
\begin{verbatim}
http://www.emsl.pnl.gov/docs/parsoft/armci/index.html
\end{verbatim}

\end{document}
