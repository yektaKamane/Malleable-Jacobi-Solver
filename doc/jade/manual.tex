\documentclass[10pt]{article}
\usepackage{../pplmanual}
\NeedsTeXFormat{LaTeX2e}
\typeout{^^J^^J
Parallel Programming Laboratory^^J
Manual Style^^J
Written by Milind A. Bhandarkar, 12/00^^J}

%%% Make it possible for both ps and pdf to be generated
\newif\ifpdf
\ifx\pdfoutput\undefined
  \pdffalse
\else
  \pdfoutput=1
  \pdftrue
\fi

\ifpdf
  \pdfcompresslevel=9
\fi

%%% Imported from fullpage.sty, since it is not always available
\topmargin 0pt
\advance \topmargin by -\headheight
\advance \topmargin by -\headsep

\textheight 8.9in

\oddsidemargin 0pt
\evensidemargin \oddsidemargin
\marginparwidth 1.0in

\textwidth 6.5in
%%% end import from fullpage

%%% Commonly Needed packages
\usepackage{graphicx,color,calc}
\usepackage{makeidx}
\usepackage{alltt}

%%% Commands for uniform looks of C++, Charm++, and Projections
\newcommand{\CC}{C\kern -0.0em\raise 0.5ex\hbox{\normalsize++}}
\newcommand{\emCC}{C\kern -0.0em\raise 0.4ex\hbox{\normalsize\em++}}
\newcommand{\charmpp}{\sc Charm++}
\newcommand{\projections}{\sc Projections}
\newcommand{\converse}{\sc Converse}
\newcommand{\ampi}{\sc AMPI}

%%% Commands to produce margin symbols
\newcommand{\new}{\marginpar{\fbox{\bf$\mathcal{NEW}$}}}
\newcommand{\important}{\marginpar{\fbox{\bf\Huge !}}}
\newcommand{\experimental}{\marginpar{\fbox{\bf\Huge $\beta$}}}

%%% Commands for manual elements
\newcommand{\zap}[1]{ }
\newcommand{\function}[1]{{\noindent{\textsf{#1}}\\}}
\newcommand{\cmd}[1]{{\noindent{\textsf{#1}}\\}}
\newcommand{\args}[1]{\hspace*{2em}{\texttt{#1}}\\}
\newcommand{\param}[1]{{\texttt{#1}}}
\newcommand{\kw}[1]{{\textsf{#1}}}
\newcommand{\uw}[1]{{\textsl{#1}}}
\newcommand{\desc}[1]{\indent{#1}}

%%% Commands needed for Maketitle
\newcommand{\@version}{}
\newcommand{\@credits}{}
\newcommand{\version}[1]{\renewcommand{\@version}{#1}}
\newcommand{\credits}[1]{\renewcommand{\@credits}{#1}}

%%% Print the License Page
\newcommand{\@license}{%
 \begin{center}
   {University of Illinois}\\
   {\charmpp/\converse\ Parallel Programming System Software}\\
   {Non-Exclusive, Non-Commercial Use License}\\
 \end{center}
 \rule{\textwidth}{1pt}
{\tiny
Upon execution of this Agreement by the party identified below (``Licensee''),
The Board of Trustees of the University of Illinois  (``Illinois''), on behalf
of The Parallel Programming Laboratory (``PPL'') in the Department of Computer
Science, will provide the \charmpp/\converse\ Parallel Programming System
software (``\charmpp'') in Binary Code and/or Source Code form (``Software'')
to Licensee, subject to the following terms and conditions. For purposes of
this Agreement, Binary Code is the compiled code, which is ready to run on
Licensee's computer.  Source code consists of a set of files which contain the
actual program commands that are compiled to form the Binary Code.

\begin{enumerate}
  \item
    The Software is intellectual property owned by Illinois, and all right,
title and interest, including copyright, remain with Illinois.  Illinois
grants, and Licensee hereby accepts, a restricted, non-exclusive,
non-transferable license to use the Software for academic, research and
internal business purposes only, e.g. not for commercial use (see Clause 7
below), without a fee.

  \item 
    Licensee may, at its own expense, create and freely distribute
complimentary works that interoperate with the Software, directing others to
the PPL server (\texttt{http://charm.cs.uiuc.edu}) to license and obtain the
Software itself. Licensee may, at its own expense, modify the Software to make
derivative works.  Except as explicitly provided below, this License shall
apply to any derivative work as it does to the original Software distributed by
Illinois.  Any derivative work should be clearly marked and renamed to notify
users that it is a modified version and not the original Software distributed
by Illinois.  Licensee agrees to reproduce the copyright notice and other
proprietary markings on any derivative work and to include in the documentation
of such work the acknowledgement:

\begin{quote}
``This software includes code developed by the Parallel Programming Laboratory
in the Department of Computer Science at the University of Illinois at
Urbana-Champaign.''
\end{quote}

Licensee may redistribute without restriction works with up to 1/2 of their
non-comment source code derived from at most 1/10 of the non-comment source
code developed by Illinois and contained in the Software, provided that the
above directions for notice and acknowledgement are observed.  Any other
distribution of the Software or any derivative work requires a separate license
with Illinois.  Licensee may contact Illinois (\texttt{kale@cs.uiuc.edu}) to
negotiate an appropriate license for such distribution.

  \item
    Except as expressly set forth in this Agreement, THIS SOFTWARE IS PROVIDED
``AS IS'' AND ILLINOIS MAKES NO REPRESENTATIONS AND EXTENDS NO WARRANTIES OF
ANY KIND, EITHER EXPRESS OR IMPLIED, INCLUDING BUT NOT LIMITED TO WARRANTIES OR
MERCHANTABILITY OR FITNESS FOR A PARTICULAR PURPOSE, OR THAT THE USE OF THE
SOFTWARE WILL NOT INFRINGE ANY PATENT, TRADEMARK, OR OTHER RIGHTS.  LICENSEE
ASSUMES THE ENTIRE RISK AS TO THE RESULTS AND PERFORMANCE OF THE SOFTWARE
AND/OR ASSOCIATED MATERIALS.  LICENSEE AGREES THAT UNIVERSITY SHALL NOT BE HELD
LIABLE FOR ANY DIRECT, INDIRECT, CONSEQUENTIAL, OR INCIDENTAL DAMAGES WITH
RESPECT TO ANY CLAIM BY LICENSEE OR ANY THIRD PARTY ON ACCOUNT OF OR ARISING
FROM THIS AGREEMENT OR USE OF THE SOFTWARE AND/OR ASSOCIATED MATERIALS.

  \item 
    Licensee understands the Software is proprietary to Illinois. Licensee
agrees to take all reasonable steps to insure that the Software is  protected
and secured from unauthorized disclosure, use, or release and  will treat it
with at least the same level of care as Licensee would use to  protect and
secure its own proprietary computer programs and/or information, but using no
less than a reasonable standard of care.  Licensee agrees to provide the
Software only to any other person or entity who has registered with Illinois.
If licensee is not registering as an individual but as an institution or
corporation each member of the institution or corporation who has access to or
uses Software must agree to and abide by the terms of this license. If Licensee
becomes aware of any unauthorized licensing, copying or use of the Software,
Licensee shall promptly notify Illinois in writing. Licensee expressly agrees
to use the Software only in the manner and for the specific uses authorized in
this Agreement.

  \item
    By using or copying this Software, Licensee agrees to abide by the
copyright law and all other applicable laws of the U.S. including, but not
limited to, export control laws and the terms of this license. Illinois  shall
have the right to terminate this license immediately by written  notice upon
Licensee's breach of, or non-compliance with, any terms of the license.
Licensee may be held legally responsible for any  copyright infringement that
is caused or encouraged by its failure to  abide by the terms of this license.
Upon termination, Licensee agrees to  destroy all copies of the Software in its
possession and to verify such  destruction in writing.

  \item
  The user agrees that any reports or published results obtained with  the
Software will acknowledge its use by the appropriate citation as  follows:

\begin{quote}
``\charmpp/\converse\ was developed by the Parallel Programming Laboratory in
the Department of Computer Science at the University of  Illinois at
Urbana-Champaign.''
\end{quote}

Any published work which utilizes \charmpp\ shall include the following
reference:

\begin{quote}
``L. V. Kale and S. Krishnan. \charmpp: Parallel Programming with Message-Driven
Objects. In 'Parallel Programming using \CC' (Eds. Gregory V. Wilson and Paul
Lu), pp 175-213, MIT Press, 1996.''
\end{quote}

Any published work which utilizes \converse\ shall include the following
reference:

\begin{quote}
``L. V. Kale, Milind Bhandarkar, Narain Jagathesan, Sanjeev Krishnan and Joshua
Yelon. \converse: An Interoperable Framework for Parallel Programming.
Proceedings of the 10th International Parallel Processing Symposium, pp
212-217, April 1996.''
\end{quote}

Electronic documents will include a direct link to the official \charmpp\ page
at \texttt{http://charm.cs.uiuc.edu/}

  \item
    Commercial use of the Software, or derivative works based thereon,
REQUIRES A COMMERCIAL LICENSE.  Should Licensee wish to make commercial use of
the Software, Licensee will contact Illinois (kale@cs.uiuc.edu) to negotiate an
appropriate license for such use. Commercial use includes: 

    \begin{enumerate}
      \item
	integration of all or part of the Software into a product for sale,
lease or license by or on behalf of Licensee to third parties, or 

      \item
	distribution of the Software to third parties that need it to
commercialize product sold or licensed by or on behalf of Licensee.
    \end{enumerate}

  \item
    Government Rights. Because substantial governmental funds have been  used
in the development of \charmpp/\converse, any possession, use or sublicense of
the Software by or to the United States government shall be subject to such
required restrictions.

  \item
    \charmpp/\converse\ is being distributed as a research and teaching tool
and as such, PPL encourages contributions from users of the code that might, at
Illinois' sole discretion, be used or incorporated to make the basic  operating
framework of the Software a more stable, flexible, and/or useful  product.
Licensees who contribute their code to become an internal  portion of the
Software agree that such code may be distributed by  Illinois under the terms
of this License and may be required to sign an  ``Agreement Regarding
Contributory Code for \charmpp/\converse\ Software'' before Illinois  can
accept it (contact \texttt{kale@cs.uiuc.edu} for a copy).
\end{enumerate}

UNDERSTOOD AND AGREED.

Contact Information:

The best contact path for licensing issues is by e-mail to
\texttt{kale@cs.uiuc.edu} or send correspondence to:

\begin{quote}
Prof. L. V. Kale\\
Dept. of Computer Science\\
University of Illinois\\
1304 W. Springfield Ave\\
Urbana, Illinois 61801 USA\\
FAX: (217) 333-3501
\end{quote}
}%tiny
 \newpage
}% end of license

\renewcommand{\maketitle}{\begin{titlepage}%
 \begin{flushright}
   {\Large
     Parallel Programming Laboratory\\
     University of Illinois at Urbana-Champaign\\
   }
 \end{flushright}
 \rule{\textwidth}{3pt}
 \vspace{\fill}
 \begin{flushright}
   \textsf{\Huge \@title \\}
 \end{flushright}
 \vspace{\fill}
 \@credits \\
 \rule{\textwidth}{3pt}
 \begin{flushright}
   {\large Version \@version}
 \end{flushright}
 \end{titlepage}
 \@license

 \tableofcontents
 \newpage
}% maketitle




\title{Jade Language Manual}
\version{1.0}
\credits{
Jade was developed by Jayant DeSouza.
}

\begin{document}
\maketitle

\section{Introduction}

This manual describes \jade, which is a new parallel programming language
developed over \charmpp{} and Java. \charmpp{} is a
\CC{}-based parallel programming library developed by Prof. L. V. Kal\'{e} 
and his students over the last 10 years at University of Illinois.

We first describe our philosophy behind this work (why we do what we do).
Later we give a brief introduction to \charmpp{} and rationale for \jade. We
describe \jade in detail. Appendices contain the details of installing
\jade, building and running \jade programs.

\subsection{Our Philosophy}

\subsection{Terminology}

\begin{description}

\item[Module] A module refers to 

\item[Thread] A thread is a lightweight process that owns a stack and machine
registers including program counter, but shares code and data with other
threads within the same address space. If the underlying operating system
recognizes a thread, it is known as kernel thread, otherwise it is known as
user-thread. A context-switch between threads refers to suspending one thread's
execution and transferring control to another thread. Kernel threads typically
have higher context switching costs than user-threads because of operating
system overheads. The policy implemented by the underlying system for
transferring control between threads is known as thread scheduling policy.
Scheduling policy for kernel threads is determined by the operating system, and
is often more inflexible than user-threads. Scheduling policy is said to be
non-preemptive if a context-switch occurs only when the currently running
thread willingly asks to be suspended, otherwise it is said to be preemptive.
\jade threads are non-preemptive user-level threads.

\item[Object] An object is just a blob of memory on which certain computations
can be performed. The memory is referred to as an object's state, and the set
of computations that can be performed on the object is called the interface of
the object.

\end{description}

\section{\charmpp{}}

\charmpp{} is an object-oriented parallel programming library for \CC{}.  It
differs from traditional message passing programming libraries (such as MPI) in
that \charmpp{} is ``message-driven''. Message-driven parallel programs do not
block the processor waiting for a message to be received.  Instead, each
message carries with itself a computation that the processor performs on
arrival of that message. The underlying runtime system of \charmpp{} is called
\converse{}, which implements a ``scheduler'' that chooses which message to
schedule next (message-scheduling in \charmpp{} involves locating the object
for which the message is intended, and executing the computation specified in
the incoming message on that object). A parallel object in \charmpp{} is a
\CC{} object on which a certain computations can be asked to performed from
remote processors.

\charmpp{} programs exhibit latency tolerance since the scheduler always picks
up the next available message rather than waiting for a particular message to
arrive.  They also tend to be modular, because of their object-based nature.
Most importantly, \charmpp{} programs can be \emph{dynamically load balanced},
because the messages are directed at objects and not at processors; thus
allowing the runtime system to migrate the objects from heavily loaded
processors to lightly loaded processors. It is this feature of \charmpp{} that
we utilize for \jade.

Since many CSE applications are originally written using MPI, one would have to
do a complete rewrite if they were to be converted to \charmpp{} to take
advantage of dynamic load balancing. This is indeed impractical. However,
\converse{} -- the runtime system of \charmpp{} -- came to our rescue here,
since it supports interoperability between different parallel programming
paradigms such as parallel objects and threads. Using this feature, we
developed \jade, an implementation of a significant subset of MPI-1.1
standard over \charmpp{}.  \jade is described in the next section.

\section{\jade}

Every mainchare's main is executed at startup.

\subsection{threaded methods}

\begin{alltt}
class C {
    public threaded void start(CProxy_CacheGroup cg) { ... }
}
\end{alltt}

\subsection{readonly}

\begin{alltt}
class C {
    public static readonly CProxy_TheMain mainChare;
    public static int readonly aReadOnly;
}
\end{alltt}

Accessed as C.aReadOnly;

Must be initialized in the main of a mainchare.  Value at the end of main is
propagated to all processors.  Then execution begins.

\subsection{msa}

\begin{alltt}
arr1.enroll();
int a = arr1[10]; // get
arr1[10] = 122; // set
arr1[10] += 2;  // accumulate
arr1.sync();    // sync
\end{alltt}

\subsection{\jade Status}

\appendix

\section{Installing \jade}

\jade is included in the source distribution of \charmpp{}. 
To get the latest sources from PPL, visit:
	http://charm.cs.uiuc.edu/

and follow the download link.
Now one has to build \charmpp{} and \jade from source.

The build script for \charmpp{} is called \texttt{build}. The syntax for this
script is:

\begin{alltt}
> build <target> <version> <opts>
\end{alltt}

For building \jade (which also includes building \charmpp{} and other
libraries needed by \jade), specify \verb+<target>+ to be \verb+jade+. And
\verb+<opts>+ are command line options passed to the \verb+charmc+ compile
script.  Common compile time options such as \texttt{-g, -O, -Ipath, -Lpath,
-llib} are accepted. 

To build a debugging version of \jade, use the option: ``\texttt{-g}''. 
To build a production version of \jade, use the options: ``\texttt{-O 
-DCMK\_OPTIMIZE=1}''.

\verb+<version>+ depends on the machine, operating system, and the underlying
communication library one wants to use for running \jade programs.
See the charm/README file for details on picking the proper version.
Following is an example of how to build \jade under linux and ethernet
environment, with debugging info produced:

\begin{alltt}
> build jade net-linux -g
\end{alltt}

\section{Compiling and Running \jade Programs}
\subsection{Compiling \jade Programs}

\charmpp{} provides a cross-platform compile-and-link script called \charmc{}
to compile C, \CC{}, Fortran, \charmpp{} and \jade programs.  This script
resides in the \texttt{bin} subdirectory in the \charmpp{} installation
directory. The main purpose of this script is to deal with the differences of
various compiler names and command-line options across various machines on
which \charmpp{} runs.

In spite of the platform-neutral syntax of \charmc{}, one may have to specify
some platform-specific options for compiling and building \jade codes.
Fortunately, if \charmc{} does not recognize any particular options on its
command line, it promptly passes it to all the individual compilers and linkers
it invokes to compile the program.

You can use \charmc{} to build your \jade program the same way as other
compilers like \texttt{cc}.  To build an \jade program, the command line 
option \emph{-language jade} should be specified. All the command line 
flags that you would use for other compilers can be used with \charmc the 
same way. For example:

\begin{alltt}
> charmc -language jade -c pgm.java -O3
> charmc -language jade -o pgm pgm.o -lm -O3 
\end{alltt}

\subsection{Running}

The \charmpp{} distribution contains a script called \texttt{charmrun} that makes
the job of running \jade programs portable and easier across all parallel
machines supported by \charmpp{}. When compiling a \jade program, \charmc{} copies \texttt{charmrun} to the directory
where the \jade program is built.  \texttt{charmrun} takes a command line
parameter specifying the number of processors to run on, and the name of the program
followed by \jade options (such as TBD) and the program arguments. A typical
invocation of \jade program \texttt{pgm} with \texttt{charmrun} is:

\begin{alltt}
> charmrun pgm +p16 +vp32 +tcharm_stacksize 3276800
\end{alltt}

Here, the \jade program \texttt{pgm} is run on 16 physical processors with
32 chunks (which will be mapped 2 per processor initially), where each
user-level thread associated with a chunk has the stack size of 3,276,800 bytes.

\section{Jade Developer documentation}

\subsection{Files}

\jade source files are spread out across several directories of the \charmpp{}
CVS tree.

\begin{tabular}{|r|l|}
\hline\\
charm/doc/jade                         & \jade user documentation files \\
charm/src/langs/jade/                  & ANTLR parser files, \jade runtime library code\\
charm/java/charm/jade/                 & \jade java code \\
charm/java/bin/                        & \jade scripts \\
charm/pgms/jade/                       & \jade example programs and tests \\
\hline
\end{tabular}

After building \jade, files are installed in:

\begin{tabular}{|r|l|}
\hline\\
charm/include/                         & \jade runtime library header files\\
charm/lib/                             & \jade runtime library\\
charm/java/bin/                        & \texttt{jade.jar} file \\
\hline
\end{tabular}

\subsection{Java packages}

The way packages work in Java is as follows: There is a ROOT directory. Within
the ROOT, a subdirectory is used which also gives the package name.  Beneath
the package directory all the \texttt{.class} files are stored.  The ROOT
directory should be placed in the java CLASSPATH.

For \jade, the ROOT is charm/java/charm/.

The \jade package name is \texttt{jade}, and is in charm/java/charm/jade.
Within here, all the jade Java files are placed, they are compiled to
\texttt{.class} files, and then jar'd up into the \texttt{jade.jar} file, which
is placed in charm/java/bin for convenience.

\end{document}
