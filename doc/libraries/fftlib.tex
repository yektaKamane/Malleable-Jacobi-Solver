The previous 3D FFT library has been deprecated and replaced with this new 3D FFT library.
The new 3D FFT library source can be downloaded with following command:
\textit{git clone https://charm.cs.illinois.edu/gerrit/libs/fft}

\section{Introduction and Motivation}
The 3D Charm-FFT library provides an interface to do parallel 3D FFT computation
in a scalable fashion.

\noindent The parallelization is achieved by splitting the 3D transform into
three phases, using 2D decomposition. First, 1D FFTs are computed over
the pencils; then a 'transform' is performed and 1D FFTs are done over
second dimension; again a 'transform' is performed and FFTs are computed
over the last dimension. So this approach takes three computation phases
and two 'transform' phases.

\noindent This library allows users to create multiple instances of the library
and perform concurrent FFTs using them. Each of the FFT instances run in
background as other parts of user code execute, and a callback is invoked
when FFT is complete.

\section{Features}
Charm-FFT library provides the following features:

\begin{itemize}
\item \textit{2D-decomposition}: Users can define fine-grained 2D-decomposition
that increases the amount of available parallelism and improves network utilization.
\item \textit{Cutoff-based smaller grid}: The data grid may have a cut off.
Charm-FFT improves performance by avoiding communication and computation of the data
beyond the cutoff.
\item \textit{User-defined mapping of library objects}: The placement of objects
that constitute the library instance can be defined by the user based on the application's
other concurrent communication and placement of other objects.
\item \textit{Overlap with other computational work}: Given the callback-based interface
and Charm++'s asynchrony, the FFTs are performed in the background while other application
work can be done in parallel.
\end{itemize}

\section{Compilation and Execution}
To install the FFT library, you will need to have charm++ installed in
you system. You can follow the Charm++ manual to do that. Then, ensure
that FFTW3 is installed.
FFTW3 can be downloaded from \textit{http://www.fftw.org}.
\
The Charm-FFT library source can be downloaded with following command:
\textit{git clone https://charm.cs.illinois.edu/gerrit/libs/fft}

\noindent Inside of Charm-FFT directory, you will find \textit{Makefile.default}. Copy
this file to \textit{Makefile.common}, change the copy's variable
\textit{FFT3\_HOME} to point your FFTW3 installation and \textit{CHARM\_DIR}
to point your Charm++ installation then run \textit{make}.
\
To use Charm-FFT library in an application, add the line
\textit{extern module fft\_Charm;} to it charm interface (.ci) file and include
\textit{fft\_charm.h} and \textit{fftw3.h} in relevant C files. Finally to
compile the program, pass \textit{-lfft\_charm} and {-lfftw3} as arguments to
\textit{charmc}.

\section{Library Interface}
To use Charm-FFT interface, the user must start by calling
\textit{Charm\_createFFT} with following parameters.

\begin{alltt}
    Charm\_createFFT(N\_x, N\_y, N\_z, z\_x, z\_y, y\_x, y\_z, x\_yz, cutoff, hmati, fft\_type, CkCallback);

    Where:
    int N\_x : X dimension of FFT calculation
    int N\_y : Y dimension of FFT calculation
    int N\_z : Z dimension of FFT calculation
    int z\_x : X dimension of Z pencil chare array
    int z\_y : Y dimension of Z pencil chare array
    int y\_x : X dimension of Y pencil chare array
    int y\_z : Z dimension of Y pencil chare array
    int x\_yz: A dimension of X pencil chare array
    double cutoff: Cutoff of FFT grid
    double *hmati: Hamiltonian matrix representing cutoff
    FFT\_TYPE: Type of FFT to perform. Either \textit{CC} for complex-to-complex or \textit{RC} for real-complex
    CkCallback: A Charm++ entry method for callback upon the completion of library initialization
\end{alltt}

This creates necessary proxies (Z,Y,X etc) for performing FFT of size $N_x \times N_y * N_z$ using 2D chare
arrays (pencils) of size $n_y \times n_x$ (ZPencils), $n_z \times n_x$ (YPencils), and $n_x \times n_y$ (XPencils).
When done, calls $myCallback$ which should receive $CProxy\_fft2d\ id$ as a unique identifier for the
newly created set of proxies.

An example of Charm-FFT initialization using Charm\_createFFT:
\begin{alltt}
// .ci

    extern module fft\_charm;

    mainchare Main \{
        entry Main(CkArgMsg *m);
    \}

    group Driver \{
        entry Driver(FFT\_Type fft\_type);
        entry void proxyCreated(idMsg *msg);
        entry void fftDone();
    \}

// .C

    Main::Main(CkArgMsg *m) \{

        ...
        /* Assume FFT of size N\_x, N\_y, N\_z */
        FFT\_Type fft\_type = CC

        Charm\_createFFT(N\_x, N\_y, N\_z, z\_x, z\_y, y\_x, y\_z, x\_yz, cutoff, hmati
, fft\_type, CkCallback(CkIndex\_Driver::proxyCreated(NULL), driverProxy));
    \}

    Driver::proxyCreated(idMsg *msg) \{

        CProxy\_fft2d fftProxy = msg-> id;
        delete msg;
    \}
\end{alltt}
In this example, an entry method \textit{Driver::proxyCreated} will be called
when an FFT instance has been created.

Using the newly received proxy, the user can identify whether a local PE has
XPencils and/or ZPencils.

\begin{alltt}
    void Driver::proxyCreated(idMsg *msg) \{
      CProxy\_fft2d fftProxy = msg->id;

      delete msg;

      bool hasX = Charm\_isOutputPE(fftProxy),
           hasZ = Charm\_isInputPE(fftProxy);

      ...
    \}
\end{alltt}

\noindent Then, the grid's dimensions on a PE can be acquired by using
\textit{Charm\_getOutputExtents} and \textit{Charm\_getInputExtents}.

\begin{alltt}
    if (hasX) \{
      Charm\_getOutputExtents(gridStart[MY\_X], gridEnd[MY\_X],
                            gridStart[MY\_Y], gridEnd[MY\_Y],
                            gridStart[MY\_Z], gridEnd[MY\_Z],
                            fftProxy);
    \}

    if (hasZ) \{
      Charm\_getInputExtents(gridStart[MY\_X], gridEnd[MY\_X],
                            gridStart[MY\_Y], gridEnd[MY\_Y],
                            gridStart[MY\_Z], gridEnd[MY\_Z],
                            fftProxy);
    \}

    for(int i = 0; i < 3; i++) \{
      gridLength[i] = gridEnd[i] - gridStart[i];
    \}
\end{alltt}

\noindent With the grid's dimension, the user must allocate and set the input and
output buffers. In most cases, this is simply the product of the three
dimensions, but for real-to-complex FFT calcaultion, FFTW-style storage for
the input buffers is used (as shown below).

\begin{alltt}
    dataSize = gridLength[MY\_X] * gridLength[MY\_Y] * gridLength[MY\_Z];

    if (hasX) \{
      dataOut = (complex*) fftw\_malloc(dataSize * sizeof(complex));

      Charm\_setOutputMemory((void*) dataOut, fftProxy);
    \}

    if (hasZ) \{
      if (fftType == RC) \{
        // FFTW style storage
        dataSize = gridLength[MY\_X] * gridLength[MY\_Y] * (gridLength[MY\_Z]/2 + 1);
      \}

      dataIn = (complex*) fftw\_malloc(dataSize * sizeof(complex));

      Charm\_setInputMemory((void*) dataIn, fftProxy);
    \}
\end{alltt}

\noindent Then, from \textit{PE0}, start the forward or backward FFT, setting
the entry method \textit{fftDone} as the callback function that will be called
when the FFT operation is complete.

\noindent For forward FFT
\begin{alltt}
    if (CkMyPe() == 0) \{
        Charm\_doForwardFFT(CkCallback(CkIndex\_Driver::fftDone(), thisProxy), fftProxy);
    \}
\end{alltt}

\noindent For backward FFT
\begin{alltt}
    if (CkMyPe() == 0) \{
        Charm\_doBackwardFFT(CkCallback(CkIndex\_Driver::fftDone(), thisProxy), fftProxy);
    \}
\end{alltt}

\noindent The sample program to run a backward FFT can be found in
\textit{Your\_Charm\_FFT\_Path/tests/simple\_tests}

\begin{alltt}
\end{alltt}

