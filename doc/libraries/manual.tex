\documentclass[11pt]{article}

\newif\ifpdf
\ifx\pdfoutout\undefined
  \pdffalse
\else
  \pdfoutput=1
  \pdftrue
\fi

\ifpdf
  \pdfcompresslevel=9
  \usepackage[pdftex,colorlinks=true,plainpages=false]{hyperref}
\else
\fi

\usepackage{fullpage}
\pagestyle{headings}
\setlength{\parskip}{0.1in}
\setlength{\textheight}{9.5in}
\setlength{\textwidth}{6.5in}
\setlength{\parindent}{0in}
\setlength{\topmargin}{-.5in}
\parskip 0.1in

%
% Constants
%
%\newcommand{\version}{1.0}		%%% The current version number
%\newcommand{\prevversion}{}	%%% The previous version number

%
% Commands
%
\newcommand{\zap}[1]{ }
\newcommand{\fcmd}{\bf}		%%% Font for Charm commands
\newcommand{\fparm}{\it\sf}	%%% Font for parameters to Charm commands
\newcommand{\fexec}{\bf}	%%% Font for compile/execute cmds/options
\newcommand{\atitle}[1]{{\it #1}}
\newcommand{\keyword}[1]{{\textbf{#1}}}
\newcommand{\userword}[1]{{\fparm \textsf{#1}}}
\newcommand{\constraint}[1]{Note: {\it #1}}
\newcommand{\note}[1]{Note: {\it #1}}

%
% Conveniences
%
\newcommand{\uw}[1]{\userword{#1}}
\newcommand{\kw}[1]{\keyword{#1}}
\newcommand{\CLocalBranch}{\keyword{CLocalBranch}}

%
%       \CC gives "C++" that looks good.
%
\newcommand{\CC}{C\kern -0.0em\raise 0.5ex\hbox{\normalsize++}}
\newcommand{\emCC}{C\kern -0.0em\raise 0.4ex\hbox{\normalsize\em++}}
\newcommand{\charmpp}{{\sc Charm++}}

\makeindex

\usepackage{alltt}

\begin{document}

\begin{titlepage}
\vspace*{2in}
\Huge
\begin{center}
Converse and \charmpp \\
Libraries
\vspace*{0.5in}
Version 1.0\\
\vspace*{0.7in}
\today
\end{center}
\normalsize
\end{titlepage}

\newpage
\tableofcontents

\newpage
\section{Introduction}
This manual describes \charmpp\ and Converse libraries.\cite{InterOpIPPS96}
This is a work in progress towards a standard library for parallel
programming on top of the Converse and \charmpp\ system.

For a description of the C-based {\sc Charm} parallel programming system,
please refer to the {\sl Charm Programming Language Manual} and the
{\sl Tutorial Introduction to Charm}\footnote{{\sc Charm} is no longer
actively supported and maintained, and these manuals are kept only for
offering the historical perspectives.}.


\newpage
\section{iRecv Library}
\section{Irecv Library}

{\sc Irecv} library provides asynchronous communication mode for chare array. The program breaks into two parts("split phase" structure): nonblocking send, receives and iwait, and callback functions. It provides a style that mpi programmers may find intuitive. 

There are three functions in Irecv library.

1. void send(void *buf, int size, int dest, int tag, int refno);
"buf" is a message buffer containing the data to be sent; "size" is the total size of the message in byte. 
Like in MPI\_send, the tag is used for matching message with dest. The integer "refno" is a reference number, usually the iteration number.
  

2. void irecv(void *buf, int size, int source, int tag, int refno);
irecv will register tags and pointer to the library, that it copy the matching message(with tags) into the location given by the "pointer".

3 void iwaitAll(recvCallBack f, void *data, int refno);
Wait for all the previously issued irecvs to complete, and then it calls the callback function "f" with "data" as argument.

To use Irecv library, create a chare array which is inherited from class "receiver". The sender need to allocate the "receiverMsg" message, fill in fields and invoke 'generic' method in the remote object; The receiver will specify the matching tags and buffer to get the message. After irecv, calling iwaitAll to wait all the receive functions calls complete. However, the iwaitAll is nonblocking and the callback functions will be called after receive calls complete.

Here is an example:

\begin{verbatim}
     int size = 100;
     for (int i=0; i<size; i++) buf[i] = data[i];
     send(buf, size*sizeof(double), neighbor, tag, iter);

     irecv(buf, size*sizeof(double), neighbor, tag, iter);

     iwaitAll(callfunc, this, iter);
\end{verbatim}

and callback function can be declared as:

\begin{verbatim}
     void callfunc(void *obj)
     {
     }
\end{verbatim}


\newpage
\section{FEM Framework}
The Finite Element Method is a commonly used modeling and analysis tool for
predicting behavior of real-world objects with respect to mechanical stresses,
vibrations, heat conduction etc. It works by modeling a structure with a
complex geometry into a number of small ``elements'' connected at ``nodes''.
This process is called ``meshing''. Meshes can be structured or unstructured.
Structured meshes have equal connectivity for each node, while unstructured
meshes have different connectivity.  Unstructured meshes are more commonly used
for FEM.

The FEM solver solves a set of matrix equations that approximate the physical
phenomena under study. For example, in stress analysis, the system of equations
is:

$$
F = k x
$$

where F is the stress, k is the stiffness matrix, and x is the displacement.
The stiffness matrix is formed by superposing such equations for all the
elements. When using FEM for dynamic analysis, these equations are solved in
every iteration, advancing time by a small ``timestep''. In explicit FEM,
forces on elements are calculated from displacements of nodes (strain) in the
previous iteration, and they in turn cause displacements of nodes in the next
iteration.

In order to reduce errors of these approximations, the element size is very
small, resulting in a large number of elements (typically hundreds of
thousands). The sheer amount of computations lends this problem to parallel
computing. Also, since connectivity of elements is small, this results in good
computation to communication ratios. However, the arbitrary connectivity of
unstructured meshes, alongwith dynamic physical phenomena introduce
irregularity and dynamic behavior, which are difficult to handle using
traditional parallel computing methods.

The FEM Framework presented in this section is an attempt to provide the right
infrastructure for programming parallel FEM applications. It is built on top of
\charmpp.

\subsection{Design Priciples}

The basic philosophy of the FEM framework is to hide most details of
parallelization from the FEM application developer. Thus, the FEM application
developer writes sequential code for performing computations over a partition
of the mesh. Partitioning, communication, and synchronization are performed by
the framework.

Computational structure of most FEM codes resembles the following pseudocode:

\begin{alltt}
  read connectivity information from a grid file
  read initial values for nodes
  read boundary conditions
  repeat until termination criterion is satisfied
    for all elements e, do
      for each node n connected to element e, do
        contribute to n some value based solely on e.
      end for
    end for
    for all nodes n, do
      update node values
    end for
  end repeat
  write final values for nodes
\end{alltt}

A node may have different instances of values that are contributed from an
element, such as forces, temperatures, pressures etc. Also, the element and
node loops may be repeated more than once in the program operating on different
value. The termination criterion may be error based, which requires a reduction
across all elements, or all nodes; or it may be based on a fixed number of such
iterations.

FEM framework is developed in \charmpp\ in order to facilitate dynamic
applications, where the runtime, or computational load of each partition may
vary at runtime. When load imbalance occurs, \charmpp\ moves these partitions
automatically in order to rebalance the load.

Most of the existing FEM codes are written using fortran77 or fortran90.  We
provide interefaces for these languages. This avoids having to write the entire
application in a different languages, and makes conversion from exisiting
programs easier.


\subsection{Writing an FEM Application}

\subsection{Compiling and Running}

\subsection{Support for Dynamic Applications}


\nocite{InterOpIPPS96}  % Just put a citation in here, so the makefile 
                        % doesn't complain

\newpage
\addtocontents{toc}{\contentsline {section}{\numberline {}References}{46}}
\bibliographystyle{plain}
\bibliography{group}

\newpage
\addtocontents{toc}{\contentsline {section}{\numberline {}Index}{47}}
\include{index}

\end{document}
