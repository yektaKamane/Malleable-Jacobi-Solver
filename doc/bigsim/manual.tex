\documentclass[10pt]{article}
\usepackage{pplmanual}
\NeedsTeXFormat{LaTeX2e}
\typeout{^^J^^J
Parallel Programming Laboratory^^J
Manual Style^^J
Written by Milind A. Bhandarkar, 12/00^^J}

%%% Make it possible for both ps and pdf to be generated
\newif\ifpdf
\ifx\pdfoutput\undefined
  \pdffalse
\else
  \pdfoutput=1
  \pdftrue
\fi

\ifpdf
  \pdfcompresslevel=9
\fi

%%% Imported from fullpage.sty, since it is not always available
\topmargin 0pt
\advance \topmargin by -\headheight
\advance \topmargin by -\headsep

\textheight 8.9in

\oddsidemargin 0pt
\evensidemargin \oddsidemargin
\marginparwidth 1.0in

\textwidth 6.5in
%%% end import from fullpage

%%% Commonly Needed packages
\usepackage{graphicx,color,calc}
\usepackage{makeidx}
\usepackage{alltt}

%%% Commands for uniform looks of C++, Charm++, and Projections
\newcommand{\CC}{C\kern -0.0em\raise 0.5ex\hbox{\normalsize++}}
\newcommand{\emCC}{C\kern -0.0em\raise 0.4ex\hbox{\normalsize\em++}}
\newcommand{\charmpp}{\sc Charm++}
\newcommand{\projections}{\sc Projections}
\newcommand{\converse}{\sc Converse}
\newcommand{\ampi}{\sc AMPI}

%%% Commands to produce margin symbols
\newcommand{\new}{\marginpar{\fbox{\bf$\mathcal{NEW}$}}}
\newcommand{\important}{\marginpar{\fbox{\bf\Huge !}}}
\newcommand{\experimental}{\marginpar{\fbox{\bf\Huge $\beta$}}}

%%% Commands for manual elements
\newcommand{\zap}[1]{ }
\newcommand{\function}[1]{{\noindent{\textsf{#1}}\\}}
\newcommand{\cmd}[1]{{\noindent{\textsf{#1}}\\}}
\newcommand{\args}[1]{\hspace*{2em}{\texttt{#1}}\\}
\newcommand{\param}[1]{{\texttt{#1}}}
\newcommand{\kw}[1]{{\textsf{#1}}}
\newcommand{\uw}[1]{{\textsl{#1}}}
\newcommand{\desc}[1]{\indent{#1}}

%%% Commands needed for Maketitle
\newcommand{\@version}{}
\newcommand{\@credits}{}
\newcommand{\version}[1]{\renewcommand{\@version}{#1}}
\newcommand{\credits}[1]{\renewcommand{\@credits}{#1}}

%%% Print the License Page
\newcommand{\@license}{%
 \begin{center}
   {University of Illinois}\\
   {\charmpp/\converse\ Parallel Programming System Software}\\
   {Non-Exclusive, Non-Commercial Use License}\\
 \end{center}
 \rule{\textwidth}{1pt}
{\tiny
Upon execution of this Agreement by the party identified below (``Licensee''),
The Board of Trustees of the University of Illinois  (``Illinois''), on behalf
of The Parallel Programming Laboratory (``PPL'') in the Department of Computer
Science, will provide the \charmpp/\converse\ Parallel Programming System
software (``\charmpp'') in Binary Code and/or Source Code form (``Software'')
to Licensee, subject to the following terms and conditions. For purposes of
this Agreement, Binary Code is the compiled code, which is ready to run on
Licensee's computer.  Source code consists of a set of files which contain the
actual program commands that are compiled to form the Binary Code.

\begin{enumerate}
  \item
    The Software is intellectual property owned by Illinois, and all right,
title and interest, including copyright, remain with Illinois.  Illinois
grants, and Licensee hereby accepts, a restricted, non-exclusive,
non-transferable license to use the Software for academic, research and
internal business purposes only, e.g. not for commercial use (see Clause 7
below), without a fee.

  \item 
    Licensee may, at its own expense, create and freely distribute
complimentary works that interoperate with the Software, directing others to
the PPL server (\texttt{http://charm.cs.uiuc.edu}) to license and obtain the
Software itself. Licensee may, at its own expense, modify the Software to make
derivative works.  Except as explicitly provided below, this License shall
apply to any derivative work as it does to the original Software distributed by
Illinois.  Any derivative work should be clearly marked and renamed to notify
users that it is a modified version and not the original Software distributed
by Illinois.  Licensee agrees to reproduce the copyright notice and other
proprietary markings on any derivative work and to include in the documentation
of such work the acknowledgement:

\begin{quote}
``This software includes code developed by the Parallel Programming Laboratory
in the Department of Computer Science at the University of Illinois at
Urbana-Champaign.''
\end{quote}

Licensee may redistribute without restriction works with up to 1/2 of their
non-comment source code derived from at most 1/10 of the non-comment source
code developed by Illinois and contained in the Software, provided that the
above directions for notice and acknowledgement are observed.  Any other
distribution of the Software or any derivative work requires a separate license
with Illinois.  Licensee may contact Illinois (\texttt{kale@cs.uiuc.edu}) to
negotiate an appropriate license for such distribution.

  \item
    Except as expressly set forth in this Agreement, THIS SOFTWARE IS PROVIDED
``AS IS'' AND ILLINOIS MAKES NO REPRESENTATIONS AND EXTENDS NO WARRANTIES OF
ANY KIND, EITHER EXPRESS OR IMPLIED, INCLUDING BUT NOT LIMITED TO WARRANTIES OR
MERCHANTABILITY OR FITNESS FOR A PARTICULAR PURPOSE, OR THAT THE USE OF THE
SOFTWARE WILL NOT INFRINGE ANY PATENT, TRADEMARK, OR OTHER RIGHTS.  LICENSEE
ASSUMES THE ENTIRE RISK AS TO THE RESULTS AND PERFORMANCE OF THE SOFTWARE
AND/OR ASSOCIATED MATERIALS.  LICENSEE AGREES THAT UNIVERSITY SHALL NOT BE HELD
LIABLE FOR ANY DIRECT, INDIRECT, CONSEQUENTIAL, OR INCIDENTAL DAMAGES WITH
RESPECT TO ANY CLAIM BY LICENSEE OR ANY THIRD PARTY ON ACCOUNT OF OR ARISING
FROM THIS AGREEMENT OR USE OF THE SOFTWARE AND/OR ASSOCIATED MATERIALS.

  \item 
    Licensee understands the Software is proprietary to Illinois. Licensee
agrees to take all reasonable steps to insure that the Software is  protected
and secured from unauthorized disclosure, use, or release and  will treat it
with at least the same level of care as Licensee would use to  protect and
secure its own proprietary computer programs and/or information, but using no
less than a reasonable standard of care.  Licensee agrees to provide the
Software only to any other person or entity who has registered with Illinois.
If licensee is not registering as an individual but as an institution or
corporation each member of the institution or corporation who has access to or
uses Software must agree to and abide by the terms of this license. If Licensee
becomes aware of any unauthorized licensing, copying or use of the Software,
Licensee shall promptly notify Illinois in writing. Licensee expressly agrees
to use the Software only in the manner and for the specific uses authorized in
this Agreement.

  \item
    By using or copying this Software, Licensee agrees to abide by the
copyright law and all other applicable laws of the U.S. including, but not
limited to, export control laws and the terms of this license. Illinois  shall
have the right to terminate this license immediately by written  notice upon
Licensee's breach of, or non-compliance with, any terms of the license.
Licensee may be held legally responsible for any  copyright infringement that
is caused or encouraged by its failure to  abide by the terms of this license.
Upon termination, Licensee agrees to  destroy all copies of the Software in its
possession and to verify such  destruction in writing.

  \item
  The user agrees that any reports or published results obtained with  the
Software will acknowledge its use by the appropriate citation as  follows:

\begin{quote}
``\charmpp/\converse\ was developed by the Parallel Programming Laboratory in
the Department of Computer Science at the University of  Illinois at
Urbana-Champaign.''
\end{quote}

Any published work which utilizes \charmpp\ shall include the following
reference:

\begin{quote}
``L. V. Kale and S. Krishnan. \charmpp: Parallel Programming with Message-Driven
Objects. In 'Parallel Programming using \CC' (Eds. Gregory V. Wilson and Paul
Lu), pp 175-213, MIT Press, 1996.''
\end{quote}

Any published work which utilizes \converse\ shall include the following
reference:

\begin{quote}
``L. V. Kale, Milind Bhandarkar, Narain Jagathesan, Sanjeev Krishnan and Joshua
Yelon. \converse: An Interoperable Framework for Parallel Programming.
Proceedings of the 10th International Parallel Processing Symposium, pp
212-217, April 1996.''
\end{quote}

Electronic documents will include a direct link to the official \charmpp\ page
at \texttt{http://charm.cs.uiuc.edu/}

  \item
    Commercial use of the Software, or derivative works based thereon,
REQUIRES A COMMERCIAL LICENSE.  Should Licensee wish to make commercial use of
the Software, Licensee will contact Illinois (kale@cs.uiuc.edu) to negotiate an
appropriate license for such use. Commercial use includes: 

    \begin{enumerate}
      \item
	integration of all or part of the Software into a product for sale,
lease or license by or on behalf of Licensee to third parties, or 

      \item
	distribution of the Software to third parties that need it to
commercialize product sold or licensed by or on behalf of Licensee.
    \end{enumerate}

  \item
    Government Rights. Because substantial governmental funds have been  used
in the development of \charmpp/\converse, any possession, use or sublicense of
the Software by or to the United States government shall be subject to such
required restrictions.

  \item
    \charmpp/\converse\ is being distributed as a research and teaching tool
and as such, PPL encourages contributions from users of the code that might, at
Illinois' sole discretion, be used or incorporated to make the basic  operating
framework of the Software a more stable, flexible, and/or useful  product.
Licensees who contribute their code to become an internal  portion of the
Software agree that such code may be distributed by  Illinois under the terms
of this License and may be required to sign an  ``Agreement Regarding
Contributory Code for \charmpp/\converse\ Software'' before Illinois  can
accept it (contact \texttt{kale@cs.uiuc.edu} for a copy).
\end{enumerate}

UNDERSTOOD AND AGREED.

Contact Information:

The best contact path for licensing issues is by e-mail to
\texttt{kale@cs.uiuc.edu} or send correspondence to:

\begin{quote}
Prof. L. V. Kale\\
Dept. of Computer Science\\
University of Illinois\\
1304 W. Springfield Ave\\
Urbana, Illinois 61801 USA\\
FAX: (217) 333-3501
\end{quote}
}%tiny
 \newpage
}% end of license

\renewcommand{\maketitle}{\begin{titlepage}%
 \begin{flushright}
   {\Large
     Parallel Programming Laboratory\\
     University of Illinois at Urbana-Champaign\\
   }
 \end{flushright}
 \rule{\textwidth}{3pt}
 \vspace{\fill}
 \begin{flushright}
   \textsf{\Huge \@title \\}
 \end{flushright}
 \vspace{\fill}
 \@credits \\
 \rule{\textwidth}{3pt}
 \begin{flushright}
   {\large Version \@version}
 \end{flushright}
 \end{titlepage}
 \@license

 \tableofcontents
 \newpage
}% maketitle



\title{Bluegene Emulator}
\version{0.01}
\credits{Charm++ BlueGene Emulator was developed by Arun Singla, Neelam Saboo
and Joshua Unger under the guidance of Prof. L. V. Kale. The new Converse BlueGe
ne Emulator is completely rewritten by Gengbin Zheng. Converse BlueGene 
Emulator is the only version maintained now.}

\begin{document}
\maketitle

\section{Introduction}

Blue Gene is a proposed one million processor machine from IBM.

The Blue Gene emulator environment is designed with the following
objectives:

\begin{enumerate}
\item To support a realistic Blue Gene API on existing parallel machines

\item To obtain first-order performance estimates of algorithms

\item To facilitate implementations of alternate programming models for
      Blue Gene
\end{enumerate}

The ``Blue Gene'' machine supported by the emulator consists of
three-dimensional grid of 1-chip nodes.  The user may specify the size
of the machine along each dimension (e.g. 34x34x36).  The chip supports
$k$ threads (e.g. 200), each with its own integer unit.  The proximity of
the integer unit with individual memory modules within a chip is not
currently modeled.

\section{History}
 The first version of BlueGene emulator was first written on Charm++, a 
parallel object language. The second version of Blue Gene emulator is now 
completely rewritten on top of Converse instead of Charm++, while the API 
supported by the original emulator is kept without major changes. The new 
emulator is implemented on a lower layer communication library - Converse 
in order to achieve better performance by avoiding the cross layer overhead. 
Switching to Converse Blue Gene emulator allows further porting of Charm++ 
parallel language on the emulator. 
   New features are also added in the Converse Blue Gene emulator including 
supporting thread-committed messages that can be send to a specific thread 
in a Blue Gene node; supporting Bluegene node level broadcast. 

The API supported by the emulator can be broken down into several
components:

\begin{enumerate}
\item Low-level API for chip-to-chip communication
\item Mid-level API that supports local micro-tasking with a chip level
scheduler with features such as: read-only variables, reductions, broadcasts,
distributed tables, get/put operations
\item Migratable objects with automatic load balancing support
\end{enumerate}

Of these, the first two have been implemented.  The simple time stamping
algorithm, without error correction, has been implemented.  More
sophisticated timing algorithms, specifically aimed at error correction,
and more sophisticated features (2, 3, and others), as well as libraries
of commonly needed parallel operations are part of the proposed work for
future.

The following sections define the appropriate parts of the API, with
example programs and instructions for executing them.

\section{Blue Gene Programming Environment}

The basic philosophy of the Blue Gene Emulator is to hide intricate details
of Blue Gene machine from
application developer.Thus, the application developer needs to provide
intialization details and handler
functions only and gets the result as though running on a real machine.
Communication, Thread creation,
Time Stamping, etc are done by the emulator.

\subsection{Blue Gene API: Level 0}

\function{void putMessage(PacketMsg *)}
\desc{
        chip-to-chip communication function, invoked by Blue Gene
        environment when a node calls sendPacket
        to put the message in the inBuffer of target node.
}

\function{boolean checkReady()}
\desc{
        invoked by communication thread to see if there is any unattended
        message in inBuffer.
}

\function{PacketMsg *getMessage()}
\desc{
        invoked by communication thread to retrieve the unattended message
        in inBuffer.
}

\subsection{Initialization API: Level 1a}

Execution starts at BgInit(Main *), where Blue Gene machine parameters are
initialized.

\function{void CreateBlueGene(CreateBgNodeMsg *msg)}
\desc{
Specifies the machines configuration in CreateBgNodeMsg.
}

Data required for initialization of Node is specified in a system
defined type CreateBgNodeMsg.

\begin{alltt}
        class CreateBgNodeMsg
        \{
                public:
                  int numCTh ;
                  int numWTh ;
                  int numBgX ;
                  int numBgY ;
                  int numBgZ ;
        \} ;
\end{alltt}

\function{int  getNumArgs()}
\desc{
Return the number of command line arguments
}

\function{const char** getArgs()}
\desc{
Return command line arguments
}

\function{typedef void (*BgHandler)(void*)}
\desc{
This is type defined in BlueGene.h. It represents a handler function
that returns nothing and takes a (void *)
}

The Runtime system calls BgNodeInit(BgNode *) for each node, where
application handlers are registered and
computation is triggered by creating a task at required nodes. The Node
Variables are encapsulated in a struct
definition and a pointer to this (node private variable) is returned.

\function{void  registerHandler(int handlerID, BgHandler h)}
\desc{
Register a Handler with each node
}

\function{void addMessage(PacketMsg *msgPtr, int handlerID, int threadCategory)}
\desc{
Create a micro-task, specifiy the handler function to be
used for this message i.e. handlerID, and
specify the thread category:
\begin{description}
\item[1:] a small piece of work that can be done by
communication thread itself, so NO scheduling overhead.
\item[0:] a large piece of work, so communication thread
schedules it for a worker thread
\end{description}
}

InterNode communication messages are to be inherited from PacketMsg

\begin{alltt}
        class PacketMsg
        \{
            public:
                  int srcX ;
                  int srcY ;
                  int srcZ ;
                  int destX ;
                  int destY ;
                  int destZ ;
                  int numBytes ;
                  double sendTime ;
                  double recvTime ;
        \} ;
\end{alltt}

After completion of execution, Blue Gene environment invokes a user defined
function BgFinish().

\subsection{Handler Function API: Level 1a}

\function{void sendPacket(int x, int y, int z, PacketMsg *msgPtr, int handlerID, int threadCategory)}
\desc{
Sends a PacketMsg pointer to Node[x,y,z] and also specifies the
handler function to be used for this message ie. handlerID,
specify the thread category:
\begin{description}
\item[1:] a small piece of work that can be done by
communication thread itself, so NO scheduling overhead.
\item[0:] a large piece of work, so communication thread
schedules it for a worker thread
\end{description}
}

\function{void getXYZ(int\& x, int\& y, int\& z)}
\desc{
Gets which Blue Gene node do the invoking thread belongs to
}

\function{double getTime()}
\desc{
Returns current time of thread in microseconds (execution time since
the application started)
}


\section{Writing a Blue Gene Application}

\subsection{Application Skeleton}

\begin{alltt}
Handler declarations
Application specific messages, inherited from PacketMsg
Struct definitions encapsulating Node (specific) variables

void  BgInit(Main *)  function
  Make a Blue Gene node creation message, CreateBgNodeMsg
  Initialize number of Communication Threads and 
    number of Worker Threads per node
  Initialize number of Blue Gene nodes in X, Y, and Z dimension.
  Initialize the emulator by calling CreateBlueGene(CreateBgNodeMsg)

void *BgNodeInit(BgNode *) function
  Register handlers, registerHandler(handlerID, handler)
  Send initialization message packets to node, 
    addMessage(messagePointer, handlerID)
  Declare Node Variables (struct) and return a pointer to it.

void  BgFinish()  function
  detects Quiescence

Handler Function 1, void handlerName(ThreadInfo *info)
Hanlder Function 2, void handlerName(ThreadInfo *info)
..
Handler Function N, void handlerName(ThreadInfo *info)

\end{alltt}

\subsection{Sample Application 1}

\begin{alltt}
/* Application: 
 *   Each node starting at [0,0,0] sends a packet to next node in
 *   the ring order.
 *   After node [0,0,0] gets message from last node
 *   in the ring, the application ends.
 */

#include "BlueGene.h"
#define  computeID 1

extern "C" void compute(ThreadInfo *) ;

class MyMsg : public PacketMsg
\{
public:
  int dummy ;
\} ;


void BgInit(Main *main)
\{
  int num_comm = 1, num_work = 2;       // default number of communication
                                        // and worker threads per node
  int num_args = main->getNumArgs();
  if (num_args < 4) \{ 
    // Abort application: insufficient number of arguments
    CkAbort("Usage: ring <x> <y> <z> [<numCommTh> <numWorkTh>]\verb+\n+"); 
  \}

  if (num_args > 5) \{ num_work = atoi(main->getArgs()[5]); \}
  if (num_args > 4) \{ num_comm = atoi(main->getArgs()[4]); \}

  CreateBgNodeMsg *bgNodeMsg = new CreateBgNodeMsg;
  bgNodeMsg->numBgX = atoi(main->getArgs()[1]);
  bgNodeMsg->numBgY = atoi(main->getArgs()[2]);
  bgNodeMsg->numBgZ = atoi(main->getArgs()[3]);
  bgNodeMsg->numCTh = num_comm;
  bgNodeMsg->numWTh = num_work;

  main->CreateBlueGene(bgNodeMsg);
  return;
\}

void* BgNodeInit(BgNode *bgNode)
\{
  bgNode->registerHandler(computeID, compute) ;

  // trigger computation at Node[0,0,0]
  if(bgNode->thisIndex.x==0 \&\& bgNode->thisIndex.y==0 \&\& bgNode->thisIndex.z==0)
  \{
   MyMsg *msg = new MyMsg ;
   msg->dummy = 0 ;
   bgNode->addMessage(msg, computeID, 0) ;
  \}
  return NULL;    // No node variables for this example
\}

void BgFinish() \{\}

void compute(ThreadInfo *info)
\{
  int i, j, k ;
  int ni, nj, nk ;

  info->bgNode->getXYZ(i,j,k) ;

  if(i==info->bgNode->numBgX-1 \&\& j==info->bgNode->numBgY-1 \&\& 
     k==info->bgNode->numBgZ-1)
  \{
           ckout << "Exiting" << endl ;
           info->bgNode->finish() ;
           return ;
  \}

  nk = k + 1 ;
  nj = j ;
  ni = i ;
  if( nk==info->bgNode->numBgZ )
  \{
          nk = 0 ;
         nj = j+1 ;
         if ( nj==info->bgNode->numBgY )
         \{
                nj = 0 ;
                ni = i+1 ;
                if ( ni==info->bgNode->numBgX ) \{
                    ni = 0 ;
                \}
         \}
  \}
  MyMsg *msg = new MyMsg ;
  msg->dummy = i+j+k ;
  info->bgNode->sendPacket(ni, nj, nk, msg, computeID, 0) ;
\}
\end{alltt}


\subsection{Sample Application 2}

\begin{alltt}

/* Application: 
 *   Find the maximum element.
 *   Each node computes maximum of it's elements and
 *   the max values it received from other nodes
 *   and sends the result to next node in the reduction sequence.
 * Reduction Sequence: Reduce max data to X-Y Plane
 *   Reduce max data to Y Axis
 *   Reduce max data to origin.
 */

#include "BlueGene.h"

#define A_SIZE 4

#define reduceID                    1
#define computeMaxID        2
#define contributeID             3

//handler declarations
void reduce(ThreadInfo *) ;
void computeMax(ThreadInfo *) ;
void contribute(ThreadInfo *) ;

//Application specifice messages inherited from PacketMsg
class contributeMsg: public PacketMsg
\{
public:
  int max ;
\} ;

class computeMaxMsg: public PacketMsg
\{\} ;

class reduceMsg: public PacketMsg
\{\} ;

//Node variables encapsulated in a struct definition
typedef struct userDataStruct
\{
  int data[A_SIZE] ;
  int count ;
\} userData ;

void BgInit(Main *main)
\{
  int num_comm = 1, num_work = 2;       // default number of communication
and worker threads
  int num_args = main->getNumArgs();
  if (num_args < 4) \{ 
    CkAbort("Usage: maxReduceNV <x> <y> <z> [<numCommTh> <numWorkTh>]\verb+\n+"); 
  \}
  if (num_args > 5) \{ num_work = atoi(main->getArgs()[5]); \}
  if (num_args > 4) \{ num_comm = atoi(main->getArgs()[4]); \}

  CreateBgNodeMsg *bgNodeMsg = new CreateBgNodeMsg;
  bgNodeMsg->numBgX = atoi(main->getArgs()[1]);
  bgNodeMsg->numBgY = atoi(main->getArgs()[2]);
  bgNodeMsg->numBgZ = atoi(main->getArgs()[3]);
  bgNodeMsg->numCTh = num_comm;
  bgNodeMsg->numWTh = num_work;

  main->CreateBlueGene(bgNodeMsg);
  return;
\}

void* BgNodeInit(BgNode *bgNode)
\{
  //register handlers
  bgNode->registerHandler(reduceID, reduce) ;
  bgNode->registerHandler(computeMaxID, computeMax) ;
  bgNode->registerHandler(contributeID, contribute) ;

  //triger computer at each node
  computeMaxMsg *msg = new computeMaxMsg;
  bgNode->addMessage(msg, computeMaxID, 0);

  //declare node variable and return a pointer
  userData *ud = new userData ;
  ud->count = 0 ;
  for(int i=0; i<A_SIZE; i++)
   ud->data[i] = 0 ;

  return (void*)ud ;
\}

void BgFinish()
\{\}

void computeMax(ThreadInfo *info)
\{
  int A[A_SIZE][A_SIZE];
  int i, j;
  int max = 0;

  int x,y,z;
  info->bgNode->getXYZ(x,y,z);

  // Initialize data in each node
  for (i=0;i<A_SIZE;i++)
    for (j=0;j<A_SIZE;j++)
      A[i][j] = info->bgNode->numBgX * info->bgNode->numBgY * 
                info->bgNode->numBgZ - x*y*z - i*j ;

  // Find Max
  for (i=0;i<A_SIZE;i++)
    for (j=0;j<A_SIZE;j++)
      if (max < A[i][j])
      \{
             max = A[i][j];
      \}

  // contribute the results for reduction
  contributeMsg *msg = new contributeMsg;
  msg->max = max;
  info->bgNode->addMessage(msg,contributeID,1);

  ckout << "computeMax in " << x << ", " << y << ", " << z << endl;
  ckout << "contributed max value " << max << endl;
\}

void contribute(ThreadInfo *info)
\{
  int x,y,z;
  info->bgNode->getXYZ(x,y,z);

  // Accessing node variables : get number of data values received at this node
  int count = ((userData*)(info->bgNode->nvData))->count++ ;
  // Store new contribution to node variables
  ((userData*)(info->bgNode->nvData))->data[count++] =
((contributeMsg*)(info->msg))->max ;

 // Compute the required Count of data values at each node to start reduction 
 // at that node, depends on reduction sequence
  int reqCount ;

  if(z==info->bgNode->numBgZ-1)
  \{
         reqCount = 1 ;
  \}
  else if(z>0 || (z==0 \&\& x==info->bgNode->numBgX-1))
  \{
         reqCount = 2 ;
  \}
  else if(x>0 || (x==0 \&\& y==info->bgNode->numBgY-1))
  \{
         reqCount = 3 ;
  \}
  else
         reqCount = 4 ;

  if(count==reqCount)     //if data for reduction is ready
  \{
       reduceMsg *msg = new reduceMsg ;
       info->bgNode->addMessage(msg, reduceID, 0) ;
       ckout << "contribute in Node " << x << ", " << y << ", " << z << endl;
       ckout << "Values collected " << count << ", calling reduction " << endl;
       return ;
  \}

  ckout << "contribute in Node " << x << ", " << y << ", " << z << endl ;
  ckout << "Values collected " << count << ", reqCount " << reqCount << endl;
\}

void reduce(ThreadInfo *info)
\{
  int x,y,z;
  info->bgNode->getXYZ(x,y,z);
  ckout << "reduce in " << x << ", " << y << ", " << z << endl;

  //do reduction
  int max = 0 ;
  int count = ((userData*)(info->bgNode->nvData))->count ;
  for(int i=0; i<count; i++)
  \{
           if(max<((userData*)(info->bgNode->nvData))->data[i])
           max = ((userData*)(info->bgNode->nvData))->data[i] ;
  \}

  if(x==0 \&\& y==0 \&\& z==0)
  \{
         ckout << "Exiting: max value is " << max << endl ;
         info->bgNode->finish() ;
         return ;
  \}

  //send max to destination, depends on reduction sequence
  if(z>0)
   z-- ;
  else if(x>0)
   x-- ;
  else
   y-- ;

  contributeMsg *msg = new contributeMsg;
  msg->max = max;
  info->bgNode->sendPacket(x,y,z,msg,contributeID,1);

  ckout << "sending max value " << max << " to " << x << ", " << y << ", " << z << endl ;
\}
\end{alltt}

\section{Compiling and Running}

Compile Blue Gene emulator programs using {\tt charmc} as one would
in the case of normal \charmpp{} programs. In order to link the
Blue Gene programs, use \texttt{-language bluegene} as an argument
to the {\tt charmc} linker.

\input{index}
\end{document}
