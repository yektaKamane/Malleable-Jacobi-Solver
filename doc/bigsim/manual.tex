\documentclass[10pt]{article}
\usepackage{../pplmanual}
\NeedsTeXFormat{LaTeX2e}
\typeout{^^J^^J
Parallel Programming Laboratory^^J
Manual Style^^J
Written by Milind A. Bhandarkar, 12/00^^J}

%%% Make it possible for both ps and pdf to be generated
\newif\ifpdf
\ifx\pdfoutput\undefined
  \pdffalse
\else
  \pdfoutput=1
  \pdftrue
\fi

\ifpdf
  \pdfcompresslevel=9
\fi

%%% Imported from fullpage.sty, since it is not always available
\topmargin 0pt
\advance \topmargin by -\headheight
\advance \topmargin by -\headsep

\textheight 8.9in

\oddsidemargin 0pt
\evensidemargin \oddsidemargin
\marginparwidth 1.0in

\textwidth 6.5in
%%% end import from fullpage

%%% Commonly Needed packages
\usepackage{graphicx,color,calc}
\usepackage{makeidx}
\usepackage{alltt}

%%% Commands for uniform looks of C++, Charm++, and Projections
\newcommand{\CC}{C\kern -0.0em\raise 0.5ex\hbox{\normalsize++}}
\newcommand{\emCC}{C\kern -0.0em\raise 0.4ex\hbox{\normalsize\em++}}
\newcommand{\charmpp}{\sc Charm++}
\newcommand{\projections}{\sc Projections}
\newcommand{\converse}{\sc Converse}
\newcommand{\ampi}{\sc AMPI}

%%% Commands to produce margin symbols
\newcommand{\new}{\marginpar{\fbox{\bf$\mathcal{NEW}$}}}
\newcommand{\important}{\marginpar{\fbox{\bf\Huge !}}}
\newcommand{\experimental}{\marginpar{\fbox{\bf\Huge $\beta$}}}

%%% Commands for manual elements
\newcommand{\zap}[1]{ }
\newcommand{\function}[1]{{\noindent{\textsf{#1}}\\}}
\newcommand{\cmd}[1]{{\noindent{\textsf{#1}}\\}}
\newcommand{\args}[1]{\hspace*{2em}{\texttt{#1}}\\}
\newcommand{\param}[1]{{\texttt{#1}}}
\newcommand{\kw}[1]{{\textsf{#1}}}
\newcommand{\uw}[1]{{\textsl{#1}}}
\newcommand{\desc}[1]{\indent{#1}}

%%% Commands needed for Maketitle
\newcommand{\@version}{}
\newcommand{\@credits}{}
\newcommand{\version}[1]{\renewcommand{\@version}{#1}}
\newcommand{\credits}[1]{\renewcommand{\@credits}{#1}}

%%% Print the License Page
\newcommand{\@license}{%
 \begin{center}
   {University of Illinois}\\
   {\charmpp/\converse\ Parallel Programming System Software}\\
   {Non-Exclusive, Non-Commercial Use License}\\
 \end{center}
 \rule{\textwidth}{1pt}
{\tiny
Upon execution of this Agreement by the party identified below (``Licensee''),
The Board of Trustees of the University of Illinois  (``Illinois''), on behalf
of The Parallel Programming Laboratory (``PPL'') in the Department of Computer
Science, will provide the \charmpp/\converse\ Parallel Programming System
software (``\charmpp'') in Binary Code and/or Source Code form (``Software'')
to Licensee, subject to the following terms and conditions. For purposes of
this Agreement, Binary Code is the compiled code, which is ready to run on
Licensee's computer.  Source code consists of a set of files which contain the
actual program commands that are compiled to form the Binary Code.

\begin{enumerate}
  \item
    The Software is intellectual property owned by Illinois, and all right,
title and interest, including copyright, remain with Illinois.  Illinois
grants, and Licensee hereby accepts, a restricted, non-exclusive,
non-transferable license to use the Software for academic, research and
internal business purposes only, e.g. not for commercial use (see Clause 7
below), without a fee.

  \item 
    Licensee may, at its own expense, create and freely distribute
complimentary works that interoperate with the Software, directing others to
the PPL server (\texttt{http://charm.cs.uiuc.edu}) to license and obtain the
Software itself. Licensee may, at its own expense, modify the Software to make
derivative works.  Except as explicitly provided below, this License shall
apply to any derivative work as it does to the original Software distributed by
Illinois.  Any derivative work should be clearly marked and renamed to notify
users that it is a modified version and not the original Software distributed
by Illinois.  Licensee agrees to reproduce the copyright notice and other
proprietary markings on any derivative work and to include in the documentation
of such work the acknowledgement:

\begin{quote}
``This software includes code developed by the Parallel Programming Laboratory
in the Department of Computer Science at the University of Illinois at
Urbana-Champaign.''
\end{quote}

Licensee may redistribute without restriction works with up to 1/2 of their
non-comment source code derived from at most 1/10 of the non-comment source
code developed by Illinois and contained in the Software, provided that the
above directions for notice and acknowledgement are observed.  Any other
distribution of the Software or any derivative work requires a separate license
with Illinois.  Licensee may contact Illinois (\texttt{kale@cs.uiuc.edu}) to
negotiate an appropriate license for such distribution.

  \item
    Except as expressly set forth in this Agreement, THIS SOFTWARE IS PROVIDED
``AS IS'' AND ILLINOIS MAKES NO REPRESENTATIONS AND EXTENDS NO WARRANTIES OF
ANY KIND, EITHER EXPRESS OR IMPLIED, INCLUDING BUT NOT LIMITED TO WARRANTIES OR
MERCHANTABILITY OR FITNESS FOR A PARTICULAR PURPOSE, OR THAT THE USE OF THE
SOFTWARE WILL NOT INFRINGE ANY PATENT, TRADEMARK, OR OTHER RIGHTS.  LICENSEE
ASSUMES THE ENTIRE RISK AS TO THE RESULTS AND PERFORMANCE OF THE SOFTWARE
AND/OR ASSOCIATED MATERIALS.  LICENSEE AGREES THAT UNIVERSITY SHALL NOT BE HELD
LIABLE FOR ANY DIRECT, INDIRECT, CONSEQUENTIAL, OR INCIDENTAL DAMAGES WITH
RESPECT TO ANY CLAIM BY LICENSEE OR ANY THIRD PARTY ON ACCOUNT OF OR ARISING
FROM THIS AGREEMENT OR USE OF THE SOFTWARE AND/OR ASSOCIATED MATERIALS.

  \item 
    Licensee understands the Software is proprietary to Illinois. Licensee
agrees to take all reasonable steps to insure that the Software is  protected
and secured from unauthorized disclosure, use, or release and  will treat it
with at least the same level of care as Licensee would use to  protect and
secure its own proprietary computer programs and/or information, but using no
less than a reasonable standard of care.  Licensee agrees to provide the
Software only to any other person or entity who has registered with Illinois.
If licensee is not registering as an individual but as an institution or
corporation each member of the institution or corporation who has access to or
uses Software must agree to and abide by the terms of this license. If Licensee
becomes aware of any unauthorized licensing, copying or use of the Software,
Licensee shall promptly notify Illinois in writing. Licensee expressly agrees
to use the Software only in the manner and for the specific uses authorized in
this Agreement.

  \item
    By using or copying this Software, Licensee agrees to abide by the
copyright law and all other applicable laws of the U.S. including, but not
limited to, export control laws and the terms of this license. Illinois  shall
have the right to terminate this license immediately by written  notice upon
Licensee's breach of, or non-compliance with, any terms of the license.
Licensee may be held legally responsible for any  copyright infringement that
is caused or encouraged by its failure to  abide by the terms of this license.
Upon termination, Licensee agrees to  destroy all copies of the Software in its
possession and to verify such  destruction in writing.

  \item
  The user agrees that any reports or published results obtained with  the
Software will acknowledge its use by the appropriate citation as  follows:

\begin{quote}
``\charmpp/\converse\ was developed by the Parallel Programming Laboratory in
the Department of Computer Science at the University of  Illinois at
Urbana-Champaign.''
\end{quote}

Any published work which utilizes \charmpp\ shall include the following
reference:

\begin{quote}
``L. V. Kale and S. Krishnan. \charmpp: Parallel Programming with Message-Driven
Objects. In 'Parallel Programming using \CC' (Eds. Gregory V. Wilson and Paul
Lu), pp 175-213, MIT Press, 1996.''
\end{quote}

Any published work which utilizes \converse\ shall include the following
reference:

\begin{quote}
``L. V. Kale, Milind Bhandarkar, Narain Jagathesan, Sanjeev Krishnan and Joshua
Yelon. \converse: An Interoperable Framework for Parallel Programming.
Proceedings of the 10th International Parallel Processing Symposium, pp
212-217, April 1996.''
\end{quote}

Electronic documents will include a direct link to the official \charmpp\ page
at \texttt{http://charm.cs.uiuc.edu/}

  \item
    Commercial use of the Software, or derivative works based thereon,
REQUIRES A COMMERCIAL LICENSE.  Should Licensee wish to make commercial use of
the Software, Licensee will contact Illinois (kale@cs.uiuc.edu) to negotiate an
appropriate license for such use. Commercial use includes: 

    \begin{enumerate}
      \item
	integration of all or part of the Software into a product for sale,
lease or license by or on behalf of Licensee to third parties, or 

      \item
	distribution of the Software to third parties that need it to
commercialize product sold or licensed by or on behalf of Licensee.
    \end{enumerate}

  \item
    Government Rights. Because substantial governmental funds have been  used
in the development of \charmpp/\converse, any possession, use or sublicense of
the Software by or to the United States government shall be subject to such
required restrictions.

  \item
    \charmpp/\converse\ is being distributed as a research and teaching tool
and as such, PPL encourages contributions from users of the code that might, at
Illinois' sole discretion, be used or incorporated to make the basic  operating
framework of the Software a more stable, flexible, and/or useful  product.
Licensees who contribute their code to become an internal  portion of the
Software agree that such code may be distributed by  Illinois under the terms
of this License and may be required to sign an  ``Agreement Regarding
Contributory Code for \charmpp/\converse\ Software'' before Illinois  can
accept it (contact \texttt{kale@cs.uiuc.edu} for a copy).
\end{enumerate}

UNDERSTOOD AND AGREED.

Contact Information:

The best contact path for licensing issues is by e-mail to
\texttt{kale@cs.uiuc.edu} or send correspondence to:

\begin{quote}
Prof. L. V. Kale\\
Dept. of Computer Science\\
University of Illinois\\
1304 W. Springfield Ave\\
Urbana, Illinois 61801 USA\\
FAX: (217) 333-3501
\end{quote}
}%tiny
 \newpage
}% end of license

\renewcommand{\maketitle}{\begin{titlepage}%
 \begin{flushright}
   {\Large
     Parallel Programming Laboratory\\
     University of Illinois at Urbana-Champaign\\
   }
 \end{flushright}
 \rule{\textwidth}{3pt}
 \vspace{\fill}
 \begin{flushright}
   \textsf{\Huge \@title \\}
 \end{flushright}
 \vspace{\fill}
 \@credits \\
 \rule{\textwidth}{3pt}
 \begin{flushright}
   {\large Version \@version}
 \end{flushright}
 \end{titlepage}
 \@license

 \tableofcontents
 \newpage
}% maketitle



\title{Parallel Simulator for Large Parallel Machines}
\version{1.00}
\credits{Charm++ BlueGene Emulator was developed by Arun Singla, Neelam Saboo
and Joshua Unger under the guidance of Prof. L. V. Kale. The new Converse 
Blue Gene Emulator is completely rewritten by Gengbin Zheng. Converse Blue Gene 
Emulator is the only version under maintenance now. Charm++ and Adaptive MPI
(AMPI) was ported onto the Blue Gene Emulator by Gengbin Zheng. The parallel
performance simulator was developed by Gengbin Zheng and 
Gunavardhan Kakulapati.}

\begin{document}
\maketitle

\section{Introduction}

One approach for building the next generation of parallel computers
is based on large aggregates of multiprocessor chips with support
for hardware multithreading. 
An initial design for IBM's Blue Gene/C project exemplifies this approach.
Blue Gene/C was a proposed one million processor machine from IBM.

It is important that one can study the programming issues and performance
of paralle applications on such machines even before the machine is built.
Thus, we have developed a parallel simulator to facilitate this research.

Since our research was initiated by Blue Gene/C project, in this manual, 
we also call our simulator as Blue Gene Simulator.
Our simulator is capable of simulating a broad class of "Massively Parallel
Processors-In-Memory", or MPPIMs machines. 

\subsection{Simulator system components:}

Our simulator system includes these components: 
\begin{enumerate}
\item a parallel emulator which emulates the low level machine API; 
\item a message driven programming language (Charm++) running on top of the emulator; 
\item adaptive MPI (an implementation of MPI on top of Charm++) environment; 
\item a parallel simulator for performance prediction. 
\end{enumerate}

\subsection{History}

The first version of Blue Gene emulator was first written in Charm++, a 
parallel object language in fall, 2001. 

The second version of Blue Gene emulator was completely rewritten on top 
of Converse instead of Charm++ in spring 2002. 
While the API supported by the original emulator remains almost same,
many new features were added - support of thread-committed messages that 
can be send to a specific thread in a Blue Gene node; support of Blue Gene 
node level broadcast.
The new emulator was implemented on a lower layer communication 
library - Converse in order to achieve better 
performance by avoiding the cross layer overhead. 
Another advantage is that the lighter weighted emulator makes it easier
to port higher level of programming language onto the emulator.

In the following sections, we will first describe how to download and compile
Blue Gene simulator(Sec. ~\ref{install}). Section~\ref{bgemulator} will describe
the low level machine API in detail. 


\section{BigSim Simulator Installation and Usage}
\label{install}

\subsection{Installing Charm++ and BigSim}

BigSim Simulator is distributed as a part of the Charm++ standard distribution.
One needs to download Charm++ and compile BigSim simulator.
One should begin with downloading Charm++ from the website:
http://charm.cs.uiuc.edu.

Please refer to ``Charm++ Installation and Usage Manual" and also the file
README in the source code for detailed instructions on how to compile Charm++.
In short, the ``build" script is the main tool for compiling \charmpp{}.  You
need to provide target and platform options: \begin{verbatim} ./build <target>
<platform> [options ...] [charmc-options ...] \end{verbatim}

For example, to compile on a Linux machine, type:
\begin{verbatim}
./build charm++ net-linux -O
\end{verbatim}

which builds essential \charmpp{} kernel using UDP sockets as 
communication method, 
alternatively, you can build Charm++ kernel on MPI using:
\begin{verbatim}
./build charm++ mpi-linux -O
\end{verbatim}

For other platforms, change net-linux to whatever platform you are compiling 
on. See the charm/README file for a complete list of supported platforms.

\subsubsection{Build Only the BigSim Emulator}

BigSim emulator is implemented on top of Converse in Charm++.
To compile BigSim emulator, one can compile emulator libraries
directly on top of normal Charm++ using ``bigsim'' as the compilation
target, like
\begin{verbatim}
./build bigsim net-linux -O
\end{verbatim}

With emulator libraries, one can write BigSim applications using its
low level emulator message passing API.

\subsubsection{Build Charm++ on BigSim Emulator}

In order to build Charm++ on top of BigSim Emulator (which itself is 
implemented on top of Converse), a special build option ``bigsim''
needs to be specified:
\begin{verbatim}
./build bigsim net-linux bigsim -O
\end{verbatim}

The first ``bigsim" is the compilation target that tells ``build" to
compile BigSim emulator libraries in addition to \charmpp{} kernel libraries;
The second ``bigsim" is a build option to platform ``net-linux", which tells
``build" to build the Charm++ on top of BigSim Emulator. 
To build AMPI on BigSim, use ``bgampi" as make target, which subsumes target
of ``bigsim":
\begin{verbatim}
./build bgampi net-linux bigsim -O
\end{verbatim}

For the above ``build" command, it creates a directory named 
``net-linux-bigsim" under charm, which contains all the header files and
libraries needed for compiling a user application.

\subsection{Compiling BigSim Applications}

\charmpp{} provides a compiler script {\tt charmc} to compile all programs.

There are three methods to write a BigSim application:

\subsubsection{Writing a BigSim application using low level machine API}
The low level machine API mimics the actual machine low level programming
API. It is defined in section~\ref{bgemulator}. Writing a program in the 
low level machine API, you just need to link \charmpp{}'s BigSim emulator
libraries, which provide the emulation of the machine API using Converse as
the communication layer.

In order to link against the BigSim library, specify 
\texttt{-language bigsim} as an argument to the {\tt charmc} linker, 
for example:
\begin{verbatim}
charmc -o hello hello.C -language bigsim
\end{verbatim}

Sample applications in low level machine API can be found under directory
charm/pgms/converse/bluegene.

\subsubsection{Writing a BigSim application in Charm++}

One can write a normal \charmpp{} application which can automatically 
run on the emulator after compilation. \charmpp{} implements
an object-based message-driven execution model. In \charmpp{} applications,
there are collections of C++ objects, which communicate by remotely invoking
methods on other objects by messages.

In order to compile a program written in \charmpp{} on Blue Gene simulator, 
specify \texttt{-language charm++} as an argument to the {\tt charmc} linker:
\begin{verbatim}
charmc -o hello hello.C -language charm++
\end{verbatim}
This will link both \charmpp{} runtime libraries and BigSim simulator 
libraries.

Sample applications in \charmpp{} can be found under directory
charm/pgms/charm++, specifically charm/pgms/charm++/littleMD.

\subsubsection{Writing a Blue Gene application in MPI}

One can also write a MPI application for Blue Gene Simulator.
The Adaptive MPI, or AMPI is implemented on top of Charm++ that supports
dynamic load balancing and multithreading for MPI applications. This is based
on the user-level migrating threads and load balancing capabilities provided
by the \charmpp{} framework. This allows legacy MPI programs to run 
on top of Blue Gene \charmpp{} and take advantage of the \charmpp{}'s
virtualization and adaptive load balancing capability.

Current AMPI implements most features in the MPI version 1.0, with a few
extensions for migrating threads and asynchronous reduction.

In order to compile an AMPI application on Blue Gene simulator, you need 
to link against the AMPI library as well as Blue Gene \charmpp{} runtime
libraries by specifying \texttt{-language ampi} as an argument to 
the {\tt charmc} linker:
\begin{verbatim}
charmc -o hello hello.C -language ampi
\end{verbatim}

Sample applications in AMPI can be found under directory
charm/pgms/charm++/ampi, specifically charm/pgms/charm++/Cjacobi3D.

\subsection{Run a BigSim Application}

To run a parallel Blue Gene application, \charmpp{} provides a utility program
called {\tt charmrun} to start the parallel program. 
For detailed description on how to run a \charmpp{} application, 
refer to the file charm/README in the source code distribution.

To run a Blue Gene application, you need to specify these parameters to 
{\tt charmrun} to define the simulated Blue Gene machine size:
\begin{enumerate}
\item {\tt +x, +y} and {\tt +z}:  define the size of the machine in three dimensions, these define the number of nodes along each dimension of the machine;
\item {\tt +wth} and {\tt +cth}:  For one node, these two parameters define the number of worker processors({\tt +wth}) and the number of communication processors({\tt +cth}).
\item {\tt +bgcorrect}: starts the simulation mode for performance prediction. Otherwise the program runs without doing parallel event simulation for performance prediction of the application.
\item {\tt +bgwalltime}: used only in simulation mode, when specified, use wallclock measurement of the time taken on the simulating machine to estimate the time it takes to run on the target machine.
\item {\tt +bgcounter}:  used only in simulation mode, when specified, use the performance counter to estimate the time on target machine. This is currently only supported when perfex is installed, like Origin2000.
\end{enumerate}

For example, to simulate a Blue Gene/L machine of size 64K in 40x40x40, with 
one worker processor and one I/O processor on each node, and use 100 
real processors to simulate:
\begin{verbatim}
./charmrun +p100 ./hello +x40 +y40 +z40 +cth1 +wth1
\end{verbatim}

To run an AMPI program, you may also want to specify the number of virtual 
processors to run the MPI by using {\tt +vp}, for example:
\begin{verbatim}
./charmrun +p100 ./hello +x40 +y40 +z40 +cth1 +wth1 +vp 128000
\end{verbatim}
starts the simulation of Blue Gene/L of size 40x40x40 with 2 processors 
in each node, running 128000 MPI threads (2 MPI threads on each Blue Gene node),
 using 100 real processors to simulate. In this case, {\tt MPI\_Comm\_size()}
returns 128000 for {\tt MPI\_COMM\_WORLD}. If you donot specify the {\tt +vp}
option, the number of virtual processors will be equal to the number of 
processors of the simulated machine, in this case 64000.




\section{Blue Gene Emulator}
\label{bgemulator}

The Blue Gene emulator environment is designed with the following
objectives:

\begin{enumerate}
\item To support a realistic Blue Gene API on existing parallel machines

\item To obtain first-order performance estimates of algorithms

\item To facilitate implementations of alternate programming models for
      Blue Gene
\end{enumerate}

The ``Blue Gene'' machine supported by the emulator consists of
three-dimensional grid of 1-chip nodes.  The user may specify the size
of the machine along each dimension (e.g. 34x34x36).  The chip supports
$k$ threads (e.g. 200), each with its own integer unit.  The proximity of
the integer unit with individual memory modules within a chip is not
currently modeled.

The API supported by the emulator can be broken down into several
components:

\begin{enumerate}
\item Low-level API for chip-to-chip communication
\item Mid-level API that supports local micro-tasking with a chip level
scheduler with features such as: read-only variables, reductions, broadcasts,
distributed tables, get/put operations
\item Migratable objects with automatic load balancing support
\end{enumerate}

Of these, the first two have been implemented.  The simple time stamping
algorithm, without error correction, has been implemented.  More
sophisticated timing algorithms, specifically aimed at error correction,
and more sophisticated features (2, 3, and others), as well as libraries
of commonly needed parallel operations are part of the proposed work for
future.

The following sections define the appropriate parts of the API, with
example programs and instructions for executing them.

\subsection{Blue Gene Programming Environment}

The basic philosophy of the Blue Gene Emulator is to hide intricate details
of the Blue Gene machine from the
application developer. Thus, the application developer needs to provide
intialization details and handler
functions only and gets the result as though running on a real machine.
Communication, Thread creation,
Time Stamping, etc are done by the emulator.

\subsubsection{Blue Gene API: Level 0}

\function{void addBgNodeInbuffer(bgMsg *msgPtr, int nodeID)}
\desc{
        low-level primitive invoked by Blue Gene emulator to put the 
        message to the inbuffer queue of a node.

        msgPtr - pointer to the message to be sent to target node; 

        nodeID - node ID of the target node, it is the serial number of a 
                 bluegene node in the emulator's physical node.
}

\function{void addBgThreadMessage(bgMsg *msgPtr, int threadID)}
\desc{
        add a message to a thread's affinity queue, these messages can be 
 	only executed by a specific thread indicated by threadID.
}

\function{void addBgNodeMessage(bgMsg *msgPtr)}
\desc{
	add a message to a node's non-affinity queue, these messages can be 
	executed by any thread in the node.
}

\function{boolean checkReady()}
\desc{
        invoked by communication thread to see if there is any unattended
        message in inBuffer.
}

\function{bgMsg * getFullBuffer()}
\desc{
	invoked by communication thread to retrieve the unattended message 
	in inBuffer.
}

\function{CmiHandler msgHandlerFunc(char *msg)}
\desc{
	Handler function type that user can register to handle the message.
}

\function{void sendPacket(int x, int y, int z, int msgSize,bgMsg *msg)}
\desc{
	chip-to-chip communication function. It send a message to Node[x][y][z].
        
	bgMsg is the message type with message envelop used internally.
}

\subsubsection{Initialization API: Level 1a}

All the functions defined in API Level 0 are used internally for the 
implementation of bluegene node communication and worker threads.

From this level, the functions defined are exposed to users to write bluegene
programs on the emulator.

Considering that the emulator machine will emulate several Bluegene nodes on
each physical node, the emulator program defines this function 
\function{BgEmulatorInit(int argc, char **argv)} to initialize each emulator
node. In this function, user program can define the Bluegene machine size,
number of communication/worker threads, and check the command line arguments.

The size of the Blue Gene machine being emulated and the number of thread per
node is determined either by the command line arguments or calling following
functions:

\function{void BgSetSize(int sx, int sy, int sz)}
\desc{
	set Blue Gene Machine size;
}

\function{void BgSetNumWorkThread(int num)}
\desc{
	set number of worker threads per node;
}

\function{void BgSetNumCommThread(int num)}
\desc{
	set number of communication threads per node;
}

\function{int BgRegisterHandler(BgHandler h)}
\desc{
	register user message handler functions; 
}

For each Blue Gene node, the execution starts at 
\function{BgNodeStart(int argc, char **argv)} called by the emulator,
where application handlers can be registered and computation 
is triggered by creating a task at required nodes.

Similar to pthread's thread specifc data, each bluegene node has its
own node specific data associated with it. To do this, the user needs to define its 
own node-specific variables encapsulated in a struct definition and register
 the pointer to the data with the emulator by following function:

\function{void BgSetNodeData(char *data)}

To retrieve the node specific data, call:

\function{char *BgGetNodeData();}

After completion of execution, user program invokes a function:

\function{void BgShutdown()}

to terminate the emulator.

\subsubsection{Handler Function API: Level 1a}

The following functions can be called in user's application program to retrieve
the Blue Gene machine information, get thread execution time, and perform
the communication.

\function{void BgGetSize(int *sx, int *sy, int *sz);}

\function{int BgGetNumWorkThread();}

\function{int BgGetNumCommThread();}

\function{int BgGetThreadID();}

\function{double BgGetTime();}

\function{void BgSendPacket(int x, int y, int z, int threadID, int handlerID, WorkType type, int numbytes, char* data);}
\desc{
This sends a trunk of data to Node[x, y, z] and also specifies the
handler function to be used for this message i.e. the handlerID;
threadID specifes the desired thread to handle the message, ANYTHREAD means
no preference.

To specify the thread category:
\begin{description}
\item[1:] a small piece of work that can be done by
communication thread itself, so NO scheduling overhead.
\item[0:] a large piece of work, so communication thread
schedules it for a worker thread
\end{description}
}


\subsection{Writing a Blue Gene Application}

\subsubsection{Application Skeleton}

\begin{alltt}
Handler function prototypes;
Node specific data type declarations;

void  BgEmulatorInit(int argc, char **argv)  function
  Configure bluegene machine parameters including size, number of threads, etc.
  You also neet to register handlers here.

void *BgNodeStart(int argc, char **argv) function
  The usual practice in this function is to send an intial message to trigger 
  the execution.
  You can also register node specific data in this function.

Handler Function 1, void handlerName(char *info)
Hanlder Function 2, void handlerName(char *info)
..
Handler Function N, void handlerName(char *info)

\end{alltt}

\subsubsection{Sample Application 1}

\begin{verbatim}
/* Application: 
 *   Each node starting at [0,0,0] sends a packet to next node in
 *   the ring order.
 *   After node [0,0,0] gets message from last node
 *   in the ring, the application ends.
 */


#include "blue.h"

#define MAXITER 2

int iter = 0;
int passRingHandler;

void passRing(char *msg);

void nextxyz(int x, int y, int z, int *nx, int *ny, int *nz)
{
  int numX, numY, numZ;

  BgGetSize(&numX, &numY, &numZ);
  *nz = z+1; *ny = y; *nx = x;
  if (*nz == numZ) {
    *nz = 0; (*ny) ++;
    if (*ny == numY) {
      *ny = 0; (*nx) ++;
      if (*nx == numX) *nx = 0;
    }
  }
}

void BgEmulatorInit(int argc, char **argv)
{
  passRingHandler = BgRegisterHandler(passRing);
}

/* user defined functions for bgnode start entry */
void BgNodeStart(int argc, char **argv)
{
  int x,y,z;
  int nx, ny, nz;
  int data, id;

  BgGetXYZ(&x, &y, &z);
  nextxyz(x, y, z, &nx, &ny, &nz);
  id = BgGetThreadID();
  data = 888;
  if (x == 0 && y==0 && z==0) {
    BgSendPacket(nx, ny, nz, -1,passRingHandler, LARGE_WORK, 
				sizeof(int), (char *)&data);
  }
}

/* user write code */
void passRing(char *msg)
{
  int x, y, z;
  int nx, ny, nz;
  int id;
  int data = *(int *)msg;

  BgGetXYZ(&x, &y, &z);
  nextxyz(x, y, z, &nx, &ny, &nz);
  if (x==0 && y==0 && z==0) {
    if (++iter == MAXITER) BgShutdown();
  }
  id = BgGetThreadID();
  BgSendPacket(nx, ny, nz, -1, passRingHandler, LARGE_WORK, 
				sizeof(int), (char *)&data);
}

\end{verbatim}


\subsubsection{Sample Application 2}

\begin{verbatim}

/* Application: 
 *   Find the maximum element.
 *   Each node computes maximum of it's elements and
 *   the max values it received from other nodes
 *   and sends the result to next node in the reduction sequence.
 * Reduction Sequence: Reduce max data to X-Y Plane
 *   Reduce max data to Y Axis
 *   Reduce max data to origin.
 */


#include <stdlib.h>
#include "blue.h"

#define A_SIZE 4

#define X_DIM 3
#define Y_DIM 3
#define Z_DIM 3

int REDUCE_HANDLER_ID;
int COMPUTATION_ID;

extern "C" void reduceHandler(char *);
extern "C" void computeMax(char *);

class ReductionMsg {
public:
  int max;
};

class ComputeMsg {
public:
  int dummy;
};

void BgEmulatorInit(int argc, char **argv)
{
  if (argc < 2) { 
    CmiAbort("Usage: <program> <numCommTh> <numWorkTh>\n"); 
  }

  /* set machine configuration */
  BgSetSize(X_DIM, Y_DIM, Z_DIM);
  BgSetNumCommThread(atoi(argv[1]));
  BgSetNumWorkThread(atoi(argv[2]));

  REDUCE_HANDLER_ID = BgRegisterHandler(reduceHandler);
  COMPUTATION_ID = BgRegisterHandler(computeMax);

}

void BgNodeStart(int argc, char **argv) {
  int x, y, z;
  BgGetXYZ(&x, &y, &z);

  ComputeMsg *msg = new ComputeMsg;
  BgSendLocalPacket(ANYTHREAD, COMPUTATION_ID, LARGE_WORK, 
			sizeof(ComputeMsg), (char *)msg);
}

void reduceHandler(char *info) {
  // assumption: THey are initialized to zero?
  static int max[X_DIM][Y_DIM][Z_DIM];
  static int num_msg[X_DIM][Y_DIM][Z_DIM];

  int i,j,k;
  int external_max;

  BgGetXYZ(&i,&j,&k);
  external_max = ((ReductionMsg *)info)->max;
  num_msg[i][j][k]++;

  if ((i == 0) && (j == 0) && (k == 0)) {
    // master node expects 4 messages:
    // 1 from itself;
    // 1 from the i dimension;
    // 1 from the j dimension; and
    // 1 from the k dimension
    if (num_msg[i][j][k] < 4) {
      // not ready yet, so just find the max
      if (max[i][j][k] < external_max) {
	max[i][j][k] = external_max;
      }
    } else {
      // done. Can report max data after making last comparison
      if (max[i][j][k] < external_max) {
	max[i][j][k] = external_max;
      }
      CmiPrintf("The maximal value is %d \n", max[i][j][k]);
      BgShutdown();
      return;
    }
  } else if ((i == 0) && (j == 0) && (k != Z_DIM - 1)) {
    // nodes along the k-axis other than the last one expects 4 messages:
    // 1 from itself;
    // 1 from the i dimension;
    // 1 from the j dimension; and
    // 1 from the k dimension
    if (num_msg[i][j][k] < 4) {
      // not ready yet, so just find the max
      if (max[i][j][k] < external_max) {
	max[i][j][k] = external_max;
      }
    } else {
      // done. Forwards max data to node i,j,k-1 after making last comparison
      if (max[i][j][k] < external_max) {
	max[i][j][k] = external_max;
      }
      ReductionMsg *msg = new ReductionMsg;
      msg->max = max[i][j][k];
      BgSendPacket(i,j,k-1,ANYTHREAD,REDUCE_HANDLER_ID,LARGE_WORK, 
				sizeof(ReductionMsg), (char *)msg);
    }
  } else if ((i == 0) && (j == 0) && (k == Z_DIM - 1)) {
    // the last node along the k-axis expects 3 messages:
    // 1 from itself;
    // 1 from the i dimension; and
    // 1 from the j dimension
    if (num_msg[i][j][k] < 3) {
      // not ready yet, so just find the max
      if (max[i][j][k] < external_max) {
	max[i][j][k] = external_max;
      }
    } else {
      // done. Forwards max data to node i,j,k-1 after making last comparison
      if (max[i][j][k] < external_max) {
	max[i][j][k] = external_max;
      }
      ReductionMsg *msg = new ReductionMsg;
      msg->max = max[i][j][k];
      BgSendPacket(i,j,k-1,ANYTHREAD,REDUCE_HANDLER_ID,LARGE_WORK, 
				sizeof(ReductionMsg), (char *)msg);
    }
  } else if ((i == 0) && (j != Y_DIM - 1)) {
    // for nodes along the j-k plane except for the last and first row of j,
    // we expect 3 messages:
    // 1 from itself;
    // 1 from the i dimension; and
    // 1 from the j dimension
    if (num_msg[i][j][k] < 3) {
      // not ready yet, so just find the max
      if (max[i][j][k] < external_max) {
	max[i][j][k] = external_max;
      }
    } else {
      // done. Forwards max data to node i,j-1,k after making last comparison
      if (max[i][j][k] < external_max) {
	max[i][j][k] = external_max;
      }
      ReductionMsg *msg = new ReductionMsg;
      msg->max = max[i][j][k];
      BgSendPacket(i,j-1,k,ANYTHREAD,REDUCE_HANDLER_ID,LARGE_WORK, 
				sizeof(ReductionMsg), (char *)msg);
    }
  } else if ((i == 0) && (j == Y_DIM - 1)) {
    // for nodes along the last row of j on the j-k plane,
    // we expect 2 messages:
    // 1 from itself;
    // 1 from the i dimension;
    if (num_msg[i][j][k] < 2) {
      // not ready yet, so just find the max
      if (max[i][j][k] < external_max) {
	max[i][j][k] = external_max;
      }
    } else {
      // done. Forwards max data to node i,j-1,k after making last comparison
      if (max[i][j][k] < external_max) {
	max[i][j][k] = external_max;
      }
      ReductionMsg *msg = new ReductionMsg;
      msg->max = max[i][j][k];
      BgSendPacket(i,j-1,k,ANYTHREAD,REDUCE_HANDLER_ID,LARGE_WORK, 
				sizeof(ReductionMsg), (char *)msg);
    }
  } else if (i != X_DIM - 1) {
    // for nodes anywhere the last row of i,
    // we expect 2 messages:
    // 1 from itself;
    // 1 from the i dimension;
    if (num_msg[i][j][k] < 2) {
      // not ready yet, so just find the max
      if (max[i][j][k] < external_max) {
	max[i][j][k] = external_max;
      }
    } else {
      // done. Forwards max data to node i-1,j,k after making last comparison
      if (max[i][j][k] < external_max) {
	max[i][j][k] = external_max;
      }
      ReductionMsg *msg = new ReductionMsg;
      msg->max = max[i][j][k];
      BgSendPacket(i-1,j,k,ANYTHREAD,REDUCE_HANDLER_ID,LARGE_WORK, 
				sizeof(ReductionMsg), (char *)msg);
    }
  } else if (i == X_DIM - 1) {
    // last row of i, we expect 1 message:
    // 1 from itself;
    if (num_msg[i][j][k] < 1) {
      // not ready yet, so just find the max
      if (max[i][j][k] < external_max) {
	max[i][j][k] = external_max;
      }
    } else {
      // done. Forwards max data to node i-1,j,k after making last comparison
      if (max[i][j][k] < external_max) {
	max[i][j][k] = external_max;
      }
      ReductionMsg *msg = new ReductionMsg;
      msg->max = max[i][j][k];
      BgSendPacket(i-1,j,k,-1,REDUCE_HANDLER_ID,LARGE_WORK, 
				sizeof(ReductionMsg), (char *)msg);
    }
  }
}

void computeMax(char *info) {
  int A[A_SIZE][A_SIZE];
  int i, j;
  int max = 0;

  int x,y,z; // test variables
  BgGetXYZ(&x,&y,&z);

  // Initialize
  for (i=0;i<A_SIZE;i++) {
    for (j=0;j<A_SIZE;j++) {
      A[i][j] = i*j;
    }
  }

//  CmiPrintf("Finished Initializing %d %d %d!\n",  x , y , z);

  // Find Max
  for (i=0;i<A_SIZE;i++) {
    for (j=0;j<A_SIZE;j++) {
      if (max < A[i][j]) {
	max = A[i][j];
      }
    }
  }

  // prepare to reduce
  ReductionMsg *msg = new ReductionMsg;
  msg->max = max;
  BgSendLocalPacket(ANYTHREAD, REDUCE_HANDLER_ID, LARGE_WORK, 
				sizeof(ReductionMsg), (char *)msg);

//  CmiPrintf("Sent reduce message to myself with max value %d\n", max);
}


\end{verbatim}



\input{index}
\end{document}


