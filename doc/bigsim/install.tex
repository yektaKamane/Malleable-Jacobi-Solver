\section{Blue Gene Simulator Installation and Usage}
\label{install}

\subsection{Installing Charm++ and Blue Gene}

Blue Gene Simulator now is integrated into Charm++ distribution as a runtime 
library. Unfortunately, the precompiled binary package distribution doesnot 
contain the Blue Gene Simulator, thus you need to download source code
and compile yourself. 

You begin by downloading Charm++ from our website:
http://charm.cs.uiuc.edu/beta.html

Please refer to "Charm++ Installation and Usage Manual" and also the README
in the source code for detailed instruction on how to compile Charm++.
In short, the "build" script is the main tool for compiling Charm++.
You need to provide target and platform options:
\begin{verbatim}
./build <target> <platform> [options ...] [charmc-options ...]
\end{verbatim}

For example, to compile on a Linux machine, type:
\begin{verbatim}
./build charm++ net-linux
\end{verbatim}

which builds basic Charm++ kernel using UDP as communication method, 
alternatively, you can build Charm++ kernel on MPI:
\begin{verbatim}
./build charm++ mpi-linux
\end{verbatim}

For other platforms, change net-linux to whatever platform you are compiling 
on.

However, those commands donot automatically compile for Blue Gene Simulator, 
thus to compile both Charm++ kernel and Blue Gene Simulator, you need 
to provide a special target and an extra option for "build" script, 
for example:
\begin{verbatim}
./build bluegene net-linux bluegene
\end{verbatim}

The first "bluegene" is the compilation target, it tells "build" to
compile Blue Gene Simulator libraries as well as Charm++ basic kernel;
The second "bluegene" is an option to platform "net-linux", which tells
"build" to compile the Charm++ kernel upon Blue Gene Emulator. 
To compile AMPI on Blue Gene, use "bgampi" as target, which subsumes target
"bluegene":
\begin{verbatim}
./build bgampi net-linux bluegene
\end{verbatim}

For the above "build" command, it creates a directory named 
"net-linux-bluegene" under charm. 

\subsection{Compiling Blue Gene Applications}

\charmpp{} provides a compiler script {\tt charmc} to compile all programs.

There are three methods to write a Blue Gene applicaiton:

\subsubsection{Writing a Blue Gene application using low level machine API}
The low level machine API mimics the actual machine low level programming
API. It is defined in section~\ref{bgemulator}.

In order to link against the Blue Gene library, specify 
\texttt{-language bluegene} as an argument to the {\tt charmc} linker, 
for example:
\begin{verbatim}
charmc -o hello hello.C -language bluegene
\end{verbatim}

Sample applications in low level machine API can be found under directory
charm/pgms/converse/bluegene.

\subsubsection{Writing a Blue Gene application using Charm++}

One can also write a normal \charmpp{} application which can automatically
run on the emulator after compilation. In order to link against the Blue Gene 
library as well as \charmpp{} kernel libraries, specify 
\texttt{-language charm++} as an argument to the {\tt charmc} linker:
\begin{verbatim}
charmc -o hello hello.C -language charm++
\end{verbatim}

Sample applications in \charmpp{} can be found under directory
charm/pgms/charm++, specifically charm/pgms/charm++/littleMD.

\subsubsection{Writing a Blue Gene application using MPI}

One can also write a MPI application to run on Blue Gene Simulator.
The Adaptive MPI, or AMPI is implemented on top of Charm++.

In order to link against the Blue Gene library as well as AMPI and \charmpp{} 
kernel libraries, specify 
\texttt{-language ampi} as an argument to the {\tt charmc} linker:
\begin{verbatim}
charmc -o hello hello.C -language ampi
\end{verbatim}

Sample applications in AMPI can be found under directory
charm/pgms/charm++/ampi, specifically charm/pgms/charm++/Cjacobi3D.

\subsection{Running a Blue Gene Application}

To run a parallel Blue Gene application, \charmpp{} provides a utility program
{\tt charmrun} to start the parallel program. For detailed description on
how to run a \charmpp{} application, refer to file charm/README in the
source code distribution.

For Blue Gene applications, you need to provide these parameters to 
{\tt charmrun} to define the simulated Blue Gene machine size:
\begin{enumerate}
\item {\tt +x, +y} and {\tt +z}:  define the size of of machine in three dimensions, these define the number of nodes of a machine;
\item {\tt +wth} and {\tt +cth}:  For one node, these parameters define the number of worker processors({\tt +wth}) and the number of communication processors({\tt +cth}).
\end{enumerate}

For example, to simulate a Blue Gene/L machine of size 64K in 40X40X40, with 
one worker processor and one I/O processor on each node, using 100 
real processors:
\begin{verbatim}
./charmrun +p100 ./hello +x40 +y40 +z40 +cth1 +wth1
\end{verbatim}


