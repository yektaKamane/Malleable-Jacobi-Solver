\documentclass[10pt]{article}
\usepackage{../pplmanual}
\NeedsTeXFormat{LaTeX2e}
\typeout{^^J^^J
Parallel Programming Laboratory^^J
Manual Style^^J
Written by Milind A. Bhandarkar, 12/00^^J}

%%% Make it possible for both ps and pdf to be generated
\newif\ifpdf
\ifx\pdfoutput\undefined
  \pdffalse
\else
  \pdfoutput=1
  \pdftrue
\fi

\ifpdf
  \pdfcompresslevel=9
\fi

%%% Imported from fullpage.sty, since it is not always available
\topmargin 0pt
\advance \topmargin by -\headheight
\advance \topmargin by -\headsep

\textheight 8.9in

\oddsidemargin 0pt
\evensidemargin \oddsidemargin
\marginparwidth 1.0in

\textwidth 6.5in
%%% end import from fullpage

%%% Commonly Needed packages
\usepackage{graphicx,color,calc}
\usepackage{makeidx}
\usepackage{alltt}

%%% Commands for uniform looks of C++, Charm++, and Projections
\newcommand{\CC}{C\kern -0.0em\raise 0.5ex\hbox{\normalsize++}}
\newcommand{\emCC}{C\kern -0.0em\raise 0.4ex\hbox{\normalsize\em++}}
\newcommand{\charmpp}{\sc Charm++}
\newcommand{\projections}{\sc Projections}
\newcommand{\converse}{\sc Converse}
\newcommand{\ampi}{\sc AMPI}

%%% Commands to produce margin symbols
\newcommand{\new}{\marginpar{\fbox{\bf$\mathcal{NEW}$}}}
\newcommand{\important}{\marginpar{\fbox{\bf\Huge !}}}
\newcommand{\experimental}{\marginpar{\fbox{\bf\Huge $\beta$}}}

%%% Commands for manual elements
\newcommand{\zap}[1]{ }
\newcommand{\function}[1]{{\noindent{\textsf{#1}}\\}}
\newcommand{\cmd}[1]{{\noindent{\textsf{#1}}\\}}
\newcommand{\args}[1]{\hspace*{2em}{\texttt{#1}}\\}
\newcommand{\param}[1]{{\texttt{#1}}}
\newcommand{\kw}[1]{{\textsf{#1}}}
\newcommand{\uw}[1]{{\textsl{#1}}}
\newcommand{\desc}[1]{\indent{#1}}

%%% Commands needed for Maketitle
\newcommand{\@version}{}
\newcommand{\@credits}{}
\newcommand{\version}[1]{\renewcommand{\@version}{#1}}
\newcommand{\credits}[1]{\renewcommand{\@credits}{#1}}

%%% Print the License Page
\newcommand{\@license}{%
 \begin{center}
   {University of Illinois}\\
   {\charmpp/\converse\ Parallel Programming System Software}\\
   {Non-Exclusive, Non-Commercial Use License}\\
 \end{center}
 \rule{\textwidth}{1pt}
{\tiny
Upon execution of this Agreement by the party identified below (``Licensee''),
The Board of Trustees of the University of Illinois  (``Illinois''), on behalf
of The Parallel Programming Laboratory (``PPL'') in the Department of Computer
Science, will provide the \charmpp/\converse\ Parallel Programming System
software (``\charmpp'') in Binary Code and/or Source Code form (``Software'')
to Licensee, subject to the following terms and conditions. For purposes of
this Agreement, Binary Code is the compiled code, which is ready to run on
Licensee's computer.  Source code consists of a set of files which contain the
actual program commands that are compiled to form the Binary Code.

\begin{enumerate}
  \item
    The Software is intellectual property owned by Illinois, and all right,
title and interest, including copyright, remain with Illinois.  Illinois
grants, and Licensee hereby accepts, a restricted, non-exclusive,
non-transferable license to use the Software for academic, research and
internal business purposes only, e.g. not for commercial use (see Clause 7
below), without a fee.

  \item 
    Licensee may, at its own expense, create and freely distribute
complimentary works that interoperate with the Software, directing others to
the PPL server (\texttt{http://charm.cs.uiuc.edu}) to license and obtain the
Software itself. Licensee may, at its own expense, modify the Software to make
derivative works.  Except as explicitly provided below, this License shall
apply to any derivative work as it does to the original Software distributed by
Illinois.  Any derivative work should be clearly marked and renamed to notify
users that it is a modified version and not the original Software distributed
by Illinois.  Licensee agrees to reproduce the copyright notice and other
proprietary markings on any derivative work and to include in the documentation
of such work the acknowledgement:

\begin{quote}
``This software includes code developed by the Parallel Programming Laboratory
in the Department of Computer Science at the University of Illinois at
Urbana-Champaign.''
\end{quote}

Licensee may redistribute without restriction works with up to 1/2 of their
non-comment source code derived from at most 1/10 of the non-comment source
code developed by Illinois and contained in the Software, provided that the
above directions for notice and acknowledgement are observed.  Any other
distribution of the Software or any derivative work requires a separate license
with Illinois.  Licensee may contact Illinois (\texttt{kale@cs.uiuc.edu}) to
negotiate an appropriate license for such distribution.

  \item
    Except as expressly set forth in this Agreement, THIS SOFTWARE IS PROVIDED
``AS IS'' AND ILLINOIS MAKES NO REPRESENTATIONS AND EXTENDS NO WARRANTIES OF
ANY KIND, EITHER EXPRESS OR IMPLIED, INCLUDING BUT NOT LIMITED TO WARRANTIES OR
MERCHANTABILITY OR FITNESS FOR A PARTICULAR PURPOSE, OR THAT THE USE OF THE
SOFTWARE WILL NOT INFRINGE ANY PATENT, TRADEMARK, OR OTHER RIGHTS.  LICENSEE
ASSUMES THE ENTIRE RISK AS TO THE RESULTS AND PERFORMANCE OF THE SOFTWARE
AND/OR ASSOCIATED MATERIALS.  LICENSEE AGREES THAT UNIVERSITY SHALL NOT BE HELD
LIABLE FOR ANY DIRECT, INDIRECT, CONSEQUENTIAL, OR INCIDENTAL DAMAGES WITH
RESPECT TO ANY CLAIM BY LICENSEE OR ANY THIRD PARTY ON ACCOUNT OF OR ARISING
FROM THIS AGREEMENT OR USE OF THE SOFTWARE AND/OR ASSOCIATED MATERIALS.

  \item 
    Licensee understands the Software is proprietary to Illinois. Licensee
agrees to take all reasonable steps to insure that the Software is  protected
and secured from unauthorized disclosure, use, or release and  will treat it
with at least the same level of care as Licensee would use to  protect and
secure its own proprietary computer programs and/or information, but using no
less than a reasonable standard of care.  Licensee agrees to provide the
Software only to any other person or entity who has registered with Illinois.
If licensee is not registering as an individual but as an institution or
corporation each member of the institution or corporation who has access to or
uses Software must agree to and abide by the terms of this license. If Licensee
becomes aware of any unauthorized licensing, copying or use of the Software,
Licensee shall promptly notify Illinois in writing. Licensee expressly agrees
to use the Software only in the manner and for the specific uses authorized in
this Agreement.

  \item
    By using or copying this Software, Licensee agrees to abide by the
copyright law and all other applicable laws of the U.S. including, but not
limited to, export control laws and the terms of this license. Illinois  shall
have the right to terminate this license immediately by written  notice upon
Licensee's breach of, or non-compliance with, any terms of the license.
Licensee may be held legally responsible for any  copyright infringement that
is caused or encouraged by its failure to  abide by the terms of this license.
Upon termination, Licensee agrees to  destroy all copies of the Software in its
possession and to verify such  destruction in writing.

  \item
  The user agrees that any reports or published results obtained with  the
Software will acknowledge its use by the appropriate citation as  follows:

\begin{quote}
``\charmpp/\converse\ was developed by the Parallel Programming Laboratory in
the Department of Computer Science at the University of  Illinois at
Urbana-Champaign.''
\end{quote}

Any published work which utilizes \charmpp\ shall include the following
reference:

\begin{quote}
``L. V. Kale and S. Krishnan. \charmpp: Parallel Programming with Message-Driven
Objects. In 'Parallel Programming using \CC' (Eds. Gregory V. Wilson and Paul
Lu), pp 175-213, MIT Press, 1996.''
\end{quote}

Any published work which utilizes \converse\ shall include the following
reference:

\begin{quote}
``L. V. Kale, Milind Bhandarkar, Narain Jagathesan, Sanjeev Krishnan and Joshua
Yelon. \converse: An Interoperable Framework for Parallel Programming.
Proceedings of the 10th International Parallel Processing Symposium, pp
212-217, April 1996.''
\end{quote}

Electronic documents will include a direct link to the official \charmpp\ page
at \texttt{http://charm.cs.uiuc.edu/}

  \item
    Commercial use of the Software, or derivative works based thereon,
REQUIRES A COMMERCIAL LICENSE.  Should Licensee wish to make commercial use of
the Software, Licensee will contact Illinois (kale@cs.uiuc.edu) to negotiate an
appropriate license for such use. Commercial use includes: 

    \begin{enumerate}
      \item
	integration of all or part of the Software into a product for sale,
lease or license by or on behalf of Licensee to third parties, or 

      \item
	distribution of the Software to third parties that need it to
commercialize product sold or licensed by or on behalf of Licensee.
    \end{enumerate}

  \item
    Government Rights. Because substantial governmental funds have been  used
in the development of \charmpp/\converse, any possession, use or sublicense of
the Software by or to the United States government shall be subject to such
required restrictions.

  \item
    \charmpp/\converse\ is being distributed as a research and teaching tool
and as such, PPL encourages contributions from users of the code that might, at
Illinois' sole discretion, be used or incorporated to make the basic  operating
framework of the Software a more stable, flexible, and/or useful  product.
Licensees who contribute their code to become an internal  portion of the
Software agree that such code may be distributed by  Illinois under the terms
of this License and may be required to sign an  ``Agreement Regarding
Contributory Code for \charmpp/\converse\ Software'' before Illinois  can
accept it (contact \texttt{kale@cs.uiuc.edu} for a copy).
\end{enumerate}

UNDERSTOOD AND AGREED.

Contact Information:

The best contact path for licensing issues is by e-mail to
\texttt{kale@cs.uiuc.edu} or send correspondence to:

\begin{quote}
Prof. L. V. Kale\\
Dept. of Computer Science\\
University of Illinois\\
1304 W. Springfield Ave\\
Urbana, Illinois 61801 USA\\
FAX: (217) 333-3501
\end{quote}
}%tiny
 \newpage
}% end of license

\renewcommand{\maketitle}{\begin{titlepage}%
 \begin{flushright}
   {\Large
     Parallel Programming Laboratory\\
     University of Illinois at Urbana-Champaign\\
   }
 \end{flushright}
 \rule{\textwidth}{3pt}
 \vspace{\fill}
 \begin{flushright}
   \textsf{\Huge \@title \\}
 \end{flushright}
 \vspace{\fill}
 \@credits \\
 \rule{\textwidth}{3pt}
 \begin{flushright}
   {\large Version \@version}
 \end{flushright}
 \end{titlepage}
 \@license

 \tableofcontents
 \newpage
}% maketitle



\makeindex

\title{\charmpp\\ Iterative Finite Element Matrix (IFEM) Library\\ Manual}
\version{1.2}
\credits{
Initial version of \charmpp{} Finite Element Framework was developed
by Orion Lawlor in the spring of 2003.
}

\begin{document}

\maketitle

\section{Introduction}

This manual presents the Iterative Finite Element Matrix (IFEM) library,
a library for easily solving matrix problems derived from 
finite-element formulations.  The library is designed to be matrix-free,
in that the only matrix operation required is matrix-vector product,
and hence the entire matrix need never be assembled

IFEM is built on the mesh and communication capabilities of the Charm++ 
FEM Framework, so for details on the basic runtime, problem setup, and 
partitioning see the FEM Framework manual.


\subsection{Terminology}

A FEM program manipulates elements and nodes. An \term{element} is a portion of
the problem domain, also known as a cell, and is typically some simple shape 
like a triangle, square, or hexagon in 2D; or tetrahedron or rectangular solid in 3D.  
A \term{node} is a point in the domain, and is often the vertex of several elements.  
Together, the elements and nodes form a \term{mesh}, which is the 
central data structure in the FEM framework.  See the FEM manual for details.

% --------------- Solvers -----------------
\section{Solvers}
\label{sec:solver}
A IFEM \term{solver} is a subroutine that controls the search for the solution.

Solvers often take extra parameters, which are listed in a type called 
in C \kw{ILSI\_Param}, which in Fortran is an array of \kw{ILSI\_PARAM} doubles.  
You initialize these solver parameters using the subroutine \kw{ILSI\_Param\_new}, 
which takes the parameters as its only argument.  The input and output
parameters in an \kw{ILSI\_Param} are listed in Table~\ref{table:solverIn}
and Table~\ref{table:solverOut}.

\begin{table}[hh]
\begin{center}
\begin{tabular}{|l|l|l|}\hline
C Field Name & Fortran Field Offset & Use\\\hline
maxResidual & 1 & If nonzero, termination criteria: magnitude of residual. \\
maxIterations & 2 & If nonzero, termination criteria: number of iterations. \\
solverIn[8] & 3-10 & Solver-specific input parameters. \\
\hline
\end{tabular}
\end{center}
\caption{\kw{ILSI\_Param} solver input parameters.}
\label{table:solverIn}
\end{table}

\begin{table}[hh]
\begin{center}
\begin{tabular}{|l|l|l|}\hline
C Field Name & Fortran Field Offset & Use\\\hline
residual & 11 & Magnitude of residual of final solution. \\
iterations & 12 & Number of iterations actually taken. \\
solverOut[8] & 13-20 & Solver-specific output parameters. \\
\hline
\end{tabular}
\end{center}
\caption{\kw{ILSI\_Param} solver output parameters.}
\label{table:solverOut}
\end{table}


\subsection{Conjugate Gradient Solver}

The only solver currently written using IFEM is the conjugate gradient solver.
This linear solver requires the matrix to be real, symmetric and positive definite.

Each iteration of the conjugate gradient solver requires one matrix-vector product
and two global dot products.  For well-conditioned problems, the solver typically 
converges in some small multiple of the diameter of the mesh--the number of elements
along the largest side of the mesh.

You access the conjugate gradient solver via the subroutine name \kw{ILSI\_CG\_Solver}.


% -------------- Shared-node ----------------
\section{Solving Shared-Node Systems}

Many problems encountered in FEM analysis place the entries of the
known and unknown vectors at the nodes--the vertices of the domain.
Elements provide linear relationships between the known and unknown node values, 
and the entire matrix expresses the combination of all these element relations.

For example, in a structural statics problem, we know the net force at
each node, $f$, and seek the displacements of each node, $u$.  Elements 
provide the relationship, often called a stiffness matrix $K$, between 
a nodes' displacments and its net forces:

\[
	f=K u
\]

We normally label the known vector $b$ (in the example, the forces), the unknown
vector $x$ (in the example, the displacments), and the matrix $A$:

\[
	b=A x
\]


IFEM provides two routines for solving problems of this type.  The first routine, \kw{IFEM\_Solve\_shared}, solves for the entire $x$ vector based on the known values of the $b$ vector.  The second, \kw{IFEM\_Solve\_shared\_bc}, allows certain entries in the $x$ vector to be given specific values before the problem is solved, creating values for the $b$ vector.


\newpage
\subsection{IFEM\_Solve\_shared}

\prototype{IFEM\_Solve\_shared}
\function{void IFEM\_Solve\_shared(ILSI\_Solver s,ILSI\_Param *p,
	int fem\_mesh, int fem\_entity,int length,int width,
	IFEM\_Matrix\_product\_c A, void *ptr, const double *b, double *x);}
\function{subroutine IFEM\_Solve\_shared(s,p,
	fem\_mesh,fem\_entity,length,width,
	A,ptr,b,x)}
	\args{external solver subroutine :: s}
	\args{double precision, intent(inout) :: p(ILSI\_PARAM)}
	\args{integer, intent(in) :: fem\_mesh, fem\_entity, length,width}
	\args{external matrix-vector product subroutine :: A}
	\args{TYPE(varies), pointer :: ptr }
        \args{double precision, intent(in) :: b(width,length)}
        \args{double precision, intent(inout) :: x(width,length)}

This routine solves the linear system $A x = b$ for the unknown vector $x$.  \uw{s} and \uw{p} give the particular linear solver to use, and are described in more detail in Section~\ref{sec:solver}.  \uw{fem\_mesh} and \uw{fem\_entity} give the FEM framework mesh (often \kw{FEM\_Mesh\_default\_read()}) and entity (often \kw{FEM\_NODE}) with which the known and unknown vectors are listed.

\uw{width} gives the number of degrees of freedom (entries in the vector) per node. For example, if there is one degree of freedom per node, width is one.  \uw{length} should always equal the number of FEM nodes.

\uw{A} is a local matrix-vector product routine you must write.  Its interface is described in Section~\ref{sec:mvp}. \uw{ptr} is a pointer passed down to \uw{A}--it is not otherwise used by the framework.  

\uw{b} is the known vector.  \uw{x}, on input, is the initial guess for the unknown vector.  On output, \uw{x} is the final value for the unknown vector.  \uw{b} and \uw{x} should both have length * width entries.  In C, DOF $i$ of node $n$ should be indexed as $x[n*$\uw{width}$+i]$.  In Fortran, these arrays should be allocated like \uw{x(width,length)}.

When this routine returns, \uw{x} is the final value for the unknown vector, and the output values of the solver parameters \uw{p} will have been written.

\begin{alltt}
// C++ Example
  int mesh=FEM_Mesh_default_read();
  int nNodes=FEM_Mesh_get_length(mesh,FEM_NODE);
  int width=3; //A 3D problem
  ILSI_Param solverParam;
  struct myProblemData myData;
  
  double *b=new double[nNodes*width];
  double *x=new double[nNodes*width];
  ... prepare solution target b and guess x ...
  
  ILSI_Param_new(&solverParam);
  solverParam.maxResidual=1.0e-4;
  solverParam.maxIterations=500; 
  
  IFEM_Solve_shared(IFEM_CG_Solver,&solverParam,
         mesh,FEM_NODE, nNodes,width,
         myMatrixVectorProduct, &myData, b,x);
  
! F90 Example
  include 'ifemf.h'
  INTEGER :: mesh, nNodes,width
  DOUBLE PRECISION, ALLOCATABLE :: b(:,:), x(:,:)
  DOUBLE PRECISION :: solverParam(ILSI_PARAM)
  TYPE(myProblemData) :: myData
  
  mesh=FEM_Mesh_default_read()
  nNodes=FEM_Mesh_get_length(mesh,FEM_NODE)
  width=3   ! A 3D problem
  
  ALLOCATE(b(width,nNodes), x(width,nNodes))
  ... prepare solution target b and guess x ..
  
  ILSI_Param_new(&solverParam);
  solverParam(1)=1.0e-4;
  solverParam(2)=500; 
  
  IFEM_Solve_shared(IFEM_CG_Solver,solverParam,
         mesh,FEM_NODE, nNodes,width,
         myMatrixVectorProduct, myData, b,x);

\end{alltt}

\newpage
\subsubsection{Matrix-vector product routine}
\label{sec:mvp}

IFEM requires you to write a matrix-vector product routine that will evaluate $A x$ for various vectors $x$.  You may use any subroutine name, but it must take these arguments:

\prototype{IFEM\_Matrix\_product}
\function{void IFEM\_Matrix\_product(void *ptr,int length,int width,
	const double *src, double *dest);}
\function{subroutine IFEM\_Matrix\_product(ptr,length,width,src,dest)}
  \args{TYPE(varies), pointer  :: ptr}
  \args{integer, intent(in)  :: length,width}
  \args{double precision, intent(in) :: src(width,length)}
  \args{double precision, intent(out) :: dest(width,length)}

The framework calls this user-written routine when it requires a matrix-vector product.  This routine should compute $dest = A \, src$, interpreting $src$ and $dest$ as vectors.  \uw{length} gives the number of nodes and \uw{width} gives the number of degrees of freedom per node, as above.

In writing this routine, you are responsible for choosing a representation for the matrix $A$. For many problems, there is no need to represent $A$ explicitly--instead, you simply evaluate $dest$ by looping over local elements, taking into account the values of $src$.  This example shows how to write the matrix-vector product routine for simple 1D linear elastic springs, while solving for displacement given net forces.

After calling this routine, the framework will handle combining the overlapping portions of these vectors across processors to arrive at a consistent global matrix-vector product.

\begin{alltt}
// C++ Example
#include "ifemc.h"

typedef struct \{
  int nElements; //Number of local elements
  int *conn; // Nodes adjacent to each element: 2*nElements entries
  double k; //Uniform spring constant
\} myProblemData;

void myMatrixVectorProduct(void *ptr,int nNodes,int dofPerNode,
          const double *src,double *dest) 
\{
  myProblemData *d=(myProblemData *)ptr;
  int n,e;
  // Zero out output force vector:
  for (n=0;n<nNodes;n++) dest[n]=0;
  // Add in forces from local elements
  for (e=0;e<d->nElements;e++) {
    int n1=d->conn[2*e+0]; // Left node
    int n2=d->conn[2*e+1]; // Right node
    double f=d->k * (src[n2]-src[n1]); //Force
    dest[n1]+=f;
    dest[n2]-=f;
  }
\}


! F90 Example
  TYPE(myProblemData) 
    INTEGER :: nElements
    INTEGER, ALLOCATABLE :: conn(2,:)
    DOUBLE PRECISION :: k
  END TYPE
  
SUBROUTINE myMatrixVectorProduct(d,nNodes,dofPerNode,src,dest)
  include 'ifemf.h'
  TYPE(myProblemData), pointer :: d
  INTEGER :: nNodes,dofPerNode
  DOUBLE PRECISION :: src(dofPerNode,nNodes), dest(dofPerNode,nNodes)
  INTEGER :: e,n1,n2
  DOUBLE PRECISION :: f
  
  dest(:,:)=0.0
  do e=1,d%nElements
    n1=d%conn(1,e)
    n2=d%conn(2,e)
    f=d%k * (src(1,n2)-src(1,n1))
    dest(1,n1)=dest(1,n1)+f
    dest(1,n2)=dest(1,n2)+f
  end do
END SUBROUTINE
\end{alltt}


\newpage
\subsection{IFEM\_Solve\_shared\_bc}

\prototype{IFEM\_Solve\_shared\_bc}
\function{void IFEM\_Solve\_shared\_bc(ILSI\_Solver s,ILSI\_Param *p,
	int fem\_mesh, int fem\_entity,int length,int width,
	int bcCount, const int *bcDOF, const double *bcValue,
	IFEM\_Matrix\_product\_c A, void *ptr, const double *b, double *x);}
\function{subroutine IFEM\_Solve\_shared\_bc(s,p,
	fem\_mesh,fem\_entity,length,width,
	bcCount,bcDOF,bcValue,
	A,ptr,b,x)}
	\args{external solver subroutine :: s}
	\args{double precision, intent(inout) :: p(ILSI\_PARAM)}
	\args{integer, intent(in) :: fem\_mesh, fem\_entity, length,width}
	\args{integer, intent(in) :: bcCount}
	\args{integer, intent(in) :: bcDOF(bcCount)}
	\args{double precision, intent(in) :: bcValue(bcCount)}
	\args{external matrix-vector product subroutine :: A}
	\args{TYPE(varies), pointer :: ptr }
        \args{double precision, intent(in) :: b(width,length)}
        \args{double precision, intent(inout) :: x(width,length)}

Like \kw{IFEM\_Solve\_shared}, this routine solves the linear system $A x = b$ for the unknown vector $x$.  This routine, however, adds support for boundary conditions associated with $x$. These so-called "essential" boundary conditions restrict the values of some unknowns. For example, in structural dynamics, a fixed displacment is such an essential boundary condition.  

The only form of boundary condition currently supported is to impose a fixed value on certain unknowns, listed by their degree of freedom--that is, their entry in the unknown vector.  In general, the $i$'th DOF of node $n$ has DOF number $n*width+i$ in C and $(n-1)*width+i$ in Fortran.  The framework guarantees that, on output, for all $bcCount$ boundary conditions, $x(bcDOF(f))=bcValue(f)$.  

For example, if $width$ is 3 in a 3d problem, we would set node $ny$'s y coordinate to 4.6 and node $nz$'s z coordinate to 7.3 like this:

\begin{alltt}
// C++ Example
  int bcCount=2;
  int bcDOF[bcCount];
  double bcValue[bcCount];
  // Fix node ny's y coordinate
  bcDOF[0]=ny*width+1; // y is coordinate 1
  bcValue[0]=4.6;
  // Fix node nz's z coordinate
  bcDOF[1]=nz*width+2; // z is coordinate 2
  bcValue[1]=2.0;

! F90 Example
// C++ Example
  integer :: bcCount=2;
  integer :: bcDOF(bcCount);
  double precision :: bcValue(bcCount);
  // Fix node ny's y coordinate
  bcDOF(1)=(ny-1)*width+2; // y is coordinate 2
  bcValue(1)=4.6;
  // Fix node nz's z coordinate
  bcDOF(2)=(nz-1)*width+3; // z is coordinate 3
  bcValue(2)=2.0;
\end{alltt}



Mathematically, what is happening is we are splitting the partially unknown vector $x$ into a completely unknown portion $y$ and a known part $f$:
\[ A x = b \]
\[ A (y + f) = b \]
\[ A y = b - A f \]

We can then define a new right hand side vector $c=b-A f$ and solve the new linear system $A y=c$ normally.  Rather than renumbering, we do this by zeroing out the known portion of $x$ to make $y$.  The creation of the new linear system, and the substitution back to solve the original system are all done inside this subroutine.

One important missing feature is the ability to specify general linear constraints on the unknowns, rather than imposing specific values.


\section{Index}
\input{index}
\end{document}
