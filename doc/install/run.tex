\section{Executing \converse{}/\charmpp{} Programs}
\label{executing charm programs}

The charmc linker produces one executable file.  On most machines
a link to the proper host program
{\tt charmrun} is created in the user program directory.  Sample
execution example is given below (the executable is called {\tt
pgm}). Exact details will differ from site to site.  The list of {\tt charmrun}
command line options is in Section~\ref{command line options}.

\begin{alltt}
charmrun pgm +p4
\end{alltt}

Executes {\tt pgm} on 4 nodes.  In a network environment, {\tt charmrun} must
be able to locate the directory of the executable.  If all workstations
share a common file name space this is trivial.  If they don't, {\tt charmrun}
will attempt to find the executable in a directory with the same path
from the {\bf \$HOME} directory.  Pathname resolution is performed as 
follows:
\begin{enumerate}
	\item The system computes the absolute path of {\tt pgm}.
	\item If the absolute path starts with the equivalent of {\bf \$HOME} 
	or the current working directory, the beginning part of the 
        path 
	is replaced with the environment variable {\bf \$HOME} or the 
	current working directory. However, if {\tt exec\_home} is 
        specified in the nodes file (see below), the beginning part of
        the path is replaced with {\tt exec\_home}.
	\item The system tries to locate this program (with modified 
	pathname and appended extension if specified) on all nodes.
\end{enumerate}

For network of workstations,
the list of nodes must be specified in a file.  The format of this file
allows you to define groups of machines, giving each group a name.
Each line of the nodes file is a command.  The most important command
is:

\begin{alltt}
host <hostname> <qualifiers>
\end{alltt}

which specifies a host.  The other commands are qualifiers: they modify
the properties of all hosts that follow them.  The qualifiers are:


\begin{tabbing}
{\tt group <groupname>}~~~\= - subsequent hosts are members of specified group\\
{\tt login <login>  }     \> - subsequent hosts use the specified login\\
{\tt shell <shell>  }     \> - subsequent hosts use the specified remote 
shell\\
%{\tt passwd <passwd>}     \> - subsequent hosts use the specified password\\
{\tt setup <cmd>  }       \> - subsequent hosts should execute cmd\\
{\tt home <dir> }         \> - subsequent hosts should find programs under dir\\
{\tt cpus <n>}            \> - subsequent hosts should use N light-weight processes\\
{\tt speed <s>}           \> - subsequent hosts have relative speed rating\\
{\tt ext <extn>}          \> - subsequent hosts should append extn to the pgm name\\
\end{tabbing}

{\bf Note:}
By default, charmrun uses a remote shell ``rsh'' to spawn node processes
on the remote hosts. The {\tt shell} qualifier can be used to override
it with say, ``ssh''. One can set the {\tt CONV\_RSH} environment variable
or use charmrun option {\tt ++remote-shell} to override the default remote 
shell for all hosts with unspecified {\tt shell} qualifier.

All qualifiers accept ``*'' as an argument, this resets the modifier to
its default value.  Note that currently, the passwd, cpus, and speed
factors are ignored.  Inline qualifiers are also allowed:

\begin{alltt}
host beauty ++cpus 2 ++shell ssh
\end{alltt}

Except for ``group'', every other qualifier can be inlined, with the
restriction that if the ``setup'' qualifier is inlined, it should be
the last qualifier on the ``host'' or ``group'' statement line.

Here is a simple nodes file:

\begin{alltt}
        group kale-sun ++cpus 1
          host charm.cs.uiuc.edu ++shell ssh
          host dp.cs.uiuc.edu
          host grace.cs.uiuc.edu
          host dagger.cs.uiuc.edu
        group kale-sol
          host beauty.cs.uiuc.edu ++cpus 2
        group main
          host localhost
\end{alltt}

This defines three groups of machines: group kale-sun, group kale-sol,
and group main.  The ++nodegroup option is used to specify which group
of machines to use.  Note that there is wraparound: if you specify
more nodes than there are hosts in the group, it will reuse
hosts. Thus,

\begin{alltt}
        charmrun pgm ++nodegroup kale-sun +p6
\end{alltt}

uses hosts (charm, dp, grace, dagger, charm, dp) respectively as
nodes (0, 1, 2, 3, 4, 5).

If you don't specify a ++nodegroup, the default is ++nodegroup main.
Thus, if one specifies

\begin{alltt}
        charmrun pgm +p4
\end{alltt}

it will use ``localhost'' four times.  ``localhost'' is a Unix
trick; it always find a name for whatever machine you're on.

Since the new nodes file is incompatible with the old nodes file, it has
been renamed.  It now is called ``.nodelist'', and all the options and
environment variables pertaining to it have also been renamed {\tt NODELIST}.

The user is required to set up remote login permissions on all nodes
using the ``.rhosts'' file in the home directory if ``rsh'' is used for remote
login into the hosts. If ``ssh'' is used, the user will have to setup
password-less login to remote hosts either using ``.shosts'' file, or using
RSA authentication based on a key-pair and adding public keys to ``.ssh/authorized\_keys'' file. See ``ssh'' documentation for more information.

Note that the charmc linker will provide the correct 
executable. The user, however, needs to know how programs are run for
the particular machine.

\subsection{Running with the simulator}

\converse{} provides a simple parallel machine simulator for developing
and debugging purposes.  It simulates a message passing system
composed of a collection of processing nodes connected with a
communication network. Each node is composed of an application
processor, local memory, and a communication coprocessor.  The
simulator is a beta version, and it is not yet proven that the
simulator timers for performance measurements produce realistic
results.


In order to run \charmpp{} programs with the simulator:
\begin{itemize}

\item prepare a configuration file as described below

\item to run, type \verb#pgm +pN# (and possibly other runtime options) where
   N is the number of processors.

\end{itemize}
Currently only \charmpp{} programs can take advantage of the
simulator features.
In the future, a method to allow any \converse{} based program
to use the simulator features will be devised.

The basic task of the simulator is to manage the message passing
obeying various machine and network parameters.  A message experiences
delays in various components of the machine. These include: 1) sender
application processor, 2) sender communication coprocesssor, 3)
network, 4) receiver communication processor, and 5) receiver
application processor.  Each component of the delayed is modelled by
the widely used formula $\alpha + n\beta$ where $\alpha$ is the
startup cost, and $\beta$ is the cost per byte.  In addition to
message delay parameters, there are others related to the network
capacity and random variations in network delays. These parameters are
specified in a configuration file named "sim.param" in the directory
of the user program. If the simulator can't find this file, it assumes
default values (mostly zero latencies).  Figure~\ref{fig:simconfig}
lists a sample configuration. The lines starting with the \# sign are
treated as comments. Each line contains a keyword followed by some
numbers. The explanation of each keyword is given below:

\begin{description}

\item[{\tt cpu\_recv\_cost}] $\alpha$ and  $\beta$ values  for the software
                            cost of a message-receive at the application
                            processor.

\item[{\tt cpu\_send\_cost}] $\alpha$ and  $\beta$ values  for the software
                            cost of a message-send at the application
                            processor.

\item[{\tt rcp\_cost}] $\alpha$ and  $\beta$ values for a message-receive 
                       at the communication processor.

\item[{\tt scp\_cost}] $\alpha$ and  $\beta$ values for a message-send
                       at the communication processor.

\item[{\tt net\_cost}] $\alpha$ and  $\beta$ values for a message-send
                       in the netowrk.

\item[{\tt cpu\_queue\_threshold\_number}] max number of messages queued
                       at the application processors's incoming message queue.

\item[{\tt cpu\_queue\_threshold\_size}] max cumulative size of
		       messages in bytes queued at the application
		       processors's incoming message queue.


\item[{\tt cpu\_queue\_threshold\_number}] max number of messages in the incoming
                       message queue of communication processor.


\item[{\tt rcp\_queue\_threshold\_number}] max number of messages in the 
                       incoming-message-queue of communication processors.                    
\item[{\tt rcp\_queue\_threshold\_size}] max cumulative size of messages in bytes
                       in the incoming-message-queue of communication 
                       processors.

\item[{\tt net\_queue\_threshold\_number}] max number of transient messages in 
                       the network.

\item[{\tt net\_queue\_threshold\_size}] max cumulative size of transient 
                       messages in bytes in the network.

\item[{\tt latency-fixed}] no random variations in the network latency 
                           ($\alpha$)

\item[{\tt latency-rand}] network latency ($\alpha$) is incremented by
                       a random value distributed exponentially. The first
                       number after the keyword is the mean of the
                       exponential distribution. The second number is the
                       initial seed vbalue for the random number generator.


\item[{\tt processor\_scale}] The simulator scales the measured time
                      execution of code-blocks by this value.

\item[{\tt periodic\_interval}] Converse has periodic checks for
                      various purposes. This is the time on seconds
                      those checks are called.
\end{description}


\begin{figure}
\begin{verbatim}
#latency parameters
cpu_recv_cost 1E-6 1E-7              
cpu_send_cost 1E-6 1E-7
rcp_cost      1E-3 1E-7
scp_cost      1E-6 1E-7
net_cost      1E-6 1E-7


#capacity parameters
# choose one 
cpu_nolimit
#cpu_queue_threshold_number 100000
#cpu_queue_threshold_size   100000


#choose one
scp_nolimit
#scp_queue_threshold_number 100000
#scp_queue_threshold_size   100000

#choose one
rcp_net_nolimit
#rcp_queue_threshold_number 100000
#rcp_queue_threshold_size   100000
#net_queue_threshold_number 100000
#net_queue_threshold_size   100000

#random variations in latency
#choose one
latency-fixed
#latency-rand   0.0001 123456

processor_scale 1.0
periodic_interval 0.1
\end{verbatim}
\caption{A sample configuration file for the simulator}
\label{fig:simconfig}
\end{figure}

\subsection[Command Line Options]{Command Line Options}
\label{command line options}
\index{command line options}

A \charmpp{} program accepts the following command line options:
\begin{description}

\item[{\tt +pN}] Run the program with N processors. The default is 1.

\item[{\tt +ss}] Print summary statistics about chare creation.  This option
prints the total number of chare creation requests, and the total number of
chare creation requests processed across all processors.

\item[{\tt +cs}] Print statistics about the number of create chare messages
requested and processed, the number of messages for chares requested and 
processed, and the number of messages for branch office chares requested and
processed, on a per processor basis.  Note that the number of messages 
created and processed for a particular type of message on a given node 
may not be the same, since a message may be processed by a different
processor from the one originating the request.

\item[{\tt user\_options}] Options that are be interpreted by the user
program may be included after all the system options. 
However, {\tt user\_options} cannot start with +.
The {\tt user\_options} will be passed as arguments to the user program 
via the usual {\tt argc/argv} construct to the {\tt main}
entry point of the main chare. 
\charmpp{} system options will not appear in {\tt argc/argv}.

\end{description}

\subsubsection{Additional Uniprocessor Command Line Options}
\label{uniprocessor command line options}

The uniprocessor versions can be used to simulate multiple
processors on a single workstation
Any number of processors between 1 and 32 can be simulated by
using the {\tt +p} option, limited only by the available memory on the
uniprocessor workstation.  By default, the uniprocessor versions handle
a single message from each processor, going in order from processor 0
thru $P-1$ (where $P$ is the number of processors) repeatedly.  

\subsubsection{Additional Network Command Line Options}
\label{network command line options}

The following {\tt ++} command line options are available in
the network version:
\begin{description}

\item[{\tt ++debug}] Run each node under gdb in an xterm window, prompting
the user to begin execution.

\item[{\tt ++debug-no-pause}] Run each node under gdb in an xterm window
immediately (i.e. without prompting the user to begin execution).

\item[{\tt ++maxrsh}] Maximum number of {\tt rsh}'s to run at a
time.

\item[{\tt ++resend-wait}] Timeout before retransmitting datagrams
(in msec).

\item[{\tt ++resend-fail}] Timeout before retransmission fails (in
msec).
This parameter can help the user kill ``runaway'' processes, which may not
be killed otherwise when the user interrupts the program before it 
completes execution.
Currently a bug exists in the network version that may cause programs to
terminate prematurely if this value is set too low and {\tt scanf} 
operations are being performed.

\item[{\tt ++nodelist}] File containing list of nodes.

\end{description}

If using the {\tt ++debug} option, the user must ensure the
following:
\begin{enumerate}

\item {\tt xterm}, {\tt xdpyinfo},  and {\tt gdb} must be in
the user's path.

\item The path must be set in the {\tt .cshrc} file, not the {\tt .login}
file, because {\tt rsh} does not run the {\tt .login} file. 

\item The nodes must be authorized to create windows on the host machine (see
man pages for {\tt xhost} and {\tt xauth}).

\end{enumerate}
