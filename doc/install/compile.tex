\section{Compiling \charmpp{} Programs}

The {\tt charmc} program, located in ``charm/bin'', standardizes 
compiling and linking procedures
among various machines and operating systems.  ``charmc'' is
a general-purpose tool for compiling and
linking, not only restricted to \charmpp{} programs.

Charmc can perform the following tasks.  The (simplified) syntax for
each of these modes is shown.  Caution: in reality, one almost always
has to add some command-line options in addition to the simplified
syntax shown below.  The options are described next.

\begin{alltt}
 * Compile C                        charmc -o pgm.o pgm.c
 * Compile C++                      charmc -o pgm.o pgm.C
 * Link                             charmc -o pgm   obj1.o obj2.o obj3.o...
 * Compile + Link                   charmc -o pgm   src1.c src2.ci src3.C
 * Create Library                   charmc -o lib.a obj1.o obj2.o obj3.o...
 * CPM preprocessing                charmc -gen-cpm file.c
 * Translate Charm++ Interface File charmc file.ci
\end{alltt}

Charmc automatically figures out where the charm lib and include
directories are --- at no point do you have to configure this
information.  However, the code that finds the lib and include
directories can be confused if you remove charmc from its normal
directory, or rearrange the directory tree.  Thus, the files in the
\kw{charm/bin}, \kw{charm/include}, and \kw{charm/lib} directories 
must be left where they are relative to each other.  

The following command-line options are available to users of charmc:

\begin{description}

\item[{\tt -o} {\em output-file}:]

Output file name.  Note: charmc only ever produces one output file at
a time.  Because of this, you cannot compile multiple source files at
once, unless you then link or archive them into a single output-file.
If exactly one source-file is specified, then an output file will be
selected by default using the obvious rule (eg, if the input file if
pgm.c, the output file is pgm.o).  If multiple input files are
specified, you must manually specify the name of the output file,
which must be a library or executable.

\item[{\tt -c}:]

Ignored.  There for compatibility with {\tt cc}.

\item[{\tt -D{\em symbol}[={\em value}]}:]

Defines preprocessor variables from the command line at compile time.

\item[{\tt -I}:]

Add a directory to the search path for preprocessor include files.

\item[{\tt -g}:]

Causes compiled files to include debugging information.

\item[{\tt -L*}:]

Add a directory to the search path for libraries selected by
the {\tt -l} command.

\item[{\tt -l*}:]

Specifies libraries to link in.

\item[{\tt -module {\em m1}[,{\em m2}[,...]]}]
Specifies additional \charmpp{} modules to link in. 
Similar to {\tt -l}, but also registers \charmpp{} parallel objects.
See the library's documentation for whether to use {\tt -l} or {\tt -module}.

\item[{\tt -optimize}:]

Causes files to be compiled with maximum optimization.

\item[{\tt -no-optimize}:]

If this follows -O on the command line, it turns optimization back off.
This is just a convenience for simple-minded makefiles.

\item[{\tt -s}:]

Strip the executable of debugging symbols.  Only meaningful when
producing an executable.

\item[{\tt -verbose}:]

All commands executed by charmc are echoed to stdout.

\item[{\tt -seq}:]

Indicates that we're compiling sequential code.  On parallel machines
with front ends, this option also means that the code is for the front
end.  This option is only valid with C and C++ files.

\item[{\tt -use-fastest-cc}:]

Some environments provide more than one C compiler (cc and gcc, for
example).  Usually, charmc prefers the less buggy of the two.  This
option causes charmc to switch to the most aggressive compiler,
regardless of whether it's buggy or not.

\item[{\tt -use-reliable-cc}:]

Some environments provide more than one C compiler (cc and gcc, for
example).  Usually, charmc prefers the less buggy of the two, but
not always.  This option causes charmc to switch to the most reliable
compiler, regardless of whether it produces slow code or not.

\item[{\tt -language \{converse|charm++|sdag|ampi|fem|f90charm\}}:]

When linking with charmc, one must specify the ``language''.  This
is just a way to help charmc include the right libraries.  Pick the
``language'' according to this table:

\begin{itemize}
\item{{\bf Charm++} if your program includes \charmpp{}, C++, and C.}
\item{{\bf Converse} if your program includes C or C++.}
\item{{\bf sdag} if your program includes structured dagger.}
\item{{\bf f90charm} if your program includes f90 Charm interface.}
\end{itemize}

\item[{\tt -balance} {\em seed load-balance-strategy}:]

When linking any \converse{} program (including any \charmpp{} or sdag program),
one must include a seed load-balancing
library.  There are currently three to choose from: {\tt rand}, {\tt
test}, and {\tt neighbor} are supported.  Default is {\tt -balance rand}.

When linking with {\tt neighbor} seed load balancer, one can also specify
a virtual tolpogy for constructing neighbors during run-time using 
{\tt +LBTopo topo}, where {\em topo} can be one of (a) ring, (b) mesh2d,
(c) mesh3d and (d) graph. The default is mesh2d.

\item[{\tt -tracemode} {\em tracing-mode}:]

Selects the desired degree of tracing for \charmpp{} programs.
See the \charmpp{} manual and the \projections{} manuals for
more information.  Currently supported modes are {\tt none}, {\tt
summary}, and {\tt projections}. Default is {\tt -tracemode none}.


\item[{\tt -memory} {\em memory-mode}:]
Selects the implementation of \kw{malloc} and \kw{free}
to use.  Select a memory mode from the table below.

\begin{itemize}
\item{{\bf os} Use the operating system's standard memory routines.}
\item{{\bf gnu} Use a set of GNU memory routines.}
\item{{\bf paranoid} Use an error-checking set of routines.
These routines will detect common mistakes such as buffer
overruns, underruns, double-deletes, and use-after-delete.
The extra checks slow down programs, so this version should
not be used in production code.}
\item{{\bf verbose} Use a tracing set of memory routines.
Every memory-related call results in a line printed to 
standard out. This version is useful for detecting memory leaks.}
\item{{\bf default} Use the default, which depends on the version of \charmpp.}
\end{itemize}


\item[{\tt -c++} {\em C++ compiler}:]

Forces the specified C++ compiler to be used.

\item[{\tt -cc} {\em C-compiler}:]

Forces the specified C compiler to be used.

\item[{\tt -cp} {\em copy-file}:]

Creates a copy of the output file in {\em copy-file}.

\item[{\tt -cpp-option} {\em options}:]

Options passed to the C pre-processor.

\item[{\tt -ld} {\em linker}:]

Use this option only when compiling programs that do not include C++
modules.  Forces charmc to use the specified linker.

\item[{\tt -ld++} {\em linker}:]

Use this option only when compiling programs that include C++
modules.  Forces charmc to use the specified linker.

\item[{\tt -ld++-option} {\em options}:]

Options passed to the linker for {\tt -language charm++}.

\item[{\tt -ld-option} {\em options}:]

Options passed to the linker for {\tt -language converse}.

\item[{\tt -ldro-option} {\em options}:]

Options passes to the linker when linking {\tt .o} files.


\end{description}
