

\chapter{Timers, periodic checks, and conditions}

This module provides functions that allow users to insert hooks, that
is user supplied functions that are called by the system at specified
times, or as specific conditions arise.

{\bf  CcdCallOnCondition}
\begin{verbatim}
void CcdCallOnCondition(condnum,fnp,arg)
    int condnum;
    CcdVoidFn fnp;
    void *arg;      
\end{verbatim}

This called instructs the system to call the function indicated by the
function pointer fnp, with the specified argument arg, when the
condition indicated by condnum is raised next. Multiple functions may
be registered for the same condition number. A total of 511
conditions, numbered one through 511, are supported.  Currently, users
must make sure that various condition numbers used in a program are
disjoint.  (In future, we may provide a call to allocate the next
available condition number.)

The system supports a predefined condition, with condition number 1
(CcdPROCESSORIDLE, BUT I AM AM NOT SURE THAT IS EXPORTED TO THE USER
PROGRAM), which is raised by the system when they are no entities
(ready threads, messages for objects, posted handlers, etc.) in the
scheduler's queue.

\vspace*{0.2in}
{\bf  CcdRaiseCondition}
\begin{verbatim}
void CcdRaiseCondition(condNum)
     int condNum;
\end{verbatim}

When this function is called, it invokes all the functions whose
pointers were registered for the \verb#condNum# via a {\em prior} call
to CcdCallOnCondition(..).  All calls to CcdCallOnCondition(..) {\em
during} this function's execution take effect only after this function
returns.  Once a user-registered function is called, it loses its
registration.  So, if users want the function to be called the next
time the same condition arises, they must register the function again.

\vspace*{0.2in}
{\bf  CcdPeriodicallyCall}
\begin{verbatim}
void CcdPeriodicallyCall(fnp, arg)
    CcdVoidFn fnp;
    void *arg;  
\end{verbatim}

To function registered through this call is called periodically by the
system's scheduler.  Typically, it would be called every time the
scheduler gets control, such as while switching context to a new
thread or before scheduling a message for an object, or before calling
a posted handler. These functions don't have to be re-registered after
every call, unlike the functions for the "conditions". 

\vspace*{0.2in}
{\bf  CcdCallFnAfter}
\begin{verbatim}
void CcdCallFnAfter(fnp, arg, deltaT)
    CcdVoidFn fnp;
    void *arg;
    unsigned int deltaT; 
\end{verbatim}

This call registers a function via a pointer to it, fnp,  that will be
called at least deltaT milliseconds later. 
The registered function fnp is actually called the first time
scheduler gets control after deltaT milliseconds have elapsed. 
(This function may not work correctly at the moment! When corrected,
the time units may switch to seconds expressed as a "double"). 
