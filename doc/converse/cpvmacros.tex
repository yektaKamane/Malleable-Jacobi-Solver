%%%%%%%%%%%%%%%%%%%%%%%%%%%%%%%%%%%%%%%%%%%%%%%%%%%%%%%%%%%%%%%%%%%%%%%%%%%%
% RCS INFORMATION:
%
%       $RCSfile$
%       $Author$        $Locker$                $State$
%       $Revision$      $Date$
%
%%%%%%%%%%%%%%%%%%%%%%%%%%%%%%%%%%%%%%%%%%%%%%%%%%%%%%%%%%%%%%%%%%%%%%%%%%%%
% DESCRIPTION:
%
%%%%%%%%%%%%%%%%%%%%%%%%%%%%%%%%%%%%%%%%%%%%%%%%%%%%%%%%%%%%%%%%%%%%%%%%%%%%
% REVISION HISTORY:
%
% $Log$
% Revision 1.7  1997-06-26 05:16:22  jyelon
% Still improving manual.
%
% Revision 1.6  1997/06/25 06:53:48  jyelon
% Just trying to make this whole manual make more sense.
%
% Revision 1.5  1995/11/16 22:11:11  milind
% removed hardwired reference.
%
% Revision 1.4  1995/11/01  21:16:23  milind
% Added index entries.
%
% Revision 1.3  1995/10/30  17:33:54  milind
% Changed Cth variables macros to Ctv macros.
% Added "More on Shared Memory Machines" section.
%
% Revision 1.2  1995/10/27  21:02:54  milind
% Integrated into Manual.
%
% Revision 1.1  1995/10/26  21:08:23  gursoy
% Initial revision
%
%%%%%%%%%%%%%%%%%%%%%%%%%%%%%%%%%%%%%%%%%%%%%%%%%%%%%%%%%%%%%%%%%%%%%%%%%%%%

\chapter{Machine Model and Global Variables}

Converse treats the parallel machine as a collection of nodes, where
each node comprises of a number of processors that share memory.  In
some cases, the number of processors per node may be exactly one.  In
addition, each of the processors may have multiple threads running on
them which share code and data but have different stacks.

Different vendors are not consistent about how they treat global and static
variables.  Most vendors write C compilers in which global variables are
shared among all the processors in the node.  A few vendors write C
compilers where each processor has its own copy of the global variables.
In theory, it would also be possible to design the compiler so that
each thread has its own copy of the global variables.

The lack of consistency across vendors, makes it very hard to write a
portable program.  The fact that most vendors make the globals shared
is inconvenient as well, usually, you don't want your globals to be
shared.  For these reasons, we added ``pseudoglobals'' to Converse.
These act much like C global and static variables, except that you have
explicit control over the degree of sharing.

\section{Converse PseudoGlobals}

Three classes of pseudoglobal variables are supported: node-private,
process-private, and thread-private variables.

\begin{description}
\item[Node-private global variables] are specific to a node. They are
     shared among all the processes within the node.
\item[Process-private global variables]  are specific to a process. They are
     shared among all the threads within the process.
\item[Thread-private global variables] are specific to a thread. They are 
     truely private.
\end{description}

There are five macros for each class. These macros are for
declaration, static declaration, extern declaration, initialization,
and access. The declaration, static and extern specifications have the
same meaning as in C. In order to support portability, however, the
global variables must be installed properly, by using the
initialization macros.  For example, if the underlying machine is a
simulator for the machine model supported by Converse, then the
thread-private variables must be turned into arrays of variables.
Initialize and Access macros hide these details from the user.  It is
possible to use global variables without these macros, as supported by
the underlying machine, but at the expense of portability.

Macros for node-private variables:
\begin{verbatim}
CsvDeclare(type,variable)
CsvStaticDeclare(type,variable)
CsvExtern(type,variable)
CsvInitialize(type,variable)
CsvAccess(variable)
\end{verbatim}

\index{CsvDeclare}
\index{CsvStaticDeclare}
\index{CsvExtern}
\index{CsvInitialize}
\index{CsvAccess}

Macros for process-private variables:
\begin{verbatim}
CpvDeclare(type,variable)
CpvStaticDeclare(type,variable)
CpvExtern(type,variable)
CpvInitialize(type,variable)
CpvAccess(variable)
\end{verbatim}
\index{CpvDeclare}
\index{CpvStaticDeclare}
\index{CpvExtern}
\index{CpvInitialize}
\index{CpvAccess}

Macros for thread-private variables:
\begin{verbatim}
CtvDeclare(type,variable)
CtvStaticDeclare(type,variable)
CtvExtern(type,variable)
CtvInitialize(type,variable)
CtvAccess(variable)
\end{verbatim}
\index{CtvDeclare}
\index{CtvStaticDeclare}
\index{CtvExtern}
\index{CtvInitialize}
\index{CtvAccess}


A sample code to illustrate the usage of the macros is provided
in Figure~\ref{fig:cpv}.
There are a few rules that user must pay attention: The
{\tt type} and {\tt variable} fields of the macros must be a single
word. Therefore, structures or pointer types can be used by defining
new types with the {\tt typedef}. In the sample code, for example,
a {\tt struct point} type is redefined with a {\tt typedef} as {\tt Point}
in order to use it in the macros. Similarly,  the access macros contain
only the name of the global variable. Any indexing or member access
must be outside of the macro as shown in the sample code 
(function {\tt func1}). Finally, all the global variables must be
installed before they are used. One way to do this systematically is
to provide a module-init function for each file (in the sample code - 
{\tt ModuleInit()}. The module-init functions of each file, then, can be 
called at the beginning of execution to complete the installations of 
all global variables.

\begin{figure}
\begin{verbatim}
File Module1.c

    typedef struct point
    {
         float x,y;
    } Point;


    CpvDeclare(int, a);
    CpvDeclare(Point, p);

    void ModuleInit()
    {
         CpvInitialize(int, a)
         CpvInitialize(Point, p);

         CpvAccess(a) = 0;
    }

    int func1() 
    {
         CpvAccess(p).x = 0;
         CpvAccess(p).y = CpvAccess(p).x + 1;
    }
\end{verbatim}
\caption{An example code for global variable usage}
\label{fig:cpv}
\end{figure}

\section{Utility Functions}
\label{utility}

To further simplify programming with global variables on shared memory machines,
Converse provides following functions and/or macros. \note{These functions
are defined on machines other than shared-memory machines also, and
have the effect of only one processor per node and only one thread per 
processor.}

\function{int CmiMyRank()}
\index{CmiMyRank}
\desc{Returns the rank of the calling processor within a shared memory node.}

\function{void CmiNodeBarrier()}
\index{CmiNodeBarrier}
\desc{Provide barrier synchronization at the node level, i.e. all the 
processors belonging to the node participate in this barrier.}

\function{void *CmiSvAlloc(int size)}
\index{CmiSvAlloc}
\desc{Allocates a block of memory of \param{size} bytes from the heap in node's 
local memory and returns pointer to the start of this block. \note{On 
machines other than shared-memory machines, this is equivalent to 
\param{CmiAlloc}.}}
