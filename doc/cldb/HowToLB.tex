\documentstyle[11pt,fullpage]{article}
\pagestyle{headings}
\setlength{\textwidth}{6.5in}
\setlength{\textheight}{9in}
\setlength{\parindent}{0in}
\setlength{\topmargin}{-.5in}
\parskip 0.1in


%
% Constants
%
\newcommand{\version}{4.5}		%%% The current version number
\newcommand{\prevversion}{4.3}	%%% The previous version number

%
% Commands
%
\newcommand{\zap}[1]{ }
\newcommand{\fcmd}{\bf}		%%% Font for Charm commands
\newcommand{\fparm}{\it\sf}	%%% Font for parameters to Charm commands
\newcommand{\fexec}{\bf}	%%% Font for compile/execute cmds/options
\newcommand{\atitle}[1]{{\it #1}}
\newcommand{\keyword}[1]{{\textbf{#1}}}
\newcommand{\userword}[1]{{\fparm \textsc{#1}}}
\newcommand{\constraint}[1]{Note: {\it #1}}
\newcommand{\note}[1]{Note: {\it #1}}

\title{How To Write A Converse Load Balancer}
\author{Terry L. Wilmarth}

\begin{document}

\maketitle

\section{Introduction}

This manual details how to write your own load balancer in Converse.
A Converse load balancer can be used by any Converse program, but also
serves as the balancer of Charm++ chare creation messages.
Specifically, to use a load balancer, you would pass messages to
CldEnqueue rather than directly to the scheduler.  This is the default
behavior with chare creation message in Charm++.  Thus, the primary
provision of a new load balancer is an implementation of the
CldEnqueue function.

\section{Existing Load Balancers and Provided Utilities}

Throughout this manual, we will occasionally refer to the source code
of two provided load balancers, the random initial placement load balancer
({\tt cldb.rand.c}) and the graph-based load balancer ({\tt
cldb.graph.c}).  The functioning of these balancers will be described
in detail later.

In addition, a special utility is provided that allows us to add and
remove load-balanced messages from the scheduler's queue.  The source
code for this is available in {\tt cldb.c}.  The usage of this utility
will also be described here in detail.

\section{A Sample Load Balancer}

This manual steps through the design of a load balancer using an
example which we will call HELP!  The HELP! load balancer has each processor
periodically send half of its load to its neighbor in a ring.
Specifically, for N processors, processor K will send approximately half of
its load to (K+1)\%N, every 100 milliseconds (this is an example only;
we leave the genius approaches up to you).

\section{CldEnqueue}

The prototype for the {\bf CldEnqueue} function is as follows:

{\tt void {\bf CldEnqueue}(int pe, void *msg, int infofn);}

Here, {\tt pe} is the intended destination of the {\tt msg}.  It may
take on the values of:

\begin{itemize}
\item Any particular processor number - the message must be sent to
that processor
\item {\tt CLD\_ANYWHERE} - the message can be placed on any processor
\item {\tt CLD\_BROADCAST} - the message must be sent to all processors
excluding the local processor
\item {\tt CLD\_BROADCAST\_ALL} - the message must be sent to all processors
including the local processor
\end{itemize}

{\bf CldEnqueue} must handle all of these possibilities.  The only
case in which the load balancer should get control of a mesage is when
{\tt pe = CLD\_ANYWHERE}.  All other messages must be sent off to their
intended destinations and passed on to the scheduler as if they never
came in contact with the load balancer. 

The integer parameter {\tt infofn} is a handler index for a
user-provided function that supplies CldEnqueue with information about
the message {\tt msg}.  We will describe this in more detail later.

Thus, an implementation of the {\bf CldEnqueue} function might have
the following structure:

\begin{verbatim}
void CldEnqueue(int pe, void *msg, int infofn)
{
  ...
  if (pe == CLD_ANYWHERE)
    /* These messages can be load balanced */
  else if (pe == CmiMyPe())
    /* Enqueue the message in the scheduler locally */
  else if (pe==CLD_BROADCAST) 
    /* Broadcast to all but self */
  else if (pe==CLD_BROADCAST_ALL)
    /* Broadcast to all plus self */
  else /* Specific processor number was specified */
    /* Send to specific processor */
}
\end{verbatim}

In order to fill in the code above, we need to know more about the
message before we can send it off to a scheduler's queue, either
locally or remotely.  For this, we have the info function.  The
prototype of an info function must be as follows:

\noindent{\tt void ifn(void *msg, CldPackFn *pfn, int *len, int
*queueing, int *priobits, unsigned int **prioptr);}

Thus, to use the info function, we need to get the actual function via
the handler index provided to {\bf CldEnqueue}.  Typically, {\bf
CldEnqueue} would contain the following declarations:

\begin{verbatim}
  int len, queueing, priobits; 
  unsigned int *prioptr;
  CldPackFn pfn;
  CldInfoFn ifn = (CldInfoFn)CmiHandlerToFunction(infofn);
\end{verbatim}

\noindent Subsequently, a call to {\tt ifn} would look like this:

\begin{verbatim}
  ifn(msg, &pfn, &len, &queueing, &priobits, &prioptr);
\end{verbatim}

The info function extracts information from the message about its size,
queuing strategy and priority, and also a pack function, which will be
used when we need to send the message elsewhere.  For now, consider
the case where the message is to be locally enqueued:

\begin{verbatim}
  ...
  else if (pe == CmiMyPe())
    {
      ifn(msg, &pfn, &len, &queueing, &priobits, &prioptr);
      CsdEnqueueGeneral(msg, queueing, priobits, prioptr);
    }
  ...
\end{verbatim}

Thus, we see the info function is used to extract info from the
message that is necessary to pass on to {\bf CsdEnqueueGeneral}.

In order to send the message to a remote destination and enqueue it in
the scheduler, we need to pack it up with a special pack function so
that it has room for extra handler information and a reference to the
info function.  Therefore, before we handle the last three cases of
{\bf CldEnqueue}, we have a little extra work to do:

\begin{verbatim}
  ...
  else
    {
      ifn(msg, &pfn, &len, &queueing, &priobits, &prioptr);
      if (pfn) {
	pfn(&msg);
	ifn(msg, &pfn, &len, &queueing, &priobits, &prioptr);
      }
      CldSwitchHandler(msg, CpvAccess(CldHandlerIndex));
      CmiSetInfo(msg,infofn);
      ...
\end{verbatim}

Calling the info function once gets the pack function we need, if
there is one.  We then call the pack function which rearranges the
message leaving space for the info function, which we will need to
call on the message when it is received at its destination, and also
room for the extra handler that will be used on the receiving side to
do the actual enqueuing.  {\bf CldSwitchHandler} is used to set this extra
handler, and the receiving side must restore the original handler.

In the above code, we call the info function again because some of the
values may have changed in the packing process.  

Finally, we handle our last few cases:

\begin{verbatim}
  ...
      if (pe==CLD_BROADCAST) 
	CmiSyncBroadcastAndFree(len, msg);
      else if (pe==CLD_BROADCAST_ALL)
	CmiSyncBroadcastAllAndFree(len, msg);
      else CmiSyncSendAndFree(pe, len, msg);
    }
}
\end{verbatim}

\section{Other Functions}

A CldHandler function is necessary to receive messages forwarded by
CldEnqueue:

\begin{verbatim}
CpvDeclare(int, CldHandlerIndex);

void CldHandler(void *msg)
{
  CldInfoFn ifn; CldPackFn pfn;
  int len, queueing, priobits; unsigned int *prioptr;
  
  CmiGrabBuffer((void **)&msg);
  CldRestoreHandler(msg);
  ifn = (CldInfoFn)CmiHandlerToFunction(CmiGetInfo(msg));
  ifn(msg, &pfn, &len, &queueing, &priobits, &prioptr);
  CsdEnqueueGeneral(msg, queueing, priobits, prioptr);
}
\end{verbatim}

Note that the {\bf CldHandler} properly restores the message's original
handler using {\bf CldRestoreHandler}, and calls the info function to obtain
the proper parameters to pass on to the scheduler.

Also required is a CldModuleInit function:

\begin{verbatim}
void CldModuleInit()
{
  CpvInitialize(int, CldHandlerIndex);
  CpvAccess(CldHandlerIndex) = CmiRegisterHandler(CldHandler);
  CldModuleGeneralInit();

  /* call other init processes here */
  CldGraphModuleInit();
}
\end{verbatim}

Here's an example of an additional init function:

\begin{verbatim}
void CldGraphModuleInit()
{
  CpvInitialize(int, CldRelocatedMessages);
  CpvInitialize(int, CldLoadBalanceMessages);
  CpvInitialize(int, CldMessageChunks);

  CpvAccess(CldRelocatedMessages) = CpvAccess(CldLoadBalanceMessages) = 
    CpvAccess(CldMessageChunks) = 0;

  CldBalance();
}
\end{verbatim}

You may want to provide the three status variables, which get
initialized in your own module init function (called CldGraphModuleInit
above).  These can be used to keep track of what your LB is doing (see usage
in cldb.graph.c and itc++queens program).

\begin{verbatim}
CpvDeclare(int, CldRelocatedMessages);
CpvDeclare(int, CldLoadBalanceMessages);
CpvDeclare(int, CldMessageChunks);
\end{verbatim}

A method for queueing balanceable messages is provided in cldb.c.  That file
contains instructions for its use, and examples of its use can be found in
cldb.graph.c.  Its primary function is to provide a way to retrieve messages
from the scheduler queue that have not yet been processed, so that they may be
moved to another processor.


\section{The HELP! Load Balancer}

The HELP! Load Balancer is available in
charm/src/Common/conv-ldb/cldb.test.c.  To try out your own load balancer you
can use this filename and SUPER\_INSTALL will compile it and you can link it
into your Charm++ programs with -balance test.  (To add your own new balancers
permanently and give them another name other than "test" you will need to
change the Makefile used by SUPER\_INSTALL. Don't worry about this for now.)
The cldb.test.c provides a good starting point for new load balancers.

Look at the code for the HELP! balancer, starting with the {\bf CldEnqueue}
function.  This is almost exactly as described earlier.  One exception is the
handling of a few extra cases: specifically if we are running the program on
only one processor, we don't want to do any load balancing.  The other obvious
difference is in the first case: how do we handle messages that can be load
balanced?  Rather than enqueuing the message directly with the scheduler, we
make use of the token queue.  This means that messages can later be removed
for relocation.  {\bf CldPutToken} adds the message to the token queue on the
local processor.

Now look two functions up from {\bf CldEnqueue}.  We have an additional
handler besides the {\bf CldHandler}: the {\bf CldBalanceHandler}.  The
purpose of this special handler is to receive messages that can be still be
relocated again in the future.  Just like the first case of {\bf CldEnqueue}
uses {\bf CldPutToken} to keep the message retrievable, {\bf
CldBalanceHandler} does the same with relocatable messages it receives.

Next we look at our initialization functions to see how the process gets
started.  The {\bf CldModuleInit} function gets called by the common Converse
initialization code, and in turn, it calls our {\bf CldHelpModuleInit}.  This
function starts off the periodic load distribution process by making a call to
{\bf CldDistributeTokens}.  This function computes an approximation of half of
its total load ({\bf CsdLength}()), and if that amount exceeds the number of
movable messages ({\bf CldCountTokens}()), we attempt to move all of the
movable messages.  To do this, we pass this number of messages to move and the
number of the PE to move them to, to the {\bf CldMultipleSend} function.

{\bf CldMultipleSend} is generally useful for any load balancer that sends
multiple messages to one processor.  It takes parameters {\sl pe} and {\sl
numToMove}, and handles the packing and transmission of as many messages up to
{\sl numToMove} as it can find, to the processor {\sl pe}.  If the number
and/or size of the messages sent is very large, {\bf CldMultipleSend} will
transmit them in reasonably sized parcels.

That's all there is to the HELP! balancer.  Make the test version of
itc++queens and try it out.

\end{document}
